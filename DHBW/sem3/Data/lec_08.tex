\documentclass[a4paper]{article}

%\usepackage{url}

%% Math
\usepackage{mathtools}
%% For Mengen like natural numbers
\usepackage{amsfonts}
%% Für spezielle Symbole
\usepackage{amssymb}

%% Images
\usepackage{import}
\usepackage{xifthen}
\usepackage{pdfpages}
%\usepackage{transparent}

%%% Command for simpler images
\newcommand{\incfig}[1]{%
    \def\svgwidth{\columnwidth}
    \import{./fig/}{#1.pdf_tex}
}

%% Links
\usepackage{hyperref}
\hypersetup{
    colorlinks=true,
    linkcolor=black,
    filecolor=magenta,
    urlcolor=cyan
}

%% Formatting
\usepackage{parskip}
\usepackage{listings}

\title{Datenbanken}
\author{Moritz}
\date{November 5, 2025}

\begin{document}
\maketitle
\tableofcontents

\section{Normalisierung}

\subsection{Zerlegungen}

% Habe erneut nicht richtig aufgepasst

\subsection{Hüllen}

\begin{align*}
    a\to ed\\
    b\to ed\\
    ab\to c\\
    a\to b
\end{align*}

Die Hülle von b ist: b, e, d, denn die Hülle sind nur die direkten und sich selbst.

In den Kommentaren ist eine Funktion, um Hülle zu berechnen.

\begin{lstlisting}
Funktion(F, a):
    Erg:= a
    while (Aenderungen an Erg) do
        Foreach FD X \to Y in F do
            if X\subset Erg then Erg := Erg \cup Y
    Ausgabe a^+ = Erg
\end{lstlisting}

\subsection{Synthesealgorithmus}

Es gibt diesen wohl als Programm von der \href{https://normalizer.db.in.tum.de/}{TUM}.

\end{document}
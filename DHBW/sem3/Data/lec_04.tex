\documentclass[a4paper]{article}

%\usepackage{url}

%% Math
\usepackage{mathtools}
%% For Mengen like natural numbers
\usepackage{amsfonts}
%% Für spezielle Symbole
\usepackage{amssymb}

%% Images
\usepackage{import}
\usepackage{xifthen}
\usepackage{pdfpages}
%\usepackage{transparent}

%%% Command for simpler images
\newcommand{\incfig}[1]{%
    \def\svgwidth{\columnwidth}
    \import{./fig/}{#1.pdf_tex}
}

%% Links
\usepackage{hyperref}
\hypersetup{
    colorlinks=true,
    linkcolor=black,
    filecolor=magenta,
    urlcolor=cyan
}

%% Formatting
\usepackage{parskip}

\title{Datenbanken}
\author{Moritz}
\date{October 8, 2025}

\begin{document}
\maketitle
\tableofcontents

\section{Der konzeptionelle Datenbankentwurf}

\subsection{Kardinalitäten (Beziehungen)}

\subsubsection{Aggregationen}

Entweder "Teil von" oder "ist ein". So ist ein Motherboard ein Teil von einem Computer und ein RAM ist ein Speichermedium.

\subsubsection{Sichtenkonsolidiertung}

Wir sind ein Softwarehaus und sammeln von verschiedenen Stellen meines Kundens verschiedene Sichten auf deren Modell. Diese sind alle an sich richtig, müssen aber zusammengefügt werden.

Achtung: Es muss geprüft werden ob gleichgenannte Entitäten auch gleich sind. 

Achtung: Treten Widersprüche auf, was zwangsläufig der Fall sein wird, so müssen diese zusammen mit den Anwendern ausgeräumt werden.

Ziele: Es muss Redundanzfrei und Wiederspruchsfrei sein, sowie von Synonyme und Homonyme bereinigt sein.

\section{Prüfungsleistung}

90\% kommen von dem Projekt, die anderen 10\% kommen von der Klassifizierung von DBMs, welche als Kurs zu erledigen ist. Das Dokument ist auf moodle.

Hinweis zum Projekt. Einer soll das Dokument hochladen und es muss gekennzeichnet werden, wer was gemacht hat.

Einteilung: Niklas, Lars, Eric machen zusammen Graforientierte DBS

\section{Der logische Datenbankentwurf}

\subsection{Definitionen}

Das relationale Modell

Definition sind ab Kapitel 3 der Folien.

Eine n-stellige Relation R ist eine Teilmenge des kartesichen Produkts

\begin{equation*}
    R\subseteq \{M_1 \times M_2 \times \dots \times M_n\}
\end{equation*}

Der Wertebereiche $M_i$ heißen Domänen. Domänen sind atomar, d.h. keine zusammengesetzten und oder mengenwerigen Datentypen

\begin{align*}
    R &\subseteq D_1 \times D_2 \times \dots \times D_n\\
    Mitarbeiter &\subseteq string \times string \times int
\end{align*}

Ein Element r aus R mit $r=(a_1, a_2, a_3, ..., a_n)$ mit $a_i\in M_i$ für $i=1, \dots n$ heißt n-Tupel.

Das Relationenschemata.

Die Menge aller Attribute einer Relation (von n Domänen) ist das Schema([R]) der Relation.

Man unterscheidet:

\begin{itemize}
    \item Einer Instanz R (Entspricht dem aktuellen Zustand einer Tabelle. Bei dem Kurs ist die Instanz, die Anwesenheit)
    \item Einem Schema [R]
\end{itemize}

Ein Relationenschema wird folgendermaßen Definiert:

\begin{align*}
    [R]:& \{[A_1:D_1, \dots, A_n:D_n]\}\\
    [Mitarbeiter]:& \{[Nachname:string, Vorname:string, Gehalt:int]\}
\end{align*}

Der Primärschlüssel wird dem $Schema([Mitarbeiter])$ als erstes hinzugefügt


\subsection{Kartesichen Produkt}

Für das Beispiel von Folie S. 7 muss aufgepasst werden, denn eine Menge enthält jedes Produkt nur einmal. Somit müssen aus den Mengen $M_1, M_2, \dots, M_n$ die Doppelungen rausgenommen werden um die Gesamtanzahl des Produktes zu bestimmen.

Es dürfen somit keine Doppelten Tupel existieren.

\end{document}
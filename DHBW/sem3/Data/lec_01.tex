\documentclass[a4paper]{article}

%\usepackage{url}

%% Math
\usepackage{mathtools}
%% For Mengen like natural numbers
\usepackage{amsfonts}
%% Für spezielle Symbole
\usepackage{amssymb}

%% Images
\usepackage{import}
\usepackage{xifthen}
\usepackage{pdfpages}
%\usepackage{transparent}

%%% Command for simpler images
\newcommand{\incfig}[1]{%
    \def\svgwidth{\columnwidth}
    \import{./fig/}{#1.pdf_tex}
}

%% Links
\usepackage{hyperref}
\hypersetup{
    colorlinks=true,
    linkcolor=black,
    filecolor=magenta,
    urlcolor=cyan
}

%% Formatting
\usepackage{parskip}

\title{Datenbanken}
\author{Moritz}
\date{September 10, 2025}

\begin{document}
\maketitle
\tableofcontents

\section{Einführung und Grundlagen}

Was ist für mich in kurz der Begriff "Datenbanken":

\begin{enumerate}
    \item SQL (z.B. SELECT * from items)
    \item SQLight, PostgreSQL
    \item Relationale Datenbanken
    \item Keys (Primary und Foreign Keys)
    \item Geschwindigkeit
    \item ERM (Modelle)/ UML Modelle
    \item Skalierbarkeit
    \item Big Data
    \item Tabellen
    \item Datentypen
\end{enumerate}

\subsection{Definition}

Eine Datenbank ist ein logisch zusammengehörender
strukturierter Datenbestand.

Eine Datenbank ist eine Sammlung von Daten, die von einem
Datenbankmanagementsystem verwaltet wird.

Eigene: Somit ist eine Datenbank einfach eine Sammlung von Daten, welche von einem Datenbankmanagementsystem (DBMS) verwaltet werden. Ein Datenbanksystem beinhaltet DBMS und die Datenbanken. Siehe Slide S. 9

Ein Datenbankmanagementsystem (DBMS) ist die Systemsoftware eines Datenbanksystem (DBS) und dient zur Verwaltung der Daten (Konsistenz, Abfrage der Daten, Datenschutz, Zugriffsrechte….). 

\subsection{Literatur}

Die Vorlesung lehnt sich an Kemper an. Ist ein 900S Buch.

\begin{enumerate}
    \item Kemper, A./Eickler, A.: Datenbanksysteme: Eine Einführung, Oldenbourg Verlag
    \item Elmasri, R.A./Navathe, S.B.: Grundlagen von Datenbanksystemen, Pearson Studium
    \item Fraeskorn-Woyke, H./Bertelsmeier, B./Riemer, P./Bauer, E.: Datenbanksysteme, Pearson Studium
    \item Preiß, N.: Entwurf und Verarbeitung relationaler Datenbanken, Oldenbourg Verlag
    \item Saake, G./Sattler, K.-U./Heuer, A.: Datenbanken - Konzepte und Sprachen, mitp
\end{enumerate}

\subsection{Prüfungsleistung}

50\% Klausur am Ende des 4tem Semester über alle Themen

50\% Projekt über alle Themen bis SQL-Teil2 über beide Semester

\subsection{Grundlagen}

Daten bestehen aus Zeichen (Bits, Buchstaben, \dots). Informationen bestehen aus Daten und Informationen im Kontext sind Wissen.

Daten sind der Syntax

Informationen sind der Kontext (Struktur, Semantik)

Die Schrift ist also auch schon ein Datenbanksystem, so können Papyrus Rollen in einem Lage gespeichert werden und auch zugegriffen werden.

\know{Bonus Programm}{Zu jeder Stunde gibt es eine Fragenkatalog, ein Teil der Klausur besteht aus diesen Fragen (25\%)}

\subsubsection{Basisfunktionen nach Codd}

Die Codd'schen Regeln

\begin{enumerate}
    \item \textbf{Integration} - 
    Eine Informationen wird nur an einer Stelle gespeichert. (nicht redundant) 
    \newline- 
    Daten einheitlich, redundanzfrei ablegen
    \item \textbf{Operationen}
    CRUD Operationen müssen unterstützt sein - Erstellen, Ändern, Löschen, Lesen, diese machen was mit den Daten. \newline- 
    Abfragesprache zum Speichern, Suchen, Ändern
    \item \textbf{Katalog}
    Die Metadaten oder Datenbeschreibung von Daten. Somit "int" und "Produkt ID" .
    \newline- 
    Meta-Daten-Zugriff auf die Datenbeschreibungen "Data Dictionary"
    \item \textbf{Benutzersichten} - 
    Besser Benutzeransichten - Somit soll für unterschiedlichen Nutzungen unterschiedliche Ansichten dargestellt werden können. 
    \newline- 
    Verschiedene Rollen haben verschieden Sichten Auswahl relevanter Daten
    \item \textbf{Konsistenzüberwachung} - 
    Es soll nicht auf Daten verwiesen werden, die es nicht gibt.
    \newline- 
    logisch korrekter Zustand der Daten
    \item \textbf{Zugriffskontrolle} - 
    Es muss geprüft werden, ob ein einzelner Nutzer Zugriff auf die angefragte Daten hat. 
    \newline- 
    Datenschutz, Verhinderung unautorisierte Zugriffe
    \item \textbf{Transaktionen} - 
    Ein Transaktion sind mehrere Operationen die nur zusammen oder gar nicht ausgeführt werden.
    \newline- 
    Zusammenfassung von DB-Änderungen zu Funktionseinheiten die als Ganzes ausgeführt werden.
    \item \textbf{Synchronisation} - 
    Wenn mehrere Nutzer auf einen Datensatz zugreifen dürfen keine Konflikte entstehen.
    \newline-
    Überwachung paralleler Zugriffe vieler Nutzer u.a. Auflösen von Schreibkonflikten
    \item \textbf{Datensicherung} - 
    Dass die Daten müssen gesichert sein, selbst wenn das System nicht mehr existiert oder die Daten verloren gehen.
    \newline- 
    Wiederherstellung von Daten nach Systemfehlern
\end{enumerate}

Eine Gute Quelle um die obigen Ideen von mir zu Ergänzen ist \href{https://www.mirko-hans.de/info/gk_12/coddsche_regeln.htm}{hier}.

\subsubsection{Weitere Begriffe}

\textbf{Datenbankmodell}: Bei einem Datenbankmodell handelt es sich um die logische Struktur eines DBMS, welches beschreibt in welcher Form Daten strukturiert, modelliert und abgespeichert werden.

\textbf{Datenbankschema}: Das Datenbankschema beschreibt die Struktur des Datenbestandes, also die Metadaten. Wie beispielsweise hängen die Tabellen zusammen, welche Integritätsbedingungen bestehen usw.

\textbf{Datenbanksprache}: Für ein RDBMS ist die Datenbanksprache SQL (structured Query Language).

\begin{enumerate}
    \item DDL = Data Definition Language (Anlegen von Tabelle und der zugehörigen Struktur)
    \item DML = Data Manipulation Language (Einfügen und Löschen von Datensätzen)
    \item DQL = Data Query Language (Abfragen auf den Datenbestand)
\end{enumerate}

\subsection{Die 3-Ebenen Architektur}

Nach ANSI/SPARC auf Slide S.18:

\begin{enumerate}
    \item \textbf{externe} Ebene - Benutzeroberfläche, Benutzersichten, Schnittstellen
    \item \textbf{konzeptionelle} Ebene - Tabellen und Beziehungen
    \item \textbf{interne} Ebene - Dateiorganisation und Zugriffspfade
\end{enumerate}

Ablage der Daten ist immer noch die Aufgabe des Betriebssystem.

\end{document}
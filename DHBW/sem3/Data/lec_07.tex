\documentclass[a4paper]{article}

%\usepackage{url}

%% Math
\usepackage{mathtools}
%% For Mengen like natural numbers
\usepackage{amsfonts}
%% Für spezielle Symbole
\usepackage{amssymb}

%% Images
\usepackage{import}
\usepackage{xifthen}
\usepackage{pdfpages}
%\usepackage{transparent}

%%% Command for simpler images
\newcommand{\incfig}[1]{%
    \def\svgwidth{\columnwidth}
    \import{./fig/}{#1.pdf_tex}
}

%% Links
\usepackage{hyperref}
\hypersetup{
    colorlinks=true,
    linkcolor=black,
    filecolor=magenta,
    urlcolor=cyan
}

%% Formatting
\usepackage{parskip}

\title{Datenbanken}
\author{Moritz}
\date{October 29, 2025}

\begin{document}
\maketitle
\tableofcontents

\section{Normalisierung}

\subsection{Normalformen}

\subsubsection{2. Normalform}

Eine Relation erfüllt die zweiter Normalform (2NF), wenn diese sich in der ersten Normalform befindet und alle Nichtschlüsselattribute nur durch den gesamten Primärschlüssel festgelegt werden.

Alle Nichtschlüsselattribute müssen vollständig nur vom gesamten Primärschlüssel abhängig sein.

Wenn dies nicht der Fall ist, müssen die Attribtue die von "kleineren" Primär-schlüssel abhängig sind, kommen in Ihre eigene Relation.

\know{Hinweis}{Alle Relationen, bei denen der Primärschlüssel nur aus einem Attribut besteht, sind automatisch in der 2.NF.}

\subsubsection{3. Normalform}

Eine Relation ist in der 3.NF, wenn diese die 2.NF erfüllt und keine transitive Abhängigkeiten zwischen einem Nichtschlüsselattribute und einem Schlüssel-kandidaten besteht.

Formal: Ist in 3.NF, wenn alle FA die volgende Form haben:

\begin{equation*}
    X\to \alpha, \qquad X\subseteq [R] \text{ und } \alpha \in [R]
\end{equation*}

und mindestens ein der drei Bedingungen gilt:

\begin{enumerate}
    \item $\alpha \in X$ (Dann Triviale FA)
    \item $X$ ist ein Superschlüssel
    \item $\alpha$ ist in einem Superschlüssel enthalten ($\implies \alpha$ ist prim)
\end{enumerate}

Anleitung zur Prüfung auf 3.NF:

\begin{enumerate}
    \item Prüfen, ob eine Relation in 1NF und 2NF ist
    \item FA's aufstellen
    \item Schlüsselkandidaten definieren
    \item Auf Definition 3NF prüfen
    \item Wenn nicht in 3NF, dann zerlegen
\end{enumerate}

Tabelle um Definition 3NF zu prüfen:

\begin{center}
    \begin{tabular}{l|c|c}
        $X\to\alpha$ & Super Schl. & prim\\
        \hline
        Pro.Nr $\to$ Bez & true &  \\
        Pro.Nr $\to$ Begin & true &  \\
        Pers.Nr $\to$ VN & x & x \\
        Pers.Nr $\to$ NN & x & x \\
        Pro.Nr $\to$ Pers.Nr & true &  \\
        Pro.Nr $\to$ VN & true &  \\
        Pro.Nr $\to$ NN & true &  \\
    \end{tabular}
\end{center}

Somit ist diese Relation so noch nicht in der 3.NF. Dafür muss die Pers.Nr $\to\alpha$ noch in eine Neue Relation überführt werden.

\subsubsection{Boyce-Codd Normalform}

Eine Relation ist in Boyce-Codd Normalform, wenn jede Determinante ein Superschlüssel ist.

Identisch mit 3NF mit nur zwei, statt drei, Bedingungen:

\begin{enumerate}
    \item $\alpha \in X$ (Dann Triviale FA)
    \item $X$ ist ein Superschlüssel
\end{enumerate}

\subsection{Zerlegungen}

% Hier habe ich nicht aufgepasst

\end{document}
\documentclass[a4paper]{article}

%\usepackage{url}

%% Math
\usepackage{mathtools}
%% For Mengen like natural numbers
\usepackage{amsfonts}
%% Für spezielle Symbole
\usepackage{amssymb}

%% Images
\usepackage{import}
\usepackage{xifthen}
\usepackage{pdfpages}
%\usepackage{transparent}

%%% Command for simpler images
\newcommand{\incfig}[1]{%
    \def\svgwidth{\columnwidth}
    \import{./fig/}{#1.pdf_tex}
}

%% Links
\usepackage{hyperref}
\hypersetup{
    colorlinks=true,
    linkcolor=black,
    filecolor=magenta,
    urlcolor=cyan
}

%% Formatting
\usepackage{parskip}

\title{Datenbanken: Fragenkatalog}
\author{Moritz}

\begin{document}
\maketitle

Fragen sind auf moodle

\begin{enumerate}
    \item Datenbanken sind ein Topf an Daten, welche von einem DBMS verwaltet. DBs (Datenbank System), welches die beiden vorherige zusammenfasst.
    \item Daten sind werte (Messwerte) und Informationen sind die Daten in einem Kontext
    \item Man konnte Informationen speichern, über Lebensdauern eines Menschen hinweg. Wissen kann gespeichert und erweitert werden.
    \item Die Ordnung von Tontäfelchen hat nicht funktioniert. Daten wurden nicht wiedergefunden. Lösung: Einen Katalog erstellen.
    \item Persistente Daten gehen nicht verloren, selbst wenn das Program beendet wird. Sichere Daten sind Datensicherungen.
    \item Mann konnte nur ein Wurzelelement. Wir können nicht mehrere Wurzelelemente haben.
    \item Mit Verknüpfungen (Pfeile oder Zeiger)
    \item IBM hat das Konstrukt von Codd nicht verstanden, da es sehr theoretisch und mathematisch war.
    \item Das alte war zu kompliziert und konnte nur von Experten benutzt werden
    \item Es muss alle Bedingungen erfüllen um ein wahres DBMS nach Codd zu sein.
    \item Die Regeln von Codd: Beispiele
    \item Datenbankmodell: Wie werden Daten geordnet (Hirachisch, Relational), Datenbankschema (Die Metadaten, Tabellenüberschriften), Datenbanksprache (Operatoren ausführen: DDL, DML, DQL - Anlegen, bearbeiten, Anzeigen)
    \item externe Ebenen, konzeptionelle Ebene, interne Ebene. Die Ebenen sind Unabhängig voneinander. Es ist eine Abstraktion zur Entwicklung der Ebenen einzeln. Komplexität verringern.
    \item Data Dictionary ist die Verwaltung der Metadaten (Attribute, Datentyp, \dots). Anfrageverwaltung nimmt die Anfragen an und versucht diese besser zu strukturieren. Transaktionsmanager macht aus mehreren Eine und wird entweder gesamt oder gar nicht ausgeführt. Recovery Manager logged die Daten und kann bei Fehler die Abfragen rückwerts abwickeln.
    \item Datenbankentwurfsprozess ist die engineer mäßige herangehensweise
    \item Die einzelnen Schritte sind: Anforderungsanalyse, Konzeptioneller Entwurf, logischer Entwurf, Datenbank-Definition, Physikalischer Entwurf
    \item ERM Graphisch, Die verschieden SQL Bereiche (DDL, DML, DQL)
    \item Eine Anforderungsanalyse die Sammlung und Dokumentation von den Anforderungen des Kundens.
    \item konzeptioneller Datenbank Entwurf dient der Modellierung. Es wird also etwas aus der realen Welt modelliert.
    \item Eine Entität Typ ist das Object und die Entität ist der einzelnen z.B. Student
    \item Das ERM besteht aus dem Entitäts Typen, deren Beziehungen und die Attribute von den Entitäts Typen.
    \item Der Wertebereich eines Attributs sind die Gültigen Werte eines Attributs. So hat das Geschlecht den Wertebereich: "Männlich, Weiblich, Divers, Nicht angegeben"
    \item Schlüsselattribute sind eindeutige Indentifikator für das Tupel (Eine Zeile, eine Entität).
    \item Ein Mengenattribut wird als neuer Entitätstyp implementiert.
    \item Eine Beziehung wird als Handlung von rechts nach links gelesen und sollte ein Satz ein.
    \item Rollen hinzufügen um klar zu machen wer handelt.
    \item Chen-Notation vs. (min, max). Die Grenzen sind unterschiedlich und die leserichtung ändert sich. Chen von der Entität weg und bei (min, max) steht es bei der ausgehenden Entität.
    \item Terniär-Beziehung zwischen drei Entitäten. Einsatz, wenn binäre Beziehungen nicht konsistent dies darstellen können.
    \item Wenn eine Entität mit sich selber in Beziehung steht. Hier muss es Rollen geben, da sonst die Beziehung nicht eindeutig ist.
    \item Spezialisierung als Prüfungsfrage: Sie haben Spezialisierungen verwendet und der Kunde fragt wofür wir diese brauchen. Antwort: Ich habe Entität die nur bestimmte Ausprägungen habe und allgemeiner Ausprägungen von oberen Entitäten besitzen. Dient zur Verhinderung von leeren Feldern in der Datenbank. Also Generalisierung (unten nach oben) oder Spezialisierung (oben nach unten)
    \item Attribute und Beziehungen werden vererbt.
    \item Partionierung ist die Einteilung in verschieden Teile (Parts). Eine Aggregation ist \dots
    \item existenzabhängige Entitäten werden mit der oberen Entität gemeinsam gekillt. In dem ERM wird dies mit einer doppelte Linie gezeichnet. Erkennen kann ich das, wenn es Abhängigkeiten gibt.
    \item Sichten konsolidieren, verschieden Gruppen haben unterschiedliche Sichten in der Datenbank.
    \item Der logische Datenbankentwurf ist die übersetzung des konzeptionellem Modell ind ein bestimmtes Datenbankmodell. Bei uns ER zu Relational.
    \item $R\subseteq D_1\times D_2\times \dots \times D_n$
    \item Der Wertebereich oder Domäne sind die Datentypen.
    \item $[R] = Schema$ und $R = Instanz$
    \item Ein Schwacher Entitätstyp wird in ERM umgesetzt, indem diese einen Fremdschlüssel und zusammen mit einem anderen sind diese dann der Primärschlüssel.
    \item Wir machen aus einer Beziehung drei Relationen. Die beiden Entitäten jeweils eine und die Beziehung auch eine.
    \item Ich habe drei Relationen und lege alle Schlüssel fest und ziehe die beiden Relationen zusammen die den gleichen Primärschlüssel haben.
    \item Wir verwenden die maxmin Notation. Somit kommt der Fremdschlüssel dorthin, wo die wenigsten NULL Werte sind.
    \item Wir können alles in eine Tabelle einfügen, resultiert in viele NULL Werte. Wir können eine Fremdschlüsselbeziehung durchführen. Die Obere Tabelle weglassen und in die Spezialisierung einfügen.
    \item Das liegt daran, das m aus der einen mit n aus der anderen eine Beziehung haben. Ich habe also keine Schlüssel die ich zusammenziehen kann.
    \item Das es keine ungültige Fremdschlüssel gibt. Dies passiert, wenn ein Tupel gelöscht wird, wenn noch darauf verwiesen wird.
    \item Zum aufdecken von Mängeln im Datenbankentwurf.
    \item Eine Anomalie beschreibt die Abweichung von Normen und es gibt Einfügeanomalie, Löschanomalie und Änderungsanomalie.
    \item Eine Funktionale Abhängigkeit (FA) heißt ein oder mehrere Attribute zeigen eindeutig auf andere Attribute.
    \item Eine Triviale FA heißt es bildet auf sich selbst ab oder auf eine Teilmenge. $X\to X$ oder $XY\to X$
    \item Eine Hülle $(F^+)$ sind alle möglichen FA die aus den Grundfunktion bilden kann, mithilfe von dem Armstrong-Axiome 
    \item Ein Superschlüssel kann reduziert werden und ein Schlüsselkandidat sind vollständig reduziert.
    \item Die 1NF (1. Normalform) ist die Definition von Relationen.
    \item Wenn der Primärschlüssel nur aus einem Attribut besteht.
    \item Wenn ein Nicht Schlüsselattribute von einem anderem Nicht Schlüsselattribute abhängig ist.
    \item Ein Primattribut ist ein Attribut, welches in einem Kanditatenschlüssel.
    \item Wenn die linke Seite (Determinate) immer ein Superschlüssel ist (Erweiterung von 3NF)
    \item Zerlegungsfrage 1
    \item Zerlegungsfrage 2
    \item Wenn alle Funktionalen Abhängigkeiten aus der Ursprünglichen Relation in der Zerlegung erhalten bleiben, dann ist eine Zerlegung abhängigkeitsbewahrend. 
\end{enumerate}

\end{document}
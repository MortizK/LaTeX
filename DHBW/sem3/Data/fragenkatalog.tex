\documentclass[a4paper]{article}

%\usepackage{url}

%% Math
\usepackage{mathtools}
%% For Mengen like natural numbers
\usepackage{amsfonts}
%% Für spezielle Symbole
\usepackage{amssymb}

%% Images
\usepackage{import}
\usepackage{xifthen}
\usepackage{pdfpages}
%\usepackage{transparent}

%%% Command for simpler images
\newcommand{\incfig}[1]{%
    \def\svgwidth{\columnwidth}
    \import{./fig/}{#1.pdf_tex}
}

%% Links
\usepackage{hyperref}
\hypersetup{
    colorlinks=true,
    linkcolor=black,
    filecolor=magenta,
    urlcolor=cyan
}

%% Formatting
\usepackage{parskip}

\title{Datenbanken: Fragenkatalog}
\author{Moritz}

\begin{document}
\maketitle
\tableofcontents

Fragen sind auf moodle

\begin{enumerate}
    \item Datenbanken sind ein Topf an Daten, welche von einem DBMS verwaltet. DBs (Datenbank System), welches die beiden vorherige zusammenfasst.
    \item Daten sind werte (Messwerte) und Informationen sind die Daten in einem Kontext
    \item Man konnte Informationen speichern, über Lebensdauern eines Menschen hinweg. Wissen kann gespeichert und erweitert werden.
    \item Die Ordnung von Tontäfelchen hat nicht funktioniert. Daten wurden nicht wiedergefunden. Lösung: Einen Katalog erstellen.
    \item Persistente Daten gehen nicht verloren, selbst wenn das Program beendet wird. Sichere Daten sind Datensicherungen.
    \item Mann konnte nur ein Wurzelelement. Wir können nicht mehrere Wurzelelemente haben.
    \item Mit Verknüpfungen (Pfeile oder Zeiger)
    \item IBM hat das Konstrukt von Codd nicht verstanden, da es sehr theoretisch und mathematisch war.
    \item Das alte war zu kompliziert und konnte nur von Experten benutzt werden
    \item Es muss alle Bedingungen erfüllen um ein wahres DBMS nach Codd zu sein.
    \item Die Regeln von Codd: Beispiele
    \item Datenbankmodell: Wie werden Daten geordnet (Hirachisch, Relational), Datenbankschema (Die Metadaten, Tabellenüberschriften), Datenbanksprache (Operatoren ausführen: DDL, DML, DQL - Anlegen, bearbeiten, Anzeigen)
    \item externe Ebenen, konzeptionelle Ebene, interne Ebene. Die Ebenen sind Unabhängig voneinander. Es ist eine Abstraktion zur Entwicklung der Ebenen einzeln. Komplexität verringern.
\end{enumerate}

\end{document}
\documentclass[a4paper]{article}

%\usepackage{url}

%% Math
\usepackage{mathtools}
%% For Mengen like natural numbers
\usepackage{amsfonts}
%% Für spezielle Symbole
\usepackage{amssymb}

%% Images
\usepackage{import}
\usepackage{xifthen}
\usepackage{pdfpages}
%\usepackage{transparent}

%%% Command for simpler images
\newcommand{\incfig}[1]{%
    \def\svgwidth{\columnwidth}
    \import{./fig/}{#1.pdf_tex}
}

%% Links
\usepackage{hyperref}
\hypersetup{
    colorlinks=true,
    linkcolor=black,
    filecolor=magenta,
    urlcolor=cyan
}

%% Formatting
\usepackage{parskip}

\title{Datenbanken}
\author{Moritz}
\date{October 1, 2025}

\begin{document}
\maketitle
\tableofcontents

\section{Der konzeptionelle Datenbankentwurf}

\subsection{Kardinalitäten (Beziehungen)}

Bei einer Beziehung ist die Leserichtung wichtig. Sollte diese nicht ausreichen um den Kontext deutlich darzustellen, können Rollen zusätzlich hinzugefügt werden.

So: Ein Kunde (Räuber) raubt eine Bank (Beraubte) oder: Ein Kunde (Beraubter) wird von einer Bank (Räuber) beraubt.

\subsubsection{Notationen}

Es gibt verschiedene Notationen um Beziehungen zu beschriften. So gibt es die Chen-Notation (1:1, 1:n, n:n) und die Krähenfuß-notation:

\begin{itemize}
    \item $-|---|-$
    \item $-|---|<$
    \item $>|---|<$
\end{itemize}

\subsubsection{Min., Max. Notation}

So wird statt M und N einfach Klammern mit min und max verwendet. Somit: $(0,^*)$ und $(5,20)$

Die ließt sich wie folgt. Ein Mitarbeiter besucht $(0,^*)$ Kurse. Ein Kurs wird von $(5,20)$ Mitarbeiter besucht. wird für die min., max. Notation die Richtung vertauscht. 

Dies ist die gewünschte Notation, wenn die Konkreten Zahlen gibt.

\subsubsection{Aufgabe}

Es gab eine Aufgabe zu der Übersetzung der Chen-Notation zur Min., Max. Notation.

\textbf{Konsistenzbindungen} sind z.B.:

\begin{itemize}
    \item Ein Projektleiter betreut mehrere Kunden und Projekte und ein Projekt muss von mindestens einem Projektleiter betreut werden.
    \item weitere \dots
    \item Ein Kunde mit seinem Projekt hat genau einen Projektleiter.
\end{itemize}

\subsubsection{merhstellige Beziehungen}

Die dritte Bedingung lässt sich durch durch mehrstellige Beziehungen darstellen.

Um Dies Umzusetzen muss man in Schlüssel Denken. So bilden Ein Kunde und ein Projekt einen Schlüssel für die Projektleitung.

\subsubsection{Rekursive Beziehungen}

Ein mitarbeiter hat einen Chef, Ein Chef hat mehrere Mitarbeiter.

\subsubsection{Spezialisierung}

Ein Projektmitarbeiter ist ein Spezieller Mitarbeiter. Diese Beziehung wird durch ein Sechseck/ Hexagon gekennzeichnet. Die Allgemeinere Entität ist Oben und die Spezialisierten Mitarbeiter sind unten.

\subsubsection{Aggregation}

Ein Auto besteht aus einem Motor, 4 Rädern, usw. Diese sind vom Auto abhängig. Sollte das Auto 500km fahren, so ist auch der Motor 500km gefahren.

Ist eine Raute mit der Beschriftung: "ist Teil von"

\subsubsection{Schwache Entitäts-Typen}

Diese sind existenzabhängig. Hierzu werden die Linien einer Raute, Verbindung oder Entität werden doppelt gezeichnet: "=="

\end{document}
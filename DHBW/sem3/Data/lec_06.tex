\documentclass[a4paper]{article}

%\usepackage{url}

%% Math
\usepackage{mathtools}
%% For Mengen like natural numbers
\usepackage{amsfonts}
%% Für spezielle Symbole
\usepackage{amssymb}

%% Images
\usepackage{import}
\usepackage{xifthen}
\usepackage{pdfpages}
%\usepackage{transparent}

%%% Command for simpler images
\newcommand{\incfig}[1]{%
    \def\svgwidth{\columnwidth}
    \import{./fig/}{#1.pdf_tex}
}

%% Links
\usepackage{hyperref}
\hypersetup{
    colorlinks=true,
    linkcolor=black,
    filecolor=magenta,
    urlcolor=cyan
}

%% Formatting
\usepackage{parskip}

\title{Datenbanken}
\author{Moritz}
\date{October 22, 2025}

\begin{document}
\maketitle
\tableofcontents

\section{Der logische Datenbankentwurf}

Schritte für ERM zu Relationen

\begin{enumerate}
    \item Alle Entitäten bilden
    \item Alle Beziehung bilden
    \item Alle Relationen mit dem gleichem Primärschlüssel zusammenfügen (außer bei Spezifikation)
    \item \dots
\end{enumerate}

Das logische Model ist jetzt abgeschlossen.

\section{Normalisierung}

Anomalien:

\begin{itemize}
    \item Einfügeanomalie
    \item Löschanomalie
    \item Änderungsanomalie
\end{itemize}

Durch dämliche Struktur der Schemas, ist es gut möglich das Fehler passieren. z.B. die Anomalien.

\subsection{Projektion $(\Pi)$}

Funktioniert wie das select * oder select Vorname, Nachname

Formal (mit R als Schema):

\begin{equation*}
    R' := \Pi_{[R']}(R), [R'] \subseteq [R]
\end{equation*}

Die Funktionale Abhängigkeit (FA; oder English Functional dependency (FD)) ist die Grundlage der Normalisierung.

Für $X, Y\subseteq [R], X\to Y$ ist funktional abhängig, wenn $f(X) = Y$ gilt.

Eine Triviale FA ist Trivial, wie $X\to X$, wie auch obermengen von $X$: $Y, X\to X$ mit $X$ dem Primärschlüssel und $Y$ anderen Attributen.

Wir müssen die FAs für eine bestimmte Instanz prüfen:

\begin{center}
    \begin{tabular}{ccc|c}
        A & B & C & R\\
        \hline
        1 & 10 & 100 & 1\\
        2 & 20 & 200 & 1\\
        1 & 10 & 500 & 1\\
        \hline
        2 & 20 & 100 & 2\\
        \hline
        3 & 20 & 600 & 3\\
    \end{tabular}
\end{center}

\begin{align*}
    R_1 &= A\to B, B\to A, C\to A, C\to B\\
    R_2 &= A\to B, B\to A\\
    R_3 &= A\to B
\end{align*}

\subsection{Armstrong-Axiome}

Die Regeln sind auf S. 13 Abgebildet.

\begin{enumerate}
    \item Reflexivität: $Y\in X \implies X\to Y$
    \item Verstärkung: $X\to Y \implies XZ\to YZ$
    \item Transitivität: $X\to Y, Y\to Z\implies X\to Z$
    \item Zerlegung: $X\to YZ \implies X\to Y, X\to Z$
    \item Vereinigung: Gegenteil von 4.
    \item Pseudotransitivität: $X\to Y, WY\to Z \implies WX \to Z$
\end{enumerate}

\subsection{Schlüssel}

Formale gesprochene Schlüssel. So gibt es einen \textbf{Superschlüssel} (aus dem Englischen Subset). Mit dem Superschlüssel lässt sich auf die gesamte Relation abbilden.

Dieser muss nicht minimal sein. So kann folgendes Gelten:

\begin{equation*}
    \exists a\in s_i: s_i/\{a\}\to [R]
\end{equation*}

Es kann mehrere \textbf{Schlüsselkandidaten} $\subseteq$ Superschlüssel

\begin{align*}
    \nexists a\in s_i:s_i/\{a\}\to [R]
    \forall a\in s_i: \not(s_i/\{a\} \to [R])
\end{align*}

Ein \textbf{Primärschlüssel} ist ein (willkürlich) ausgewählter Schlüsselkandidat.

\begin{itemize}
    \item Ein Superschlüssel ist Reduzierbar und identifiziert ein Tupel
    \item Ein Schlüsselkandidat kann nicht weiter Reduziert werden
    \item Ein Primärschlüssel ist ein willkürlich gewählter Schlüsselkandidat
\end{itemize}

Ein Attribut wird als \textbf{Primattribut} oder prim bezeichnet, wenn das Attribut in einem Schlüsselkandidaten vorkommt.

\subsection{Normalformen}

\subsubsection{1. Normalform}

Eine Relation erfüllt die erste Normalform, wenn jede Entität für jedes Attribut der Relation nur einen Datenwert besitzt.

Ist durch die Definition von Relation schon erfüllt.

Wenn wir z.B. bei Abschlüssen mehrere Wirtschaftsinformatiker, welche weitere Abschlüsse haben die als Klartext drin stehen, können Rechtschreibfehler zu Anomalien führen. Dies würde die erste Normalform nicht erfüllen.

\subsubsection{2. Normalform}

Eine Relation erfüllt die zweiter Normalform (2NF), wenn diese sich in der ersten Normalform befindet und alle Nichtschlüsselattribute nur durch den gesamten Primärschlüssel festgelegt werden.

\end{document}
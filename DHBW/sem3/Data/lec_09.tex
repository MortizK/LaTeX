\documentclass[a4paper]{article}

%\usepackage{url}

%% Math
\usepackage{mathtools}
%% For Mengen like natural numbers
\usepackage{amsfonts}
%% Für spezielle Symbole
\usepackage{amssymb}

%% Images
\usepackage{import}
\usepackage{xifthen}
\usepackage{pdfpages}
%\usepackage{transparent}

%%% Command for simpler images
\newcommand{\incfig}[1]{%
    \def\svgwidth{\columnwidth}
    \import{./fig/}{#1.pdf_tex}
}

%% Links
\usepackage{hyperref}
\hypersetup{
    colorlinks=true,
    linkcolor=black,
    filecolor=magenta,
    urlcolor=cyan
}

%% Formatting
\usepackage{parskip}
\usepackage{listings}

\title{Datenbanken}
\author{Moritz}
\date{November 12, 2025}

\begin{document}
\maketitle
\tableofcontents

\section{Äquivalenz}

\begin{lstlisting}
    A -> BC
    B -> AC
    C -> B
\end{lstlisting}

\begin{lstlisting}
    A -> C
    C -> ABC
    B -> C
    A -> B
\end{lstlisting}

Sind diese Beiden Äquivalent:

Hierzu müssen wir beide Relation links und rechts reduzieren und die Hülle unverändert bleibt und schauen, ob beide Relation die gleiche Minimierung haben.

Also:

\begin{itemize}
    \item wir schauen bei A -$>$ BC ob wir einen von beiden oder beide weglassen können.
    \item Also A -$>$ C (B weg) hat immer noch die Hülle $A^+\{A,C,B\}$
    \item Noch A -$>$ \{\} (B,C weg) hat $A^+\{A\}$
    \item Weiter mit B -$>$ AC
    \item B -$>$ C (A weg) hat $B^+\{B,C\}$
    \item B -$>$ A (C weg) hat $B^+\{A, B, C\}$
    \item noch für C -$>$ B
\end{itemize}

Und bei Relation zwei:

\begin{itemize}
    \item A -$>$ \{\} (C weg) hat $A^+\{A,B,C\}$, also kann diese weg
    \item C -$>$ AB (C weg) hat $C^+\{A, B, C\}$
    \item C -$>$ A (B,C weg) hat $C^+\{A, B, C\}$, dies geht noch
    \item C -$>$ \{\} (A,B,C weg) hat $C^+\{C\}$, geht nicht mehr
    \item B -$>$ \{\} (C weg) hat $B^+\{B\}$, geht nicht
    \item A -$>$ \{\} (B weg) hat $A^+\{A\}$, geht nicht
\end{itemize}

jetzt haben wir:

\begin{lstlisting}
    A -> C
    B -> A
    C -> B
\end{lstlisting}

\begin{lstlisting}
    C -> A
    B -> C
    A -> B
\end{lstlisting}

Jetzt müssen wir für die Äquivalenz beides in beides umformen können.


\section{Dekomposition}

Wir können funktionale Abhängigkeiten durch Dekomposition zerlegen.

Dies kommt nicht in der Klausur ran.

\end{document}
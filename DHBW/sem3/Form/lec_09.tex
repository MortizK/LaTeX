\documentclass[a4paper]{article}

%\usepackage{url}

%% Math
\usepackage{mathtools}
%% For Mengen like natural numbers
\usepackage{amsfonts}
%% Für spezielle Symbole
\usepackage{amssymb}

%% Images
\usepackage{import}
\usepackage{xifthen}
\usepackage{pdfpages}
%\usepackage{transparent}

%%% Command for simpler images
\newcommand{\incfig}[1]{%
    \def\svgwidth{\columnwidth}
    \import{./fig/}{#1.pdf_tex}
}

%% Links
\usepackage{hyperref}
\hypersetup{
    colorlinks=true,
    linkcolor=black,
    filecolor=magenta,
    urlcolor=cyan
}

%% Formatting
\usepackage{parskip}

\title{Formale Sprachen}
\author{Moritz}
\date{November 14, 2025}

\begin{document}
\maketitle
\tableofcontents

\section{While Program}

ab S. 446.

Dies besteht aus einer Einfachen Grammatik, welche wir auf Papier gelöst haben.

Es lassen sich alle Turing Maschinen als While Program umsetzen.

Unser While beendet nur, wenn die Variable gleich Null gesetzt wird. Für andere Bedingungen müssen separate berechnet werden und in die Variable der While Schleife geschrieben werden.

Aus diesen Standardblöcken der while Schleife und Zuweisungen können wir auch die anderen Blöcke: for, if else umsetzen.

Wir können aber auch mithilfe von Schleifen und Konstanten eine Addition von zwei Variablen durchführen.

\section{Klausur}

Wir haben eine Übungsklausur durchgeführt und dann auch bis Aufgabe 7 Korrigiert.

\end{document}
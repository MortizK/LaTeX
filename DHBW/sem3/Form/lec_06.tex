\documentclass[a4paper]{article}

%\usepackage{url}

%% Math
\usepackage{mathtools}
%% For Mengen like natural numbers
\usepackage{amsfonts}
%% Für spezielle Symbole
\usepackage{amssymb}

%% Images
\usepackage{import}
\usepackage{xifthen}
\usepackage{pdfpages}
%\usepackage{transparent}

%%% Command for simpler images
\newcommand{\incfig}[1]{%
    \def\svgwidth{\columnwidth}
    \import{./fig/}{#1.pdf_tex}
}

%% Links
\usepackage{hyperref}
\hypersetup{
    colorlinks=true,
    linkcolor=black,
    filecolor=magenta,
    urlcolor=cyan
}

%% Formatting
\usepackage{parskip}

\title{Formale Sprachen}
\author{Moritz}
\date{October 30, 2025}

\begin{document}
\maketitle
\tableofcontents

\section{Formale Grammatiken \& Kontext freie Sprachen}

\subsection{Kellerautomat}

Aus einer Grammatik ein PDA (Kellerautomat) machen.

Für eine Grammatik $G=(N, \Sigma, P, S)$

\begin{align*}
    A_G = &(\{g\}, \Sigma, \Sigma \cup N, \Delta, q, S)\\
    \Delta = &\{(q, \epsilon, A, \gamma, q) \mid A\to \gamma\in P\} \cup\\
    &\{(q, a, a, \epsilon, q) \mid a\in \Sigma\}
\end{align*}

Für die Umwandlung kann ein PDA einfach die Entiwcklungsregeln befolgen um den Stack aufzubauen.

Wenn in dem Zusammenhang ein nicht Terminalsysbol auftaucht kann dies mit einem Buchstaben des Wortes konsumiert werden.

\begin{align*}
    S\to AA\\
    A\to a\mid AA\\
    (S, aaa, S)\to(S, aaa, AA)\to(S, aaa, aA)\\
    \to(S, aa, A)\to(S, aa, AA)\to(S, aa, aA)\\
    \to(S, a, A)\to(S, \epsilon, \epsilon)
\end{align*}

Das ist genau was die $\Delta$-Regeln oben beschreiben.

Die andere Richtung ist komplizierter! Die Trivialen, welche den Stack verbrauchen:

\begin{itemize}
    \item $0 \epsilon Z \to \epsilon 0$ wird zu $[0Z0] \to \epsilon$
    \item $0 b A \to \epsilon 1$ wird zu $[0A1] \to \epsilon$

\end{itemize}

Und die komplexeren, z.B. $0 a Z \to AZ 0$:

\begin{itemize}
    \item $[0Z0] \to a[0A0][0Z0]$
    \item $[0Z0] \to a[0A1][1Z0]$
    \item $[0Z1] \to a[0A0][0Z1]$
    \item $[0Z1] \to a[0A1][1Z1]$
\end{itemize}

Wenn wir dies für alle gemacht haben müssen wir nun:

\begin{enumerate}
    \item Die Terminal Symbole bestimmen
    \item Die Erreichbaren Symbole bestimmen
\end{enumerate}

Ein Sinnvoller Aufschriebt ist auf einem Zettel.

\subsection{Eigenschaften von Kontext Freien Sprachen}

% Fehlt mir, weil ich snake replay gemacht habe.

\subsubsection{Pumping Lemma II}

Diese kann für Kontext Freie Sprachen ein Wort aufpumpen.

Es besagt, dass in beliebig langen Wörter es ein Nicht Terminierendes Symbol (NTS) gibt, welches aufgepumpt werden kann. Diese produziert dann mehrere Terminierende Symbole (TS) in Parallel. 

Das Theorem ist auf S. 307.

Es gibt somit wieder eine einteilung in mit $s\in L$ und $|s|>k$:

\begin{equation*}
    u*v*w*x*y = s
\end{equation*}

Sodass folgendes gilt:

\begin{enumerate}
    \item $vx \neq \epsilon$, somit $|vx|>0$
    \item $|vwx| \leq k$
    \item $u*v^h*w*x^h*y \in L$ für alle $h\in \mathbb{N}$
\end{enumerate}

\subsubsection{Closure properties}

Kontext freie Sprachen sind geschlossen unter Vereinigung, concatenation und Kleene star $(\cup, ., ^*)$

Sie sind nicht geschlossen unter Gemeinsamkeiten (intersection $\cap$)

Übung S.311. Übersetzen der Sprachen in die Grammatiken

\begin{align*}
    G_1 = & (\{S, A\}, (a, b, c), P, S)\\
    P: & S \to Sc \mid A\\
    & A \to aAb \mid \epsilon\\
    G_2 = & (\{S, B\}, (a, b, c), P, S)\\
    P: & S \to Sa \mid B\\
    & B \to bBc \mid \epsilon\\
\end{align*}

Danach die Geschlossenheit über Komplement wiederlegen:

\begin{align*}
    L = L_1 \cap L_2 \mid ^c\\
    L^c = L_1^c \cup L_2^c \mid ^c\\
    L = (L_1^c \cup L_2^c)^c
\end{align*}

Dies zeigt: Das es nicht abgeschlossen ist!

\subsubsection{Decision problems}

\pagebreak
\section{Turing Machine}

Hat ein Tape was in beide Richtungen unendlich Viele Speicherplätze hat. Diese kann gelesen und beschrieben werden.

Um auf diesem Tape zu bewegen gibt es Movement Instructions (l, r, n)

Eine Turing Machine ist ein Automat, der mit dem Tape interagieren kann. Eine Transition sieht somit wie folgt aus:

\begin{center}
    p a $\to$ b d q
\end{center}

\begin{center}
    \begin{tabular}{c|l}
        p & aktueller Zustand\\
        a & einlesen von Tape\\
        \hline
        b & überschreiben auf Tape\\
        d & Zeiger bewegen [l, n, r]\\
        q & neue Zustand\\
    \end{tabular}
\end{center}

Die Turing Machine hört auf, wenn diese keine Transition mehr gehen kann. Danach schaut diese ob Sie in einem Akzeptierenden Zustand ist.

Ein Konfiguration stellt sich zusammen aus den schon eingelesenen Buchstaben, gefolgt von dem aktuellem Zustand und den noch folgenden Buchstaben.

Also: $c2ape$ oder $4tape$

Formaler Aufschrieb:

\begin{equation*}
    TM = (Q, \Sigma, \Gamma, \Delta, q_0, F)
\end{equation*}

\begin{center}
    \begin{tabular}{c|l}
        $Q$ & Menge der Zustände\\
        $\Sigma$ & Unsere Alphabet\\
        $\Gamma$ & Alphabet des Tapes\\
        $\Delta$ & Menge der Transitionen\\
        $q_0$ & Startzustand\\
        $F$ & Menge der Akzeptierenden Zustände\\
    \end{tabular}
\end{center}

Aufgabe S. 370

\begin{align*}
    TM = &(Q, \Sigma, \Gamma, \Delta, q_0, F)\\
    Q = &\{0, 1\}\\
    \Sigma = &\{a, b\}\\
    \Gamma = &\{a, b, \#\}\\
    q_0 = &0\\
    F = &\{0\}\\
    \Delta: &\{(0,a,a,r,1)\\
    &(0,b,b,r,0)\\
    &(1,a,a,r,0)\\
    &(1,b,b,r,1)\}\\
\end{align*}

\end{document}
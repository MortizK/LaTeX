\documentclass[a4paper]{article}

%\usepackage{url}

%% Math
\usepackage{mathtools}
%% For Mengen like natural numbers
\usepackage{amsfonts}
%% Für spezielle Symbole
\usepackage{amssymb}

%% Images
\usepackage{import}
\usepackage{xifthen}
\usepackage{pdfpages}
%\usepackage{transparent}

%%% Command for simpler images
\newcommand{\incfig}[1]{%
    \def\svgwidth{\columnwidth}
    \import{./fig/}{#1.pdf_tex}
}

%% Links
\usepackage{hyperref}
\hypersetup{
    colorlinks=true,
    linkcolor=black,
    filecolor=magenta,
    urlcolor=cyan
}

%% Formatting
\usepackage{parskip}
\usepackage{listings}

\title{Formale Sprachen}
\author{Moritz}
\date{October 31, 2025}

\begin{document}
\maketitle
\tableofcontents

\section{Flex und Scanners}

\subsection{Compiler}

\begin{enumerate}
    \item Source handler: Bekommt eine Sequenz von Charakteren.
    \item Lexical analysis: Sequenz von token (id, "int")
    \item Syntax analysis: Baumstruktur der Tokens
    \item Semantic analysis: Abstract Syntax Tree + Symbole table
    \item Code generation
\end{enumerate}

\subsection{Generating a scanner}

\begin{lstlisting}
    flex -t numbers.l > numbers.c
    gcc -c -o numbers.o numbers.c
    gcc numbers.o -o scan_numbers
    ./scan_numbers Numbers.txt
\end{lstlisting}

\subsection{Regular Expressions}

Sind Teil des ASCII Buchstabensets und wird unterteil in Operatoren und Buchstaben.

\begin{center}
     O = \{., *, +, ?, -, ~, |, (, ), [, ], \{,\}, <, >, /, \, \^, \$, \"\}
\end{center}

Als Definition in .l files:

\begin{lstlisting}
    DIGIT  [0-9]
    INTEGER {DIGIT}+
    PAIR    \({INTEGER}, {INTEGER}\)
\end{lstlisting}

Aufgabe S.214

\begin{enumerate}
    \item [a-z] wird zu (a+b+c+\dots+y+z)
    \item [\^0-9] wird zu \{ASCII\}$\backslash$ \{0,1,2,\dots,9\} verodert
    \item (r)+ wird zu (r)(r)$^*$
    \item (r)\{3\} wird zu (rrr)
    \item (r)\{3, 7\} wird zu (rrr + rrrrr+ rrrrr + rrrrrr + rrrrrrr)
    \item (r)? wird zu ($\epsilon$+r)
\end{enumerate}

\subsection{User code section}

Muss eine main() function haben, welche den yyin pointer aufsetzt und die function yylex() aufruft.

\begin{lstlisting}
    %%
    int main( int argc, char **argv )
    {
        ++argv, --argc; /* skip over program name */
        if ( argc > 0 )
            yyin = fopen( argv[0], "r" );
        else
            yyin = stdin;
        yylex();
        printf("Final %d : %f\n", intval, floatval);
    }
\end{lstlisting}

\subsection{Parser and Bison}

Es geht darum die Sprache einer Programmiersprache zu erkennen. Somit muss ein Programm der Grammatik entsprechen.

Ein Lexer teilt das Program in tokens ein. Dies sind die kleinste Einheit:

\begin{itemize}
    \item 1, 2, 53 - Integers
    \item i, handle, stream - Identifiers
    \item >, >=, * - Operatoren
\end{itemize}

Aus diesen Tokens wird ein Baum generiert.

\subsubsection{Bison}

Ist ein "1 token Look-Ahead Left-to-right Rightmost-derivation" Parser

Wenn wir etwas auf dem Stack haben und es reduzieren wollen, müssen wir gucken ob jede mögliche reduktion konsistent ist mit dem nächstem Symbol was auf den Stack kommt.

Es gab noch eine Übung mit einem Wissenschaftlichem Taschenrechner. Hierzu mussten wir einen Lexer und einen Parser erstellen und ausführen.

Kann ich mir von Simon schicken lassen.

\subsection{Produktionen und Ableitungen}

Hier habe ich Snake Binary als Prototyp fertig gemacht. Muss ich noch nachholen.

\section{2 Band Turing Machine}

Ab S. 422 im Skript.

\end{document}
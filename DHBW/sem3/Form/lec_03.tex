\documentclass[a4paper]{article}

%\usepackage{url}

%% Math
\usepackage{mathtools}
%% For Mengen like natural numbers
\usepackage{amsfonts}
%% Für spezielle Symbole
\usepackage{amssymb}

%% Images
\usepackage{import}
\usepackage{xifthen}
\usepackage{pdfpages}
%\usepackage{transparent}

%%% Command for simpler images
\newcommand{\incfig}[1]{%
    \def\svgwidth{\columnwidth}
    \import{./fig/}{#1.pdf_tex}
}

%% Links
\usepackage{hyperref}
\hypersetup{
    colorlinks=true,
    linkcolor=black,
    filecolor=magenta,
    urlcolor=cyan
}

%% Formatting
\usepackage{parskip}

\title{Formale Sprachen}
\author{Moritz}
\date{September 26, 2025}

\begin{document}
\maketitle
\tableofcontents

\section{Reguläre Sprachen und endliche Automaten}

\subsection{Aus Regulären Ausdrücken einen NFA bauen}

Der Lauf des Automaten von S. 117 ist:

\begin{align*}
    r_{aaaab}=&((0, aaaab), (2, aaaab), (3, aaab), (1, aaab), (6, aaab), (8, aaab), \\
    & (9, aab), (8, aab), (9, ab), (8, ab), (9, b), (7, b), (10, b), (11, \epsilon))
\end{align*}

Alternativen um die Automaten nach den Bauregeln von S.111ff zu bauen, um Zustände einzusparen.

So können wir zwei Automaten wie eine Kette miteinander verweben. "Überlappen" der beiden Automaten. Der Formale Aufschrieb dieser Kombination von Automaten wird deutlich komplizierter.

Durch dieses "Überlappen" können Fehler entstehen, wenn wir z.B. $a^*b^*$ darstellen wollen. So entstehen Loops, welche beliebige Wörter akzeptieren, da zwischen b und a gewechselt werden kann.

\subsection{DFA into RE}

Wir wollen aus einem DFA einen RE (Regex) machen.

Um dies umzusetzen brauchen wir nochmal die Rechengesetze von RE auf S. 62

\begin{enumerate}
    \item[17] $\epsilon \notin L(s)$ and $r=sr+t\to r=s^*t$ (Arden's Lemma)
\end{enumerate}

Somit wird:

\begin{align*}
    L_1 &= bL_1 + \epsilon\\
    &= b^*\epsilon\\
    &= b^*
\end{align*}

Die Übung von S.124 habe ich auf Papier.

Somit beschreiben REs, NFAs und DFAs dieselbe Sprachen.

\subsection{Vereinfachung von DFAs}

Ein DFA ist \textbf{reduziert}, wenn nur erreichbare Zustände in dem Automaten enthalten sind. Ein Zustand ist erreichbar, wenn dieser ein Teil eines akzeptierten Laufes ist.

Eine \textbf{Minimierung}, bedeutet ein Automat hat die minimale Anzahl an Zustände, der dieselbe Sprache akzeptiert. Dieser erweitert den \textbf{reduzierten} Automaten.

Hinreichende Bedingung sind: 
\begin{enumerate}
    \item Die zu vereinfachenden Knoten müssen die gleichen akzeptanz-Zustände haben.
    \item Zustände sind equivalent, wenn ihre Transitionen gleich sind - der gleicher Buchstabe enden im gleichem akzeptanz-Zustände.
\end{enumerate}

Wir nehmen also einen Buchstaben $c\in\Sigma$ sodass wenn wir aus zwei unterschiedlichen Knoten $\delta(p, c)=p'$ und $\delta(q, c)=q'$ in unterschiedlichen Knoten Enden. S. 132

Dies wird in einer Matrix gemacht. Beispiel ab S. 136 und auf Papier.

\begin{enumerate}
    \item Kreuze und Gleichheiten von Paaren nach den Hinreichenden Bedingungen in die Matrix eintragen.
    \item Jede Lücke durchgehen und für jedes Paar alle Buchstaben angucken. Dies resultiert in einem Paar, wenn dies Markiert ist, übernehmen wir dies.
    \item Für jeden Neuen eintrag müssen alle Lücken neu geprüft werden.
    \item Wenn mehrere Paare aufeinander zeigen sind diese gleich.
\end{enumerate}

Auf Papier "Formale Sprache 26.09.25 S.144" wurde der ganze Prozess von RegEx $\to$ NFA $\to$ DFA $\to$ Minimierter DFA.

\know{Theorem}{Minimale DFAs sind eindeutig}

\subsection{Gleichheit von RegEx}

Vollständig auf Papier gemacht. Hierzu können die Rechengesetze verwendet werden, oder die Gleichheit der minimalen Automaten. Hierzu wurde der Weg von RegEx $\to$ NFA $\to$ DFA $\to$ Minimierter DFA durchgeführt.

\end{document}
\documentclass[a4paper]{article}

%\usepackage{url}

%% Math
\usepackage{mathtools}
%% For Mengen like natural numbers
\usepackage{amsfonts}
%% Für spezielle Symbole
\usepackage{amssymb}

%% Images
\usepackage{import}
\usepackage{xifthen}
\usepackage{pdfpages}
%\usepackage{transparent}

%%% Command for simpler images
\newcommand{\incfig}[1]{%
    \def\svgwidth{\columnwidth}
    \import{./fig/}{#1.pdf_tex}
}

%% Links
\usepackage{hyperref}
\hypersetup{
    colorlinks=true,
    linkcolor=black,
    filecolor=magenta,
    urlcolor=cyan
}

%% Formatting
\usepackage{parskip}

\title{Formale Sprachen und Automaten}
\author{Moritz}
\date{September 12, 2025}

\begin{document}
\maketitle
\tableofcontents

\section{Vorweg}

\subsection{Unix/Linux}

Wir brauchen eine Linux Umgebung. Dies kann in WSL gemacht werden und wir brauchen mindestens: flex, bison, gcc, make

\subsection{Klausur}

Wir dürfen 2 Handbeschriebene Zettel mitnehmen.

\section{Überblick}

\subsection{Formale Sprachen}

Eine Formale Sprache besteht aus einem Alphabet (0-9 oder a-z oder Unicode).

Es gibt kontextfreien und kontextsensitiven Sprachen S.12f.

\subsection{Automaten}

Bestehen aus: states, transitions, letters or characters und memory.

Ein Beispiel gibt es auf S. 16

\subsection{Turing Machine}

Ist im Prinzip eine State-Machine, diese kann alles machen was unsere moderne Computer auch machen können. Es ist ein Theoretisches Modell.

"Wenn es berechenbar ist, kann eine Turing Machine dies auch berechnen"

\subsection{Formale Grammatiken}

Sind Grammatiken zur Erstellung von Wörtern, die aus einen Alphabet entstehen.

Somit gibt es terminal Symbole, welche im Ergebnis Wort vorkommen und non-terminal Symbole die als temporäre Symbole vorkommen um ein Wort zu generieren.

\know{Konvention}{non-terminal Symbole werden Groß geschrieben. Ein Beispiel zur erläuterung gibt es auf S.24}

\section{Definitionen}

Definitionen sind ab S.30

\know{Alphabet}{Ein Alphabet $\Sigma$ ist ein endliches, nicht leere Menge von Charakteren (Symbole oder Buchstaben)
\begin{equation*}
    \Sigma = \{c_1, \dots, c_n\}
\end{equation*}}

\know{Wort}{Ein Wort über ein Alphabet ist eine endliche Folge an Charakteren:
\begin{equation*}
    w=c_1\dots c_n \text{ mit } c_1\dots c_n\in \Sigma
\end{equation*}

Das Leere Wort hat keine Charaktere: $\epsilon$

Das gesamte set aller Wörter in einem Alphabet $\Sigma$ wird als $\Sigma *$ gekennzeichnet}

\subsection{Operatoren}

Länge: Die Länge eines Wortes lässt sich mit $|w|$ bestimmen. Die Anzahl eines Charakters eines Wortes lässt sich mit $|w|_c$ bestimmen.

Einzelnen Charaktere eines Wortes werden mit $w[i]$ mit $i\in\{1, 2, \dots, |w|\}$

Concat: $w_1*w_2$ ist auch $w_1$ gefolgt von $w_2$:

\begin{equation*}
    w_1 = 01, w_2 = 10, w_1w_2 = 0110
\end{equation*}

Power: $w^0=\epsilon$ und $w^n$ ist das wort w, n-mal wiederholt aneinandergehängt.

\subsection{Formale Sprache}

Für ein Alphabet $\Sigma$, ist eine formale Sprache über $\Sigma$ eine Teilmenge $L\subseteq \Sigma *$

Ein gutes Beispiel gibt es auf S.38 mit den Binären Zahlen ohne leading 0.

Bekommen die Zahl aus dem Binärem Wort.

\begin{equation*}
    n: L_\mathbb{N}\to \mathbb{N}
\end{equation*}

Produkt: Ein Produkt zweier Sprachen ist die Concat von allen Wörtern der unterschiedlichen Sprachen.

\begin{equation*}
    L_1*L_2 = \{w_1*w_2\mid w_1\in L_1, w_2\in L_2\}
\end{equation*}

Es gibt noch einen \textbf{Kleene Start} Operator, welche auf S.47 beschrieben wird. Dieser beinhaltet alle Potenzen einer Sprache.

\subsection{Potenzreihe}

Sind alle Subsets von $\Sigma$

\begin{equation*}
    2^{\Sigma_{bin}}=2^{\{0,1\}} = \{\{\}, \{0\}, \{1\}, \{0, 1\}\}
\end{equation*}

\section{Reguläre Sprachen und endliche Automaten}

Diese Ergänzen das Alphabet durch zusätzliche Charaktere. Diese sind: $\{\emptyset, \epsilon, +, \cdot, *, (, )\}$

Definitionen von Operatoren mit regulären ausdrücken auf S. 56

Aufgabe von S.59: Um die Aufgabe 5. zu Lösen kann man die Problemfälle bezeichnen. So lassen sich die strings die nicht 110 beinhaltet durch $(0+10)^*1^*$

\subsection{Rechengesetze}

Sind ab Seite 60

Es muss auf die Unterschiede von $\epsilon$ (dem leerem Wort) und $\emptyset$ (der leerem Menge)

\end{document}
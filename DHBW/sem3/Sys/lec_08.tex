\documentclass[a4paper]{article}

%\usepackage{url}

%% Math
\usepackage{mathtools}
%% For Mengen like natural numbers
\usepackage{amsfonts}
%% Für spezielle Symbole
\usepackage{amssymb}

%% Images
\usepackage{import}
\usepackage{xifthen}
\usepackage{pdfpages}
%\usepackage{transparent}

%%% Command for simpler images
\newcommand{\incfig}[1]{%
    \def\svgwidth{\columnwidth}
    \import{./fig/}{#1.pdf_tex}
}

%% Links
\usepackage{hyperref}
\hypersetup{
    colorlinks=true,
    linkcolor=black,
    filecolor=magenta,
    urlcolor=cyan
}

%% Formatting
\usepackage{parskip}

\title{Betriebssysteme}
\author{Moritz}
\date{November 11, 2025}

\begin{document}
\maketitle
\tableofcontents

\section{Dateisysteme}

Dieses besteht aus Dateien, Verzeichnissen, Verknüpfungen und Spezialdateien.

Es gibt verschieden. Unter WIndows wird FAT oder NTFS benutzt während bei Linux Ext3 oder Ext4 benutzt wird.

\subsection{Implementierung}

Ein Dateisystemlayout besteht aus mehreren Partitionen. Block 0 isz der MBR (Master Boot Record), dieser speichert:

\begin{itemize}
    \item Partitionstabelle mit anfangs- und End-Adresse jeder Partition
    \item Von welcher Partition gebooted wird
\end{itemize}

In jeder Systempartition gibt es zudem noch einen Bootblock, der von dem MBR ausgeführt wird.

\subsection{Dateibelegung}

Es gibt unterschiedliche Strukturen um Dateien auf eine Festplatte zu legen:

\begin{itemize}
    \item Zusammenhängend (eine Datei in mehrere Blöcke)
    \item Verkette Listen (eine Datei als Liste von mehreren Blöcken)
    \item zentrale Indexstruktur (FAT: jeder Block speichert zudem noch die Adresse des folgeblocks)
    \item verteilte Indexstruktur (jede Datei hat sine eigene Indexliste)
\end{itemize}

\subsection{Verzeichnisbelegung}

Ein Verzeichnis ist eine Datei mit Dateizeigern.

\subsection{Verwaltung von freien Blöcken}

Es gibt hierzu freilisten und Bitmaps.

Eine Freiliste ist eine Verkettete Liste die nur auf alle Freien Blöcke sequentiell zeigt. Die Freiliste wird durch das Füllen des Speichers kleiner.

Eine Bitmap hat eine Konstante Größe und gibt für jeden Block mit einem Bit an, ob dieser frei ist (1) oder belegt ist (0).

Es gibt noch Verknüpfungen, welche es erlauben von mehren Stellen auf eine Datei/ Verzeichnis zuzugreifen.

\section{Dateisystem Ext3}

Zugriffsrechte:

\begin{itemize}
    \item Benutzer ID (id -u)
    \item UID 0 ist für root
    \item jede Datei hat UID und eine Group ID (id -g)
    \item jeder Nutzer hat eine UID und kann mehrere GIDs haben
    \item Nur der Eigentümer darf Rechte einer Datei ändern.
\end{itemize}

Die Attribute einer Datei werden in einem speziellem Indexblock (Inode) gespeichert.

\subsection{Dateibaum}

Unter dem Wurzelverzeichnis verweisen alle weiteren Verzeichnisse auf sich selbst und auf das Verzeichnis nach oben. Dateien sind genauso, diese haben nur niemanden der auf diese Verweist.

\subsection{Verzeichnis}

%% Ist wieder nur eine Map mit den Codierung

\subsection{Adressierung der Datenblöcke}

Eine Inode besteht aus 12 direkten Blockadressen, sowie jeweils eine einfach, zweifach und dreifach indirekte Blockadresse. Bedeutet das die mehrfach indirekte Blockadressen auf Blöcke zeigen die auf Blöcke zeigen.

Wenn wir somit 2kByte Blöcke haben können auf:

\begin{align*}
    2^{10}Byte / 4 Byte = 512 [Adressen]\\
    12 * 2kB = 24kB\\
    512 * 2kB = 1MB\\
    512^2 * 2kB = 512MB\\
    512^3 * 2kB = 256GB\\
    \Sigma = ca. 257GB
\end{align*}

\subsection{Layout einer Partition}

%% Weiter habe ich es nicht geschafft

\end{document}
\documentclass[a4paper]{article}

%\usepackage{url}

%% Math
\usepackage{mathtools}
%% For Mengen like natural numbers
\usepackage{amsfonts}
%% Für spezielle Symbole
\usepackage{amssymb}

%% Images
\usepackage{import}
\usepackage{xifthen}
\usepackage{pdfpages}
%\usepackage{transparent}

%%% Command for simpler images
\newcommand{\incfig}[1]{%
    \def\svgwidth{\columnwidth}
    \import{./fig/}{#1.pdf_tex}
}

%% Links
\usepackage{hyperref}
\hypersetup{
    colorlinks=true,
    linkcolor=black,
    filecolor=magenta,
    urlcolor=cyan
}

%% Formatting
\usepackage{parskip}

\title{Betriebssysteme}
\author{Moritz}
\date{October 14, 2025}

\begin{document}
\maketitle
\tableofcontents

\section{Synchronisation}

Aufgabenblatt 3 hat eine Alternative zu den Semaphoren um Race Conditions zu verhindern.

\section{Kommunikation zwischen Prozessen}

Nutzung von Betriebssystem angeforderten gemeinsamen Speicherbereich.

Kommunikation über Dateien im Dateisystem.

Kommunikation übe "Pipes"

Explizierte Austausch von Signalen

Explizierter Austausch von Nachrichten - auch an Prozesse auf anderen Rechnern

\subsection{Pipes}

z.B. in der shell von Linux ls .l $|$ grep "VfB"

Es gibt Unnamed oder anonymous Pipe, Welche nur von Kindern des selben Elternprozesses funktioniert. Siehe $|$ der Linux Shell.

Kann mit "int pipe(int df[2])" initialisiert werden. fd[0] verweist auf den Ausgang und fd[1] auf den Eingang der Pipe.

durch close(fd[1]) bei dem parent und close(fd[0]) beim child kann eine Pipe eingerichtet werden. Beispiel auf S.17

Named Pipes überstehen sogar den Neustart des Betriebssystems.

\subsection{Signale}

Ein Signal ist eine kleine Zahl im Bereich von 1 und STGRTMAX (typisch 64).

Dies sind einzelnen Codes, die bestimmte Bedeutung haben. So ist SIGINT der Interrupt durch Strg + C

Signale sind für Prozesse, wenn es um Threads geht wird es schwierig umzusetzen.

\subsection{Nachrichten}

Nachrichtenaustausch über Sockets - auch zwischen verschiedenen Rechner:

\begin{itemize}
    \item TCP-Sockets (verbindungsorientierte)
    \item UDP-Sockets (verbindungslose)
    \item UNIX-Sockets
\end{itemize}

\end{document}
\documentclass[a4paper]{article}

%\usepackage{url}

%% Math
\usepackage{mathtools}
%% For Mengen like natural numbers
\usepackage{amsfonts}
%% Für spezielle Symbole
\usepackage{amssymb}

%% Images
\usepackage{import}
\usepackage{xifthen}
\usepackage{pdfpages}
%\usepackage{transparent}

%%% Command for simpler images
\newcommand{\incfig}[1]{%
    \def\svgwidth{\columnwidth}
    \import{./fig/}{#1.pdf_tex}
}

%% Links
\usepackage{hyperref}
\hypersetup{
    colorlinks=true,
    linkcolor=black,
    filecolor=magenta,
    urlcolor=cyan
}

%% Formatting
\usepackage{parskip}

\title{Betriebssysteme}
\author{Moritz}
\date{October 28, 2025}

\begin{document}
\maketitle
\tableofcontents

\section{Speicherverwaltung}

Speicherhierarchie:

\begin{itemize}
    \item Die Register haben einen Zugriff von ~0,1ns
    \item Die Taktrate einer 3,5Ghz CPU ist alle ~0,3ns
    \item Danach kommen die L1-L4 Chache bei 1ns-40ns
    \item Arbeitsspeicher bei 30ns-100ns
    \item SSD bei ~30$\mu$s-100$\mu$s
\end{itemize}

Lokalität: Eine Zeitliche Lokalität ist die Wahrscheinlichkeit, dass auf diese Speicheradresse in naher Zukunft nochmal zugegriffen wird.

Die Räumliche Lokalität beschreibt das gleicher für benachbarten Speicheradressen.

\subsection{Direkt adressierter Speicher}

\dots

Wir können den belegten pages (4kByte) eine Bitmap zuordnen, welche angibt, welche pages belegt sind und welche nicht.

Ein Programm besteht aus benachbarten pages im RAM.

\subsubsection{Verwaltungstechniken}

Monoprogramming

Multiprogramming mit festen Partitionen

Multiprogramming mit variablen Partitionen und Swapping

Multiprogramming mit virtuellem Speicher und Paging

Statt einer Bitmap zur Bestimmung von freien Speicheradressen wird eine verkettete Liste Benutzt (mit start, size, next)

Hierzu gibt es verschiedene Methoden um einen neuen Prozess in den RAM zu laden:

\begin{itemize}
    \item first fit (erster platz wo der Prozess rein kann)
    \item best fit (platz der die kleinste Lücke generiert)
    \item (buddy system)
\end{itemize}

\subsubsection{Swapping}

Swapping passiert, wenn im RAM nicht genügent Platz für weitere Prozesse ist. So wird ein Prozess vollständig auf die Festplatte geschrieben.

\subsubsection{Virtueller Speicher}

Wir haben einen Virtuellen Speicher, welcher dann eine Zuweisung auf den Physikalischen Speicher hat.

Im Virtuellem Speiche ist die Einheit eine \textbf{Seite} und im Physikalischem ist es eine \textbf{Kachel/Seitenrahmen}.

Diese Verwaltung braucht auch Speicher und CPU zeit (ein overhead)

Durch den virtuellen Speicher lassen sich auch Prozesse laufen lassen, die größer sind als der Physikalische Speicher, da nur die notwendigen Teile in dem RAM sein müssen.

Und wir können nun auch nicht zusammenhängende Seitenrahmen auf dem RAm haben, da vom Virtuellen auf den RAM übersetzt wird.

Ein Seitenfehler (page fault):

\begin{itemize}
    \item Minor Page fault: Seite im RAM, aber nicht in der aktueller Seitentabelle (kann passieren, wenn mehrere Prozesse dieselbe Bibliothek benutzten)
    \item Major Page fault: Seite nicht im RAM
    \item invalide Adresse: Segmentation Fault
\end{itemize}



\end{document}
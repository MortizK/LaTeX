\documentclass[a4paper]{article}

%\usepackage{url}

%% Math
\usepackage{mathtools}
%% For Mengen like natural numbers
\usepackage{amsfonts}
%% Für spezielle Symbole
\usepackage{amssymb}

%% Images
\usepackage{import}
\usepackage{xifthen}
\usepackage{pdfpages}
%\usepackage{transparent}

%%% Command for simpler images
\newcommand{\incfig}[1]{%
    \def\svgwidth{\columnwidth}
    \import{./fig/}{#1.pdf_tex}
}

%% Links
\usepackage{hyperref}
\hypersetup{
    colorlinks=true,
    linkcolor=black,
    filecolor=magenta,
    urlcolor=cyan
}

%% Formatting
\usepackage{parskip}

\title{Betriebssysteme}
\author{Moritz}
\date{September 30, 2025}

\begin{document}
\maketitle
\tableofcontents

\section{Prozesse und Threads}

\subsection{Prozessmodell}

EIn Prozess ist die Instanziierung eines Programms auf einem Prozessor einschließlich seiner Umgebung

Alternative wird auch "Task" für einen Prozess verwendet. Ein Job ist eine Sequenz aus mehreren nacheinander ausgeführten Prozessen.

Ein Prozess kann andere Prozesse erzeugen: Elternprozess und Kindprozess.

Es gibt drei Kontexte, alle drei bilden den Prozesskontext:

\begin{enumerate}
    \item Speicher - Kontext
    
    Daten (Heap). Programm (Code), Stack
    \item Hardware - Kontext
    
    Stackzeiger, Programmzähler, Register (CPU und MMU)
    \item System - Kontext
    
    Prozess ID, Prozesskontrollblock, Zugriffsrechte (UserID, GroupID), Dateideskriptoren, Prozesszustand
\end{enumerate}

Ein Prozess hat vier Zustände, zwischen denen gewechselt wird:

\begin{enumerate}
    \item Bereit (ready) - Prozessor ist belegt, Prozess aber ausführbar
    \item Aktiv (running) - wird ausgeführt
    \item Blockiert (blocked) - wartet auf externes Ereignis
    \item Inaktiv (Idle) - wurde gerade erzeugt oder ist terminiert
\end{enumerate}

Ein Scheduler steuert diese Übergänge und die Priorität der Prozesse.

Prozessverwaltung verwaltet die Zuteilung von Betriebsmittel von wartenden Prozessen. Die Prozesstabelle ist die Datenstruktur im Kernel.

Jeder Prozess in der Prozesstabelle hat einen Prozesskontrollblock (PCB). Dieser speichert:

\begin{itemize}
    \item Speicherplatz für Programmzähler
    \item Prozesszustand
    \item Erwartetes Ereignis für Zustand blocked
    \item Statische und dynamische Priorität
    \item Bislang verbrauchte CPU-Zeit und Systemzeit
    \item PID und PPID (Parent Prozess ID)
    \item Aktuell zugeordnete Betriebsmittel (Dateien, Netzwerkverb,)
    \item Speicherplatz für aktuelle Registerinhalte
    \item Speicherreferenzen zu COde, Daten, Stack, \dots
\end{itemize}

Für einen Prozesswechsel müssen die Zustände von Prozess $P_0$ in dem PCB gespeichert werden und die Neuen Zustände von Prozess $P_1$ werden aus dem PCB geladen und $P_1$ wird aktiv.

\subsection{Threadmodell}

Ein Thread sind quasi mehrere Aufgaben eines Prozesses quasi-parallel zu bearbeiten. Diese müssen denselben Skope haben/ denselben Prozesskontext haben.

Umsetzung: Wir haben einen Dispatcher Thread, welche auf eine Request Wartet und für jede Request einen WorkerThread erstellt. Dieser kann dann im gleichem Kontext quasi parallel die Request abarbeiten.

quasi-parallel bedeutet wir sparen uns den Kontextwechsel den ein Prozess hat, da wir ja nur in dem gleichem Prozess bleiben.

Threadkontext:

\begin{enumerate}
    \item lokaler Programmzähler
    \item lokaler Registersatz
    \item lokaler Stack
    \item erbt offene Dateien und Netzwerkverbindungen des Prozesses
    \item eigener Stackbereich für lokale Variabel
    \item eigener Hardwarekontext, nur aus PC und Registersatz
\end{enumerate}

Bei Threadwechsel muss nur der Threadkontext von Punkt 5. und 6. gewechselt werden.

Die Zustände werden durch den Zustand "tot" ergänzt. Zustände aller Threads werden intern durch verkettete Listen organisiert für die Zustände.

Thread Control Block (TCB):

\begin{itemize}
    \item Eindeutiger Identifikator
    \item Speicherplatz zur Sicherung von Hardwarekontexts
    \item GGf. Wartegrund bei blockiertem Thread
    \item Zeiger auf PCB des Prozesses
    \item Sonstige Zustandsinformationen und Statistiken für Scheduling
\end{itemize}

In der Implementierung bei den Betriebssysteme gibt es Benutzer-Threads und Kernel-Threads. Diese laufen auf der jeweiligen Ebene und haben unterschiedliche Vor- und Nachteile.

\subsection{Prozesse und Threads in Unix}

Baumartige Organisation mit dem init als Elternprozess. Dieser hat die PID von 1.

Systemaufrufe zur Prozessverwaltung:

\begin{itemize}
    \item pid-t fork(): erzeugt ein Kindprozess als Kopie des Elternprozess mit neuer PID. Gibt an das Kind einer 0 und an den Elternprozess die PID des Kindes.
    \item pid-t wait(): Ein Elternprozess wartet auf die Rückgabe eines Kindes
    \item \dots
\end{itemize}

Um als Benutzer Threads zu erstellen wird die pThread-Bibliothek verwendet. Diese verwendet die native POSIX Thread Library (NPTL).

\subsection{Prozesse und Threads in Windows}

Es wird zwischen Threads, Prozesse und Jobs unterschieden. Jobs sind eine Gruppe mehrere Prozesse, die zwecks Resourcenvergabe einheitlich angesprochen werden kann.

Es gibt auch noch System-Threads, z.B. den Idle-Thread.

Ein Prozess hat zusätzlich noch einen Object-Handle mit PID.

Ne Menge C-Code für die Prozesserzeugung in Windows: Slide S. 42

Threadzustände in Windows:

\begin{itemize}
    \item init
    \item ready
    \item running
    \item standby
    \item terminate
    \item waiting
    \item transition
    \item deferred
\end{itemize}

\subsection{POSIX Threads}

Ist eine ANSI/IEEE Standard für Betriebssystem-unabhängige Thread-Programmierung. S. 49ff muss ich mir nochmal angucken!



\end{document}
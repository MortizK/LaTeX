\documentclass[a4paper]{article}

%\usepackage{url}

%% Math
\usepackage{mathtools}
%% For Mengen like natural numbers
\usepackage{amsfonts}
%% Für spezielle Symbole
\usepackage{amssymb}

%% Images
\usepackage{import}
\usepackage{xifthen}
\usepackage{pdfpages}
%\usepackage{transparent}

%%% Command for simpler images
\newcommand{\incfig}[1]{%
    \def\svgwidth{\columnwidth}
    \import{./fig/}{#1.pdf_tex}
}

%% Links
\usepackage{hyperref}
\hypersetup{
    colorlinks=true,
    linkcolor=black,
    filecolor=magenta,
    urlcolor=cyan
}

%% Formatting
\usepackage{parskip}
\usepackage{graphicx}
\usepackage{float}

\title{Angewandte Mathematik}
\author{Moritz}
\date{November 5, 2025}

\begin{document}
\maketitle
\tableofcontents

\section{Integralrechnung}

\begin{align*}
    \lim_{n\to\infty; \Delta G\to 0} \sum_{i=1}^{n} v_i
    = \lim_{n\to\infty; \Delta G\to 0} \sum_{i=1}^{n} f(x_i, y_i) \Delta G_i
    = \int\int_{G}f(x, y) dG
\end{align*}

falls der Grenzwert existiert

Anmerkung: a) Gelegentlich auch als $\int_{G}f(x, y), dG$ geschrieben

b) Ist $f(x, y)$ in abgeschlossene Gebiet, so existiert der Grenzwert und damit das Doppelintegral.

\subsection{Berechnung des Doppelintegrals in kartesischen Koordinaten}

Zurückführen des Doppelintegrals auf zwei nacheinander auszuführende gewöhnliche Integration.

Wir betrachten zunächst einen "normalen" (links nach rechts) Integrationsbereich:

\begin{align*}
    G = \{(x, y)\in \mathbb{R}^2\}\\
    a \leq x \leq b\\
    \varphi_1(x) \leq Y \leq \varphi_2(x)
\end{align*}

aber wenn die andere Richtung (unten nach oben):


\begin{align*}
    G = \{(x, y)\in \mathbb{R}^2\}\\
    a \leq y \leq b\\
    \varphi_1(x) \leq x \leq \varphi_2(x)
\end{align*}

Nun werden beide Achsen kombiniert: Zuerst Summation in y-Richtung:

\begin{align*}
    dG = dx * dy\\
    V_i = f(x_i, y_i) dx dy\\
    \left(\int_{\varphi_1(x)}^{\varphi_2(x)}f(x, y)dy\right)dx
\end{align*}

Damit haben wir nun eine Fläche. Danach Summation in x-Richtung.

\begin{align*}
    \int\int_{G}f(x, y) dG =
    \int_{x=a}^{b}\left(\int_{y=\varphi_1(x)}^{\varphi_2(x)}f(x, y) dy\right)dx =
    \int_{x=a}^{b}\int_{y=\varphi_1(x)}^{\varphi_2(x)}f(x, y) dy dx
\end{align*}

\begin{itemize}
    \item Reihenfolge ist im Allgemeinen nicht ohne weiteres vertauschbar
    \item Falls G rechteckig ist (also alle Grenzen konstant), ist die Reihenfolge vertauschbar.
    \item Es gibt auch Gebiete, bei denen man in andere Reihenfolge integrieren muss. (Siehe linke rechts oder oben unten)
\end{itemize}

\subsection{Beispiele}

geg.: $\varphi_2: y=\sqrt{x}; \varphi_1: y=x; f(x, y)=xy$ ges.: $\int\int_{G}f(x, y) dG$

Wir suchen also das Volumen der Funktion $f(x, y)$ in der xy-Ebene in den Grenzen, die von $\varphi_{1,2}$ bestimmt werden.

Lösung:

\begin{align*}
    0 \leq x \leq 1\\
    x \leq y \leq \sqrt{x}
\end{align*}

\begin{align*}
    \int\int_{G}f(x, y) dG &=
    \int_{x=0}^{1}\int_{y=x}^{\sqrt{x}} xy dy dx = 
    \int_{x=0}^{1} \left[\frac{1}{2}xy^2\right]_{x}^{\sqrt{x}}dx\\
    &= \int_{0}^{1}\left(\frac{x^2}{2}-\frac{x^3}{2}\right)dx =
    \frac{1}{2}\left[\frac{1}{3}x^3-\frac{x^4}{4}\right]_0^1 =
    \frac{1}{2}\left(\frac{1}{3}-\frac{1}{4}\right) =
    \frac{1}{24}
\end{align*}

Ist G kein Normalbereich, so versucht man eine Zerlegung in mehrere Normalbereiche. Vorstellung Mathe-stein.

geg.: $G = $ Gebiet wird von $y=x^2$ und $x=y^2$ mit $f(x, y)=x^2+y^2$

ges.: $\int\int_{G}f(x, y) dG$

Skizze: Eingeschlossener Bereich von $x=\sqrt{x}$ und $y=x^2$

\begin{align*}
    0 \leq x \leq 1\\
    x^2 \leq y \leq \sqrt{x}
\end{align*}

\begin{align*}
    \int\int_{G}(x^2+y^2) dG &= 
    \int_{x=0}^{1}\int_{y=x^2}^{\sqrt{x}} (x^2+y^2)dy dx = 
    \int_{0}^{1}\left[x^2y + \frac{1}{3}y^3\right]_{x^2}^{\sqrt{x}} dx\\
    &= \int_{0}^{1}\left(x^\frac{5}{2}+\frac{1}{3}x^\frac{3}{2}-x^4-\frac{1}{3}x^6\right)dx = \dots
\end{align*}

geg.: Parallelogramm mit den Eckpunkten (0,0); (0, 1); (2, 3); (1, 2) und $f(x, y) = x$

\begin{figure}[H]
    \centering
    \includegraphics[width=0.7\textwidth]{fig/VolumenVonParallelogram.png}
    \caption{Volumen eines Parallelogramms}
    \label{fig:volumen-parallelogram}
\end{figure}

\subsection{Berechnung des Doppelintegrals in Polarkoordinaten}

\begin{align*}
    x = r*\cos(\varphi)\\
    y = r*\sin(\varphi)
\end{align*}

Wir integrieren also zwischen der äußeren und inneren Kurve. Somit gilt:

\begin{align*}
    \varphi_1 \leq \varphi \leq \varphi_2\\
    r_i(\varphi) \leq r \leq r_a(\varphi)
\end{align*}

Somit bildet sich:

\begin{align*}
    dG = r * dr d\varphi
\end{align*}

Integriere im Polarkoordinaten:

\begin{enumerate}
    \item Funktion umrechnen.
    \item Flächenelement umschreiben.
\end{enumerate}

Beispiel geg.: und ges.: fehlen mir.

\begin{align*}
    f(x, y) &= xy = 
    (r*\cos(\varphi))(r*\sin(\varphi)) =
    r^2\cos(\varphi)\sin(\varphi)\\
    \int\int_{G}xy dG &= 
    \int_{\varphi=0}^{\frac{\pi}{4}}\int_{r=0}^{2} r^2\cos(\varphi)\sin(\varphi) =
    \int_{\varphi=0}^{\frac{\pi}{4}}\cos(\varphi)\sin(\varphi)\int_{r=0}^{2}r^3 dr d\varphi\\
    &= \int_{\varphi=0}^{\frac{\pi}{4}}\cos(\varphi)\sin(\varphi)\left[\frac{1}{4}r^4\right]_0^2 d\varphi =
    \int_{0}^{\frac{\pi}{4}}\cos(\varphi)\sin(\varphi) * 4 d\varphi\\
    &= 4 \left[\frac{1}{2}\sin^2(\varphi)\right]_0^{\frac{\pi}{4}} = 2\left(\frac{\sqrt{2}}{2}\right)^2 = \frac{2*2}{4} = 1
\end{align*}

$\mathbb{R}^3$: Drehung der Parabel $z=4-x^2$ um die z-Achse $\implies$ Paraboloid schneiden mit der xy-Ebene (z=0)

Daraus ergibt sich die Funktion $z=4 -(x^2+y^2)=f(x, y)$. $x^2+y^2$ ist der Radius des Rotationskörpers.

Somit ist unser Gebiet G:

\begin{align*}
    0 \leq \varphi \leq 2\pi\\
    0 \leq r \leq 2
\end{align*}

Somit:

\begin{align*}
    V = \int\int_{G} (4-r^2) dG &= 
    \int_{\varphi=0}^{2\pi}\int_{r=0}^{2} (4-r^2) r dr d\varphi =
    \int_{0}^{2\pi} \left[2r^2-\frac{1}{4}r^4\right]_0^2 d\varphi\\
    &= \int_{0}^{2\pi}(8-4)d\varphi = \left[\varphi\right]_0^{2\pi} =
    4 * 2\pi = 8\pi 
\end{align*}

\subsection{Anwendung}

%% Kasten start

Flächeninhalte von G: Setze $f=1$

\begin{align*}
    |g| = \int\int_{G} dG &= \int_{x=a}^{b}\int_{y=\lambda_1(x)}^{\lambda_2(x)} 1 dy dx\\
    &= \int_{\varphi=\varphi_1}^{\varphi_2}\int_{r=r_1(\varphi)}^{r_2(\varphi)} 1*r dr d\varphi
\end{align*}

%% Kasten Ende

Beispiel. geg.: Archimedische Spirale $r-r(\varphi)=a*\varphi, (a>0)$ ges.: umschlossene Fläche für $0\leq \varphi \leq 2\pi$

Unser G:

\begin{align*}
    0 \leq \varphi \leq 2\pi\\
    0 \leq r \leq a*\varphi
\end{align*}

Ergebnis: $\frac{4}{3}a^2\pi^3$

\subsection{Schwerpunkt einer homogenen Fläche}

\begin{align*}
    x_S = \frac{1}{|y|} = \int\int_{y} x dG\\
    y_S = \frac{1}{|y|} = \int\int_{y} y dG\\
\end{align*}

Schwerpunkt $(x_S, y_S)$

\end{document}
\documentclass[a4paper]{article}

%\usepackage{url}

%% Math
\usepackage{mathtools}
%% For Mengen like natural numbers
\usepackage{amsfonts}
%% Für spezielle Symbole
\usepackage{amssymb}

%% Images
\usepackage{import}
\usepackage{xifthen}
\usepackage{pdfpages}
%\usepackage{transparent}

%%% Command for simpler images
\newcommand{\incfig}[1]{%
    \def\svgwidth{\columnwidth}
    \import{./fig/}{#1.pdf_tex}
}

%% Links
\usepackage{hyperref}
\hypersetup{
    colorlinks=true,
    linkcolor=black,
    filecolor=magenta,
    urlcolor=cyan
}

%% Formatting
\usepackage{parskip}

\title{Angewandte Mathematik}
\author{Moritz}
\date{November 12, 2025}

\begin{document}
\maketitle
\tableofcontents

\section{Integralrechnung}

\subsection{Schwerpunkt}

Beispiel: Schwerpunkt der endl. Fläche, die von $y=x+2$ und $y=4-x^2$ begrenzt wird.

geg.: $y=x+2, y=y^2-4$ ges.: Schwerpunkt $S(x_S, y_S)$

Skizze anfertigen!

Die Grenzen lassen sich aus den zwei Schnittpunkten der Funktion ablesen. Der Triviale ist bei $(-2, 0)$ und ein nicht trivialer liegt bei $(1, 3)$

Schnittpunkte:

\begin{align*}
    4-x^2 = x+2\\
    x^2+x-2 = 0\\
    \text{Quadratische Ergänzung}\\
\end{align*}

Dies bestätigt die Vermutung.

Zunächst Flächeninhalt:

\begin{align*}
    |G| = \int\int_{G} dG &= \int_{x=-2}^{1}\int_{y=x+2}^{4-x^2} dy dx\\
    &= \int_{x=-2}^{1} \left[y\right]_{x+2}^{4-x^2} dx\\
    &= \int_{-2}^{1} (4-x^2-x-2) dx\\
    &= \left[2x-\frac{1}{3}x^3-\frac{1}{2}x^2\right]_{-2}^1 \\
    &= 2-\frac{1}{3} -\frac{1}{2} +4 - \frac{8}{3} +2\\
    &= \frac{9}{2}
\end{align*}

\begin{align*}
    x_S = \frac{1}{|G|} \int\int_{G} x dG &= \frac{2}{9} \int_{x=-2}^{1}\int_{y=x+2}^{4-x^2} x dy dx\\
    &= \frac{2}{9} \int_{x=-2}^{1} [xy]_{x+2}^{4-x^2} dx\\
    &= \frac{2}{9} \int_{-2}^{1} (4x - x^3) - (x^2 + 2x) dx\\
    &= \frac{2}{9} \int_{-2}^{1} 2x - x^3 -x^2 dx\\
    &= \frac{2}{9} \left[x^2 - \frac{1}{4}x^4 - \frac{1}{3}x^3\right]_{-2}^1 = -\frac{1}{2} = -0.5
\end{align*}


\begin{align*}
    x_S = \frac{1}{|G|} \int\int_{G} y dG &= \frac{2}{9} \int_{x=-2}^{1}\int_{y=x+2}^{4-x^2} y dy dx\\
    &= \frac{2}{9} \int_{x=-2}^{1} \left[\frac{1}{2}y^2\right]_{x+2}^{4-x^2} dx\\
    &= \frac{1}{9} \int_{-2}^{1} (4-x^2)^2 - (x+2)^2 dx\\
    &= \frac{1}{9} \int_{-2}^{1} 12-9x^2 -4x + x^4 dx\\
    &= \frac{1}{9}\left[12x - 3x^3 - 2x^2 + \frac{1}{5}x^5\right]_{-2}^1 = \frac{12}{5} = 2.4
\end{align*}

Schwerpunkt $S=(-0.5, 2.4)$

\subsection{Kurvenintegral (1. Art)}

auch Linien oder Wegintegral

Kurven: $\varphi(t)$ sei eine Vektorwertige Funktion

\begin{equation*}
    \varphi[a, b] \to \mathbb{R}^n \text{ mit } \varphi(t) = \begin{pmatrix}
         \varphi_1(t) \\ \vdots \\ \varphi_n(t)
    \end{pmatrix} \text{ mit } t\in[a, b]
\end{equation*}

Wir setzen voraus: $\varphi(t)$ sind stückweise stetig diffenzierbar.

\begin{align*}
    ||\dot{\varphi}(t)|| = \sqrt{(\dot{\varphi_1}(t))^2 + \dots +(\dot{\varphi_n}(t))^2} \neq 0 \text{ für } t\in[a, b] 
\end{align*}

(kann abgeschwächt werden)

$\implies x=\varphi(t)$ können wir als Kurve K im $\mathbb{R}^n$ auffassen.

Skizze mit Parameter $t$ und der Kurve im $\mathbb{R}^3$

Motivation: Ein geborgener Draht/ Leistung (entspricht eine Kurve) dessen Dichte variabel ist und vom Ort abhängt. Wir wollen die Gesamtmasse bestimmen.

$f: \mathbb{R}^n\to \mathbb{R}$ sei eine stetige Funktion die auf der Kurve definiert ist.

\begin{align*}
    \int_{K} f(x) ds \text{ (ds ist das Bogenelement der Kurve)}
\end{align*}

Berechnung:

\begin{align*}
    \int_{K} f(x) ds = \int_{a}^{b} f(\varphi(t))\quad||\dot{\varphi}(t)|| \quad dt
\end{align*}

Beispiel: geg.:

\begin{align*}
    K: \varphi(t) = \begin{pmatrix}
       \cos(t) \\ \sin(t) \\t
    \end{pmatrix}, t\in (0, 2\pi)\\
    f(x, y, z) = x^2 + y^2 + z^2
\end{align*}

ges.: Das Kurvenintegral 1. Art

Lösung:

\begin{align*}
    I = \int_{K} (x^2 + y^2 + z^2) ds\\
    \dot{\varphi}(t) = \begin{pmatrix}
        -\sin(t) \\\cos(t) \\ 1
    \end{pmatrix}\\
    ||\dot{\varphi}(t)|| = \sqrt{(-\sin(t))^2 + \cos^2(t) + 1^2} = \sqrt{2}\\
    I = \int_{0}^{2\pi} (\cos^2(t) + \sin^2(t) + t^2)\sqrt{2} dt\\
    = \sqrt{2} \int_{0}^{2\pi} 1 + t^2 dt \\
    = \sqrt{2} \left[t + \frac{1}{3}t^3\right]_0^{2\pi} = 2\sqrt{2}\pi (1+\frac{4}{3}\pi^2)
\end{align*}

Anwendung: Die Länge einer Kurve bestimmen. $L = \int_{K}1*ds$
Masse der Kurve bestimmen: $m= \int_{K} \varrho(x) ds$
Schwerpunkt: z.B. $x_1$-Koordinate: $x_{1s}=\frac{1}{m} \int_{K} x_1 \varrho(x) ds$ 

Beispiele:

geg.: 

\begin{align*}
    \varphi(t) = \begin{pmatrix}
        \cos(t) \\\sin(t)\\ t
    \end{pmatrix}, t\in[0, \pi/2]
\end{align*}

ges.: $S(x_{1b}, x_{2b}, x_{3b})$

Wir nutzen was wir schon über die Kurve wissen:

\begin{equation*}
    ||\dot{\varphi}(t)|| = \sqrt{2}
\end{equation*}

1)

\begin{align*}
    m = \int_{K} \varrho ds &= \int_{0}^{\pi/2} 1 * ||\dot{\varphi}(t)|| dt\\
    &= \int_{0}^{\pi/2} \sqrt{2} dt = \sqrt{2}[t]_0^{\pi/2} = \frac{\sqrt{2}}{2}\pi
\end{align*}

2)

\begin{align*}
    x_{1b} = \frac{1}{m} \int_{K} x_1 \varphi(x) db\\
    = \frac{2\pi}{\sqrt{2}} \int_{0}^{\pi/2} \cos(t) * 1 * ||\dot{\varphi}(t)|| dt\\
    = \frac{2}{\pi} \int_{0}^{\pi/2} \cos(t) dt \\
    = \frac{2}{\pi} \left[\sin(t)\right]_0^{\pi/2} = \frac{2}{\pi}
\end{align*}

\begin{align*}
    x_{2b} = \frac{\sqrt{2}}{\pi} \int_{0}^{\pi/2} \sin(t) * 1 * \sqrt{2} dt\\
    = \frac{2}{\pi} [-\cos(t)]_0^{\pi/2} = \frac{2}{\pi}
\end{align*}

\begin{align*}
    x_{3b} = \frac{\sqrt{2}}{\pi} \int_{0}^{\pi/2} t * 1 * \sqrt{2} dt\\
    = \frac{2}{\pi} \left[\frac{1}{2}t^2\right]_0^{\pi/2} = \frac{\pi}{4}
\end{align*}

\subsection{Länge einer Kurve}

K: $y=f(x)$ für $x\in[x_0, x_1] \to$ Parameterdarstellung

\begin{equation*}
    K: t\xrightarrow{\varphi} \begin{pmatrix}
        t\\f(t)
    \end{pmatrix}, x=t, y=f(t)
\end{equation*}

\begin{align*}
    L = \int_{K} ||\dot{\varphi}(t)|| dt\\
    \varphi(t) = \begin{pmatrix}
        1, f'(t)
    \end{pmatrix}\\
    ||\dot{\varphi}(t)|| = \sqrt{1 + (f'(t))^2}\\
    L =  \int_{x_0}^{x_1} \sqrt{1 + (f'(t))^2} dx
\end{align*}

\subsection{Kurvenintegral (2. Art)}

Habe den Start verpasst, muss die von Rosalie holen.

\begin{align*}
    \int_{c} f(x) dx := \int_{a}^{b} \langle f(c(t)), c'(t) \rangle dt 
    = \int_{a}^{b} f(c(t))^T c'(t) dt
\end{align*}

zweites Beispiel

geg.: $F: \mathbb{R}^3 \to \mathbb{R}$ mit Schwerkraft an der Erdoberfläche

\begin{align*}
    F = \begin{pmatrix}
        0 \\ 0 \\ -mg
    \end{pmatrix}
\end{align*}

ges.: Arbeit auf K

\begin{align*}
    K: \varphi: [0, 1] \to \mathbb{R}^3\\
    \varphi(t) = \begin{pmatrix}
        t \\ 0\\ 2(t-1)^2
    \end{pmatrix} \\
    \implies \varphi(t) = \begin{pmatrix}
        1 \\ 0 \\ 4(t-1)
    \end{pmatrix}
\end{align*}

Lösung:

\begin{align*}
    W = \int_{K} \vec{F} \circ \vec{ds} 
    &= \int_{0}^{1} \begin{pmatrix}
        0 \\ 0 \\ mg
    \end{pmatrix} \circ \begin{pmatrix}
        1\\ 0 \\ 4(t-1)
    \end{pmatrix} dt \\
    &= \int_{0}^{1} (-mg)4(t-1) dt\\
    &= -mg * 4 \int_{0}^{1} (t-1) dt = -mg*4 \left[\frac{1}{2}t^2 - t\right]_0^1\\
    &= -4mg (\frac{1}{2}-1) = 2mg
\end{align*}

\know{Klausur}{Hier ist der Klausurstoff zuend. Und Konvergenzbetrachtungen brauchen wir nicht.}

\end{document}
\documentclass[a4paper]{article}

%\usepackage{url}

%% Math
\usepackage{mathtools}
%% For Mengen like natural numbers
\usepackage{amsfonts}
%% Für spezielle Symbole
\usepackage{amssymb}

%% Images
\usepackage{import}
\usepackage{xifthen}
\usepackage{pdfpages}
%\usepackage{transparent}

%%% Command for simpler images
\newcommand{\incfig}[1]{%
    \def\svgwidth{\columnwidth}
    \import{./fig/}{#1.pdf_tex}
}

%% Links
\usepackage{hyperref}
\hypersetup{
    colorlinks=true,
    linkcolor=black,
    filecolor=magenta,
    urlcolor=cyan
}

%% Formatting
\usepackage{parskip}

\title{Angewandte Mathematik}
\author{Moritz}
\date{October 22, 2025}

\begin{document}
\maketitle
\tableofcontents

\section{Differentialrechnung mehrstelliger Funktionen}

\know{Für die Klausur}{Wir sollen immer angeben, was gegeben ist und was gesucht ist.}

\subsection{Eigenwerte}

Die Eigenwerte lassen sich in einer Matrix bestimmen, indem wir die Determinanten bestimmen von den kleineren Quadratmatrizen, von ober links.

Wenn die Eigenwerte positiv sind, so ist die Matrix positiv definit und somit der Punkt ein lokales Minimum.

Wenn die Eigenwerte negativ sind, so ist die Matrix negativ definit und somit der Punkt ein lokales Maximum.

Wenn die Eigenwerte sowohl positiv und negativ sind, so ist die Matrix semi definit und wir haben keine Aussage.

(Bei zwei positiven und ein negativer Eigenwert $\implies$ Matrix indefinit)

\subsection{Lokale Extrema mit Nebenbedingungen}

Im Anwendungen heben häufig Nebenbedingungen auf: Bestimmte Extrema von $z=f(x, y)$ unter der Nebenbedingung $\varphi(x, y)=0$

Manchmal: NB nach einer Variablen auflösen und in $f$ einsetzen.

\subsubsection{Methode des Lagrange'schen - Multiplikator}

Höhenlinien von $f$. und die Höhenlinie von $\varphi$ zur Höhe Null. Nun Zeichnen wir von beiden den Gradienten in dem Berührpunkt $E$ beider Höhenlinien.

Im Punkt $E$ berühren sich die Höhenlinien tangential. Dort ist das Maximum  unter der NB: 

Entlang $\varphi$ geht es bis dahin auf $f$ bergauf und danach wieder abwärts.

$\implies$ dort liegt $\nabla f \perp$ Tangenten von $\varphi$ in $E\perp \nabla \varphi \implies \nabla f \parallel \nabla \varphi$

\begin{align*}
    \nabla f &= -\lambda \nabla \varphi\\
    \implies \nabla f + \lambda \nabla \varphi &= 0 \quad\text{(notwendige Bedingung)}
\end{align*}

\subsubsection{Methode (notwendige Bedingung für kritische Punkte)}

1) Bilde Hilfspunkte:

\begin{align*}
    F(x, y, \lambda) = f(x, y) + \lambda*\varphi(x, y)
\end{align*}

2) Gradient der Funktion

\begin{align*}
    \nabla F &= \begin{pmatrix}
        0\\0\\0
    \end{pmatrix} = \nabla (f + \lambda*\varphi)\\
    F_x &= f_x + \lambda * \varphi_x = 0\\
    F_y &= f_y + \lambda * \varphi_y = 0\\
    F_{\lambda} &= \varphi = 0\\
\end{align*}

Das Gleichungssystem für $x, y, \lambda$ liefert die kritischen Punkte.

3) Außerdem können in "singulären" Punkte, die das System

\begin{align*}
    \nabla\varphi(x, y)&=\vec{0}\\
    \varphi(x, y)&=0
\end{align*}

erfüllen Extrema vorliegen $\implies$ Extrema untersuchen.

Hier versagt der Mechanismus von Lagrange, $\nabla \varphi = \vec{0}$

\begin{align*}
    \nabla f = -\lambda\nabla \varphi \implies \nabla f = -\lambda *\vec{0}
\end{align*}

Anmerkung: $\lambda$ ist eine Hilfsgröße

\subsubsection{Beispiele}

geg.: $f(X, y) = xy$, ges.: Extrema?

unter NB: $x+y=1 \implies x + y - 1 = 0$

Lösung: 1) $F(x, y, \lambda) = xy + \lambda (x+y-1)$

2)

\begin{align*}
    \nabla F = \begin{pmatrix}
        F_x\\F_y\\F_{\lambda}
    \end{pmatrix} &= \begin{pmatrix}
        y + \lambda\\ x + \lambda\\ x+y-1
    \end{pmatrix} = \begin{pmatrix}
        0\\0\\0
    \end{pmatrix}\\
    y &= -\lambda\\
    x &= -\lambda\\
    x+y &= 1\\
    x &= \frac{1}{2}\\
    \implies \lambda &= -\frac{1}{2}
\end{align*}

3) "Singuläre Punkte überprüfen

\begin{align*}
    \nabla \varphi = \begin{pmatrix}
        \varphi_x\\ \varphi_y
    \end{pmatrix} = \begin{pmatrix}
        1\\1
    \end{pmatrix} \neg \begin{pmatrix}
        0\\0
    \end{pmatrix}
\end{align*}

$\implies$ Es kann keine weiteren kritischen Punkte geben!

Andere Ansatz: Auflösung \& Einsetzen

NB: $y=y(x)=1-x$

\begin{align*}
    f(x, y) &= f(x, y(x)) = \tilde{f}(x) = x(1-x) = x- x^2\\
    \tilde{f}'(x) &= 1-2x = 0 \implies x= \frac{1}{2}\\
    \tilde{f}''(x) &= -2 < 0 \implies \text{lokales Maximum bie } (\frac{1}{2},\frac{1}{2})
\end{align*}

Anmerkung: Hinreichend ist die gerenderte Hesse Matrix.

\begin{align*}
    H_F(x, y, \lambda) = \begin{pmatrix}
        F_{\lambda\lambda} & F_{\lambda x} & F_{\lambda y}\\
        F_{x\lambda} & F_{x x} & F_{x y}\\
        F_{y\lambda} & F_{y x} & F_{y y}
    \end{pmatrix} = \begin{pmatrix}
        0 & \varphi_{x} & \varphi_{y}\\
        \varphi_{x} & F{xx} & F{yy}\\
        \varphi_{y} & F{yx} & F{yy}\\
    \end{pmatrix}
\end{align*}

Umsetzung $i$ führender Hauptminoren $i>2*m$ bei $m$ Nebenbedingungen $\implies$ d.h. für zweidimensionale Funktionen unter einer Nebenbedingung ($m=1$)

Hauptminoren $H_i$ mit $i<2*1$

\begin{align*}
    H_3 = det(H_f(x^0, y^0, \lambda^0)) < 0 \implies (x^0, y^0) \text{ Minimum}\\
    H_3 = det(H_f(x^0, y^0, \lambda^0)) > 0 \implies (x^0, y^0) \text{ Maximum}\\
\end{align*}

im Beispiel: $\varphi_x=1, \varphi_y=1, F_{xx}=0, F_{yy}=0, F_{xy}=1$

\begin{align*}
    H_3 = det\left(\begin{pmatrix}
        0 & 1 & 1\\
        1 & 0 & 1\\
        1 & 1 & 0
    \end{pmatrix}\right) = 2 > 0 \implies \text{lokales Maximum}
\end{align*}

Allgemein: 

\begin{align*}
    y = f(x_1, \dots, x_n)\\
    NB: \varphi_i(x_1, \dots, x_n) = 0, i =1, \dots, m\\
\end{align*}

1) Hilfsfunktionen:

\begin{equation*}
    F(x_1, \dots, x_n, \lambda_1, \dots, \lambda_m)
    = f(x_1, \dots, x_n) + \sum_{i=1}^{m} \varphi_i(x_1, \dots, x_n)
\end{equation*}

2) $\nabla F = \vec{0}$: Gleichungssystem

\begin{align*}
    \frac{\delta F}{\delta x_k} = 0, k=1,  ..., n\\
    \frac{\delta F}{\delta \lambda_i} = 0, i = 1, ..., m
\end{align*}

$n+m$ Gleichungen für $n+m$ Unbekannte

3) Zusätzliche mögliche Extrema an Stellen wo $\nabla \varphi_i$ lineare abhängig sind und die Nebenbedingungen unerfüllt sind!

Beispiel: geg.: Kreis $x^2+y^2-5=0$, Gerade $y=3x+30$, ges.: Minimaler Abstand zwischen Kreis und Gerade.

Lösung: 

\begin{figure}[ht]
    \centering
    \incfig{KreisGerade}
    \caption{KreisGerade}
    \label{fig:KreisGerade}
\end{figure}

Der Punkte $A=(a, b)$ und der Punkte $B=(c, d)$

1) Hilfsfunktionen:

\begin{equation*}
    F(a, b, c, d, \lambda, \mu) = (a-c)^2 + (b-d)^2 + \lambda(a^2+b^2-5) + \mu(2c-d+30) 
\end{equation*}

2) $\nabla F = 0$

\begin{align*}
    F_a &= 2(a-c) + 2\lambda a = 0\\
    F_b &= 2(b-d) + 2\lambda b = 0\\
    F_c &= -2(a-c) + 2\mu = 0\\
    F_d &= -2(b-d) - 2\mu = 0\\
    F_\lambda &= a^2 + b^2 - 5 = 0\\
    F_\mu &= 2c-d+30 = 0
\end{align*}

\begin{align*}
    \nabla \varphi_1 = \begin{pmatrix}
        2a\\2b\\0\\0
    \end{pmatrix}\\
    \nabla \varphi_2 = \begin{pmatrix}
        0\\0\\2\\-1
    \end{pmatrix}\\
\end{align*}

linear abhängig wenn $a = b = 0$, was aber nicht geht, da der Punkte $(a, b)$ auf dem Kreis mit Radius $\sqrt{5}$ liegt.

3) Lösen des Gleichungssystem

\begin{align*}
    (1.)+(3.): & 2\lambda a + 2\mu = 0 |*1\\
    (2.)+(4.): & 2\lambda b - \mu = 0 |*2\\
    \implies  & 2\lambda(a+2b) = 0
\end{align*}

Annahme $\lambda = 0$:

\begin{align*}
    (1.) & 2(a-c) = 0 \implies a=c\\
    (2.) & 2(b-d) = 0 \implies b=d\\
    \implies & (a, b)=(c, d) \text{ kann nicht sein}
\end{align*}

Der Kreis und die Gerade haben keine gemeinsame Punkte.

$a+2b=0$: $a=-2b$

\begin{align*}
    (5.): & (-2b)^2 + b^2 = 5 \implies 4b^2+b^2 = 5\\
    \implies & 5b^2 = 5\\
    \implies & b_{1/2} = \pm 1
\end{align*}

Aus der Skizze $b=+1$ und daraus folgt $b=-2$

\begin{align*}
    b*(1.)-a*(2.): 2(a-c)b + 2\lambda ab - 2(b-d)a - 2\lambda ba &=0\\
    2(a-c)b - 2(b-d)a &=0\\
    (a-c)b &= (b-d)a \text{ einsetzen}\\
    (-2-c)*1 &= (1-d)*(-2)\\
    c &= -2d\\
    (6.): d &= 2c+30\\
    c &= -2(2c+30)\\
    c &= 6
\end{align*}

\begin{align*}
    \lambda = \frac{-2(a-c)}{2a}=\frac{10}{2} = 5\\
    \mu = -2(b-d) = 10\\
\end{align*}

$\implies a=-2, b=1, c=-12, d=6, \lambda=5, \mu=10$

Abstand $l=\sqrt{(a-c)^2 + (b-d)^2} = \sqrt{125}$

\end{document}
\documentclass[a4paper]{article}

%\usepackage{url}

%% Math
\usepackage{mathtools}
%% For Mengen like natural numbers
\usepackage{amsfonts}
%% Für spezielle Symbole
\usepackage{amssymb}

%% Images
\usepackage{import}
\usepackage{xifthen}
\usepackage{pdfpages}
%\usepackage{transparent}

%%% Command for simpler images
\newcommand{\incfig}[1]{%
    \def\svgwidth{\columnwidth}
    \import{./fig/}{#1.pdf_tex}
}

%% Links
\usepackage{hyperref}
\hypersetup{
    colorlinks=true,
    linkcolor=black,
    filecolor=magenta,
    urlcolor=cyan
}

%% Formatting
\usepackage{parskip}

\title{Angewandte Mathematik}
\author{Moritz}
\date{October 1, 2025}

\begin{document}
\maketitle
\tableofcontents

\section{Differentialrechnung mehrstelliger Funktionen}

\subsection{Totale Differenzierbarkeit}

\subsubsection{Allgemeines}

$f: \mathbb{R} \to \mathbb{R}$, f differenzierbar $\to$ f ist stetig

$f: \mathbb{R}^n \to \mathbb{R}$, f partiell differenzierbar $\to$ f ist nicht zwingend stetig - f total differenzierbar $\to$ f ist stetig

Hierzu gibt es ein PDF, welches unter doc zu finden ist.

\subsubsection{Fehlerfortpflanzung}

Beispiel 1: geg: $z=x^2\sqrt{y}$

\begin{align*}
    \tilde{x}=1, \quad \Delta x = 0.1 (\Delta x = \pm 0.1)\\
    \tilde{y}=4, \quad \Delta y = 0.2 (\Delta y = \pm 0.2)
\end{align*}

Lösung:

\begin{align*}
    \Delta z &= |f_x| \Delta x + |f_y| \Delta y\\
    f_x &= 2x \sqrt{y}, \qquad f_x(\tilde{x}, \tilde{y})=4\\
    f_y &= \frac{x^2}{2\sqrt{y}}, \qquad f_y(\tilde{x}, \tilde{y})=\frac{1}{4}\\
    \Delta z &= 4 * 0.1 + \frac{1}{4} * 0.2 = 0.4 + 0.05 = 0.45\\
    & \implies \tilde{z} = f(\tilde{x}, \tilde{y}) = 1 * 2 = 2\\
    & \implies z = \tilde{z} \pm \Delta z = 2 \pm 0.45\\
    & \implies 1.55 \geq z \geq 2.45
\end{align*}

Beispiel 2: geg: $z = \frac{x}{y} + 2 \frac{y}{x}$

\begin{align*}
    \tilde{x} = 5, \quad \Delta x = 0.1\\
    \tilde{y} = 2, \quad \Delta y = 0.1\\
\end{align*}

Lösung:

\begin{align*}
    \tilde{z} &= f(\tilde{x}, \tilde{y}) = \frac{5}{2} + 2 \frac{2}{5} = \frac{33}{10} = 3.3\\
    f_x &= \frac{1}{y} - 2 \frac{y}{x^2}, \qquad f_x(\tilde{x}, \tilde{y}) = \frac{1}{2} - 2 \frac{2}{5^2} = \frac{25-8}{50} = \frac{17}{50} = 0.34\\
    f_< &= -\frac{x}{y^2} + \frac{2}{x}, \qquad f_x(\tilde{x}, \tilde{y}) = - \frac{5}{2^2} + \frac{2}{5} = \frac{-25 + 8}{20} = -\frac{17}{20} = - 0.85\\
    \Delta z &= |f_x| \Delta x + |f_y| \Delta y\\
     &= 0.34 * 0.1 + |-0.85| * 0.1 = 0.034 + 0.085 = 0.119\\
    & \implies z = \tilde{z} \pm \Delta z = 3.3 \pm 0.119 \\
    & \implies 3.181 \geq z \geq 3.419
\end{align*}

\subsection{Gradient und Richtungsableitung}

$f: \mathbb{R}^n\to \mathbb{R}$

\begin{equation*}
    grad f = \begin{pmatrix}
        \frac{\delta f}{\delta x_1} \\
        \frac{\delta f}{\delta x_2} \\
        \vdots \\
        \frac{\delta f}{\delta x_n} \\
    \end{pmatrix}
\end{equation*}

Die vereinfachte Darstellung erfolgt mit dem Nabla-Operator $\nabla$

\begin{equation*}
    \nabla = \begin{pmatrix}
        \frac{\delta}{\delta x_1} \\
        \frac{\delta}{\delta x_2} \\
        \vdots \\
        \frac{\delta}{\delta x_n} \\
    \end{pmatrix}: \nabla f = grad f
\end{equation*}

\noindent\fbox{
    \parbox{\textwidth-21pt}{
        Der Gradient von $f(\nabla f)$ ist ein Vektor in $\mathbb{R}^n$. Er steht senkrecht auf den Höhenlinien $f=const$ von $f$.

        Er liegt aber dennoch in der Tangentialebene des Punktes auf dieser Höhenlinien.
    }
}

$f: \mathbb{R}^2\to \mathbb{R}$, $f(x, y)$

totales Differential von $f$:

\begin{equation*}
    df = \frac{\delta f}{\delta x}dx + \frac{\delta f}{\delta } dy = grad f \circ \begin{pmatrix}
        dx \\ dy
    \end{pmatrix} = \nabla f \circ \begin{pmatrix}
        dx \\ dy
    \end{pmatrix}
\end{equation*}

Auf einer Höhenlinie $f=const$ ist $df=0$, da $f$ sich dort nicht ändert! $\implies$ Eigenschaft des Skalarprodukts $\implies \nabla f \perp\begin{pmatrix}
    dx \\ dy
\end{pmatrix}$ wobei $\begin{pmatrix}
    dx \\ dy
\end{pmatrix}$ in Tangentialrichtung der Höhenlinien zeigt.

Wir hatten eine Definition, diese habe ich verpasst. Muss Niklas oder Rosalie fragen.

Beispiel: geg: $f(x, y) = \frac{1 + \sin(x)}{1 + y}$

ges.: $f_{\vec{a}}(\vec{v_0})$

\begin{equation*}
    \vec{r_0}=\begin{pmatrix}
        0\\0
    \end{pmatrix}, \vec{a_0} = \begin{pmatrix}
        1\\2
    \end{pmatrix}
\end{equation*}

Lösung:

\begin{align*}
    f_{\vec{a}} = \frac{\vec{a}\circ \nabla f}{||\vec{a}||}\\
    f_x = \frac{wx}{1+y}, \qquad f_y = \frac{1 + \sin(x)}{(1+y)^2}
\end{align*}

\begin{align*}
    \nabla f |_{\vec{r_0}} = \begin{pmatrix}
        f_x|_{\vec{r_0}} \\
        f_y|_{\vec{r_0}}
    \end{pmatrix} = \begin{pmatrix}
        f_x|_{\begin{pmatrix}
        0\\0
    \end{pmatrix}} \\
        f_y|_{\begin{pmatrix}
        0\\0
    \end{pmatrix}}
    \end{pmatrix} = \begin{pmatrix}
        1 \\ -1
    \end{pmatrix}\\
    f_{\vec{a}} = \frac{\begin{pmatrix}
        1\\2
    \end{pmatrix} \circ \begin{pmatrix}
        1\\-1
    \end{pmatrix}}{\left|\left|\begin{pmatrix}
        1\\2
    \end{pmatrix}\right|\right|} = \frac{1 - 2}{\sqrt{1^2+2^2}}=-\frac{1}{\sqrt{5}}
\end{align*}

\end{document}
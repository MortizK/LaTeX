\documentclass[a4paper]{article}

%\usepackage{url}

%% Math
\usepackage{mathtools}
%% For Mengen like natural numbers
\usepackage{amsfonts}
%% Für spezielle Symbole
\usepackage{amssymb}

%% Images
\usepackage{import}
\usepackage{xifthen}
\usepackage{pdfpages}
%\usepackage{transparent}

%%% Command for simpler images
\newcommand{\incfig}[1]{%
    \def\svgwidth{\columnwidth}
    \import{./fig/}{#1.pdf_tex}
}

%% Links
\usepackage{hyperref}
\hypersetup{
    colorlinks=true,
    linkcolor=black,
    filecolor=magenta,
    urlcolor=cyan
}

%% Formatting
\usepackage{parskip}

\title{Angewandte Mathematik}
\author{Moritz}
\date{September 25, 2025}

\begin{document}
\maketitle
\tableofcontents

\section{Differentialrechnung mehrstelliger Funktionen}

\subsection{Stetigkeit}

Def.: Eine Funktion $f: \mathbb{R}^n\to \mathbb{R}$ heißt stetig im Punkt $x_0\in D_f\subset \mathbb{R}^n$, wenn für \textbf{jede} Folge $\{x^k\}$ mit $\lim_{k\to\infty}x^k=x_0$ gilt:

\begin{equation*}
    \lim_{k\to\infty} f(x^k)=f(x_0)
\end{equation*}

Beispiel stetiger Funktion:

\begin{enumerate}
    \item Polynome $f(x, y)=x^3+x^2+y-3y^3$
    \item Summen, Differenzen und Produkte stetiger Funktionen.
    \item Quotienten stetiger Funktionen, falls Nenner $\neq 0$. Hier kann dieser Punkt definiert werden, damit die Funktion stetig ist.
    \item Verkettung stetiger Funktion
\end{enumerate}

\begin{align*}
    f(x, y)&=\sqrt{\frac{(x^3-x^2y)(y^2-x)}{x^2+y^2}}\\
    g(x, y, z)&=\ln(1+x^2)\sin(xy)+z
\end{align*}

$f(x, y)$ ist stetig, außer bei $x=y=0$, dort ist es nicht definiert.

$g(x, y, z)$ ist im $\mathbb{R}^3$ stetig.

\subsubsection{Ausführliches Beispiel}

\begin{equation*}
    f(x_1, x_2)=\begin{cases}
        \frac{2x_1x_2}{x_1^2+x_2^2}, & \text{f"ur } (x_1, x_2)\neq(0,0)\\
        0, & \text{f"ur } x_1 = x_2 = 0
    \end{cases}
\end{equation*}

Die Funktion ist in $\mathbb{R}^2/\{0,0\}$ stetig, jedoch für $x^0=(0,0)$ unstetig.

Wir betrachten dafür verschiedene Punktfolgen $ \{x^k\}$ die gegen $(0,0)$ (verschieden) konvergieren.

Fall  1 und 2: entweder die $x_1$-Achse oder die $x_2$-Achse:

\begin{align*}
    x^k &=\begin{pmatrix}
        \frac{1}{k}\\0
    \end{pmatrix}, k=1, 2, \dots\\
    f(x^k)&=\frac{2*\frac{1}{k}*0}{(\frac{1}{k})^2+0^2}
            =\frac{0}{(\frac{1}{k})^2}=0\\
    &\implies \lim_{k\to\infty} f(x^k)=0
\end{align*}

Fall 3: Die Diagonale:

\begin{align*}
    x^k &=\begin{pmatrix}
        \frac{1}{k}\\\frac{1}{k}
    \end{pmatrix}, k=1, 2, \dots\\
    f(x^k)&=\frac{2*\frac{1}{k}*\frac{1}{k}}{(\frac{1}{k})^2+(\frac{1}{k})^2}
        =\frac{2*(\frac{1}{k})^2}{2*(\frac{1}{k})^2}=1\\
    &\implies \lim_{k\to\infty} f(x^k)=1
\end{align*}

Somit gibt es einen Widerspruch!

\subsubsection{Das Zweiter Beispiel}

\begin{equation*}
    f(x_1, x_2)=\begin{cases}
        x_1\sin(\frac{1}{x_1^2x_2^2}), & \text{f"ur } (x_1, x_2)\neq(0,0)\\
        0, & \text{f"ur } x_1 = x_2 = 0
    \end{cases}
\end{equation*}

Die Funktion ist eine Verkettung stetiger Funktionen $\implies$ somit ist diese in $\mathbb{R}^2/\{(0,0)\}$ stetig.

Frage: Stetigkeit in $(0,0)$?

Betrachtung beliebiger Punktfolge: 

\begin{equation*}
    x^k=\begin{pmatrix}
        x_1^k\\x_2^k
    \end{pmatrix}\xrightarrow{k\to\infty}\begin{pmatrix}
        0\\0
    \end{pmatrix}\implies\begin{matrix}
        x_1^k\xrightarrow{x\to\infty}0\\
        x_2^k\xrightarrow{x\to\infty}0
    \end{matrix}
\end{equation*}

\begin{align*}
    0\leq |f(x_1^k, x_2^k)|
    &=\left|x_1^k*\sin\left(\frac{1}{(x_1^k)^2+(x_2^k)^2}\right)\right|
    \\&=|x_1^k|*\left|\sin\left(\frac{1}{(x_1^k)^2+(x_2^k)^2}\right)\right|
    \leq |x_1^k|*1
\end{align*}

Für $\xrightarrow{k\to\infty}$ kommen wir auf $0\leq |f(x_1^k, x_2^k)|\leq 0$ somit bleibt nur die Gleichheit.

\begin{equation*}
    \implies \lim_{k\to\infty}f(x_1^k, x_2^k)=0=f(0,0)
\end{equation*}

Somit stetig in ganz $\mathbb{R}^2$

\subsection{Partielle Ableitungen}

Rückblick auf $\mathbb{R}: f: \mathbb{R}\to \mathbb{R}$

Wir haben in einer Grafik die Steigung ($f'(x)$) bestimmt. Hierzu sind die Tangente (das Ziel) und die Sekante (die Schätzung) wichtig.

\begin{equation*}
    f'(x_0):=\lim_{x\to0}\frac{f(x_0+h)-f(x_0)}{h}
\end{equation*}

\subsubsection{Verallgemeinerung auf $\mathbb{R}^2$}

Verallgemeinerung auf den $\mathbb{R}^2:f:\mathbb{R}^2\to \mathbb{R}$

Hierzu festhalten einer Variablen in $f(x,y)$, z.B $y=y_0$: $f$ kann man dann als Funktion mit nur einer Variablen betrachtet werden.

\begin{equation*}
    f(x, y_0)=\tilde{f}(x)
\end{equation*}

Es folgt wieder eine Skizze im $\mathbb{R}^3$, wo dann ein Schnitt gezeigt wird, in welchem wir die Ableitung durchführen können.

Def.: Eine Funktion $f(x, y)$ heißt im Punkt $(x_0, y_0)$ partiell nach $x$ differenzierbar, wenn der Grenzwert $\lim_{h\to0}\frac{f(x_0+h, y_0)-f(x_0, y_0)}{h}$ existiert. Er heißt dann partielle Ableitung von $f$ nach $x$ im Punkt $(x_0, y_0)$:

\begin{align*}
    \frac{\delta f}{\delta x}(x_0, y_0)\text{ oder } f_x(x_0,y_0)\\
    \frac{\delta f}{\delta x}(x_0, y_0)=f_y(x_0, y_0)=\lim_{h\to0}\frac{f(x_0, y_0+h)-f(x_0, y_0)}{h}
\end{align*}

Da man übrige Variablen konstant hält gelten alle Ableitungsregeln wie früher (wie für den "$\mathbb{R}$-Fall") - Analog für Funktionen mit mehr als 2 Variablen:

\begin{equation*}
    \frac{\delta f}{\delta x}(x, y, z), 
    f_x(x, y, z), 
    \frac{\delta f}{\delta x}, 
    f_x
\end{equation*}

\subsubsection{"partielle Differentialoperatoren"}

\begin{equation*}
    \frac{\delta}{\delta x}, 
    \frac{\delta}{\delta y}, 
    \dots
    \frac{d}{dx}, 
\end{equation*}

Erzeugen aus $f$ die entsprechenden partielle Ableitungen:

\begin{equation*}
     \frac{\delta}{\delta x}f(x, y ,z)=f_x,\quad
     \frac{\delta}{\delta y}f(x, y, z)=f_y
\end{equation*}

\subsubsection{Beispiel: paritelle Ableitung}

\begin{align*}
    f(x, y, z) &=2x^3+2xy+yz^5\\
    f_x &= 6x^2+2y\\
    f_y &= 2x+z^5\\
    f_z &= 5yz^4
\end{align*}

\subsubsection{Geometrische Interpretation}

Partielle Ableitungen von $f(x,y)$ sind die Anstiege der Tangentialebene in den Achsenrichtungen (wenn es eine Tangentialebene gibt).

\subsubsection{Weitere Beispiele}

Produktregel: $(uv)'=u'v+uv'$

\begin{align*}
    f(x,y) &= xy^2*(\sin(x)+\sin(y))\\
    \frac{\delta f}{\delta x} &= \frac{\delta u}{\delta x}v+u*\frac{\delta v}{\delta x}
    = y^2(\sin(x)+\sin(y))+xy^2\cos(x)\\
    \frac{\delta f}{\delta y} &= \frac{\delta u}{\delta y}v+u*\frac{\delta v}{\delta y}
    = 2xy(\sin(x)+\sin(y))+xy^2\cos(y)
\end{align*}

Kettenregel: $v(u)'=v'(u)*u'$

\begin{align*}
    z &= f(x, y)=\ln(x+y^2)\\
    u &= x+y^2\qquad z=ln(u)\\
    \frac{\delta z}{\delta x} 
    &= \frac{d z}{d u} * \frac{\delta u}{\delta x} 
    = \frac{1}{u}*1=\frac{1}{x+y^2}\\
    \frac{\delta z}{\delta y} 
    &= \frac{d z}{d u} * \frac{\delta u}{\delta y} 
    = \frac{1}{u}*2y=\frac{2y}{x+y^2}
\end{align*}

4tes Beispiel

\begin{align*}
    f(x, y, z) &= 2xe^{yz} + \sqrt{x^2+y^2+z^2}\\
    \frac{\delta f}{\delta y} 
    &= 2xze^{yz} + \frac{y}{\sqrt{x^2+y^2+z^2}}
\end{align*}

Quotientenregel: $(\frac{u}{v})'=\frac{u'v-uv'}{v^2}$

\begin{align*}
    z &= f(x, y)=\frac{x}{x^2+y^2}\\
    \frac{\delta f}{\delta x} 
    &= \frac{(x^2+y^2)-x2x}{(x^2+y^2)^2}
    = \frac{x^2+y^2-2x^2}{(x^2+y^2)^2}
    = \frac{y^2-x^2}{(x^2+y^2)^2}\\
    \frac{\delta f}{\delta y} 
    &= -\frac{x2y}{(x^2+y^2)^2}
    = -\frac{2xy}{(x^2+y^2)^2}
\end{align*}

6tes Beispiel

\begin{align*}
    f(x,y) &= x^y = e^{y*\ln(x)} (x>0)\\
    \frac{\delta f}{\delta x} 
    &= y*x^{y-1}\\
    \frac{\delta f}{\delta y} 
    &= \frac{\delta}{\delta y} e^{y*\ln(x)}
    = x^y*\ln(x)
\end{align*}

\subsection{Höhere Ableitungen und der Satz von Schwarz}

Können partielle Ableitungen einer mehrstelligen Funktion noch einmal partielle differenziert werden, so erhält man "partielle Ableitungen höherer Ordnung".

Bezeichnung:

\begin{align*}
    f(x, y, z)\xrightarrow{\frac{\delta}{\delta x}}
    \frac{\delta f}{\delta x}\xrightarrow{\frac{\delta}{\delta z}}
    \frac{\delta }{\delta z}\left(\frac{\delta f}{\delta x}\right)
    = \frac{\delta^2 f}{\delta x \delta z}\\
    f \qquad f_x \qquad f_{xy}
\end{align*}

Ableitungen nach mehreren unterschiedlichen Variablen heißen "gemischte Ableitungen"

\know{Höhere Ableitung}{Im Allgemeinen ist die Reihenfolge beim Differenzieren von Bedeutung}

Beispiel: Ableitung bis zur Ordnung 3 von $f(x, y) = x^3y+y$

Dieser kann ganz gut als Baum gezeichnet werden:

\begin{align*}
    f_x = 3x^2y\\
    f_{xx} = 6xy\\
    f_{xxx} = 6y\\
    f_{xxy} = 6x\\
    f_{xy} = 3x^2\\
    f_{xyx} = 6x\\
    f_{xyy} = 0\\
    f_y = x^3+1\\
    f_{yx} = 3x^2\\
    f_{yxx} = 6x\\
    f_{yxy} = 0\\
    f_{yy} = 0\\
    f_{yyx} = 0\\
    f_{yyy} = 0\\
\end{align*}

Der Satz von Schwarz besagt, dass bei Polynomen die Reihenfolge der gemischten Ableitungen das Ergebnis nicht ändert. So sind $f_{xxy}=f_{xyx}=f_{yxx}=6x$

\know{Satz von Schwarz}{Sind alle partiellen Ableitungen einer Funktion bis zur Ordnung $k$ stetig, so kommt es bei der Berechnung der gemischten Ableitungen bis zur Ordnung $k$ nicht auf die Reihenfolge des Differenzierens an.}

\end{document}
\documentclass[a4paper]{article}

%\usepackage{url}

%% Math
\usepackage{mathtools}
%% For Mengen like natural numbers
\usepackage{amsfonts}
%% Für spezielle Symbole
\usepackage{amssymb}

%% Images
\usepackage{import}
\usepackage{xifthen}
\usepackage{pdfpages}
%\usepackage{transparent}

%%% Command for simpler images
\newcommand{\incfig}[1]{%
    \def\svgwidth{\columnwidth}
    \import{./fig/}{#1.pdf_tex}
}

%% Links
\usepackage{hyperref}
\hypersetup{
    colorlinks=true,
    linkcolor=black,
    filecolor=magenta,
    urlcolor=cyan
}

%% Formatting
\usepackage{parskip}

\title{Angewandte Mathematik}
\author{Moritz}
\date{September 17, 2025}

\begin{document}
\maketitle
\tableofcontents

\section{Differentialrechnung mehrstelliger Funktionen}

\subsection{Konvergenz im $\mathbb{R}^n$}

Wir behandeln eine Folge $\{x^k\}$ von Vektoren $x^k=\left(\begin{smallmatrix}x_1^k\\ x_2^k\\ \dots\end{smallmatrix}\right)\in \mathbb{R}$ und einem Vektor $x=\left(\begin{smallmatrix}x_1\\ x_2\\ \dots\end{smallmatrix}\right)\in \mathbb{R}$ Damit definieren wir den Grenzwert der Folge:

% Definition
\begin{equation*}
    \lim_{k\to\infty}x^k=x \iff 
    ||x^k-x||=\sqrt{\sum_{i=1}^{n}(x^k_i-x_i)^2}\xrightarrow{k\to\infty}0
\end{equation*}

Wir verwenden hier meist die euklidische Norm, aber jede Norm in $\mathbb{R}^n$ wäre verwendbar.

Beispiel

\begin{equation*}
    x^k=\left(\begin{array}{r}
        1/k\cos(\frac{k\pi}{10})\\
        1/k\sin(\frac{k\pi}{10})
    \end{array}\right), k=1,2, \dots
\end{equation*}

Vermutung:

\begin{equation*}
    \lim_{k\to\infty}x^k=\left(\begin{array}{r}
        0\\
        0
    \end{array}\right)
\end{equation*}

Nachweis: Frage Rosalie für den Nachweis.

Wie in $\mathbb{R}^1$ gilt: 

\satz{Der Grenzwert einer Folge ist eindeutig. Beweis im Aufgabenblatt (Häufungspunkt)!}

\satz{
\begin{equation*}
    \lim_{x\to\infty} x^k=x \iff \forall i=1, \dots, n: \lim_{k\to\infty} x_i^k=x_i
\end{equation*}

Konvergent im $\mathbb{R}^n$ "$\iff$" Koordinatenweise Konvergenz}

Beweis: Es sei

\begin{align*}
    &\lim_{k\to\infty} x^k=x; d.h. \lim_{k\to\infty}\sqrt{\sum_{i=1}^{n}(x^k_i-x_i)^2}=0\\
    \iff &\lim_{k\to\infty}\sum_{i=1}^{n}(x_i^k-x_i)^2=0 \\
    \iff &\forall i=1, \dots, n: \lim_{k\to\infty} \sqrt{(x_i^k=x_i)^2}=\lim_{k\to\infty}|x_i^k-x_i|=0\\
    \iff &\forall i=1, \dots, n: \lim_{x\to\infty} x_i^k = x_i
\end{align*}

%Definition
Def.: $\epsilon$-Umgebung eines Punktes $x^0\in \mathbb{R}^n$:

\begin{equation*}
    U_\epsilon(x_0)=\{x\in \mathbb{R}^n\mid ||x-x^0||<\epsilon\} = B[Mittelpunkt, Radius]
\end{equation*}

n-dimensionale offene Kugel um $x^0$ in $\mathbb{R}^n$

\subsection{Mehrstelliger Funktionen}

Wir betrachte die Abbildung $f: \mathbb{R}^n\to \mathbb{R}$ oder allgemeiner $f: \mathbb{R}^n\to \mathbb{R}^n$.

Mitunter wird beispielsweise unterschieden: $f: \mathbb{R}^3\to \mathbb{R}$ ist ein "Skalarfeld", Temperaturfeld $T=T(x, y, z)$

$f: \mathbb{R}^3\to \mathbb{R}^3$ ist ein "Vektorfeld", z.B. Gravitationsfeld.

Mögliche Schreibweisen: $f(x, y, z), f(x_1, x_2, x_3), f(\vec{x})$ mit $\vec{x}=\left(\begin{smallmatrix}
    x_1\\x_2\\x_3
\end{smallmatrix}\right)$

\subsubsection{Visualisierung von Funktionen mehrerer Veränderlichen}

Funktionen mit 2 Veränderlichen (Variablen) lassen sich grafisch darstellen, bei Funktionen von mehre Variablen ist das nur eingeschränkt möglich.

Darstellung als "Fläche": $z=(f(y, x), (x, y)\in D_f)$ (definitionsbereich) 

% Visualisierung in 3D, mit der Fläche als das Ergebnis der Funktionswerte in dem Definitionsbereich.

"Fläche" im herkömmlichen Sinne falls $f$ stetig ist.

Erinnerung an Funktion in $\mathbb{R}$

\begin{equation*}
    f(x)=\begin{cases}
        1, \quad\text{falls x rational}\\
        0, \quad\text{falls x irrational} (\pi, e, ...)
    \end{cases}
\end{equation*}

Im einfachstem Fall ist der Definitionsbereich ein Achsen paralleles Rechteck. Möglich sind aber auch allgemeinere und komplexere $D_f$, z.B.

% Skizzen vgl. Rosalie
\begin{enumerate}
    \item einfach Zusammenhängend (Eine Form)
    \item Zusammenhängend (Eine Form mit "Löchern")
    \item nicht Zusammenhängend (Mehrere Formen auch mit Löchern)
\end{enumerate}

Graph: Entspricht der Menge der Punkte $(x, y, f(x, y))\in \mathbb{R}^3$

\subsubsection{Höhenlinien}

Beispiel: Höhenlinien in Topographischen Karten. Isobare auf Wetterkarten (Kurve gleichen Luftdrucks)

Höhenlinie von $f$ zur gleichen Höhe  $c=\{(x,y)\in \mathbb{R}^2\mid (x, y)\in D_f\and f(x,y)=c\}$

Schnitte des Graphen der Funktion mit Ebenen parallel zur x-y-Ebene in der Höhe c:

% Skizze mit Höhenlinien eines Berges. Dichtere Linien sind eine Höhere Steigung.

Die Berechnung ist unter Umständen nicht einfach, da $f(x,y)=c$ eine implizierte Gleichung ist.

\subsubsection{Schnittdiagramme}

Schnitte des dreidimensionalen Graphen mit Ebenen parallel zur x-z-, und y-z-Ebene.

\begin{equation*}
    z=f(x, y=c)=\tilde{f}(x)\quad || \text{ zu x-z-Ebene}
\end{equation*}

Beispiel: Ideales Gas $pV=n*R*T$, $R$-universelle Gaskonstante.

Ein Schnitt wäre: $p=f(T, V)=\frac{n*R*T}{V}$ für konkrete Temperature folgt Kurve gleicher Temperatur folgt Isotherme

\begin{equation*}
    p=n*R*\frac{T_{const}}{V}=\frac{C}{V} \implies p\sim \frac{1}{V}
\end{equation*}

\subsection{Stetigkeit}

\section{Integralrechnung mehrstelliger Funktionen}

\end{document}
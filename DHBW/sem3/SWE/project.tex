\documentclass[a4paper]{article}

%\usepackage{url}

%% Math
\usepackage{mathtools}
%% For Mengen like natural numbers
\usepackage{amsfonts}
%% Für spezielle Symbole
\usepackage{amssymb}

%% Images
\usepackage{import}
\usepackage{xifthen}
\usepackage{pdfpages}
%\usepackage{transparent}

%%% Command for simpler images
\newcommand{\incfig}[1]{%
    \def\svgwidth{\columnwidth}
    \import{./fig/}{#1.pdf_tex}
}

%% Links
\usepackage{hyperref}
\hypersetup{
    colorlinks=true,
    linkcolor=black,
    filecolor=magenta,
    urlcolor=cyan
}

%% Formatting
\usepackage{parskip}

\title{Software Engineering \\ \large Prüfungsleistung: Projekt}
\author{Moritz Köhler}

\begin{document}
\maketitle
\tableofcontents

\section{Vorbereitung/ Interview}

Situation: Ich bin Software-Entwickler und soll ein Tool zur Erstellung von Klausuren aus Prüfungsaufgaben.

\begin{enumerate}
    \item Wie liegen die einzelnen Prüfungsaufgaben vor? Pdf, Word, \LaTeX oder andere. Soll eine eigene Datenstruktur für Prüfungsaufgaben erstellt werden?
    \subitem Aktuell Libra Office mit allen Aufgaben
    \subitem Der Datensatz soll neue Teilaufgaben erstellen können mit Text (keine Grafiken) kein Import Nötig.
    \subitem Grafik Import Wäre Supper (v.2 erst mal nicht implementieren)
    \item Sind Teilaufgaben/Unteraufgaben in einer Prüfungsaufgabe enthalten oder liegen diese Einzeln for.
    \subitem Ja siehe unten.
    \item Wie sollen die Klausuren aus den Prüfungsaufgaben generiert werden? Per Zufall, Händisch oder anders?
    \subitem Wird vom Nutzer angegeben.
    \item In welcher Reihenfolge sollen die Aufgaben erscheinen? Soll die Reihenfolge manuell bearbeitbar sein?
    \subitem Die Kapitel werden angegeben.
    \item Welches Ausgabemedium soll entstehen: Pdf, Digital oder Druck.
    \subitem als PDF.
    \item Sollen nach Aufgaben White Space für Lösungen sein? Wenn Ja: Blanko, Liniert, Kariert, Gepunktet? Oder sind diese Schon in den Daten der Aufgaben hinterlegt?
    \subitem Als Kasten mit der Größe abhängig von der Punktzahl. Aber noch anpassbar.
    \item Bestehen die Daten auch aus Grafiken oder nur Text?
    \subitem nur Text, Grafiken für v.2
    \item Soll die Benotung in dem Tool möglich sein?
    \subitem 
    \item Wie soll die Punkteverteilung der Aufgaben geregelt werden.
    \subitem Sind in den Teilaufgaben. Das Kapitel hat eine Zielpunktzahl.
    \item Soll es ein Inhaltsverzeichnis der Klausur geben?
    \subitem Ja im Deckblatt, nicht teil der Anforderung.
    \item Wie ist das mit einem Deckblatt.
    \subitem Wird manuell vom Nutzer außerhalb vom Tool hinzugefügt.
    \item Sollen die Fertigen Klausuren verschlüsselt werden? Wenn Ja wie und mit welchem Standard?
    \subitem 
    \item Sollen Korrekturen von z.B. Rechtschreibung in den Prüfungsaufgaben innerhalb des Tools möglich sein. (Aufwendig, da Textbearbeitung implementiert werden muss)
    \subitem In der Datenerstellung.
    \item Soll die Software eigenständig sein, oder z.B. Word ergänzen, z.B. als neuen Reiter.
    \subitem Eigenständig, Lokal
    \item Geht es bei Klausuren auch um anderer Prüfungsleistung wie Projekte.
    \subitem 
    \item Gibt es für die Prüfungsaufgaben irgendwelche Rechte? Dürfen manche nur ihre Eigenen Aufgaben sehen?
    \subitem Es gibt nur einen Lokalen Nutzer
    \item Wer darf die Software Nutzen? Brauchen wir eine Benutzerverwaltung oder kann die bisherige Login Funktion der Hochschule genutzt werden?
    \subitem Open Source, Lokal
\end{enumerate}

\subsection{Kunde erzählt:}

Eine Software als einzelperson lokal auf dem Rechner. Mit Zufallsgenerator (Eine Klausur keine Varianten). Klausuraufgaben und Teilaufgaben in einer Datensammlung.

Verschieden Betriebssystem als Java Anwendung. Soll Open Source sein mit Java FX.

Zwei generieren Buttons: Klausur und Probeklausur.

Es gibt eine Probeklausur, die als Leitlinie gilt.


\subsection{Weitere Punkte aus dem Interview:}


Eine Aufgabe (mit Titel als \textbf{Kapitel}) besteht aus: Nummeriert (automatisch), \textbf{Teilaufgabe} (Buchstabe). Eine Teilaufgabe ist die Aufgabe mit Textfeld, Punktezahl, Musterlösung, Größe des White-Space (als Kasten) generieren durch Punktzahl. Teilaufgaben gehören zu einem Themengebiet. Musterlösung soll zusätlich als Zweit PDF gedruckt werden.

Jede Teilaufgabe hat Kategorie (Schwer, Mittel, Einfach), Wie viele Aufgaben mit Welche Kapitel es gibt. Eine Teilaufgabe kann gekennzeichnet werden, das es nur für Probeklausur.

Variante an Teilaufgaben. Es soll nur eine Variante in der Generierung vorkommen. \textbf{Variante} die Teilaufgaben haben hier die gleicher Punktzahl. Ist aber Egal.

Eine Aufgabe soll aus Teilaufgaben eines Kapitels mit einer Zielpunktzahl. Die Schwierigkeit soll in dritteln erfolgen. Wenn das Dritteln nicht möglich ist, soll der Druck verhindert werden.

Wenn die Zielpunktzahl nicht möglich ist: Es sollen nur mögliche Zielpunkte zur Auswahl sein.

Versionierung der Teilaufgabe ist nicht notwendig.

Randplatz: 2cm

Kopf und Fußzeilen. Fußzeile mit Seitenzahl 1/x, beginnend mit 1.

Jedes Kapitel ist eine Aufgabe. 1 zu 1

Die Sprache ist auf English. Es soll aber möglich sein in Zukunft weitere Sprache der Benutzeroberfläche hinzuzufügen.

Probeklausur: Ist identisch 

Code soll gut Dokumentiert sein (English)

Die Daten sollen als xml gespeichert werden. Einlesen und Laden für unterschiedliche Module.

Evtl. eine Vorschau des PDF.

Gesamtpunktzahl ist nicht relevant. Wird indirekt durch die Kapitel gesteuert.

In V.2 html fähiger Text

Reihenfolge der Teilaufgaben ist egal. Einfach nach Schwer

Benutzeroberfläche: einfach und übersichtlich mit Buttons, + hinzufügen, - löschen, exit, laden und speicher von xml (speichern unter). Jede Interaktion braucht einen Hover Text.

User Freie Software


\subsection{Begriffe für das Begriffslexikon:}

\begin{enumerate}
    \item Kapitel (eine Aufgabe)
    \item Aufgabe (Teilaufgaben)
    \item Teilaufgabe (Name, Punkte, Schwierigkeit, Varianten)
    \item Varianten (Frage, Lösung)
\end{enumerate}

\section{Anforderungsanalyse}

\know{Aufgabe}{Ordnen der Oberen Notizen.}

\subsection{Anforderungsdokument/ Lastenheft}

\subsection{Begriffslexikon}

\subsection{Spezifikation/ Pflichtenheft}

\end{document}
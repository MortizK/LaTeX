\documentclass[a4paper]{article}

%\usepackage{url}

%% Math
\usepackage{mathtools}
%% For Mengen like natural numbers
\usepackage{amsfonts}
%% Für spezielle Symbole
\usepackage{amssymb}

%% Images
\usepackage{import}
\usepackage{xifthen}
\usepackage{pdfpages}
%\usepackage{transparent}

%%% Command for simpler images
\newcommand{\incfig}[1]{%
    \def\svgwidth{\columnwidth}
    \import{./fig/}{#1.pdf_tex}
}

%% Links
\usepackage{hyperref}
\hypersetup{
    colorlinks=true,
    linkcolor=black,
    filecolor=magenta,
    urlcolor=cyan
}

%% Formatting
\usepackage{parskip}

\title{Software Engineering}
\author{Moritz}
\date{September 15, 2025}

\begin{document}
\maketitle
\tableofcontents

\section{Grundlagen}

\subsection{Wasserfallmodel}

Einzelnen Prozesse des Wasserfallmodells

\begin{enumerate}
    \item Analyse
    \item Spezifikation
    \item Grobentwurf
    \item Feinentwurf
    \item Implementierung und Modultests
    \item Integrationstest
    \item Abnahme, Betrieb und Wartung
\end{enumerate}

\subsection{SCRUM}

Die agile Entwicklung mit SCRUM. Grafik auf S. 16.

So gibt es eine Planungsphase für alles und kleiner Planungsphasen für die einzelnen sprints.

Ein Sprint bearbeitet dann mehrere Aufgaben und bespricht diese am Ende eines jeden sprints. Am Ende des sprints, sollte eine funktionierende Software als Ergebnis sein.

Der Burndownchart, zeigt ungefähr an, an welchem Datum alle Backlog Items fertig sind.

Wo wird der Kunde eingebunden. Hierfür gibt es einen \textbf{Product Owner}. Dieser entscheidet die Prioritäten der Items und entscheidet, ob die Anforderungen erreicht worden sind.

Idealgröße: 10 bis 12 Personen.

\section{Versionsverwaltung}

Hat das Ziel die Versionen und Varianten einer Software zu Organisieren und zu sichern.

Eine Variante ist eine parallelentwicklung (z.B. IOS App und Android App)

Eine Version ist das Ergebnis und löst die vorgehende Version ab.

Eine Konfiguration ist eine Zusammenstellung aus Versionen und Varianten. (z.B. Baukasten aus mehreren Features)

\subsection{Dateischutz}

Schema um zu verhindern, dass mehrere Developer gegenseitig Ihre Version überschreiben:

\begin{center}
    lock - modify - unlock
\end{center}

Dies hat das Problem, das nur ein Developer zur Zeit arbeiten kann. Eine Alternative ist folgendes:

Jeder Developer kopiert zentrale Dateien an einen Lokalen Ort (check-out). Diese Lokale Änderungen werden dann zurück an den zentralen Ort kopiert (commit).

\begin{center}
    copy - modify - merge
\end{center}

\subsection{git}

git ist die Grundlage für GitHub und ist das weitverbreiteste Versionsverwaltungstool.

\know{Löschen}{In git kann man auch alte commits ändern. z.B. wenn Secrets oder große Dateien entfernt werden soll.}

\section{Analyse und Spezifikation}

\end{document}
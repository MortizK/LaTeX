\documentclass[a4paper]{article}

%\usepackage{url}

%% Math
\usepackage{mathtools}
%% For Mengen like natural numbers
\usepackage{amsfonts}
%% Für spezielle Symbole
\usepackage{amssymb}

%% Images
\usepackage{import}
\usepackage{xifthen}
\usepackage{pdfpages}
%\usepackage{transparent}

%%% Command for simpler images
\newcommand{\incfig}[1]{%
    \def\svgwidth{\columnwidth}
    \import{./fig/}{#1.pdf_tex}
}

%% Links
\usepackage{hyperref}
\hypersetup{
    colorlinks=true,
    linkcolor=black,
    filecolor=magenta,
    urlcolor=cyan
}

%% Formatting
\usepackage{parskip}

\title{Software Engineering}
\author{Moritz}
\date{September 29, 2025}

\begin{document}
\maketitle
\tableofcontents

\section{Analyse und Spezifikation}

\subsection{Anforderung an die Spezifikation}

\begin{multicols}{2}

Anforderung an den Inhalt:

\begin{itemize}
    \item zutreffend
    \item vollständig
    \item widerspruchsfrei
    \item neutral
    \item nachvollziehbar
    \item objektivierbar
\end{itemize}

\columnbreak

Anforderungen an Darstellung und Form:

\begin{itemize}
    \item leicht verständlich
    \item präzise
    \item leicht erstellbar
    \item leicht verwendbar
\end{itemize}

von S.53

\end{multicols}

Das Begriffslexikon beschreibt die genutzten Begriffe, welche immer wieder in der Software vorkommen. Diese sollten auch für die Klassen benutzt werden.

Die Zusammenhänge der Begriffe werden im Begriffsmodell dokumentiert.

\subsection{UML}

Die Unified Modelling Language (UML) ist eine grafische Modellierungssprache. 

\href{https://www.pi.uni-hannover.de/fileadmin/pi/se/templates/UML-Poster-v2.1.pdf}{Wichtige Diagrammtypen}

\end{document}
\section{Begriffslexikon}

\subsection{Klausur (Exam)}

Wenn keine XML-Datei geladen ist, startet die Software mit einer Beispielstruktur, die vom Nutzer zur Erstellung der ersten oder weiteren XML-Dateien verwendet werden kann.

\begin{tabular}{ll}
    Name: & New Exam
\end{tabular}

Diese enthält zum Start noch keine Kapitel/Aufgaben.

Das Exam dient als Grundlage um die Klausur zu generieren.

\subsection{Kapitel (Chapter)}

Ein Kapitel ist die übergeordnete Struktur, um Aufgaben einem Thema zuzuordnen. Der vergebene Name wird bei der Erstellung der PDF-Datei als Titel der Aufgabe verwendet.

\begin{tabular}{ll}
    Name: & New Chapter
\end{tabular}

\subsection{Aufgabe (Task)}

Eine Aufgabe ist eine konkrete Aufgabenarchitektur und kann mehrere Varianten haben. Sie verwaltet zudem die zu erreichende Maximalpunktzahl, den Schwierigkeitsgrad und die Sichtbarkeit.

\begin{tabular}{ll}
    Name: & New Subtask\\
    Punktzahl: & 1\\
    Schwierigkeitsgrad: & "easy"\\
    Sichtbarkeit: & "exam"\\
    Variante: & Startet mit einer Variante (siehe Variant)
\end{tabular}

\textbf{Schwierigkeitsgrad:}

\begin{itemize}
    \item "easy"
    \item "medium"
    \item "hard"
\end{itemize}

\textbf{Sichtbarkeit:}

\begin{itemize}
    \item "exam" - echte Klausur
    \item "mock-exam" - Probeklausur
\end{itemize}

\subsection{Variante (Variant)}

Eine Variante verwaltet die konkrete Aufgabenstellung mit einer möglichen Musterlösung. Diese werden direkt für das Erstellen der PDF-Datei verwendet.

\begin{tabular}{ll}
    Frage: & New Question\\
    Antwort: & New Answer
\end{tabular}

\subsection{Punktwert / Scoring}

Jede Teilaufgabe hat eine maximale Punktzahl:

\begin{itemize}
    \item Wird beim PDF-Export angezeigt.
    \item Dient bei der Klausurgenerierung zur Zielpunktsumme.
\end{itemize}

\subsection{PDF-Datei / Export}

Die Ausgabe der Klausur erfolgt als PDF.  

\begin{itemize}
    \item Klausur ohne Lösungen
    \item Musterlösung
    \item Probeklausur (Mock-Exam) ohne Lösungen
    \item Probeklausur (Mock-Exam) mit Lösungen
\end{itemize}

\subsection{Validierung}

Prüft die Konsistenz des Exams:

\begin{itemize}
    \item Jede Teilaufgabe muss mindestens eine Variante besitzen.
    \item Punktwerte, Schwierigkeitsgrad und Sichtbarkeit werden geprüft.
    \item Bei Fehlern wird die PDF-Erstellung blockiert.
\end{itemize}

\subsection{Standardwerte}

Für neu angelegte Elemente existieren Defaultwerte:

\begin{itemize}
    \item Klausurname: "New Exam"
    \item Kapitelname: "New Chapter"
    \item Task Name: "New Subtask"
    \item Punkte: 1
    \item Schwierigkeitsgrad: "easy"
    \item Sichtbarkeit: "exam"
    \item Variante Frage/Antwort: "New Question" / "New Answer"
\end{itemize}

\subsection{Strukturdiagramm}

\begin{verbatim}
Exam (XML)
+-- Chapter
|   +-- Task
|   |   +-- Variant 1
|   |   +-- Variant 2
|   |   +-- ...
|   +-- ...
+-- ...
\end{verbatim}



\section{Begriffslexikon}

\subsection{Klausur (Exam)}

Wenn keine XML-Datei geladen ist startet die Software mit eine Beispiel Struktur, welche von dem Nutzer zur Erstellung der ersten oder weiteren XML-Dateien verwendet werden kann.

\begin{tabular}{ll}
    Name: & New Exam
\end{tabular}

Diese enthält zum Start noch keine Kapitel/ Aufgaben.

\subsection{Kapitel/ Aufgabe (Task)}

Ein Kapitel/ Aufgabe ist die übergeordnete Struktur um Teilaufgabe einem Thema zuzuordnen. Der Vergebene Name wird bei der Erstellung der PDF-Datei als Titel der Aufgabe verwendet.

\begin{tabular}{ll}
    Name: & New Chapter
\end{tabular}

\subsection{Teilaufgabe (Subtask)}

Eine Teilaufgabe ist eine Konkrete Aufgabenarchitektur und kann mehrere Varianten haben. 

Die Teilaufgabe verwaltet zudem die zu erreichende Maximalpunktzahl, den Schwierigkeitsgrad und die Sichtbarkeit.

Der Schwierigkeitsgrad kann folgende Werte annehmen: "easy", "medium", "hard".

Die Sichtbarkeit kann folgende Werte annehmen: "exam", "mock-exam".

\begin{tabular}{ll}
    Name: & New Subtask\\
    Punktzahl: & 5\\
    Schwierigkeitsgrad: & "easy"\\
    Sichtbarkeit: & "exam"\\
    Variante: & Startet mit einer Variante (Siehe Variante)
\end{tabular}

\subsection{Variante (Variant)}

Eine Variante verwaltet die konkrete Aufgabenstellung mit eine möglichen Musterlösung. Diese werden direkt für das erstellen der PDF-Datei verwendet.

\begin{tabular}{ll}
    Frage: New Question\\
    Answer: New Answer
\end{tabular}
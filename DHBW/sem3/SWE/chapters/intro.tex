\section{Einleitung}

\subsection{Zweck des Dokuments}

Dieses Dokument beschreibt die Software \textbf{genExam}, ein Werkzeug zur Verwaltung von Aufgabenpools und zur automatisierten Generierung von Klausuren. 

Es richtet sich an folgende Zielgruppen:

\begin{itemize}
    \item \textbf{Entwickler:} Verantwortlich für Implementierung, Wartung und Erweiterung der Software.
    \item \textbf{Tester:} Leiten das korrekte Verhalten der Software anhand der Spezifikation ab.
    \item \textbf{Lehrende:} Möchten die Funktionsweise nachvollziehen und Aufgabenpools erstellen.
\end{itemize}

Darüber hinaus soll das Dokument die Benutzung, Architektur und nicht-funktionale Anforderungen der Software klar und nachvollziehbar darstellen.

\subsection{Überblick über das System}

\textbf{genExam} ist eine JavaFX-basierte Desktop-Anwendung zur Verwaltung von Aufgabenpools, aus denen automatisch Klausuren generiert werden können.  

Die Aufgabenpools werden in \textbf{XML-Dateien} gespeichert und bestehen aus folgenden hierarchischen Elementen:

\begin{itemize}
    \item \textbf{Kapitel / Tasks}: Oberkategorien für Aufgaben.
    \item \textbf{Teilaufgaben / Subtasks}: Konkrete Aufgaben mit Punktzahl, Schwierigkeitsgrad und Sichtbarkeit.
    \item \textbf{Varianten / Variants}: Unterschiedliche Ausprägungen einer Teilaufgabe inklusive Musterlösung.
\end{itemize}

Beim Laden der XML-Dateien prüft das System die Struktur und Inhalte, um fehlerhafte Daten zu erkennen. Eine PDF-Generierung kann erst nach dieser Validierung und durch einen separaten Nutzerbefehl ausgelöst werden, wodurch sichergestellt wird, dass nur gültige Aufgaben exportiert werden.

\begin{itemize}
    \item Klausur ohne Musterlösung
    \item Klausur mit eingebetteten Musterlösungen
    \item Probeklausuren (nur "mock-exam"-Subtasks)
\end{itemize}

Das System validiert beim Laden der XML-Dateien die Struktur und Inhalte, um fehlerhafte Daten zu erkennen und die PDF-Generierung zu verhindern, falls ungültige Elemente vorhanden sind.

\subsection{Hauptmerkmale}

\begin{itemize}
    \item Import und Export von Aufgabenpools als XML-Dateien
    \item Verwaltung von Kapiteln (Tasks) und Teilaufgaben (Subtasks)
    \item Verwaltung von Varianten mit unterschiedlichen Aufgabenstellungen und Lösungen
    \item Automatische PDF-Generierung mit oder ohne Musterlösung
    \item Unterstützung von Probeklausuren mit spezifisch gekennzeichneten Subtasks
    \item Mehrsprachige Benutzeroberfläche
\end{itemize}

\subsection{Systemumgebung}

\textbf{genExam} ist plattformübergreifend lauffähig auf:

\begin{itemize}
    \item Windows 11 oder neuer
    \item macOS Monterey oder neuer
    \item Linux-Distributionen mit installiertem OpenJDK 17+
\end{itemize}

Die Software wird als ausführbares JAR-Paket verteilt und kann durch Doppelklick oder über die Konsole gestartet werden. Es sind Schreibrechte für XML- und PDF-Dateien erforderlich.

\subsection{Nicht-funktionale Anforderungen}

\subsubsection{Leistungsanforderungen}

\begin{itemize}
    \item Die Benutzeroberfläche reagiert innerhalb von 500\,ms auf Eingaben.
    \item Laden einer typischen XML-Datei ($<$1\,MB, ~500 Subtasks) dauert maximal 2\,s.
    \item Generierung einer PDF-Datei erfolgt in weniger als 5\,s.
    \item Skalierbarkeit: Aufgabenpools mit mehreren tausend Subtasks sollten ohne Absturz verarbeitet werden können.
\end{itemize}

\subsubsection{Wartbarkeit}

\begin{itemize}
    \item Die Software folgt der MVC-Architektur (Model-View-Controller).
    \item Öffentliche Klassen und Algorithmen sind mit JavaDoc dokumentiert.
    \item Erweiterungen oder Optimierungen der Software können modular umgesetzt werden.
    \item Automatisierte Tests (Unit-Tests) unterstützen die Stabilität bei Änderungen.
\end{itemize}

\subsubsection{Benutzerfreundlichkeit}

\begin{itemize}
    \item Intuitive Benutzeroberfläche mit Menüs, Dialogen und Übersichtslisten.
    \item Mehrsprachigkeit: UI kann auf alternative Sprachen umgestellt werden.
\end{itemize}

\subsubsection{Sicherheit und Datenintegrität}

\begin{itemize}
    \item Lokale Speicherung: Alle XML- und PDF-Dateien werden auf dem Rechner des Nutzers verwaltet.
    \item Validierung der XML-Dateien beim Laden, um fehlerhafte Strukturen zu verhindern.
    \item Schreibrechte für Speicherverzeichnisse müssen vorhanden sein.
    \item Keine Netzwerkverbindung erforderlich, wodurch externe Sicherheitsrisiken minimiert werden.
\end{itemize}

\subsubsection{Abhängigkeiten}

\begin{itemize}
    \item Java 17 oder höher
    \item JavaFX für die Benutzeroberfläche
    \item Externe Bibliotheken für PDF-Erzeugung und XML-Verarbeitung
\end{itemize}

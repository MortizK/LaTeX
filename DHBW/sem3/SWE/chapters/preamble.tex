
%% For Margins
\usepackage[margin=2.5cm]{geometry}

%% For Page Numbers
\usepackage{fancyhdr}
\usepackage{lastpage}
\pagestyle{fancy}
\fancyhf{} % clear all header/footer fields
\fancyfoot[C]{Page \thepage{} of \pageref{LastPage}}

\usepackage{changepage}  % vor den Definitionen einbinden

%% Environment für Einrückung / Margin
\newenvironment{addmargin}[1]{%
  \begin{adjustwidth}{#1}{0pt}%
}{%
  \end{adjustwidth}%
}

%% Command for Use Cases
\newcommand{\usecase}[6]{%
    \subsubsection*{#1}
    \label{#2}
    \begin{addmargin}{0.75em}

    \textbf{Kurzbeschreibung:}\\
    #3

    \textbf{Vorbedingungen:}\\
    #4

    \ifx&#5&  % Prüft, ob #5 leer ist
    \else
    \textbf{Ablauf:}
        \begin{enumerate}
            #5
        \end{enumerate}
    \fi

    \textbf{Nachbedingungen:}\\
    #6

    \end{addmargin}
}

%% Helper: Fehlerliste mit Items
\newcommand{\fehler}[1]{%
    \begin{itemize}
        #1
    \end{itemize}
}

%% Tikz for native uml rendering
\usepackage{tikz}
\usetikzlibrary{positioning, shapes, arrows, fit}

\tikzset{
    actor/.style={ellipse, draw, minimum width=2.5cm, minimum height=1cm, align=center},
    usecase/.style={rectangle, draw, rounded corners, align=center, minimum width=4cm, minimum height=1cm},
    package/.style={draw, rectangle, rounded corners, inner sep=5mm},
    include/.style={dashed, ->},
    extend/.style={dashed, ->, bend left=30},
}
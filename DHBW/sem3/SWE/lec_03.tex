\documentclass[a4paper]{article}

%\usepackage{url}

%% Math
\usepackage{mathtools}
%% For Mengen like natural numbers
\usepackage{amsfonts}
%% Für spezielle Symbole
\usepackage{amssymb}

%% Images
\usepackage{import}
\usepackage{xifthen}
\usepackage{pdfpages}
%\usepackage{transparent}

%%% Command for simpler images
\newcommand{\incfig}[1]{%
    \def\svgwidth{\columnwidth}
    \import{./fig/}{#1.pdf_tex}
}

%% Links
\usepackage{hyperref}
\hypersetup{
    colorlinks=true,
    linkcolor=black,
    filecolor=magenta,
    urlcolor=cyan
}

%% Formatting
\usepackage{parskip}

\title{Software Engineering}
\author{Moritz}
\date{September 22, 2025}

\begin{document}
\maketitle
\tableofcontents

\section{Analyse und Spezifikation}

Es muss bekannt sein, was entwickelt werden kann und soll. Es werden die Anforderungen einer Software gesammelt.

In der Spezifikationsphase werden die Anforderungen verbindlich dokumentiert.

\subsection{Spezifikation}

Die Spezifikation ist Grundlage für \dots

\begin{enumerate}
    \item die Abstimmung mit dem Kunden/dem Marketing
    \item den Entwurf und die Implementierung
    \item das Benutzerhandbuch
    \item die Testvorbereitung (Was ist das korrekte Verhalten)
    \item die Abnahme
    \item die Wiederverwendung
    \item die Klärung späterer Einwände, Regressansprüche
    \item eine spätere Neu-Implementierung
\end{enumerate}

Die Anforderungen sind unter IEEE 610.12 (1990) definiert.

Arten von Anforderungen:

\begin{enumerate}
    \item \textbf{Offene Anforderungen} müssen nur dokumentiert werden, da sie offen geäußert werden.
    \item \textbf{Latente Anforderungen} bestehen, werden dem Auftraggeber aber erst bewusst, wenn man ihn darauf hinweist.
    \item \textbf{Entwickler-Optionen} sind Anforderungen, die dem Kunden egal sind und deshalb vom Entwickler entschieden werden können.
    \item \textbf{Harte Anforderungen} sind exakt beschrieben, ihre korrekte Implementierung kann mit Ja oder Nein entschieden werden.
    \item Die korrekte Implementierung \textbf{weicher Anforderungen} kann nicht durch Ja oder Nein entschieden werden, zum Beispiel die Forderung nach Wartbarkeit.
    \item \textbf{Vage Anforderungen} sind nicht exakt bestimmt, sondern durch einen Übergang, zum Beispiel die Forderung nach einer schnellen Arbeitsweise.
\end{enumerate}

Nicht-Funktionale Anforderungen. Diese beziehen sich auf Zuverlässigkeit, Komfort, Benutzbarkeit und Wartbarkeit. Diese lassen sich nicht in einer Funktion wiederspiegeln.

Um mit diesen umzugehen, sollte die objektiv und prüfbar umformuliert werden, damit man dieser erfüllen kann. Für manche kann es Normen geben, wie die barrierefreiheit einer Webanwendung.

\subsection{Dokumente}

Im Verlauf der ersten beiden Phasen entstehen also mehre Dokumente.

Das Anforderungsdokument: Dokumentiert die Ergebnisse der Analyse, auch Lastenheft genannt.

Das Begriffslexikon: verzeichnet und erklärt die Begriffe dem Kunden.

Die Spezifikation: Dokumentiert alle verbindlichen zu realisierenden Funktionen, auch Pflichtenheft genannt.

\subsection{Analysen}

Die Ist- und Soll-Analysen sind wichtig. Hierzu sollte sich die aktuelle Lage angeguckt werden. Verwendete Software und deren Daten.

Nach dem Ist-Zustand soll der Soll-Zustand festgestellt werden. Neue Funktionen oder Verbesserungen.

\section{Projekt}

Das Projekt stellt sich aus den Unterschiedlichen Aufgaben des SWE zusammen, diese sind alle unter project.tex zusammengefasst.

\end{document}
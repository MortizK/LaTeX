\documentclass[a4paper]{article}

%\usepackage{url}

%% Math
\usepackage{mathtools}
%% For Mengen like natural numbers
\usepackage{amsfonts}
%% Für spezielle Symbole
\usepackage{amssymb}

%% Images
\usepackage{import}
\usepackage{xifthen}
\usepackage{pdfpages}
%\usepackage{transparent}

%%% Command for simpler images
\newcommand{\incfig}[1]{%
    \def\svgwidth{\columnwidth}
    \import{./fig/}{#1.pdf_tex}
}

%% Links
\usepackage{hyperref}
\hypersetup{
    colorlinks=true,
    linkcolor=black,
    filecolor=magenta,
    urlcolor=cyan
}

%% Formatting
\usepackage{parskip}

\title{Software Engineering}
\author{Moritz}
\date{September 8, 2025}


\begin{document}
\maketitle
\tableofcontents

\section{Einführung}

\begin{enumerate}
    \item Requirements Engineering
    \item Software Engineering/ Architecture
\end{enumerate}

Die Benotung des Moduls sind 50\% im 3 Semester und 50\% im 4 Semester.

In diesem ist es eine Einzelarbeit eines Programmentwurfs in Java, welche am 14.11.25 abzugeben ist und am 17.11.25 zu präsentieren ist.

\subsection{Begriffe}

\begin{enumerate}
    \item Vorgehens Modell
    \item Phasen und deren Zusammenhänge
    \item Requirement Engineering
    \item Analyse und Entwurfsmodelle (z.B. UML)
    \item Softwarearchitektur
    \item Continues Integration
    \item Versionsverwaltung
    \item Betrieb und Wartung
    \item Dokumentation
\end{enumerate}


\subsection{Wasserfallmodell}

Das ist die Gliederung der Vorlesung, welche nahe an dem Wasserfallmodell ist. Siehe S. 15 für Wasserfallmodell.

\begin{enumerate}
    \item Grundlagen
    \item Versionsverwaltung
    \item Analyse und Spezifikation
    \item Architektur und UML
    \item Codequalität, Coderichtlinien
    \item Qualitätssicherung und Software-Tests
    \item Vorgehensmodelle
\end{enumerate}

\subsection{Buchempfehlungen}

\begin{enumerate}
    \item Ian Sommerville: Software
        Engineering. Pearson Studium
        2018, 10. Auflage, 896 Seiten.
        ISBN 978-3868943443
    \item Jochen Ludewig, Horst Lichter:
        Software Engineering -
        Grundlagen, Menschen,
        Prozesse. dpunkt.verlag, 2023,
        3. Auflage, 712 Seiten.
        ISBN 978-3864905988
\end{enumerate}

\section{Grundlagen}

Software besteht also nicht nur aus dem Programm, sondern aus allen Dokumenten, Arbeitsanweisungen, Daten und Programmen, die für den Betrieb eines Rechnersystems benötigt werden.



\end{document}
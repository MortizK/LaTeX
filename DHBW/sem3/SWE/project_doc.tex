\documentclass[a4paper]{article}

%\usepackage{url}

%% Math
\usepackage{mathtools}
%% For Mengen like natural numbers
\usepackage{amsfonts}
%% Für spezielle Symbole
\usepackage{amssymb}

%% Images
\usepackage{import}
\usepackage{xifthen}
\usepackage{pdfpages}
%\usepackage{transparent}

%%% Command for simpler images
\newcommand{\incfig}[1]{%
    \def\svgwidth{\columnwidth}
    \import{./fig/}{#1.pdf_tex}
}

%% Links
\usepackage{hyperref}
\hypersetup{
    colorlinks=true,
    linkcolor=black,
    filecolor=magenta,
    urlcolor=cyan
}

%% Formatting
\usepackage{parskip}

%% For Margins
\usepackage[margin=2.5cm]{geometry}

%% For Page Numbers
\usepackage{fancyhdr}
\usepackage{lastpage}
\pagestyle{fancy}
\fancyhf{} % clear all header/footer fields
\fancyfoot[C]{Page \thepage{} of \pageref{LastPage}}

\usepackage{changepage}  % vor den Definitionen einbinden

%% Environment für Einrückung / Margin
\newenvironment{addmargin}[1]{%
  \begin{adjustwidth}{#1}{0pt}%
}{%
  \end{adjustwidth}%
}

%% Command for Use Cases
\newcommand{\usecase}[6]{%
    \subsubsection*{#1}
    \label{#2}
    \begin{addmargin}{0.75em}

    \textbf{Kurzbeschreibung:}\\
    #3

    \textbf{Vorbedingungen:}\\
    #4

    \ifx&#5&  % Prüft, ob #5 leer ist
    \else
    \textbf{Ablauf:}
        \begin{enumerate}
            #5
        \end{enumerate}
    \fi

    \textbf{Nachbedingungen:}\\
    #6

    \end{addmargin}
}

%% Helper: Fehlerliste mit Items
\newcommand{\fehler}[1]{%
    \begin{itemize}
        #1
    \end{itemize}
}


\title{Software Engineering \\ 
\large Prüfungsleistung: Projekt \\
\small Version: 1.0}
\author{Moritz Köhler}

\begin{document}
\maketitle
\tableofcontents

\pagebreak
\section{Einleitung}

\subsection{Zweck des Dokumentes}

Diese Spezifikation beschreibt die Software \textbf{genExam}. Diese Software ist ein Werkzeug zur Erstellung von Aufgabenpools und automatische Generierung von Klausuren.

Das Dokument richtet sich an: 

\begin{itemize}
    \item Entwickler, die für die technische Umsetzung und Erweiterung verantwortlich sind.
    \item Tester, die anhand der Spezifikation das Korrekte Verhalten der Software ableiten können.
    \item Lehrende, die die Funktionsweise der Software nachvollziehen möchten.
\end{itemize}

\subsection{Überblick über das System}

\textbf{genExam} ist eine JavaFx-Applikation zur Verwaltung von Aufgabenpools, aus denen Klausuren generiert werden können.

Das Aufgabenpool soll als \textbf{XML-Datei} gespeichert werden und besteht aus mehreren Aufgaben, welche als Kapitel der Klausur dienen. Jedes Kapitel kann in mehrere Teilaufgabe gegliedert werden, welche wiederrum aus mehreren Varianten bestehen können.

Aus dieser Struktur soll ein PDF generiert werden, welche random aus eine Auswahl an Kapitel eine Klausur als \textbf{PDF-Datei} exportiert.

Hauptmerkmale der Software:

\begin{itemize}
    \item Import und Export des Aufgabenpools als XML-Datei.
    \item Verwaltung der Aufgaben (Tasks).
    \item Verwaltung der Teilaufgaben (Subtasks) mit Punktzahl, Schwierigkeitsgrad und Sichtbarkeitsstufen.
    \item Verwaltung von Varianten mit unterschiedlichen Lösungen.
    \item Automatische Klausurgenerierung mit und ohne Musterlösung.
\end{itemize}

Die Software Arbeitet Lokal ohne eine Netzwerkverbindung, wodurch die Verwaltung der XML-Dateien und PDF-Dateien in der Verantwortung des Nutzer liegt.

\subsection{Systemumgebung}

\textbf{genExam} ist eine JavaFx-basierte Desktop Anwendung und somit auf mehreren System lauffähig:

\begin{itemize}
    \item Windows 11 oder neuer
    \item macOS Montery oder neuer
    \item Linux-Distribution mit installiertem OpenJDK 17+
\end{itemize}

Die Software wird als JAR-Paket veröffentlicht und kann nach dem Entpacken durch Doppelklick oder über die Konsole ausgeführt werden.

\subsection{Nicht-Funktionale Anforderungen}

\subsubsection{Leistungsanforderungen}

Die Anzahl der gespeicherten Aufgaben ist nur durch den verfügbaren Systemspeicher begrenzt.

Die Benutzeroberfläche reagiert innerhalb von 500ms nach Benutzereingaben.

Das Laden einer typische XML-Datei (< 1Mb, ~500 Teilaufgaben) darf maximal 2s dauern.

Das generieren einer PDF-Datei erfolgt in weniger als 5s.

\subsubsection{Wartbarkeit}

Alle Module folgen der MVC-Architektur (Model - View - Controller)

Es gibt für die öffentlichen Klassen und den genutzten Algorithmen eine Dokumentation mit JavaDoc.

Durch diese Schritte ist die Erweiterung der Software durch Funktionalität oder Optimierung der vorhanden Funktionalitäten einfacher möglich.

%% \section{Mengengerüst}
%% Habe ich schon unter Leistungsanforderungen beschrieben

\pagebreak
\section{Begriffslexikon}

\subsection{Klausur (Exam)}

Wenn keine XML-Datei geladen ist startet die Software mit eine Beispiel Struktur, welche von dem Nutzer zur Erstellung der ersten oder weiteren XML-Dateien verwendet werden kann.

\begin{tabular}{ll}
    Name: & New Exam
\end{tabular}

Diese enthält zum Start noch keine Kapitel/ Aufgaben.

\subsection{Kapitel/ Aufgabe (Task)}

Ein Kapitel/ Aufgabe ist die übergeordnete Struktur um Teilaufgabe einem Thema zuzuordnen. Der Vergebene Name wird bei der Erstellung der PDF-Datei als Titel der Aufgabe verwendet.

\begin{tabular}{ll}
    Name: & New Chapter
\end{tabular}

\subsection{Teilaufgabe (Subtask)}

Eine Teilaufgabe ist eine Konkrete Aufgabenarchitektur und kann mehrere Varianten haben. 

Die Teilaufgabe verwaltet zudem die zu erreichende Maximalpunktzahl, den Schwierigkeitsgrad und die Sichtbarkeit.

Der Schwierigkeitsgrad kann folgende Werte annehmen: "easy", "medium", "hard".

Die Sichtbarkeit kann folgende Werte annehmen: "exam", "mock-exam".

\begin{tabular}{ll}
    Name: & New Subtask\\
    Punktzahl: & 5\\
    Schwierigkeitsgrad: & "easy"\\
    Sichtbarkeit: & "exam"\\
    Variante: & Startet mit einer Variante (Siehe Variante)
\end{tabular}

\subsection{Variante (Variant)}

Eine Variante verwaltet die konkrete Aufgabenstellung mit eine möglichen Musterlösung. Diese werden direkt für das erstellen der PDF-Datei verwendet.

\begin{tabular}{ll}
    Frage: New Question\\
    Answer: New Answer
\end{tabular}

\pagebreak
\section{Benutzeroberfläche}

%% Darstellung und Beschreibung der einzelnen Komponenten der Benutzeroberfläche


\pagebreak
\section{Anwenderfälle}

\subsection{Aktoren}

Die primären Akteure der Software sind Dozierende oder anderes lehrendes Personal,
welches Aufgabenpools verwalten und daraus Klausuren generieren möchte.
Es werden lediglich grundlegende Computerkenntnisse vorausgesetzt.

\begin{multicols}{2}
\raggedcolumns

\subsection{Übersicht der Anwenderfälle}

\begin{tabular}{ll}
    UC-01 & \hyperref[uc:01]{XML-Datei neu anlegen} \\
    UC-02 & \hyperref[uc:02]{XML-Datei laden} \\
    UC-03 & \hyperref[uc:03]{XML-Datei speichern / Speichern unter} \\
    UC-04 & \hyperref[uc:04]{Kapitel anlegen} \\
    UC-05 & \hyperref[uc:05]{Kapitel bearbeiten} \\
    UC-06 & \hyperref[uc:06]{Kapitel löschen} \\
    UC-07 & \hyperref[uc:07]{Teilaufgabe anlegen} \\
    UC-08 & \hyperref[uc:08]{Teilaufgabe bearbeiten} \\
    UC-09 & \hyperref[uc:09]{Teilaufgabe löschen} \\
    UC-10 & \hyperref[uc:10]{Variante anlegen} \\
    UC-11 & \hyperref[uc:11]{Variante bearbeiten} \\
    UC-12 & \hyperref[uc:12]{Variante löschen} \\
    UC-13 & \hyperref[uc:13]{Klausurkonfiguration öffnen} \\
    UC-14 & \hyperref[uc:14]{Klausur generieren (PDF)} \\
    UC-15 & \hyperref[uc:15]{Musterlösung generieren (PDF)} \\
    UC-16 & \hyperref[uc:16]{Probeklausur generieren} \\
    UC-17 & \hyperref[uc:17]{PDF-Vorschau anzeigen} \\
    UC-18 & \hyperref[uc:18]{Sprache ändern} \\
\end{tabular}

\subsection{Anwenderfälle im Detail}

% -------------------------------
% UC-01
\usecase
{UC-01: XML-Datei neu anlegen}
{uc:01}
{Der Nutzer erstellt einen neuen, leeren Aufgabenpool.}
{Keine XML-Datei muss geladen sein.}
{
    \item Nutzer klickt auf "New Exam".
    \item genExam erzeugt eine interne Standardstruktur.
}
{Eine neue, leere Exam-Struktur ist im Speicher.}
\textbf{Fehlerbetrachtung:} Nicht relevant (Operation kann nicht fehlschlagen).

\columnbreak
% -------------------------------
% UC-02
\usecase
{UC-02: XML-Datei laden}
{uc:02}
{Eine bestehende XML-Datei wird geladen.}
{Datei muss existieren und korrektes Format besitzen.}
{
    \item Nutzer wählt mittels Dateidialog eine XML-Datei.
    \item System lädt und validiert die Struktur.
}
{Der Aufgabenpool ist vollständig sichtbar.}
\textbf{Fehlerbetrachtung:}
\fehler{
    \item Datei korrupt → Fehlermeldung, Abbruch.
    \item Datei hat falsche Struktur → Fehlermeldung, Abbruch.
}

% -------------------------------
% UC-03
\usecase
{UC-03: XML-Datei speichern / Speichern unter}
{uc:03}
{Der aktuelle Aufgabenpool wird gespeichert.}
{Ein gültiger Aufgabenpool wurde geladen oder erstellt.}
{
    \item Nutzer wählt "Save" oder "Save as".
    \item System schreibt XML-Datei.
}
{Alle Daten sind persistent gespeichert.}
\textbf{Fehlerbetrachtung:}
\fehler{
    \item Schreibfehler (z. B. fehlende Rechte) → Meldung.
}

\columnbreak
% -------------------------------
% UC-04
\usecase
{UC-04: Kapitel anlegen}
{uc:04}
{Ein neues Kapitel wird zum Aufgabenpool hinzugefügt.}
{Ein Aufgabenpool ist geladen oder neu erstellt.}
{
    \item Nutzer klickt auf "Add Chapter".
    \item Kapitel wird mit Standardname angelegt.
}
{Ein neues Kapitel existiert im XML-Modell.}
\textbf{Fehlerbetrachtung:} Keine - trivialer Vorgang.

% -------------------------------
% UC-05
\usecase
{UC-05: Kapitel bearbeiten}
{uc:05}
{Kapitelname oder Metadaten werden geändert.}
{Kapitel existiert.}
{
    \item Nutzer wählt Kapitel aus.
    \item Nutzer ändert Namen oder Kürzel.
}
{Kapitel wurde aktualisiert.}
\textbf{Fehlerbetrachtung:} Ungültiger Name → Hinweis, Änderung wird nicht übernommen.

% -------------------------------
% UC-06
\usecase
{UC-06: Kapitel löschen}
{uc:06}
{Ein Kapitel inklusive aller Unterelemente wird gelöscht.}
{Kapitel existiert.}
{
    \item Nutzer klickt auf "Delete".
    \item Sicherheitsabfrage erscheint.
}
{Kapitel ist unwiderruflich entfernt.}
\textbf{Fehlerbetrachtung:} Keine - außer Nutzer bricht ab.

\columnbreak
% -------------------------------
% UC-07
\usecase
{UC-07: Teilaufgabe anlegen}
{uc:07}
{Innerhalb eines Kapitels wird ein Subtask erstellt.}
{Mindestens ein Kapitel existiert.}
{
    \item Nutzer wählt Kapitel.
    \item Nutzer klickt "Add Subtask".
}
{Neuer Subtask mit Standardvariant existiert.}
\textbf{Fehlerbetrachtung:} Keine.

% -------------------------------
% UC-08
\usecase
{UC-08: Teilaufgabe bearbeiten}
{uc:08}
{Ein Subtask wird editiert.}
{Subtask existiert.}
{
    \item Nutzer öffnet Subtask.
    \item Nutzer ändert Punktzahl, Difficulty, Scope.
}
{Alle Änderungen sind übernommen.}
\textbf{Fehlerbetrachtung:}
\fehler{
    \item Punktzahl nicht im erlaubten Bereich.
    \item Difficulty fehlt.
    \item Scope ungültig.
}

% -------------------------------
% UC-09
\usecase
{UC-09: Teilaufgabe löschen}
{uc:09}
{Ein Subtask wird gelöscht.}
{Subtask existiert.}
{}
{Subtask ist unwiderruflich geloescht.}
\textbf{Fehlerbetrachtung:} Keine.

\columnbreak
% -------------------------------
% UC-10
\usecase
{UC-10: Variante anlegen}
{uc:10}
{Eine neue Variante wird einem Subtask hinzugefügt.}
{Subtask existiert.}
{
    \item Nutzer klickt "Add Variant".
}
{Variante existiert mit Standardwerten.}
\textbf{Fehlerbetrachtung:} Keine.

% -------------------------------
% UC-11
\usecase
{UC-11: Variante bearbeiten}
{uc:11}
{Frage- oder Antworttext einer Variante wird geändert.}
{Variante existiert.}
{
    \item Nutzer öffnet Variante.
    \item Nutzer bearbeitet Texte.
}
{Änderungen sind gespeichert.}
\textbf{Fehlerbetrachtung:}
\fehler{
    \item Leere Frage → Fehlermeldung.
}

% -------------------------------
% UC-12
\usecase
{UC-12: Variante löschen}
{uc:12}
{Eine Variante wird gelöscht.}
{Variante existiert.}
{}
{Variante gelöscht. Falls letzte Variante gelöscht → Subtask wird ebenfalls gelöscht.}
\textbf{Fehlerbetrachtung:} Keine.

% -------------------------------
% UC-13
\usecase
{UC-13: Klausurkonfiguration öffnen}
{uc:13}
{Der Nutzer öffnet das Menü zur Generierung einer Klausur.}
{Mindestens ein Kapitel mit gültigen Subtasks existiert.}
{}
{Konfigurationsdialog sichtbar.}
\textbf{Fehlerbetrachtung:} Keine generierbaren Kapitel → Hinweis.

% -------------------------------
% UC-14
\usecase
{UC-14: Klausur generieren (PDF)}
{uc:14}
{System erstellt eine PDF-Klausur ohne Musterlösung.}
{\begin{itemize}
    \item Klausur ist gültig (Difficulty 33\%-Regel).
    \item Ausgabeort vom Nutzer gewählt.
\end{itemize}}
{}
{PDF-Datei existiert.}
\textbf{Fehlerbetrachtung:}
\fehler{
    \item Zielpunktzahl nicht möglich.
    \item Keine gültigen Aufgaben pro Kapitel.
}

% -------------------------------
% UC-15
\usecase
{UC-15: Musterlösung generieren}
{uc:15}
{Erzeugt eine PDF-Datei mit eingebetteten Musterlösungen.}
{Mindestens eine Variante hat eine Antwort.}
{}
{"exam.lsg.pdf" existiert.}
\textbf{Fehlerbetrachtung:}
\fehler{
    \item Keine Lösungen vorhanden → Warnung.
}

% -------------------------------
% UC-16
\usecase
{UC-16: Probeklausur generieren}
{uc:16}
{Erzeugt eine rein aus Mock-Subtasks bestehende Klausur.}
{Mindestens ein Subtask ist als "mock-exam" markiert.}
{}
{Mock-Klausur-PDF existiert.}
\textbf{Fehlerbetrachtung:}
\fehler{
    \item Keine Mock-Aufgaben vorhanden → Fehlermeldung.
}

\columnbreak
% -------------------------------
% UC-17
\usecase
{UC-17: PDF-Vorschau anzeigen}
{uc:17}
{Vor dem Export wird eine Vorschau gerendert.}
{PDF muss generierbar sein.}
{}
{}
\textbf{Fehlerbetrachtung:}
\fehler{
    \item Fehlende PDF-Engine → Meldung.
}

\columnbreak
% -------------------------------
% UC-18
\usecase
{UC-18: Sprache ändern}
{uc:18}
{Die Benutzeroberfläche kann auf eine alternative Sprache umgestellt werden.}
{Alternative Sprache installiert.}
{}
{UI lädt neue Übersetzungen.}
\textbf{Fehlerbetrachtung:}
\fehler{
    \item Fehlende Übersetzung → Fallback zu Englisch.
}

\end{multicols}

\end{document}
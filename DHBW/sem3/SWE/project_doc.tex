\documentclass[a4paper]{article}

%\usepackage{url}

%% Math
\usepackage{mathtools}
%% For Mengen like natural numbers
\usepackage{amsfonts}
%% Für spezielle Symbole
\usepackage{amssymb}

%% Images
\usepackage{import}
\usepackage{xifthen}
\usepackage{pdfpages}
%\usepackage{transparent}

%%% Command for simpler images
\newcommand{\incfig}[1]{%
    \def\svgwidth{\columnwidth}
    \import{./fig/}{#1.pdf_tex}
}

%% Links
\usepackage{hyperref}
\hypersetup{
    colorlinks=true,
    linkcolor=black,
    filecolor=magenta,
    urlcolor=cyan
}

%% Formatting
\usepackage{parskip}

\title{Software Engineering \\ 
\large Prüfungsleistung: Projekt \\
\small Version: 1.0}
\author{Moritz Köhler}

\begin{document}
\maketitle
\tableofcontents

\pagebreak
\section{Einleitung}

\subsection{Zweck des Dokumentes}

Diese Spezifikation beschreibt die Software \textbf{genExam}. Diese Software ist ein Werkzeug zur Erstellung von Aufgabenpools und automatische Generierung von Klausuren.

Das Dokument richtet sich an: 

\begin{itemize}
    \item Entwickler, die für die technische Umsetzung und Erweiterung verantwortlich sind.
    \item Tester, die anhand der Spezifikation das Korrekte Verhalten der Software ableiten können.
    \item Lehrende, die die Funktionsweise der Software nachvollziehen möchten.
\end{itemize}

\subsection{Überblick über das System}

\textbf{genExam} ist eine JavaFx-Applikation zur Verwaltung von Aufgabenpools, aus denen Klausuren generiert werden können.

Das Aufgabenpool soll als \textbf{XML-Datei} gespeichert werden und besteht aus mehreren Aufgaben, welche als Kapitel der Klausur dienen. Jedes Kapitel kann in mehrere Teilaufgabe gegliedert werden, welche wiederrum aus mehreren Varianten bestehen können.

Aus dieser Struktur soll ein PDF generiert werden, welche random aus eine Auswahl an Kapitel eine Klausur als \textbf{PDF-Datei} exportiert.

Hauptmerkmale der Software:

\begin{itemize}
    \item Import und Export des Aufgabenpools als XML-Datei.
    \item Verwaltung der Aufgaben (Tasks).
    \item Verwaltung der Teilaufgaben (Subtasks) mit Punktzahl, Schwierigkeitsgrad und Sichtbarkeitsstufen.
    \item Verwaltung von Varianten mit unterschiedlichen Lösungen.
    \item Automatische Klausurgenerierung mit und ohne Musterlösung.
\end{itemize}

Die Software Arbeitet Lokal ohne eine Netzwerkverbindung, wodurch die Verwaltung der XML-Dateien und PDF-Dateien in der Verantwortung des Nutzer liegt.

\subsection{Systemumgebung}

\textbf{genExam} ist eine JavaFx-basierte Desktop Anwendung und somit auf mehreren System lauffähig:

\begin{itemize}
    \item Windows 11 oder neuer
    \item macOS Montery oder neuer
    \item Linux-Distribution mit installiertem OpenJDK 17+
\end{itemize}

Die Software wird als JAR-Paket veröffentlicht und kann nach dem Entpacken durch Doppelklick oder über die Konsole ausgeführt werden.

\subsection{Nicht-Funktionale Anforderungen}

\subsubsection{Leistungsanforderungen}

Die Anzahl der gespeicherten Aufgaben ist nur durch den verfügbaren Systemspeicher begrenzt.

Die Benutzeroberfläche reagiert innerhalb von 500ms nach Benutzereingaben.

Das Laden einer typische XML-Datei (< 1Mb, ~500 Teilaufgaben) darf maximal 2s dauern.

Das generieren einer PDF-Datei erfolgt in weniger als 5s.

\subsubsection{Wartbarkeit}

Alle Module folgen der MVC-Architektur (Model - View - Controller)

Es gibt für die öffentlichen Klassen und den genutzten Algorithmen eine Dokumentation mit JavaDoc.

Durch diese Schritte ist die Erweiterung der Software durch Funktionalität oder Optimierung der vorhanden Funktionalitäten einfacher möglich.

%% \section{Mengengerüst}
%% Habe ich schon unter Leistungsanforderungen beschrieben

\pagebreak
\section{Begriffslexikon}

\subsection{Klausur (Exam)}

Wenn keine XML-Datei geladen ist startet die Software mit eine Beispiel Struktur, welche von dem Nutzer zur Erstellung der ersten oder weiteren XML-Dateien verwendet werden kann.

\begin{tabular}{ll}
    Name: & New Exam
\end{tabular}

Diese enthält zum Start noch keine Kapitel/ Aufgaben.

\subsection{Kapitel/ Aufgabe (Task)}

Ein Kapitel/ Aufgabe ist die übergeordnete Struktur um Teilaufgabe einem Thema zuzuordnen. Der Vergebene Name wird bei der Erstellung der PDF-Datei als Titel der Aufgabe verwendet.

\begin{tabular}{ll}
    Name: & New Chapter
\end{tabular}

\subsection{Teilaufgabe (Subtask)}

Eine Teilaufgabe ist eine Konkrete Aufgabenarchitektur und kann mehrere Varianten haben. 

Die Teilaufgabe verwaltet zudem die zu erreichende Maximalpunktzahl, den Schwierigkeitsgrad und die Sichtbarkeit.

Der Schwierigkeitsgrad kann folgende Werte annehmen: "easy", "medium", "hard".

Die Sichtbarkeit kann folgende Werte annehmen: "exam", "mock-exam".

\begin{tabular}{ll}
    Name: & New Subtask\\
    Punktzahl: & 5\\
    Schwierigkeitsgrad: & "easy"\\
    Sichtbarkeit: & "exam"\\
    Variante: & Startet mit einer Variante (Siehe Variante)
\end{tabular}

\subsection{Variante (Variant)}

Eine Variante verwaltet die konkrete Aufgabenstellung mit eine möglichen Musterlösung. Diese werden direkt für das erstellen der PDF-Datei verwendet.

\begin{tabular}{ll}
    Frage: New Question\\
    Answer: New Answer
\end{tabular}

\pagebreak
\section{Anwenderfälle}

\subsection{Aktoren}

Dozenten oder anderes Lehrende Personal von primär Hochschulen und Universitäten. 

Diese Person sollte minimales Technisches und Computerkenntnisse haben, um die Software vollständig bedienen zu können.

\subsection{Übersicht}

\begin{tabular}{ll}
    Dateiverwaltung &  XML-Dateien anlegen, speichern und laden\\
    Aufgabenverwaltung &  Task und Subtasks erzeugen, bearbeiten und löschen\\
    Variantenverwaltung &  Varianten erzeugen, bearbeiten und löschen\\
    Exam-Erstellung &  Eine Klausur konfigurieren und exportieren\\
\end{tabular}

%% Die Verschiedene Anwenderfälle (Use-Cases)

\subsection{Dateiverwaltung}

\subsection{Aufgabenverwaltung}

\subsection{Variantenverwaltung}

\subsection{Exam-Erstellung}

\pagebreak
\section{Benutzeroberfläche}

%% Darstellung und Beschreibung der einzelnen Komponenten der Benutzeroberfläche

\end{document}
\documentclass[a4paper]{article}

%\usepackage{url}

%% Math
\usepackage{mathtools}
%% For Mengen like natural numbers
\usepackage{amsfonts}
%% Für spezielle Symbole
\usepackage{amssymb}

%% Images
\usepackage{import}
\usepackage{xifthen}
\usepackage{pdfpages}
%\usepackage{transparent}

%%% Command for simpler images
\newcommand{\incfig}[1]{%
    \def\svgwidth{\columnwidth}
    \import{./fig/}{#1.pdf_tex}
}

%% Links
\usepackage{hyperref}
\hypersetup{
    colorlinks=true,
    linkcolor=black,
    filecolor=magenta,
    urlcolor=cyan
}

%% Formatting
\usepackage{parskip}

\title{Rechnerarchitektur}
\author{Moritz}
\date{September 22, 2025}

\begin{document}
\maketitle
\tableofcontents

\section{Arithmetik}

\subsection{Festkommanzahlen}

Wir Teilen unsere Wertigkeit einer 8-Bit Zahl einfach in einen Bereich mit Exponenten $0$ bis $n-k$ und $-1$ bis $-k$.

\begin{equation*}
    \sum_{i=-k}^{n-k-1}z_i*2^i
\end{equation*}

Dies hat zwei Nachteile:

\begin{enumerate}
    \item Der Wertebereich halbiert sich mit jedem weiterem $k$
    \item Ganz kleine Zahlen lassen sich mit festem $k$ teilweise nicht darstellen.
\end{enumerate}

Bei der Umrechnung muss gerundet werden. Im allgemeinem wird gerundet, wenn die zu rundende $\geq$ die Hälfte ist. Im 2er System ist das exakt die nächste Stelle.

So muss für $k=4$ bei der Umrechnung von $0,2_{10}$ genau $k+1=5$ Nachkommastellen ausgerechnet werden und dann korrekt Runden.

\end{document}
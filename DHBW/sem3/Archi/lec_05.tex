\documentclass[a4paper]{article}

%\usepackage{url}

%% Math
\usepackage{mathtools}
%% For Mengen like natural numbers
\usepackage{amsfonts}
%% Für spezielle Symbole
\usepackage{amssymb}

%% Images
\usepackage{import}
\usepackage{xifthen}
\usepackage{pdfpages}
%\usepackage{transparent}

%%% Command for simpler images
\newcommand{\incfig}[1]{%
    \def\svgwidth{\columnwidth}
    \import{./fig/}{#1.pdf_tex}
}

%% Links
\usepackage{hyperref}
\hypersetup{
    colorlinks=true,
    linkcolor=black,
    filecolor=magenta,
    urlcolor=cyan
}

%% Formatting
\usepackage{parskip}

\title{Rechnerarchitektur}
\author{Moritz}
\date{September 22, 2025}

\begin{document}
\maketitle
\tableofcontents

\section{Arithmetik}

\subsection{Addierwerk}

Vergleich der Addierwerke an der Addition einer 64-Bit Zahl:

\begin{enumerate}
    \item Serielles Addierwerk \begin{equation*}
        t_{ges} = (t_{FA} + t_{FF}) * n = 5\tau * n = 32ns
    \end{equation*}
    \item RCA: Ripple Carry Adder \begin{equation*}
        t_{ges} = t_{FA} * n = 2\tau * n = 2 * 0,1ns * 64 = 12,8ns
    \end{equation*}
    \item CLA: Carry Look Ahead Adder \begin{equation*}
        t_{ges} = t_{g,p} + t_{carry} + t_{s} = 6\tau
    \end{equation*}
    Nur praktikabel für kleine $n (< 16)$
    \item RCA aus CLAs (m-bit) \begin{equation*}
        t_{ges} = \left\lceil\frac{n}{m}\right\rceil * 2\tau + 4\tau
    \end{equation*}
    für $m=16$ ist $t_{ges} = 12\tau = 1,2ns$
\end{enumerate}

Wir müssen von allen die Formel der Zeit können und einen Kritischen Pfad einzeichnen können.

\subsection{Subtraktion}

Die Subtraktion funktioniert indem wir $a + (-b)$ rechnen. Das Funktioniert im 2er Komplement.

Die Beispielrechnungen sind im Script S.96.

Wenn die Beiden Zahl unterschiedliche Vorzeichen haben (Also $a, -b$), so hat die Summe dieser ($a + (-b)$) immer das Richtige Ergebnis.

Realisierung. Wir ergänzen eine Steuerleitung, welche in das $c_{in}$ kommt. Dies ist die plus 0 oder 1.

Die $b$-Eingänge werden mit der Steuerleitung $S$ XOR, quasi negiert.

\begin{center}
    \begin{tabular}{cc|c}
        b & S & \\
        \hline
        0 & 0 & 0 \\
        1 & 0 & 1 \\
        0 & 1 & 1 \\
        1 & 1 & 0 \\
    \end{tabular}
    \quad
    \begin{tabular}{cc|c}
        b & S & \\
        \hline
        $b$ & 0 & $b$ \\
        $b$ & 1 & $\bar{b}$ \\
    \end{tabular} 
\end{center}

Dieser braucht dann für das XOR, $2\tau$ mehr Zeit, als der Verbauter Addierer. 

\subsubsection{Overflow Prüfen}

Nun muss nur noch geprüft werden, ob das Ergebnis korrekt ist. Dies kann mit den letzten beiden Carry-Bits gemacht werden.

An den Beispielen auf Slide S.98 wird deutlich, dass hierfür ein XOR geeignet ist.

\subsubsection{Condition-Code-Register CCR}

Dieser hat für jede Arithmetische Funktion eine Reihe an Flags, die bestimmt werden:

\begin{center}
    N(Negativ) - C (Carry) - Z (Zero) - V (Overflow)
\end{center}

\subsection{Multiplikation}

Hierfür gucken wir uns erst die Schriftliche Multiplikation an. Diese besteht aus einer einfachen Multiplikation und einem Shiften für jede Stelle von $b$ in $a*b$.

Statt nun aber immer eine $2n$-bit Zahl und eine $n$-bit Zahl zu addieren, schieben wir statt den zu addierenden Zahlen, das Ergebnis $p$. Slide S.104

\know{Aufgabe}{Wie lange benötig die erste Addition von a, b nach p? Und die Folgenden? Slide S. 105}

\subsection{Festkommanzahlen}

\subsection{Fließkommazahlen}

\end{document}
\documentclass[a4paper]{article}

%\usepackage{url}

%% Math
\usepackage{mathtools}
%% For Mengen like natural numbers
\usepackage{amsfonts}
%% Für spezielle Symbole
\usepackage{amssymb}

%% Images
\usepackage{import}
\usepackage{xifthen}
\usepackage{pdfpages}
%\usepackage{transparent}

%%% Command for simpler images
\newcommand{\incfig}[1]{%
    \def\svgwidth{\columnwidth}
    \import{./fig/}{#1.pdf_tex}
}

%% Links
\usepackage{hyperref}
\hypersetup{
    colorlinks=true,
    linkcolor=black,
    filecolor=magenta,
    urlcolor=cyan
}

%% Formatting
\usepackage{parskip}
\usepackage{listings}

\title{Rechnerarchitektur}
\author{Moritz}
\date{October 14, 2025}

\begin{document}
\maketitle
\tableofcontents

\section{RISC-V}

\subsection{Hello World}

Wir haben mit Hello World angefangen.

\lstinputlisting{assembly/test.s}

Das sind wir im detail durchgegangen. Bei dem Assembeln wurde schon "Hello World" in den Speicher geladen.

\subsection{Little- vs. Big-Endian}

Die Byte-order spielt bei der Umwandlung von n-Bit-Zahl auf m-Bit-Zahl eine Rolle.

Dies resultiert darin, dass "Hello World" in umgedrehte 4 Byte abschnitte im Arbeitsspeicher liegt. Also: "lleH" "W wo" "dlro" "000!"

\know{Aufgabe}{Wir bekommen add-in.s und sollen dies verstehen und erweitern}

\end{document}
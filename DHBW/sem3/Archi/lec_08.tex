\documentclass[a4paper]{article}

%\usepackage{url}

%% Math
\usepackage{mathtools}
%% For Mengen like natural numbers
\usepackage{amsfonts}
%% Für spezielle Symbole
\usepackage{amssymb}

%% Images
\usepackage{import}
\usepackage{xifthen}
\usepackage{pdfpages}
%\usepackage{transparent}

%%% Command for simpler images
\newcommand{\incfig}[1]{%
    \def\svgwidth{\columnwidth}
    \import{./fig/}{#1.pdf_tex}
}

%% Links
\usepackage{hyperref}
\hypersetup{
    colorlinks=true,
    linkcolor=black,
    filecolor=magenta,
    urlcolor=cyan
}

%% Formatting
\usepackage{parskip}

\title{Rechnerarchitektur}
\author{Moritz}
\date{October 6, 2025}

\begin{document}
\maketitle
\tableofcontents

\section{Arithmetik}

\subsection{Fließkommazahlen}

Die Aufgabe 1 von Aufgabenblatt 3 ist eine Klausuraufgabe. Diese sollte verstanden sein.

Der Zahlenbereich von Fließkommazahlen nach IEEE754 und hat per definition nur die Zahlen nach $1.f$. In dem Beispiel von S.120 ist $1.f = \{1.0, 1.25, 1.5, .75\}$ mit $m=2$ und mit $b=2$ haben wir $E=\{-1, 0, 1, 2\}$

\subsubsection{Denormalisierte Zahlen}

Wir haben uns für den Exponenten die werte $0\dots0$ und $1\dots1$ rausgenommen um Sonderfälle von 0 und inf zu berücksichtigen.

Um nun die Zahlen zwischen 0 und 1 darzustellen werden \textbf{denormalisierte} Zahlen verwendet. Dies passiert, wenn $e = 0\dots0$ ist. Dann werden die Zahlen ohne hidden Bit kodiert. Also $\{0.0, 0.25, 0.5, 0.75\} * 2^{-1}$. Die $-1$ ist der minimalwert von $E$

Das ist das was ich in matplotlib mit symlog gemacht habe. Zwischen -1 und 1 wird Linear skaliert und sonst exponential.

\subsubsection{Zusammenfassung}

Auf S. 121 gibt es eine schöne Tabelle für single, double und quad precision.

Kleinste Zahl $>0$ bei single Precision ist:

\begin{align*}
    x_{min}^{Norm} = 2^{-126} = 10^{-38}
\end{align*}

\begin{align*}
    x_{min}^{DeNorm} &= a_{min} * 2^{E_{min}}\\
     & = 2^{-m} * 2^{-b+1}\\
     & = 2^{-23} * 2^{-126}\\
     & = 2^{-149} = 1,4 * 10^{-45}
\end{align*}

\subsubsection{Sonderfälle}

Alle Sonderfälle von IEEE754 sind auf S.123 zu sehen.

\subsubsection{Beispielumrechnung}

Ist auf S.125:

IEEE 754 single precision Darstellung von folgender Dezimalzahl: $-2.5_{10}$

Lösung:

\begin{enumerate}
    \item Darstellung als Binäre Festkommazahl: $2.5_{10}=1*2^1+1*2^{-1}=10,1_2$
    \item Normalisierung
    \item Berechnung von $e$: $E=e-b$ mit $b=127$
\end{enumerate}

\end{document}
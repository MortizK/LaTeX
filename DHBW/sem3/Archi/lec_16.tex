\documentclass[a4paper]{article}

%\usepackage{url}

%% Math
\usepackage{mathtools}
%% For Mengen like natural numbers
\usepackage{amsfonts}
%% Für spezielle Symbole
\usepackage{amssymb}

%% Images
\usepackage{import}
\usepackage{xifthen}
\usepackage{pdfpages}
%\usepackage{transparent}

%%% Command for simpler images
\newcommand{\incfig}[1]{%
    \def\svgwidth{\columnwidth}
    \import{./fig/}{#1.pdf_tex}
}

%% Links
\usepackage{hyperref}
\hypersetup{
    colorlinks=true,
    linkcolor=black,
    filecolor=magenta,
    urlcolor=cyan
}

%% Formatting
\usepackage{parskip}

\title{Rechnerarchitektur}
\author{Moritz}
\date{November 3, 2025}


\begin{document}
\maketitle
\tableofcontents

\section{Berechnung der Leistung}

Wir können die Leistung einer CPU berechnen. Hierzu müssen wir die Arbeit pro Zeiteinheit berechnen.

Die Zeit lässt sich verschieden Messen. So kann extern mit einer Stoppuhr gestoppt werden, Die Zeit in der CPU oder deren Takte.

Da verschieden Instruktionen verschieden viele Zyklen der CPU brauchen, lässt sich die gesamtanzahl der Zyklen durch die Anzahl der Instruktionen und deren Taktverbrauch berechnen.

Aus dieser Information lässt sich die Durchschnittliche Taktanzahl eines Programmes ermitteln. Dies wird auch als CPI bezeichnet.

Aus dieser Information lässt sich die genaue Zeit ermitteln:

\begin{equation*}
    Time = \frac{Seconds}{Program} 
    = \frac{Instructions}{Program} * \frac{Clock cycles}{Instructions} * \frac{Seconds}{Clock cycle}
\end{equation*}

Die Leistung kann dann in MIPS (million instructions per second) ausgedrückt werden.

\end{document}
\documentclass[a4paper]{article}

%\usepackage{url}

%% Math
\usepackage{mathtools}
%% For Mengen like natural numbers
\usepackage{amsfonts}
%% Für spezielle Symbole
\usepackage{amssymb}

%% Images
\usepackage{import}
\usepackage{xifthen}
\usepackage{pdfpages}
%\usepackage{transparent}

%%% Command for simpler images
\newcommand{\incfig}[1]{%
    \def\svgwidth{\columnwidth}
    \import{./fig/}{#1.pdf_tex}
}

%% Links
\usepackage{hyperref}
\hypersetup{
    colorlinks=true,
    linkcolor=black,
    filecolor=magenta,
    urlcolor=cyan
}

%% Formatting
\usepackage{parskip}

\title{Rechnerarchitektur}
\author{Moritz}
\date{September 8, 2025}

\begin{document}
\maketitle
\tableofcontents

\section{Informationen}

Unser Dozent ist die Studienleitung für die kommenden zwei Semester. Hierzu müssen die Krankmeldungen angepasst werden. So ist auch das Sekretariat ein anderes.

Die Kontakte sind auf der Präsentation in Teams oder Moodle.

\section{Einleitung}

Ziel dieser Vorlesung ist es, zu verstehen, wie der Computer funktioniert. Also was macht er, um 5+3 zu berechnen.

\subsection{Fünf Bereiche}

\begin{enumerate}
    \item input
    \item output
    \item CPU: Datenpfad (beinhaltet Rechenwerke)
    \item CPU: Steuerwerk
    \item Speicher
\end{enumerate}

\subsection{Kapitel}

\begin{enumerate}
    \item Historie und Zahlendarstellung
    \item Arithmetik
    \item Instruktionen: Die Sprache des Computers
    \item Assembly Programmierung
    \item CPU
    \item Leistung
    \item Speicher
\end{enumerate}

Was wir am Ende Können sollten: Einfache Python Zeile in Assembly und dann Binärcode übersetzten, welcher direkt auf der CPU laufen kann.

Literatur: Computer Organization and Design MIPS Edition: The Hardware Software Interface 6. Edition Morgan Kaufmann, 2020

Weitere Empfehlungen: Crash-Course Computer Science auf YouTube

\subsection{Bonus-Programm}

Wir können in diesem Teil der Klausur bis zu 10\% der möglichen Punkte als Bonuspunkte erarbeiten. Hierzu muss die Aufgabe nur vorgestellt werden können.

\section{Historie und Zahlendarstellung}

\subsection{Rechnerarchitektur: Motivation}

\begin{enumerate}
    \item Digitale Logik
        
        Beschreibt Implementierung aller Komponenten in Gattern (AND, OR, \dots)
    \item Mikroarchitektur (µA)
        
        Beschreibt Struktur eines Rechners mit Registern, ALU, Steuerwerk
    \item Instruktionssatz-Architektur
        
        Beschreibt Instruktionssatz, Registersatz und Arbeitsspeicher-Adressierungsarten

        Stellt Schnittstelle zwischen Software und Hardware dar
    \item Systemarchitektur
    
        Organisation des Arbeitsspeicher
\end{enumerate}

\subsection{Historische Entwicklung}

Der Abakus ist wohl die erste Rechenmaschine. Funktioniert mit 1er und 5er und stellt die decimal-stellen dar.

\subsection{Technologische Grundlagen: Entwicklung}

\subsection{Komponenten heutiger Rechner}

\subsection{Zahlendarstellung im Rechner}

\end{document}
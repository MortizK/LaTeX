\documentclass[a4paper]{article}

%\usepackage{url}

%% Math
\usepackage{mathtools}
%% For Mengen like natural numbers
\usepackage{amsfonts}
%% Für spezielle Symbole
\usepackage{amssymb}

%% Images
\usepackage{import}
\usepackage{xifthen}
\usepackage{pdfpages}
%\usepackage{transparent}

%%% Command for simpler images
\newcommand{\incfig}[1]{%
    \def\svgwidth{\columnwidth}
    \import{./fig/}{#1.pdf_tex}
}

%% Links
\usepackage{hyperref}
\hypersetup{
    colorlinks=true,
    linkcolor=black,
    filecolor=magenta,
    urlcolor=cyan
}

%% Formatting
\usepackage{parskip}
\usepackage{listings}

\title{Rechnerarchitektur}
\author{Moritz}
\date{November 11, 2025}

\begin{document}
\maketitle
\tableofcontents

\section{Pipelining}

\subsection{Forwarding}

Löst alle Konflikte außer lw. da hierfür in die Vergangenheit gegangen werden müsste.

\subsection{Aufgabe S.242}

Welche Datenkonflikte können hier auftreten und welche lassen sich mit Forwarding lösen:

\begin{lstlisting}
    lw t1, 0(t0)    # kein
    lw t2, 4(t0)    # kein
    add t3, t1, t2  # Data Hazard t2, t1 - 2 Pausen
    sw t3, 12(t0)   # Data Hazard t3 - 2 Pausen
    lw t4, 8(t0)    # kein
    add t5, t1, t4  # Data Hazard t4 - 2 Pausen
    sw t5, 16(t0)   # Data Hazard t5 - 2 Pausen
\end{lstlisting}

Mit Forwarding:

\begin{lstlisting}
    lw t1, 0(t0)    # kein
    lw t2, 4(t0)    # kein
    add t3, t1, t2  # Data Hazard t2 - 1 Pause
    sw t3, 12(t0)   # kein
    lw t4, 8(t0)    # kein
    add t5, t1, t4  # Data Hazard t4 - 1 Pause
    sw t5, 16(t0)   # kein
\end{lstlisting}

Eine Skizze hilft immens bei dieser Aufgabe

\subsection{Control Hazards}

Bei Branch wird immer einfach dass gemacht, was zuletzt in dieser Zeile gemacht wurde. dies ist zu 90\% Richtig und es muss nicht gewartet werden.

\section{Speicher}



\end{document}
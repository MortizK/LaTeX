\documentclass[a4paper]{article}

%\usepackage{url}

%% Math
\usepackage{mathtools}
%% For Mengen like natural numbers
\usepackage{amsfonts}
%% Für spezielle Symbole
\usepackage{amssymb}

%% Images
\usepackage{import}
\usepackage{xifthen}
\usepackage{pdfpages}
%\usepackage{transparent}

%%% Command for simpler images
\newcommand{\incfig}[1]{%
    \def\svgwidth{\columnwidth}
    \import{./fig/}{#1.pdf_tex}
}

%% Links
\usepackage{hyperref}
\hypersetup{
    colorlinks=true,
    linkcolor=black,
    filecolor=magenta,
    urlcolor=cyan
}

%% Formatting
\usepackage{parskip}

\title{Rechnerarchitektur}
\author{Moritz}
\date{November 10, 2025}

\begin{document}
\maketitle
\tableofcontents

\section{Berechnungen}

Formeln zum Auswendiglernen:

\begin{align*}
    Time = \frac{Seconds}{Program} 
    &= \frac{Instructions}{Program} * \frac{Clock cycles}{Instructions} * \frac{Seconds}{Clock cycle}\\
    Leistung: P_x &= \frac{f_{clk}}{CPI_x}*10^{-6}[MIPS]\\
    t_x &= IC_x * CPI_x * t_{clk}
\end{align*}

Der $CPI_x$ ist ein Durchschnittswert:

\begin{equation*}
    CPI = \sum_{i} \frac{IC_i}{IC}*CPI_i
\end{equation*}

\section{Pipelining}

Hierzu müssen die Phasen verstanden werden:

\begin{enumerate}
    \item Instruktion aus dem Speicher lesen: Instruction fetch
    \item Register lesen und gleichzeitig Instruktion entschlüsseln: Regster read
    \item Arithmetische/ logische Operation ausführen oder eine Adresse berechnen: ALU operation
    \item Einen Operanden aus dem Speicher lesen oder dorthin schreiben: Data access
    \item Register mit Daten befüllen: Register write
\end{enumerate}

Das Pipelining erfolgt indem die Clock kleiner Wird und wir in jeder Phase eine Instruktion ausführen.

Somit wird $t_{clk}$ kleiner und der $CPI$ bleibt gleich

\subsection{3 Architekturen}

\begin{enumerate}
    \item 1 Instruktion pro clock cycle
    
    CPI: 1
    
    \item 1 Phase pro clock cycle
    
    CPI: $3 \leq CPI \leq 5$
    
    \item Pipelining
    
    CPI: 1
\end{enumerate}

Wenn das add kein Data access braucht, benutzt es diese Phase einfach nicht. Somit bleibt dies als Blank einfach frei

\begin{center}
    fetch | read | ALU | \qquad | write
\end{center}

Bei sw wird nicht in ein Register geschrieben also

\begin{center}
    fetch | read | ALU | Data | \qquad
\end{center}

\subsection{Datenpfad für Pipelining}

Aufgabe Seite 234

\begin{equation*}
    CPI_x = 0,4 * 4 + 0,2 * 3 + 0,4 * 5 = 4,2  
\end{equation*}

\begin{align*}
    t_x &= IC_x * CPI_x * t_{clk}\\
    t_x^s &= IC_x * 4,2 * 200ps\\
    t_x^p &= IC_x * 1 * (200ps + 40ps)\\
    S &= \frac{t_x^s}{t_x^p} = \frac{4,2 * 200}{240} = 3,5
\end{align*}

Bei Pipelining ist der CPI immer 1, wir haben aber durch zusätzliche zwischenregister einen overhead von 40ps pro Takt.

\subsection{Konflikte}

\begin{enumerate}
    \item Structural: Begrenzte Hardware z.B. ALU
    \item Data: Ein Instruktion hängt von den Daten einer anderen ab und kann noch nicht beginnen.
    \item Control: Ausgelöst durch Verzögerungen, bis fest steht welche die nächste Instruktion ist (branches)
\end{enumerate}

Dat Hazards können gelöst werden, indem die ALU direkte Verbindungen zu den zwischenregister hat (Forwarding)

\end{document}
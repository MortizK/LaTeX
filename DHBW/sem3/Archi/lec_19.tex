\documentclass[a4paper]{article}

%\usepackage{url}

%% Math
\usepackage{mathtools}
%% For Mengen like natural numbers
\usepackage{amsfonts}
%% Für spezielle Symbole
\usepackage{amssymb}

%% Images
\usepackage{import}
\usepackage{xifthen}
\usepackage{pdfpages}
%\usepackage{transparent}

%%% Command for simpler images
\newcommand{\incfig}[1]{%
    \def\svgwidth{\columnwidth}
    \import{./fig/}{#1.pdf_tex}
}

%% Links
\usepackage{hyperref}
\hypersetup{
    colorlinks=true,
    linkcolor=black,
    filecolor=magenta,
    urlcolor=cyan
}

%% Formatting
\usepackage{parskip}

\title{Rechnerarchitektur}
\author{Moritz}
\date{November 18, 2025}

\begin{document}
\maketitle
\tableofcontents

\section{Speicher}

Wir haben die verschieden Speicher durchgesprochen, DRAM, SRAM.

\subsection{DRAM}

\subsection{SRAM}

\subsection{Caches}

Hier gibt es Direkt abbildender Cache. Um die tatsächliche größe des Caches auszurechnen, um 16KiB Daten zu speichern.

Hierzu brauchen wir die Wortlänge $w=32$, die Bytes pro Zeile $4=2^2=2^d$ und die Anzahl der Zeilen $16*2^{10} / 4 = 2^{12} = 2^k$.

Daraus ergibt sich dann die Tag Länge $t$:

\begin{equation*}
    t = w - k - d
\end{equation*}

Eine Zeile besteht somit aus 1 Bit $v$, t Bit $t$ und 4 Byte Daten.

In diesem Fall also:

\begin{align*}
    (1 + 18 + 48) * 2^{12}\qquad [bits]\\
    = 26.112 \qquad [Bytes]\\
    = 25.5 \qquad [KiB]
\end{align*}

\end{document}
\documentclass[a4paper]{article}

%\usepackage{url}

%% Math
\usepackage{mathtools}
%% For Mengen like natural numbers
\usepackage{amsfonts}
%% Für spezielle Symbole
\usepackage{amssymb}

%% Images
\usepackage{import}
\usepackage{xifthen}
\usepackage{pdfpages}
%\usepackage{transparent}

%%% Command for simpler images
\newcommand{\incfig}[1]{%
    \def\svgwidth{\columnwidth}
    \import{./fig/}{#1.pdf_tex}
}

%% Links
\usepackage{hyperref}
\hypersetup{
    colorlinks=true,
    linkcolor=black,
    filecolor=magenta,
    urlcolor=cyan
}

%% Formatting
\usepackage{parskip}

\title{Rechnerarchitektur}
\author{Moritz}
\date{October 7, 2025}

\begin{document}
\maketitle
\tableofcontents

\section{Instruktionen}

\subsection{Einleitung}

Es gab einen Kurzen Überblick S.130 bis S. 135

\subsection{Wir bauen eine CPU}

Wir bauen die Intel 8-Bit Architektur. Unser Baukasten besteht aus wichtigen Bauteilen wie das ALU, Register und RAM.

Das Register sind einfach flankengesteuerte D-Flip-Flops die in $3\tau$ beschrieben werden können. Diese liegen in der CPU.

Im RAM liegt an jeder Adresse genau ein Byte (8-Bit)

Somit haben wir innerhalb der CPU die Register und das ALU. Außerhalb gibt es noch den RAM.

\subsubsection{Die Sprache}

Die ISA (Instruktionssatz Architektur) besteht aus einem opcode (Operation Code). Der opcode ist sind 4 Bit aus dem Byte. Die anderen 4 Bit sind die RAM Adresse.

Somit ist eine Instruktion: 4-Bit Opcode | 4-Bit-Adresse

\subsubsection{Aufbau}

S. 140ff werden noch zwei Register ergänzt. Den Befehlszähler, welcher weis \textbf{WO} was passiert und den Befehlsspeicher der dann weis, \textbf{WAS} passiert.

\subsubsection{Decode}

Um nun einen Opcode aus dem Befehlsspeicher auszuführen, wird dieser mithilfe von ANDs rausgefunden.

Diese ANDs sind das \textbf{Steuerwerk}. Diese Steuern dann, was passiert. In den Visualisierungen S. 144 sind alle Steuerleitungen Blau.

Die Adressbits werden direkt an den RAM gegeben. Dieser macht dann nur etwas damit, wenn MemRead = true kommt.

Register haben auch noch ein RegWrite und nur wenn das true ist, wird in dem Register was geladen.

Die Phasen:

\begin{enumerate}
    \item Fetch Phase
    \item Decode Phase
    \item Execute Phase
\end{enumerate}

\subsubsection{Steuerwerk}

Dies besteht aus dem Befehlsspeicher, dem Befehlszähler und allen ANDs, die die einzelnen Check if \dots Instruktionen beinhalten.

Neben dem Steuerwerk ist die ALU. Diese gibt den Outputs und die Ergebnis Flags zurück und nimmt den ALU-Opcode und zwei Inputs.

\subsection{ISA am Beispiel RISC-V}

\subsection{RISC-V-Instruktionswörter}

\subsection{Zusammenfassung}

\end{document}
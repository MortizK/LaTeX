\documentclass[a4paper]{article}

%\usepackage{url}

%% Math
\usepackage{mathtools}
%% For Mengen like natural numbers
\usepackage{amsfonts}
%% Für spezielle Symbole
\usepackage{amssymb}

%% Images
\usepackage{import}
\usepackage{xifthen}
\usepackage{pdfpages}
%\usepackage{transparent}

%%% Command for simpler images
\newcommand{\incfig}[1]{%
    \def\svgwidth{\columnwidth}
    \import{./fig/}{#1.pdf_tex}
}

%% Links
\usepackage{hyperref}
\hypersetup{
    colorlinks=true,
    linkcolor=black,
    filecolor=magenta,
    urlcolor=cyan
}

%% Formatting
\usepackage{parskip}

\title{Rechnerarchitektur}
\author{Moritz}
\date{October 21, 2025}

\begin{document}
\maketitle
\tableofcontents

\section{RISC-V}

\subsection{Funktionen}

Wir können auch zwischen call by value und call by reference unterscheiden.

Rückgabe von Funktionen/ Programmen erfolgt über a0.

Call by Reference ist notwendig, da call by value nur register Inhalte a0 bis a7 als Inputs hat. Um größere Datenmengen, z.B. Arrays und Strings, verarbeiten zu können.

Ein Beispiel Programm liegt auf Teams oder Moodle.

\subsection{Gleitkommazahlen}

Ist Separat als Floating Point Unit (FPU) implementiert. Dies sind eigene 32 Register.

Der Code für ecall ist die 2 für printFloat und 6 für readFloat. Gibt den Wert aus fa0 aus. der Code muss trotzdem auf a7 liegen.

z.B. fadd.s f2, f3, f4 oder fsub.d f6, f8, f12. Es wird zwischen .s und .d unterschieden: single und double.

\end{document}

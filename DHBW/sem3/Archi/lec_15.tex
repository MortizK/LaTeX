\documentclass[a4paper]{article}

%\usepackage{url}

%% Math
\usepackage{mathtools}
%% For Mengen like natural numbers
\usepackage{amsfonts}
%% Für spezielle Symbole
\usepackage{amssymb}

%% Images
\usepackage{import}
\usepackage{xifthen}
\usepackage{pdfpages}
%\usepackage{transparent}

%%% Command for simpler images
\newcommand{\incfig}[1]{%
    \def\svgwidth{\columnwidth}
    \import{./fig/}{#1.pdf_tex}
}

%% Links
\usepackage{hyperref}
\hypersetup{
    colorlinks=true,
    linkcolor=black,
    filecolor=magenta,
    urlcolor=cyan
}

%% Formatting
\usepackage{parskip}

\title{Rechnerarchitektur}
\author{Moritz}
\date{October 28, 2025}

\begin{document}
\maketitle
\tableofcontents

\section{CPU - Central Processing Unit}

Wir wollen uns das Steuerwerk genauer anschauen und nun für 32Bit RISC-V CPU.

In einem Takt wollen wir: Fetch, Decode, Execute ausführen.

Dies wollen wir nur für die integer Operatoren machen.

Unser Registersatz wird als ein Block dargestellt. Dieser hat die Register IDs für rs1, rs2, rd (register destination) und Daten Leitungen für schreiben und lesen von rs1 und rs2. Steuerleitung: RegWrite.

Immideat Generierung von 12Bit auf 32Bit. Es schreibt das MSB auf die 20 Höheren Stellen (um positiv und negative Zahlen zu konvertieren)

Der RAM besteht aus Adresse, Schreibe Daten, Lese Daten und den Steuerleitungen: MemWrite, MemRead

\subsection{Befehlszähler/ -speicher}

Diese werden vor die Register geschaltet. 

Um die branch Befehle umzusetzen wird das Immideat left shift 1 (mal 2) und dann auf den Programm Counter addiert.

Un das Inkrement um 4 auf den Programm Counter wird auch implementiert.

Alles auf S. 199

Aufgabe S.201:

Wie müssen alle Steuerbefehle gesetzt werden für:

\begin{center}
    beq rs1, rs2, <offset>
\end{center}

Das Steuerwerk besteht nur aus dem Program Counter und Program Speicher, wie auch den Main Control Unit wie alles ander um den PC zu bearbeiten.

Alles andere (Register, ALU, Imm, DataMem und ALU-Control) ist der Datapfad.

\end{document}
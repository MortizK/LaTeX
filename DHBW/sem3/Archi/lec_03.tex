\documentclass[a4paper]{article}

%\usepackage{url}

%% Math
\usepackage{mathtools}
%% For Mengen like natural numbers
\usepackage{amsfonts}
%% Für spezielle Symbole
\usepackage{amssymb}

%% Images
\usepackage{import}
\usepackage{xifthen}
\usepackage{pdfpages}
%\usepackage{transparent}

%%% Command for simpler images
\newcommand{\incfig}[1]{%
    \def\svgwidth{\columnwidth}
    \import{./fig/}{#1.pdf_tex}
}

%% Links
\usepackage{hyperref}
\hypersetup{
    colorlinks=true,
    linkcolor=black,
    filecolor=magenta,
    urlcolor=cyan
}

%% Formatting
\usepackage{parskip}

\title{Rechnerarchitektur}
\author{Moritz}
\date{September 15, 2025}

\begin{document}
\maketitle
\tableofcontents

\section{Arithmetik}

Wir bauen eine ALU, welche die verschiedenen Operatoren ausführen kann.

\subsection{Grundlagen: Boolsche Gatter}

Siehe Digitaltechnik. Wir sprechen von Spannungen. Erinnerung:

\begin{center}
    \begin{tabular}{c|c|c|c}
        Operator & Symbol & Beispiel & Zeit\\
        \hline
        AND & \& & $A\cdot B$ & $\tau$\\
        OR & $\geq 1$ & $A+B$ & $\tau$\\
        NOT & $\circ$ & $\bar{A}$ & $\tau$\\
        XOR & $=1$ & $A\oplus B$ & $2\tau$\\
    \end{tabular}
\end{center}

Ein NICHT Gatter kann aus einem Transistor gebaut werden.

Ein UND Gatter kann aus zwei Transistoren gebaut werden, welche in Reihe geschaltet werden.

Ein ODER Gatter brauche ich zwei Transistoren, welche parallel geschaltet sind.

Ein Gatter braucht eine Zeit von $\tau=1ns$. Mit diesem lässt sich ein Kritischer Pfad einzeichnen, welcher die längste Zeit braucht. Es können auch Konflikte entstehen (Siehe Digitaltechnik)

\subsection{Addition}

\subsubsection{Halbaddierer}

Ein Halbaddierer nimmt $a$ und $b$ als inputs und gibt $s_0$ und $c_0$ als output. Er addiert also zwei Bits zu einer zwei Bit Output Zahl und nur das LSB einer Zahl.

\begin{center}
    \begin{tabular}{cc|cc}
        $a_0$ & $b_0$ & $c_0$ & $s_0$\\
        \hline
        0 & 0 & 0 & 0\\
        0 & 1 & 0 & 1\\
        1 & 0 & 0 & 1\\
        1 & 1 & 1 & 0
    \end{tabular}
\end{center}

Aus dieser Wahrheitstabelle ergibt sich ein Halbaddierer, wo $c_0=a_0\cdot b_0$ ein UND ist und $s_0=a_0\oplus b_0$ ein XOR.

Daraus ergibt sich eine Laufzeit von $2\tau$, da das XOR $2\tau$ braucht.

\subsubsection{Volladdierer}

\begin{center}
    \begin{tabular}{ccc|cc}
        $a_i$ & $b_i$ & $c_{i-1}$ & $c_i$ & $s_0$\\
        \hline
        0 & 0 & 0 & 0 & 0\\
        0 & 0 & 1 & 0 & 1\\
        0 & 1 & 0 & 0 & 1\\
        0 & 1 & 1 & 1 & 0\\
        1 & 0 & 0 & 0 & 1\\
        1 & 0 & 1 & 1 & 0\\
        1 & 1 & 0 & 1 & 0\\
        1 & 1 & 1 & 1 & 1\\
    \end{tabular}
\end{center}

Somit hängen wir einen zweiten Halbaddierer hintenran, um drei Zahlen miteinander zu addieren.

Die Carry's der beiden Halbaddierer kommen in ein ODER. Und die Summe des erstens Halbaddierer kommt in den zweiten und dessen Summe ist die Ergebnis Summe.

% \begin{figure}[ht]
% \centering
% \incfig{Volladdierer}
% \caption{Volladdierer}
% \label{fig:Volladdierer}
% \end{figure}

\end{document}
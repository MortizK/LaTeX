\documentclass[a4paper]{article}

%\usepackage{url}

%% Math
\usepackage{mathtools}
%% For Mengen like natural numbers
\usepackage{amsfonts}
%% Für spezielle Symbole
\usepackage{amssymb}

%% Images
\usepackage{import}
\usepackage{xifthen}
\usepackage{pdfpages}
%\usepackage{transparent}

%%% Command for simpler images
\newcommand{\incfig}[1]{%
    \def\svgwidth{\columnwidth}
    \import{./fig/}{#1.pdf_tex}
}

%% Links
\usepackage{hyperref}
\hypersetup{
    colorlinks=true,
    linkcolor=black,
    filecolor=magenta,
    urlcolor=cyan
}

%% Formatting
\usepackage{parskip}

\title{Rechnerarchitektur}
\author{Moritz}
\date{September 30, 2025}

\begin{document}
\maketitle
\tableofcontents

\section{Arithmetik}

\subsection{Fließkommazahlen}

Nach IEEE 754. Dies erstellt Zahlen nach der Scientific Notation.

\begin{equation*}
    x = a * r^e
\end{equation*}

Mit der Mantisse: $a$, dem Exponenten $e$ für einen Radix $r=10$

Für die Fließkommazahlen im Binärem ist der Radix $r=2$.

\begin{equation*}
    x = (-1)^s * a * 2^{e-b}
\end{equation*}

Hierbei ist $a$ die Festkommazahl zwischen $1\geq a \geq 2$, $e$ ein unsigned int und das Bias $b = 2^{p-1}-1$.

Somit ist die range der Exponenten:

\begin{equation*}
    -b \geq E = e - b \geq 2^{p-1} = b + 1
\end{equation*}

Aufgrund von Reservierten Slots gibt es noch eine Weitere Einschränkungen, da $e = 0\dots0$ und $e=1\dots1$ für None und Inf reserviert sind:

\begin{equation*}
    -b + 1 \geq E \geq b
\end{equation*}

Da die Binärzahlen von $a$ immer die Form $1,xxxxx$ haben, können wir die $1$ einfach impliziert weglassen und nur die Nachkommastellen $f$ speichern.

\begin{equation*}
    a = 1,f
\end{equation*}

\subsubsection{Beispiel}

Einmal Durchrechnen von $10,5_{10}$ als Fließkommazahlen. mit $p=4$ und $m=5$:

\begin{enumerate}
    \item Bias 
    \begin{equation*}
        b=2^{p}-1 = 7
    \end{equation*}
    \item Festkommazahl 
    \begin{equation*}
        10,5_{10} = 1010,1_2
    \end{equation*}
    \item Normieren/ Kommaverschiebung
    \begin{equation*}
        1010,1 = 1,0101 * 2^3
    \end{equation*}
    \item $f$ ablesen und auf $m$-bit runden
    \begin{equation*}
        f = 01010
    \end{equation*}
    \item $e$ bestimmen
    \begin{align*}
        e &= E + b = 3 + b = 3 + 7 = 10\\
        10_{10} &= 1010_2
    \end{align*}
    \item Zusammenfügen
    \begin{center}    
        \begin{tabular}{c|c|c}
            $s$ & $e$ & $f$ \\\hline
            $0$ & $1010$ & $01010$
        \end{tabular}
    \end{center}
\end{enumerate}

Andere Richtung Umformen durch die Formel:

\begin{equation*}
    (-1)^s * 1,f * 2^{e - b}
\end{equation*}

\subsubsection{Wertebereich von IEEE 754}

Die positiven Zahlen lassen sich von:

\begin{equation*}
    1,0 * 2^{-b+1} \dots (1 + 1 - \frac{1}{2^m}) * 2^{b}
\end{equation*}

Somit gibt es noch keine Null, da wir in der Mantisse die $1,f$ implizieren.

Für einen Festen Exponenten gibt es insgesamt $2^{m}$ verschiedene Zahlen.

Der Abstand zwischen zwei benachbarten Zahlen ist somit:

\begin{equation*}
    \Delta x=\frac{1}{2^m}*2^{E} = 2^{E-m}
\end{equation*}

\end{document}
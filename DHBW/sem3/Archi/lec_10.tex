\documentclass[a4paper]{article}

%\usepackage{url}

%% Math
\usepackage{mathtools}
%% For Mengen like natural numbers
\usepackage{amsfonts}
%% Für spezielle Symbole
\usepackage{amssymb}

%% Images
\usepackage{import}
\usepackage{xifthen}
\usepackage{pdfpages}
%\usepackage{transparent}

%%% Command for simpler images
\newcommand{\incfig}[1]{%
    \def\svgwidth{\columnwidth}
    \import{./fig/}{#1.pdf_tex}
}

%% Links
\usepackage{hyperref}
\hypersetup{
    colorlinks=true,
    linkcolor=black,
    filecolor=magenta,
    urlcolor=cyan
}

%% Formatting
\usepackage{parskip}

\title{Rechnerarchitektur}
\author{Moritz}
\date{October 13, 2025}

\begin{document}
\maketitle
\tableofcontents

\section{Instruktion}

\subsection{ISA am Beispiel RISC-V}

Eine ISA ist der Bauplan und die Instruktionen um eine CPU bauen zu können.

x86 und ARM sind von intel und amd lizenziert und nicht jeder darf diese Benutzen.

Wir behandeln die 32-Bit Wortbreite für RISC-V und die genaue Bezeichnung ist: RISC-V ISA eine (0, 3)-Register/Register-ISA (0 Operanden aus dem Speicher und immer 3 Operanden für arithmetische Operationen aus dem Register)

Dadurch, dass alles in Register gespeichert wir, hat RISC-V mehr Register (32 bei 32-Bit) als z.B. x86 (hat nur 8).

Es gibt ein Nullregister und 31 Universalregister.

\subsection{RISC-V-Instruktionswörter}

Folie S. 158

Die sehen immer gleich aus: operator, ziel, register1, register2

\begin{center}
    sub s0, s1, s2 \# Differenz s1-s2 nach s0
\end{center}

Wir haben unsere Register x0 bis x31, diese haben aber auch andere Name, die je nach Verwendung anders heißen S.159.

\begin{itemize}
    \item s0-s11 zur Speicherung von Variablen
    \item t0-t6 zur Speicherung temporärer Werte
    \item \dots
\end{itemize}

Die t-Register können von Funktionen überschrieben werden. s-Register müssen zwischengespeichert, wenn Funktionen diese ändern.

\subsubsection{Lesen und Schreiben}

Beide Operatoren bearbeiten Wörter, somit werden 32-Bit bearbeitet, was hinter vier Adressen liegt.

Load Word lw rd, rs1

\begin{center}
    lw t1, s1 \# Fülle t1 mit Daten hinter Adresse in s1
\end{center}

Store Word sw rs2, rs1

\begin{center}
    sw t1, s1 \# Inhalte von t1 nach Adresse in s1
\end{center}

Bei rw und sw ist das zweite Argument immer die Arbeitsspeicher-Adresse!

\subsubsection{Displacement}

Wenn wir Arrays im Speicher haben, zeigt der Pointer nur zum Start des Arrays. Um dann auf die anderen Stellen zuzugreifen wir ein Displacement verwendet. S. 163

lw rd, disp(rs1)

\begin{center}
    lw t0, 32(s1) \# t0$<=$Daten hinter Adresse A[0+8]=s1+32/4
\end{center}

Wie division durch 4 kommt von der Wortlänge, da ein Wort 32-Bit hat und hinter jeder Adresse 1-Byte = 8-Bit gespeichert wird.

\subsubsection{Register und Speicher}

Häufig Verwendete Variablen lässt der Compiler im Register, da diese schnelle und Stromeffizienz sind.

Design Prinzipien für RISC-V:

\begin{enumerate}
    \item Kleiner ist schneller
    \item Der häufigste Fall muss schnell sein
    \item Einfach und regelmäßig ist besser
    \item Ein gutes Design erfordert Kompromisse
\end{enumerate}

\subsubsection{Konstanten}

Hierzu wird dem operator ein "i" hinzugefügt:

\begin{center}
    addi s1, s2, 100 \# S. 166
\end{center}

Sollte wir Konstanten addieren, welche über das 2er Komplement hinaus gehen. Bei 12-Bit: -2048 bis 2047. Dann werden einfach mehrere addi ausgeführt.

\subsubsection{Bedingte Sprünge}

\begin{center}
    loop: addi t1, t1, 1
    bn t1, s1, loop
\end{center}

Der Assembler macht aus dem string "loop" ein Integer, der den Sprung des Befehlszähler repräsentiert.

\begin{equation*}
    loop = Befehlszaehler + 4 + const * 2
\end{equation*}

Die $const$ ist wodurch der loop ersetzt wird. 

\subsubsection{Unbedingte Sprünge}

sind im \textbf{J (jump)-Format}.

\begin{center}
    jal x1, 16 \# Rücksprungadresse: rd=PC+4 in x1 und PC+16*2
\end{center}

\subsection{Umwandlung Assembly $\to$ Binär}

Hierzu gibt es verschiedene Formate. Das \textbf{R (register)-Format} hat drei Register.

\begin{center}
    func7 | rs2 (5-Bit) | rs1 (5-Bit) | func3 | rd (5-Bit) | opcode (7-Bit)
\end{center}

Diese werden von recht nach links gelesen.

Das \textbf{I (immediate)-Format}:

\begin{center}
    constant (12-Bit) | rs1 (5-Bit) | func3 | rd (5-Bit) | opcode (7-Bit)
\end{center}

Die anderen Typen sind auf S.171 und bekommen wir mit in die Klausur.

Zwischen den Formaten wird mit dem Opcode entschieden.

Das \textbf{J (jump)-Formate} wird für unbedingte Sprünge verwendet.

\subsection{Klausuraufgabe}

S. 176

\subsection{Zusammenfassung}

\end{document}
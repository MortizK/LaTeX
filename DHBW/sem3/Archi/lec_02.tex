\documentclass[a4paper]{article}

%\usepackage{url}

%% Math
\usepackage{mathtools}
%% For Mengen like natural numbers
\usepackage{amsfonts}
%% Für spezielle Symbole
\usepackage{amssymb}

%% Images
\usepackage{import}
\usepackage{xifthen}
\usepackage{pdfpages}
%\usepackage{transparent}

%%% Command for simpler images
\newcommand{\incfig}[1]{%
    \def\svgwidth{\columnwidth}
    \import{./fig/}{#1.pdf_tex}
}

%% Links
\usepackage{hyperref}
\hypersetup{
    colorlinks=true,
    linkcolor=black,
    filecolor=magenta,
    urlcolor=cyan
}

%% Formatting
\usepackage{parskip}

\title{Rechnerarchitektur}
\author{Moritz}
\date{September 9, 2025}

\begin{document}
\maketitle
\tableofcontents

\section{Historie und Zahlendarstellung}

\subsection{Von-Neumann Architektur}

Die Grundlegende Architektur eines modernen Computers. Slide S. 30 ist ne gute Grafik.

\subsection{Technologische Grundlagen: Entwicklung}

Transistoren sind echt geil. Sehr schnell und zuverlässig. Das Problem, was moor's law aufhält ist die Kühlung.

Denn: Mehr Transistoren produzieren mehr Hitze, welche gekühlt werden muss (Power Wall)

\subsection{Komponenten heutiger Rechner}

Die 5 Hauptkomponenten:

\begin{enumerate}
    \item Daten Eingang (input)
    \item Daten Ausgang (output)
    \item Datenpfad (datapath)
    \item Steuerwerk (control)
    \item Speicher (memory)
\end{enumerate}

Die Eigentliche CPU ist nur der Datenpfad und das Steuerwerk.

\subsection{Zahlendarstellung im Rechner}

2-er Komplement. So sollt die Addition von 1 und -1 (Binäraddition) gleich 0 sein.

So ist $1_{10}=0001_{2}$ und welche Zahl addiert ergibt $0000_{2}$. Das ist $-1_{10}=1111_{2}$

Daraus ergibt sich das Muster: Die negative Zahl $z$ addiert mit dem Komplement $\bar{z}$ und $1$ ergibt $0$

\begin{equation}
    -z=\bar{z}+1
\end{equation}

\know{2er-Komplement}{Zur Umrechnung IMMER invertieren und 1 addieren. NIEMALS 1 subtrahieren!}



\end{document}
\documentclass[a4paper]{article}

%\usepackage{url}

%% Math
\usepackage{mathtools}
%% For Mengen like natural numbers
\usepackage{amsfonts}
%% Für spezielle Symbole
\usepackage{amssymb}

%% Images
\usepackage{import}
\usepackage{xifthen}
\usepackage{pdfpages}
%\usepackage{transparent}

%%% Command for simpler images
\newcommand{\incfig}[1]{%
    \def\svgwidth{\columnwidth}
    \import{./fig/}{#1.pdf_tex}
}

%% Links
\usepackage{hyperref}
\hypersetup{
    colorlinks=true,
    linkcolor=black,
    filecolor=magenta,
    urlcolor=cyan
}

%% Formatting
\usepackage{parskip}

\title{Analysis}
\author{Moritz}
\date{May 14, 2025}

\begin{document}
\maketitle
\tableofcontents

\section{Folgen und Reihen}

\subsection{Spezielle Folgen}

Beispiel

\begin{align*}
      & \sqrt[n]{5*3^n+n^7+1}
    = \sqrt[n]{3^n*(5+\frac{n^7}{3^n}+\frac{1}{3^n})}          \\
    = & \sqrt[n]{3^n}\sqrt[n]{5+\frac{n^7}{3^n}+\frac{1}{3^n}}
    \xrightarrow{n\to\infty}3,
\end{align*}

Nach Lemma Vorlesung ist $\sqrt[n]{a_n}\xrightarrow{n\to\infty} 1$

Lemma: Für eine Folge $(a_n)_{n\in \mathbb{N}}$ mit der Eigenschaft:

\begin{center}
    $0<m\leq a_n\leq N$, für fast alle $n\in \mathbb{N}$

    $M$ und $N$ sind quasi unsere Grenzen $a-\epsilon$ und $a+\epsilon$
\end{center}

Dann gilt $\sqrt[n]{a_n}\xrightarrow{n\to\infty}1$

Insbesondere, falls $a_n\xrightarrow{n\to\infty}a>0$

Beweis:

Da $\sqrt[n]{M}\xrightarrow{n\to\infty}1$ und $\sqrt[n]{N}\xrightarrow{n\to\infty}1$ gilt nach Sandwich-Prinzip $\sqrt[n]{a_n}\xrightarrow{n\to\infty}1$.

\subsection{Folgen die wir noch nicht behandeln können}

Noch nicht mit unseren Rechenregeln.

\begin{enumerate}
    \item $\sqrt{2}$ \quad HERON-Verfahren: im Heft Ana 14.5
    \item $e_n:=(1+\frac{1}{n})^n\xrightarrow{n\to\infty}e$ ?
\end{enumerate}

\subsection{Reihen}

Sei $(a_n)_{n\in \mathbb{N}}$ eine Nullfolge. Dann ist die zugehörige Folge

\begin{equation*}
    S_n:=\sum_{k=1}^{n}a_k
\end{equation*}

Falls $\lim_{n\to\infty}\sum_{k=1}^{n}a_k$ existiert, bezeichnet man den GRenzwert mit:

\begin{equation*}
    \sum_{k=1}^{\infty}a_k:=\lim_{n\to\infty}\sum_{k=1}^{n}a_k
\end{equation*}

Beispiel $(\sum_{k=1}^{n}\frac{1}{k^2})_{n\in \mathbb{N}}$ ist die Reihe, meist einfach durch die Summe $\sum_{k=1}^{n}\frac{1}{k^2}$

\begin{equation*}
    \sum_{k=1}^{\infty}\frac{1}{k^2}:=\lim_{n\to\infty}\sum_{k=1}^{n}\frac{1}{k^2}
\end{equation*}

\subsubsection{Definitionen}

$A\subseteq \mathbb{R}$ heißt nach oben beschränkt, wenn es eine obere Schranke $M\in \mathbb{R}$ gibt, d.h. $\forall_{x\in A}x\leq M$

$A\subseteq \mathbb{R}$ heißt nach unten beschränkt, wenn es eine untere Schranke $M\in \mathbb{R}$ gibt, d.h. $\forall_{x\in A}x\geq M$

Falls es eine kleinste obere Schranke von $A$ gibt heißt sie \textit{Supremum} von $A$ $sup A$

Falls es eine größte untere Schranke von $A$ gibt heißt sie \textit{Infimum} von $A$ $inf A$

Falls $sup A\in A$, hat $A$ ein Maximum $sup A = max A$.

Falls $inf A\in A$, hat $A$ ein Minimum $inf A = min A$.

Für reelle Folgen $(a_n)_{n\in \mathbb{N}}: sup$ $a_n{_{n\in \mathbb{N}}}:= sup\{a_n\mid n\in \mathbb{N}\}$

\know{Vollständigkeitsaxiom von $\mathbb{R}$}{Jede nach oben beschränkte Teilmenge von $A$ hat ein Supremum.

    Also: Wenn es eine obere Schranke gibt, so gibt es eine beste obere Schranke}

\textbf{Satz}: von der monotonen Konvergenz

Eine monoton wachsende und nach oben beschränkte Folge ist konvergent (mit dem Supremum als Grenzwert)

Eine monoton fallend und nach unter beschränkte Folge ist konvergent (mit dem Infimum als Grenzwert)

\newpage

Beweis:

\begin{figure}[ht]
    \centering
    \incfig{monotone_konvergenz}
    \caption{Monotone Konvergenz}
    \label{fig:monotone_konvergenz}
\end{figure}

Da die Reihe monoton steigen ist und ein Supremum hat, gibt es ein ${a_n}_\epsilon\in U$

\subsubsection{Beispiel}

\begin{equation*}
    S_n=\sum_{k=1}^{n}\frac{1}{k^2}\text{ ist als Summe positiver Zahlen automatisch monoton wachsend.}
\end{equation*}

Zur Konvergenz fehlt nur noch eine obere Schranke, d.h. ein $M>0$ mit:

\begin{equation*}
    \sum_{k=1}^{n}\frac{1}{k^2}\leq M\quad\text{f.a. } n\in \mathbb{N}
\end{equation*}

Obere Schranke durch Abschätzung nach oben:

\begin{align*}
    \sum_{k=2}^{n}\frac{1}{k^2}
     & \leq \sum_{k=2}^{n}\frac{1}{k(k-1)}                                                                   \\
    \sum_{k=2}^{n}\left(\frac{1}{k-1}-\frac{1}{k}\right)
     & = 1 - \frac{1}{2} + \frac{1}{2} - \frac{1}{3} \dots - \frac{1}{n - 1} + \frac{1}{n - 1} - \frac{1}{n} \\
    \sum_{k=2}^{n}\frac{1}{k(k-1)}
     & = 1 - \frac{1}{n} < 1
\end{align*}

Somit gibt es eine obere Schranke und diese ist $\sum_{k=1}^{\infty}\frac{1}{k^2}=\frac{\pi^2}{6}$

\subsubsection{Allgemeines Konvergenzkriterium}

Wie könnte ein allgemeines grenzwertunabhängiges Konvergenzkriterium aussehen?

Angenommen: $a_n\xrightarrow{n\to\infty}a$:

\begin{align*}
    \forall_{\epsilon > 0}\exists_{n_\epsilon}\forall_{n\geq n_\epsilon}|a-a_n|<\epsilon                     \\
    \forall_{\epsilon > 0}\exists_{n_\epsilon}\forall_{n\geq n_\epsilon, m\geq n_\epsilon}|a_n-a_m|<\epsilon \\
    \text{CAUCHY-Kriterium}
\end{align*}

\know{CHAUCHY-Kriterium}{besagt, dass sich der Abstand von Punkten immer kleiner wird. Somit brauchen wir nicht mehr den Grenzwert.}

Ziel: Zeigen, dass das CAUCHY-Kriterium für Konvergenz hinreichend ist.

Weg:

\begin{enumerate}
    \item CK $\to$ Beschränkung
    \item Satz von BOLZANO-WEIERSTRASS: jede beschränkte Menge hat mindestens einen Häufungspunkt

          $\implies\{a_n\mid n\in \mathbb{N}\}$ hat einen Häufungspunkt
    \item CK + BOLZ.WEI $\implies$ nur ein Häufungspunkt (HP)
    \item $\implies$ HP = Grenzwert
\end{enumerate}

\subsubsection{Bolz. WEIERST. Idee}

Intervallhalbierungsverfahren. Wir nehmen ein Intervall (unsere beschränkte Menge) und halbieren diese. Dies wiederholen wir in der Menge, wo $\infty$-Viele Folgeglieder sind.

Diese Verfahren schrumpft auf einen einziegen Punkt, welche ein Häufungspunkt sein muss. Wenn nach einer Halbierung beide $\infty$-Viele Folgeglieder haben, so befinden sich in beide mindestens ein Häufungspunkt.

\textbf{Satz}: Intervallschachtelung

Immer kleiner werdende Intervalle welche sich gegenseitig enthalten.

\begin{equation*}
    I_1=[a_1, b_1]\subseteq I_2=[a_2, b_2]\subseteq I_3=[a_3, b_3] \dots
\end{equation*}

Intervalle $(I_n)_{n\in \mathbb{N}}$ mit der Eigenschaft

\begin{enumerate}[i.]
    \item $I_{n+1}=[a_{n+1}, b_{n+1}]\subseteq I_n=[a_n, b_n]$
    \item $|I_n|:=b_n-a_n\xrightarrow{n\to\infty}0$
\end{enumerate}

Dann gibt es genau ein $x\in \mathbb{R}$ mit: $x\in I_n\quad f.a. n\in \mathbb{N}$

Alle $a_n$ sind monoton steigend und durch $b_1$ beschränkt. $\to \lim_{n\to\infty}a_n=x_1\leq b_n$.

Alle $b_n$ sind monoton fallend und durch $a_1$ beschränkt. $\to \lim_{n\to\infty}b_n=x_2$.

\begin{center}
    $\implies x_1\leq x_2$
\end{center}

Diese beiden können keinen echten Abstand voneinander haben, da dies ein Wiederspruch mit $b_n-a_n\xrightarrow{n\to\infty}0$ (Abstand der Grenzen geht gegen 0)

\end{document}
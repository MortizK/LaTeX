\documentclass[a4paper]{article}

%\usepackage{url}

%% Math
\usepackage{mathtools}
%% For Mengen like natural numbers
\usepackage{amsfonts}
%% Für spezielle Symbole
\usepackage{amssymb}

%% Images
\usepackage{import}
\usepackage{xifthen}
\usepackage{pdfpages}
%\usepackage{transparent}

%%% Command for simpler images
\newcommand{\incfig}[1]{%
    \def\svgwidth{\columnwidth}
    \import{./fig/}{#1.pdf_tex}
}

%% Links
\usepackage{hyperref}
\hypersetup{
    colorlinks=true,
    linkcolor=black,
    filecolor=magenta,
    urlcolor=cyan
}

%% Formatting
\usepackage{parskip}

\title{Analysis}
\author{Moritz}
\date{June 25, 2025}

\begin{document}
\maketitle
\tableofcontents

\section{Differentialrechnung}

\subsection{TAYLOR-Entwicklung}

Beispiel

\begin{align*}
    e^x-\sum_{k=0}^{n}\frac{1}{k!}
    = \frac{e^\xi}{(n+1)!}x^{n+1}\leq e^x\frac{x^{n+1}}{(n+1)!}
    \xrightarrow{n\to\infty}0 \\
    \implies e^x=\lim_{n\to\infty}\sum_{k=0}^{n}\frac{1}{k!}
\end{align*}

\subsubsection{Euler Formel}

\begin{align*}
    e^{ix}=\sum_{k=0}^{\infty}\frac{(ix)^k}{k!}
     & =1+ix+\frac{i^2x^2}{2}+\frac{i^3x^3}{3!}+\frac{i^4x^4}{4!}+... \\
     & = 1 -\frac{x^2}{2!} + \frac{x^4}{4!} -\frac{x^6}{6!} + \dots +
    i(x-\frac{x^3}{3!}+\frac{x^5}{5!} - \dots)                        \\
     & = \sum_{k=0}^{\infty}\frac{(-1)^kx^{2k}}{(2k!)}
    +i\sum_{k=0}^{\infty} \frac{(-1)kx^{2k+1}}{(2k+1)!}               \\
     & =\cos(x) + i \sin(x)
\end{align*}

\subsubsection{Gausche Glockenkurve}

Ist wichtig in der Statistik, da dies die Normalverteilung ist.

\begin{align*}
    f(x) = e^{-x^2} \\
    f(0)=1          \\
    f'(0)=0         \\
    \vdots
\end{align*}

Alternative:

\begin{align*}
    e^{-x^2}=e^z
    =\sum_{k=0}^{\infty}\frac{z^k}{k!}
    =\sum_{k=0}^{\infty}\frac{(-x^2)^k}{k!}
    =\sum_{k=0}^{\infty}\frac{(-1)^kx^{2k}}{k!}
\end{align*}

Insgesamt heute viele Übungen in meine Heft unter: Ana 25.06 gemacht.

\begin{align*}
    arcsin'(x)   & = \frac{1}{\sqrt{1-x^2}}                 \\
    (1+x)^\alpha & =\sum_{k=0}^{\infty}\binom{\alpha}{k}x^k
\end{align*}

\begin{align*}
    f(x)   & =(1+x)^\alpha,
    \quad f(0)=1                                \\
    f'(x)  & =\alpha(1+x)^{\alpha-1},
    \quad f'(0)=\alpha                          \\
    f''(x) & =\alpha(\alpha-1)(1+x)^{\alpha-2},
    \quad f'(0)=\alpha(\alpha-1)                \\
           & ,
    \quad f^{(n)}(0)=\alpha(\alpha-1)(\alpha-2)\dots(\alpha-(n-1))
\end{align*}

\end{document}
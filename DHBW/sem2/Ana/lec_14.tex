\documentclass[a4paper]{article}

%\usepackage{url}

%% Math
\usepackage{mathtools}
%% For Mengen like natural numbers
\usepackage{amsfonts}
%% Für spezielle Symbole
\usepackage{amssymb}

%% Images
\usepackage{import}
\usepackage{xifthen}
\usepackage{pdfpages}
%\usepackage{transparent}

%%% Command for simpler images
\newcommand{\incfig}[1]{%
    \def\svgwidth{\columnwidth}
    \import{./fig/}{#1.pdf_tex}
}

%% Links
\usepackage{hyperref}
\hypersetup{
    colorlinks=true,
    linkcolor=black,
    filecolor=magenta,
    urlcolor=cyan
}

%% Formatting
\usepackage{parskip}

\title{Analysis}
\author{Moritz}
\date{June 24, 2025}

\begin{document}
\maketitle
\tableofcontents

\section{Differentialrechnung}

\subsection{Ableitung von Umkehrfunktionen}

\begin{equation}
    f^{-1'}(y)=\frac{1}{f'(f^{-1}(y))}
\end{equation}

Am Beispiel von $\sin$ mithilfe $\sin^2+\cos^2=1$ als $\cos=\pm\sqrt{1-\sin^2}$, welches im Bereich, den wir umkehren wollen nur die positive wurzel infrage kommt.

\begin{align*}
    sin^{-1'}(y)
     & =\frac{1}{\cos(\sin^{-1}(y))}            \\
     & =\frac{1}{\sqrt{1-\sin^2(\sin^{-1}(y))}} \\
     & =\frac{1}{\sqrt{1-y^2}}                  \\
    tan^{-1'}(y)
     & = \frac{1}{1+y^2}=arctan'(y)             \\
    ln'(y)
     & = \frac{1}{y}                            \\
    sinh^{-1'}
     & = \frac{1}{\sqrt{1+y^2}}                 \\
    cosh^{-1'}
     & = \frac{1}{\sqrt{y^2-1}}
\end{align*}

\begin{align*}
    sinh(x)=\frac{1}{2}(e^x)-e^{-x}                               \\
    cosh(x)=\frac{1}{2}(e^x+e^{-x})                               \\
    tanh(x)=\frac{sinh(x)}{cosh(x)}=\frac{e^x-e^{-x}}{e^x+e^{-x}} \\
    cosh^2-sinh^2=1
    sinh'=cosh; cosh'=sinh
\end{align*}

$sinh^{-1}(y)$ lässt sich ausrechnen: Es ist $sinh(x)=y$ nach $x$ auflösen.

\begin{align*}
    e^-e^{-x}        & =2y\quad\mid *e^x       \\
    e^{2x}-1         & =2ye^x                  \\
    e^{2x}-2ye^x-1   & =0\qquad u = e^x        \\
    u^2-2yu-1        & =0                      \\
    u_{1/2}          & =y\pm\sqrt{y^2+1} = e^x \\
    u                & =y+\sqrt{y^2+1} = e^x   \\
    sinh^{-1}(y) = x & =ln(y+\sqrt{1+y^2})
\end{align*}

Wenn $f$ diffbar in $x_0$

\begin{align*}
    f(x)-f(x_0)
     & =\Theta(x)(x-x_0)                          \\
    \Theta(x_0)
     & =f'(x_0) \neq 0                            \\
    y-y_0
     & = \Theta(f^{-1}(y))(f^{-1}(y)-f^{-1}(y_0)) \\
    f^{-1}(y)-f^{-1}(y_0)
     & = \frac{1}{\Theta(f^{-1}(y))}(y-y_0)       \\
    f^{-1'}
     & =\frac{1}{\Theta(f^{-1}(y_0))}             \\
     & =\frac{1}{\Theta(x_0)}                     \\
     & =\frac{1}{f'(x_0)}                         \\
     & =\frac{1}{f'(f^{-1}(y_0))}                 \\
\end{align*}

\subsection{Die Regeln von del'Hospital}

\subsubsection{Erweiterete Mittelwertsatz}

$f, g: [a,b] \to \mathbb{R}$ stetig in $(a,b)$ diffbar.

$g'(x)\neq 0$ f.a. $x\in (a, b)$

\begin{equation}
    \exists_{\xi\in (a,b)} \frac{f(b)-f(a)}{g(b)-g(a)}=\frac{f'(\xi)}{g'(\xi)}
\end{equation}

\subsubsection{Die Regeln}

\begin{align*}
    f(x)\xrightarrow{x\to a^+}0 \\
    g(x)\xrightarrow{x\to a^+}0
\end{align*}

Frage: $\frac{f(x)}{g(x)}\xrightarrow{x\to a^+}"\frac{0}{0}"$ unbestimmter Ausdruck

Erw. MWS:

\begin{equation*}
    \frac{f(x)}{g(x)}
    =\frac{f(x)-f(a)}{g(x)-g(a)}
    =\frac{f'(\xi)}{g'(\xi)}
    \xrightarrow[(\xi\to a^+)]{x\to a^+} c
\end{equation*}

Nach dem SWP gilt: $x\to a^+\implies \xi\to a^+$. Annahme:

\begin{equation*}
    \lim_{x\to a^+}\frac{f'(x)}{g'(x)} = c \text{ existiert}
\end{equation*}

\subsection{Die Regeln}

1.

\begin{align*}
    f(x)\xrightarrow{x\to a^{+(-)}}0,\pm\infty \\
    g(x)\xrightarrow{x\to a^{+(-)}}0,\pm\infty \\
\end{align*}

Falls:

\begin{equation*}
    \lim_{x\to a^{+(-)}}\frac{f'(x)}{g'(x)}=c
\end{equation*}

existiert, dann auch:

\begin{equation*}
    \lim_{x\to a^{+(-)}}\frac{f(x)}{g(x)}=c
\end{equation*}

2.

\begin{align*}
    f(x)\xrightarrow{x\to \infty (-\infty)}0,\pm\infty \\
    g(x)\xrightarrow{x\to \infty (-\infty)}0,\pm\infty \\
\end{align*}

Kurz:

\begin{equation}
    \lim_{x\to \infty}\frac{f'(x)}{g'(x)}=c
    \implies
    \lim_{x\to \infty}\frac{f(x)}{g(x)}=c
\end{equation}

Beispiel

\begin{align*}
    \lim_{x\to \infty}\frac{x^2}{e^x}
    =\lim_{x\to\infty}\frac{2x}{e^x}
    =\lim_{x\to\infty}\frac{2}{e^x}
    =0
\end{align*}

weitere Beispiele in meinem Heft unter Ana 24.06. l'Hospital.

\subsection{TAYLOR-Entwicklung einer Funktion}

Ziel: Eine Funktion $f$ durch eine Potenzreihe berechenbar zu machen.

\begin{equation*}
    f(x)=\sum_{k=0}^{\infty}a_k x^k
    = a_0+a_1x+a_2x^2+a_3x^3+\dots+a_nx^n+\dots
\end{equation*}

Wie müssen die $a_k$ aussehen?

\begin{align*}
    f(0)=a_0;
     & \quad f'(x) = a_1 + 2a_2x + 3a_3x^2+4a_4x^3\dots \\
    a_1=f'(0);
     & \quad f''(x) = 2a_2 + 3*2a_3x+ 4*3a_4x^2+\dots   \\
    a_2=\frac{f''(0)}{2};
     & \quad f''(x) = 3*2a_3+ 4*3*2a_4x+\dots           \\
    a_3=\frac{f'''(0)}{3!}                              \\
    a_4=\frac{f''''(0)}{4!}
\end{align*}

Notwendig

\begin{equation}
    f(x)-\sum_{k=0}^{N}\frac{f^{(k)}(0)}{k!}x^k=R_N(\xi)
\end{equation}

\end{document}
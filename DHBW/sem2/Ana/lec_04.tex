\documentclass[a4paper]{article}

%\usepackage{url}

%% Math
\usepackage{mathtools}
%% For Mengen like natural numbers
\usepackage{amsfonts}
%% Für spezielle Symbole
\usepackage{amssymb}

%% Images
\usepackage{import}
\usepackage{xifthen}
\usepackage{pdfpages}
%\usepackage{transparent}

%%% Command for simpler images
\newcommand{\incfig}[1]{%
    \def\svgwidth{\columnwidth}
    \import{./fig/}{#1.pdf_tex}
}

%% Links
\usepackage{hyperref}
\hypersetup{
    colorlinks=true,
    linkcolor=black,
    filecolor=magenta,
    urlcolor=cyan
}

%% Formatting
\usepackage{parskip}

\title{Analysis}
\author{Moritz}
\date{May 20, 2025}

\begin{document}
\maketitle
\tableofcontents

\section{Folgen und Reihen}

\subsection{Monotones Wachstum}

\begin{align*}
    b_{n+1}-b_n\geq 0 \\
    \frac{b_{n+1}}{b_n}\geq 1
\end{align*}

Mit diesen beiden Formeln können wir monotones Wachstum nachweisen.

\subsection{Vollständigkeitsaxiom von $\mathbb{R}$}

Existenz vom Supremum (einer oberen Grenze). Davon konnten wir den Satz der monotonen konvergenz.

\begin{center}
    ${a_n}n\in \mathbb{N}$ monoton wachsend/ fallend und
    nach oben/ unten beschränkt $\implies$ konvergent.
\end{center}

grenzwert-unabhängiges Konvergenzkriterium (ich muss den Grenzwert nicht kennen, um zu zeigen das es ihn gibt)

\subsection{CAUCHY-Kriterium}

\begin{equation*}
    \forall_{\epsilon>0}\exists_{n_\epsilon}\forall_{n,m\geq n_\epsilon}\quad |a_n-a_m|<\epsilon
\end{equation*}

Heißt: 2 Punkte in einer $\epsilon$-Umgebung haben höchsten den Abstand $\epsilon$ und dies muss für fast alle Punkte gelten.

Somit haben wir: Konvergent $\implies$ CAUCHY-Kriterium

\subsubsection{Umgekehrt: CK $\implies$ Kgz.}

\begin{enumerate}
    \item CL $\implies$ beschränkt
    \item Beschränkte Folgen haben mindestens einen Häufungspunkt (HP) (Satz von BOLZANO-WEIERSTRASS)
    \item Beobachtung: Beschränkte Folge \& nur ein HP $\implies$ HP = Grenzwert
    \item Beschränkt \& CK $\implies$ nur ein HP $\implies$ Kgz.
\end{enumerate}

\subsubsection{Satz von BOLZANO-WEIERSTRASS}

Jede beschränkte Folge in $\mathbb{R}(\mathbb{C})$ hat wenigstens einen Häufungspunkt.

Ich habe den visuellen Beweis verpasst und nicht aufgeschrieben.

Im Groben war ein Teil, rekursiv ein Intervall so halbieren, das in einer Hälfte noch $\infty$ viele Folgenglieder beinhalten. Bis wir einen Punkt haben, welcher in allen dieser Teilintervalle ist.

Annahme: Es gibt zwei HP $a$ und $b$, dennoch müssten sich zwei Punkte $a_n$ und $a_m$ beliebig nahe kommen. Dies geht visuell nicht, da $a$ und $b$ einen Abstand von $>\epsilon$ haben, was ein wiederspruch zum CAUCHY-Kriterium ist.

\subsection{Geometrische Reihe}

\begin{align*}
    S_n:= \sum_{k=0}^{n}q^k
     & =1+q+q^2+q^3+\dots+q^n                   \\
    q*S_n
     & =\qquad q+q^2+q^3+\dots+q^n+q^{n+1}      \\
    (1-q)S_n
     & = S_n-qS_n = 1-q^{n+1}                   \\
    S_n
     & =\sum_{k=0}^{n}q^k=\frac{1-q^{n+1}}{1-q}
    \xrightarrow[|q|<1]{n\to\infty}\frac{1}{1-q}
\end{align*}

\subsubsection{Dezimalentwicklung}

\begin{align*}
    \sqrt{2}
     & =1.414213\dots                                                \\
     & =\sum_{k=0}^{\infty}\frac{a_k}{10^k}\quad,a_0=1, a_1=4, \dots \\
     & =a_0,a_1a_2a_3\dots                                           \\
    a_i
     & \in\{0,1, \dots, 9\}, i\geq 1
\end{align*}

$\sum_{k=1}^{n}\frac{a_k}{10^k}$ Diese folge ist monoton wachsend. Wir brauchen einer obere Schranke. Und die ist 1:

\begin{align*}
    \sum_{k=1}^{n}\frac{a_k}{10^k}
    \leq\sum_{k=1}^{n}\frac{9}{10^k}
    =9*\sum_{k=1}^{n}\left(\frac{1}{10}\right)^k
    <9*\sum_{k=1}^{\infty}\left(\frac{1}{10}\right)^k
    =9*\frac{\frac{1}{10}}{1-\frac{1}{10}}=1
\end{align*}

\subsubsection{Absolut Konvergente Reihen}

Gegeben eine Folge $(a_n)_n\in \mathbb{N}$. Die zugehörige Reihe ist die Folge der \textit{Partialsummen}:

\begin{equation*}
    \left(\sum_{k=1}^{n}a_k\right)_{n\in \mathbb{N}}, \qquad \sum_{k=1}^{n}a_k
\end{equation*}

meist nur durch die Summe angegeben.

Eine Reihe heißt \textit{absolut konvergent}, falls die Reihe $\sum_{k=1}^{n}|a_k|$ der Beträge konvergiert, andernfalls \textit{bedingt konvergent}.

\textbf{Satz}: Für eine Reihe gilt: absolut konvergent $\implies$ konvergent

$\textbf{Beweis}$: absolut Konvergent $\implies$ CK für $\sum_{k=1}^{n}|a_k|$ gilt. Zu zeigen: dann CK für $\sum a_k$ gilt.

d.h. mit $m > n$

\begin{align*}
    \forall_{\epsilon>0}\exists_{n_\epsilon}\forall{n, m\geq n_\epsilon}
     & |\sum_{k=1}^{m}|a_k|-\sum_{k=1}^{n}|a_k|| = \sum_{k=n+1}^{m}|a_k| < \epsilon                                              \\
    \forall_{\epsilon>0}\exists_{n_\epsilon}\forall{n, m\geq n_\epsilon}
     & |\sum_{k=1}^{m}a_k-\sum_{k=1}^{n}a_k| = |\sum_{k=n+1}^{m}a_k| \leq \sum_{k=n+1}^{m}|a_k| < \epsilon \implies \text{ kgz.}
\end{align*}

\end{document}
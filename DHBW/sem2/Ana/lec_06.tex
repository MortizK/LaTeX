\documentclass[a4paper]{article}

%\usepackage{url}

%% Math
\usepackage{mathtools}
%% For Mengen like natural numbers
\usepackage{amsfonts}
%% Für spezielle Symbole
\usepackage{amssymb}

%% Images
\usepackage{import}
\usepackage{xifthen}
\usepackage{pdfpages}
%\usepackage{transparent}

%%% Command for simpler images
\newcommand{\incfig}[1]{%
    \def\svgwidth{\columnwidth}
    \import{./fig/}{#1.pdf_tex}
}

%% Links
\usepackage{hyperref}
\hypersetup{
    colorlinks=true,
    linkcolor=black,
    filecolor=magenta,
    urlcolor=cyan
}

%% Formatting
\usepackage{parskip}

\title{Analysis}
\author{Moritz}
\date{May 27, 2025}

\begin{document}
\maketitle
\tableofcontents

\section{Folgen und Reihen}

Euler Folge mit zeigen der monoton wachsenden Folge is im Heft Ana 27.05 Euler Folge.

\begin{equation*}
    e_n:=\left(1+\frac{1}{n}\right)^{n+1}=\left(\frac{n+1}{n}\right)^{n+1}
\end{equation*}

\subsection{Majoranten}

Eine Reihe $\sum_{k=1}^{n}a_k$ heißt \textit{absolut konvergent}, falls $\sum_{k=1}^{n}|a_k|$ konvergiert.

$\sum_{k=1}^{n}|a_k|$ wird mit Hilfe on Majoranten untersucht, d.h. mit Reihen $\sum_{k=1}^{n}b_k$, sodass $|a_k|\leq b_k$ für fast alle $k\in \mathbb{N}$ und $\sum_{k=1}^{\infty}b_k$ ex. Dann ist nämlich $\sum_{k=1}^{\infty}b_k$ eine obere Schranke von $\sum_{k=1}^{n}|a_k|$: denn:

\begin{equation*}
    \sum_{k=1}^{n}|a_k|
    \leq \sum_{k=1}^{n}b_k
    \leq \sum_{k=1}^{\infty}b_k
\end{equation*}

Mögliche (da uns bekannte) Majoranten sind:

\begin{align*}
     & \sum_{k=1}^{n}\frac{1}{k^2}    \\
     & \sum_{k=1}^{n}q^k\quad (0<q<1)
\end{align*}

Wir möchten Nachweisen:

\begin{align*}
    |a_k|           & \leq q^k \\
    \sqrt[k]{|a_k|} & \leq q
\end{align*}

Und dies wandeln wir um:

\begin{equation*}
    \lim_{k\to\infty}\sqrt[k]{|a_k|}=\tilde{q}<1
\end{equation*}

Wenn wir dies Ungleichung nachweisen, ist die Reihe $\sum_{k=1}^{n}a_k$ absolut konvergent

\subsection{Potenzreihen}

Polynome: $p(x)=a_0+a_1x+a_2+x^2+a_3x^3+\dots+a_nx^n=\sum_{k=0}^{n}a_kx^k$ n-ten Grades

Potenzreihen sind "Polynome $\infty$-Grades" $\sum_{k=0}^{\infty}a_kx^k$ ist eine Funktion von $x\in \mathbb{C}$

Das ist alles für jedes $x$ ein Grenzwert. Wie sieht also die Menge aller $x\in \mathbb{C}$ aus, für die die Reihe konvergiert?

Antwort: Potenzreihen konvergieren in Kreisscheiben. D.h. nur der Radius ist zu bestimmen. In der Kreisscheibe gilt \textit{absolute Konvergenz} und außerhalb \textit{Divergenz}.

\subsubsection{Beipsiele}

\begin{equation*}
    \sum_{k=1}^{\infty}\frac{1}{k^2}*x^k,\quad
    \sum_{k=1}^{\infty}k^2x^k,\quad
    \sum_{k=0}^{\infty}2^kx^k
    = \sum_{k=0}^{\infty}(2x)^k,\quad
    \sum_{k=0}^{\infty}\frac{x^k}{k!}
\end{equation*}

Konvergenzradien: $R_1=1, R_2=1, R_3=\frac{1}{2}, R_4=\infty$

\subsubsection{Satz}

Eine Potenzreihe $\sum_{k=0}^{\infty}a_kx^k$ konvergiert im Innern einer Kreisscheibe mit Radius $R$ absolut und ist außerhalb divergent. Auf dem Rand haben wir keine Aussage.

$R$ heißt Konvergenzradius der Potenzreihe und berechnet sich (fast immer)

\begin{equation*}
    R=\frac{1}{\lim_{k\to\infty} \sqrt[k]{|a_k|}}, \quad
    R=\frac{1}{\lim_{k\to\infty}\frac{|a_{k+1}|}{|a_k|}}
\end{equation*}

Den Fall $\lim_{k\to\infty} \sqrt[k]{|a_k|}=0$ oder $\lim_{k\to\infty}\frac{|a_{k+1}|}{|a_k|}$ bezeichnet man durch "$R=\infty$" und sagt: Der Konvergenzradius ist unendlich (meint: jedes $x\in \mathbb{C}$ liefert ein absolut konvergent Reihe)

\subsubsection{Beweis}

$\sum_{k=0}^{\infty}a_kx^k$ Idee: Wir wandeln das Wurzelkriterium (Quotientenk.) auf $a_kx^k$ an.

Konvergenzbedingung ist also:

\begin{equation*}
    \lim_{k\to\infty}\sqrt[k]{|a_kx^k|}
    =\lim_{k\to\infty}\sqrt[k]{|a_k||x^k|}
    =\lim_{k\to\infty}|x|*\sqrt[k]{|a_k|}
    =|x|*\lim_{k\to\infty}\sqrt[k]{|a_k|}<1
\end{equation*}

\begin{equation*}
    \implies |x| < \frac{1}{\lim_{k\to\infty}\sqrt[k]{|a_k|}} =: R
    \quad\text{konvergent}
\end{equation*}

Die Wahl des Quotientenkriterium:

\begin{equation*}
    \lim_{k\to\infty}\frac{|a_{k+1}x^{k+1}|}{|a_kx^k|}
    =\lim_{k\to\infty}\frac{|a_{k+1}x|}{|a_k|}
    =|x|*\lim_{k\to\infty}\frac{|a_{k+1}|}{|a_k|}<1
\end{equation*}

\begin{equation*}
    \implies |x|=\frac{1}{\lim_{k\to\infty}\frac{|a_{k+1}|}{|a_k|}}=:R
\end{equation*}

\subsection{Funktionen}

$g(x):=f(x-1)$ Wir setzen bei $g(1)$, den wert $f(0)$ ein. Dies verschiebt die Funktion $g(x)$ nach rechts, beziehungsweise die \textbf{Y-Achse} von der Funktion $f(x)$ nach links.

\end{document}
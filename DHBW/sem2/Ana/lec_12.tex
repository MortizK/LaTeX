\documentclass[a4paper]{article}

%\usepackage{url}

%% Math
\usepackage{mathtools}
%% For Mengen like natural numbers
\usepackage{amsfonts}
%% Für spezielle Symbole
\usepackage{amssymb}

%% Images
\usepackage{import}
\usepackage{xifthen}
\usepackage{pdfpages}
%\usepackage{transparent}

%%% Command for simpler images
\newcommand{\incfig}[1]{%
    \def\svgwidth{\columnwidth}
    \import{./fig/}{#1.pdf_tex}
}

%% Links
\usepackage{hyperref}
\hypersetup{
    colorlinks=true,
    linkcolor=black,
    filecolor=magenta,
    urlcolor=cyan
}

%% Formatting
\usepackage{parskip}

\title{Analysis}
\author{Moritz}
\date{June 17, 2025}

\begin{document}
\maketitle
\tableofcontents

\section{Differentialrechnung}

\subsection{Stetige Funktionen Argumentation}

\begin{align*}
    f(x)=\begin{cases}
             exp(-\frac{1}{x^2} & \quad, x\neq 0) \\
             a                  & \quad, x=0
         \end{cases}
\end{align*}

$x\mapsto f(x)$ ist für $x\neq 0$ stetig, denn $x\mapsto -\frac{1}{x^2}$ ist stetig, die Exponentialfunktion ist stetig, also ist auch $x\mapsto exp(-\frac{1}{x^2})$ als Verkettung stetiger Funktionen stetig.

$x\to\infty$: $-\frac{1}{x^2}\xrightarrow{x\to\infty}0$, da exp stetig ist, $exp(-\frac{1}{x^2})\xrightarrow{x\to\infty}exp(0)=1$

$x\to0$: $-\frac{1}{x^2}\xrightarrow{x\to0}-\infty$ und daher $exp(-\frac{1}{x^2})\xrightarrow{x\to0}0$

Wir wählen $a=0$ damit die Funktion stetig ist. HIER: SKIZZE kann helfen.

\subsection{Mittelwertsatz}

$f$ auf $[a, b]$ stetig und in $(a, b)$ diffbar. Dann gibt es eine Stelle $\xi\in(a, v)$, sodass:

\begin{equation}
    \frac{f(b)-f(a)}{b-a}=f'(\xi)
\end{equation}

Skizzen sind in meinem Heft unter: Ana 17.06 Mittelwertsatz.

\textbf{Beweis}: Zunächst nur für $f(a)=f(b)$

\begin{enumerate}
    \item Max. und Min liegen beide auf dem Rand. $\implies f(a)=f(x)=$konst.
          \begin{center}
              Jedes $\xi\in(a, b)$ liefert $f(\xi)=0$
          \end{center}
    \item Max. oder Min liegt bei $\xi\in(a,b)$
          \begin{center}
              $\implies f'(\xi)=0$
          \end{center}
\end{enumerate}

\textbf{Hilfsfunktion}: Jetzt der allgemeine Fall, welche wir auf diesen Spezial zurück.

\begin{align*}
    h(x) := & f(x)-\frac{f(b)-f(a)}{b-a}(x-a)                      \\
    h(a) =  & f(a),\quad h(b)=f(b)-\frac{f(b)-f(a)}{b-a}(b-a)=f(a)
\end{align*}

\begin{align*}
    \implies \exists\xi\in(a,b): h'(\xi)=0=f'(\xi)-\frac{f(b)-f(a)}{b-a}
\end{align*}

\textbf{Abstaubersatz}

\begin{align*}
    h(x) := & f(x)-\frac{f(b)-f(a)}{g(b)-g(a)}(g(x)-g(a))                      \\
    h(a) =  & f(a),\quad h(b)=f(b)-\frac{f(b)-f(a)}{g(b)-g(a)}(g(b)-g(a))=f(a)
\end{align*}

\begin{equation*}
    \implies \exists\xi\in(a,b): h'(\xi)=0=f'(\xi)-\frac{f(b)-f(a)}{g(b)-g(a)}*g'(\xi)
\end{equation*}

$g$ auf $[a,b]$ stetig und in $(a,b)$ diffbar und $g'(x)\neq 0$ in $(a,b)$ (monoton wachsend/ fallend).

$f$ auf $[a, b]$ stetig und in $(a, b)$ diffbar. Dann gibt es eine Stelle $\xi\in(a, v)$, sodass:

\begin{equation}
    \frac{f(b)-f(a)}{g(b)-g(a)}=\frac{f'(\xi)}{g'(\xi)}
\end{equation}

\subsection{L'Hôpital}

\begin{align*}
    f(x)\xrightarrow{x\to 0}0=f(0)              \\
    g(x)\xrightarrow{x\to 0}0=g(0)              \\
    \frac{f(x)}{g(x)}\xrightarrow{?}
    "\frac{0}{0}" \text{ unbestimmter Ausdruck} \\
    \frac{f(x)}{g(x)}=\frac{f(x)-f(0)}{g(x)-f(0)}
    =\frac{f'(\xi)}{g'(\xi)}
\end{align*}

Falls:

\begin{equation*}
    \lim_{\xi\to0}\frac{f'(\xi)}{g'(\xi)}=c
\end{equation*}

\textbf{Beispiel}

\begin{equation*}
    \lim_{x\to0}\frac{\sin(x)}{x}
    =\lim_{x\to0}\frac{\cos(x)}{1}
    =\cos(0)=1
\end{equation*}

\subsection{Kurvendiskussion}

\subsubsection{Symmetrien}

Achsensymmetrie: $f(-x)=f(x)$

Punktsymmetrie: $f(-x)=-f(x)$

\textbf{Rechenregeln}

\begin{enumerate}
    \item $f*g:$ $p*p=a;p*a=p;a*p=p;a*a=a$
    \item $\frac{f}{g}:$ $\frac{p}{p}=a;\frac{p}{a}=\frac{a}{p}=p;\frac{a}{a}=a;$
\end{enumerate}

Um dies zu erkennen muss man genau hingucken. Bei manchen Polynomen lässt es sich erkennen.

\begin{equation*}
    x\mapsto \frac{3x^3-7x}{x^4-2x^2+5} = \frac{p}{a} = p
\end{equation*}

\subsubsection{Untersuchung}

Phantasyfunktion hierzu ist im Heft: Ana 17.06 Kurvendiskussion, sowie Beispiele zur Vorzeichenmethode.

\begin{enumerate}
    \item Symmetrie, Asymptoten
    \item Nullstellen: $f(x)=0$ , $x_1$
    \item Extrema: $f'(x)=0$ , $ \tilde{x}_1 (\tilde{y}_1=f(\tilde{x}_1))$

          Identifikation: (Hinreichend:)
          \begin{align*}
              f''(\tilde{x}_1)<0 &
              \implies H=[\tilde{x}_1, \tilde{y}_1] \\
              f''(\tilde{x}_2)>0 &
              \implies T=[\tilde{x}_2, \tilde{y}_2] \\
              f''(\tilde{x}_3)=0 & :
              \text{ keine Info} \to \text{Vorzeichenmethode}
          \end{align*}
          Alternativ: Vorzeichenmethode (VZM) (Notwendig und hinreichend)
    \item Wendepunkte: $f''(x)=0$ , $\hat{x}_1 (\hat{y}_1=f(\hat{x}_1))$

          Identifikation:
          \begin{align*}
              f'''(\hat{x}_1)\neq 0 &
              \implies W_1=[\hat{x}_1, \hat{y}_1] \\
              f'''(\hat{x}_4)=0     &
              \implies\text{?}\to\text{VZM für } f''
          \end{align*}
          Alternativ: VZM für $f''$ (Notwendig und hinreichend)
\end{enumerate}

\textbf{Vorzeichenmethode}

Wir untersuchen einfach die Ableitung auf das Vorzeichen $x+\nu$ und $x-\nu$ sodass: $f'(x-\nu)\neq0\neq f'(x+\nu)$. Falls die Vorzeichen gleich sind, so haben wir einen Sattelpunkt, wenn nicht ist es ein Hochpunkt $(+\to-)$ oder Tiefpunkt $(-\to+)$.

\end{document}
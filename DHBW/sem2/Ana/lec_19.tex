\documentclass[a4paper]{article}

%\usepackage{url}

%% Math
\usepackage{mathtools}
%% For Mengen like natural numbers
\usepackage{amsfonts}
%% Für spezielle Symbole
\usepackage{amssymb}

%% Images
\usepackage{import}
\usepackage{xifthen}
\usepackage{pdfpages}
%\usepackage{transparent}

%%% Command for simpler images
\newcommand{\incfig}[1]{%
    \def\svgwidth{\columnwidth}
    \import{./fig/}{#1.pdf_tex}
}

%% Links
\usepackage{hyperref}
\hypersetup{
    colorlinks=true,
    linkcolor=black,
    filecolor=magenta,
    urlcolor=cyan
}

%% Formatting
\usepackage{parskip}

\title{Analysis}
\author{Moritz}
\date{July 9, 2025}

\begin{document}
\maketitle
\tableofcontents

\section{Integralrechnung}

\subsection{Integrale in anderen Bereichen}

\begin{enumerate}
    \item Länge einer Kurve
          \begin{equation*}
              \int_{a}^{b} \sqrt{1+f'^2(x)} dx
          \end{equation*}
    \item Volumen (von Rotationskörpern)
          \begin{align*}
              V & =\pi \sum_{k=0}^{n}f^2(x_k)*\Delta x_k            \\
                & \xrightarrow{n\to\infty} \pi\int_{a}^{b}f^2(x) dx
          \end{align*}
    \item Oberfläche (von Rotationskörpern)
          \begin{equation*}
              O = \int_{a}^{b} 2\pi f(x)\sqrt{1+f'^2(x)} dx
          \end{equation*}
\end{enumerate}

Diese Integrale sind für den Kreis und Kugel in meinem Heft unter Ana 7.8.

\subsection{Satz von TAYLOR}

\begin{align*}
    \int_{0}^{x}f'(t) dt
     & = [f(t)]_0^x = f(x) -f(0)                                                                              \\
    \implies f(x)
     & = f(0) + \int_{0}^{x}f'(t) dt                                                                          \\
     & = f(0) + [(t-x)f'(t)]_0^x - \int_{0}^{x}(t-x)f''(x) dt                                                 \\
     & = f(0) + f'(0)x - \left[\frac{1}{2}(t-x)^2f''(t)\right]_0^x + \int_{0}^{x}\frac{1}{2}(t-x)^2f'''(x) dt \\
     & = \dots
\end{align*}

\begin{equation*}
    f(x)=\sum_{k=0}^{n}\frac{f_{(0)}^{(k)}}{k!}x^k + (-1)^n\int_{0}^{x}\frac{(t-x)^n}{n!}f_{(t)}^{(n+1)} dt
\end{equation*}

Das Restglied nach der Reihe $\xrightarrow{n\to\infty}0$ geht gegen null.

\end{document}
\documentclass[a4paper]{article}

%\usepackage{url}

%% Math
\usepackage{mathtools}
%% For Mengen like natural numbers
\usepackage{amsfonts}
%% Für spezielle Symbole
\usepackage{amssymb}

%% Images
\usepackage{import}
\usepackage{xifthen}
\usepackage{pdfpages}
%\usepackage{transparent}

%%% Command for simpler images
\newcommand{\incfig}[1]{%
    \def\svgwidth{\columnwidth}
    \import{./fig/}{#1.pdf_tex}
}

%% Links
\usepackage{hyperref}
\hypersetup{
    colorlinks=true,
    linkcolor=black,
    filecolor=magenta,
    urlcolor=cyan
}

%% Formatting
\usepackage{parskip}

\title{Analysis}
\author{Moritz}
\date{May 21, 2025}

\begin{document}
\maketitle
\tableofcontents

\section{Folgen und Reihen}

Eine Reihe $\sum_{k=1}^{n}a_k$ heißt \textit{absolut konvergent}, falls $\sum_{k=1}^{n}|a_k|$ konvergiert.

\begin{equation*}
    \implies \sum_{k=1}^{n} \text{ ist konvergent}
\end{equation*}

\subsection{Majorantenkriterium}

gegeben $\sum_{k=1}^{n}a_k$. Falls es eine konvergente Reihe $\sum_{k=1}^{n}b_k$ mit $b_k\geq 0$ gibt, sodass:

\begin{center}
    $|a_k|\leq b_k$
\end{center}

gilt, dann $\sum_{k=1}^{n}b_k$ \textit{Majorante} von $\sum_{k=1}^{n}a_k$. Dann ist $\sum_{k=1}^{n}a_k$ \textit{absolut konvergent}. Denn:

\begin{equation*}
    \sum_{k=1}^{n}|a_k|\leq \sum_{k=1}^{n}b_k\leq \sum_{k=1}^{\infty}a_k=M\qquad\text{obere Schranke}
\end{equation*}

\begin{enumerate}
    \item Idee (immer): Versuche die Geometrische Reihe: $|a_k|\leq q^k$ mit $0<q<1$

          \begin{center}
              $\implies \sqrt[k]{|a_k|}\leq q < 1$
          \end{center}

          Idee: Berechne $\lim_{k\to\infty}\sqrt[k]{|a_k|}<1$

          Hoffnung: Dann ist die Reihe \textit{absolut konvergent}:

          Also: Annahme $\lim_{k\to\infty}\sqrt[k]{|a_k|}=:\tilde{q}<1$ gibt es.

          D.h. für fast alle $k\in \mathbb{N}$ gilt $\sqrt[k]{|a_k|}\leq q (=\tilde{q} + \epsilon)<1$

          \begin{center}
              $\implies |a_k| \leq q^k$ für fast alle k
          \end{center}
\end{enumerate}

\subsection{Satz (Wurzel/ Quotientenkriterium)}

Gegeben eine Reihe $\sum_{k=1}^{n}a_k$.

Falls

\begin{enumerate}[i.]
    \item $\lim_{k\to\infty}\sqrt[k]{|a_k|}=q$ oder
    \item $\lim_{k\to\infty}|\frac{a_{k+1}}{a_k}|=q$ existiert
\end{enumerate}

und $q<1$ ist, dann ist die Reihe \textit{absolut konvergent}.

Falls $q=1$ ist, haben wir keine Aussage.

Falls $q>1$ ist, dann ist die Reihe sicher \textit{divergent} (normalerweise wenig informativ)

Diesen Satz finde ich im \href{https://elearning.dhbw-stuttgart.de/moodle/pluginfile.php/812513/mod_folder/content/0/Vorlesungsskript/Analysis_kurz.pdf#page=21}{Skriptum} unter \textbf{1.2.8 Satz}. Den Beweis für ii. (Quotientenkriterium) habe ich nicht, hat aber Rosalie.

Beobachtung: Damit eine Reihe $\sum_{k=1}^{n}a_k$ konvergieren kann, muss $a_k\xrightarrow{k\to\infty}0$ gelten.

Denn: Falls $\sum_{k=1}^{n}a_k$ konvergiert, muss CAUCHY-Kriterium erfüllt sein:

\begin{equation*}
    \forall_{\epsilon>0}\exists_{n_\epsilon}\forall_{n, m\geq n_\epsilon}
    |\sum_{k=1}^{m}(a_n)-\sum_{k=1}^{n}(a_n)|<\epsilon
\end{equation*}

insbesondere für $m=n+1$ und $n\geq n_\epsilon$:

\begin{equation*}
    \forall_{\epsilon>0}\exists_{n_\epsilon}\forall_{n\geq n_\epsilon}
    |\sum_{k=1}^{n+1}(a_n)-\sum_{k=1}^{n}(a_n)|=|a_{n+1}|<\epsilon
\end{equation*}

$\implies (a_n)_{n\in \mathbb{N}}$ ist eine Nullfolge.

\subsection{Satz (LEIBNIZ-Reihen)}

Reihen der Form $\sum_{k=0}^{n}(-1)^k*a_k$ heißt LEIBNIZ-Reihe, wenn die $(a_k)_{k\in \mathbb{N}}$ eine monoton fallende Nullfolge bildet.

Alle LEIBNIZ-Reihen sind konvergent.

\end{document}
\documentclass[a4paper]{article}

%\usepackage{url}

%% Math
\usepackage{mathtools}
%% For Mengen like natural numbers
\usepackage{amsfonts}
%% Für spezielle Symbole
\usepackage{amssymb}

%% Images
\usepackage{import}
\usepackage{xifthen}
\usepackage{pdfpages}
%\usepackage{transparent}

%%% Command for simpler images
\newcommand{\incfig}[1]{%
    \def\svgwidth{\columnwidth}
    \import{./fig/}{#1.pdf_tex}
}

%% Links
\usepackage{hyperref}
\hypersetup{
    colorlinks=true,
    linkcolor=black,
    filecolor=magenta,
    urlcolor=cyan
}

%% Formatting
\usepackage{parskip}

\title{Analysis}
\author{Moritz}
\date{May 6, 2025}

\begin{document}
\maketitle
\tableofcontents

\section{Folgen und Reihen}

\begin{align*}
    (x_n)_n\in \mathbb{N} & : x_1, x_2, x_3, \dots                                            \\
    \sqrt{2}              & = 1.4142315\dots < 1.5                                            \\
    x_1 = 1, x_2          & = 1.4, x_3 = 1.41, x_4 = 1.414, \dots                             \\
    a_1 = 1, a_2          & = 0.4, a_3 = 0.01, a_4 = 0.004, \dots                             \\
    x_1 = a_1, x_2        & = x_1 + a_2, x_3 = x_2 + a_3, x_n=\sum_{k=1}^{n}a_k \text{ Reihe}
\end{align*}

Es gibt 2 Wege $\mathbb{Q}\to \mathbb{R}$:

\begin{enumerate}
    \item DEDEKINDsche Schnitte:
          Wir bilden Menge welche den obersten Wert haben welcher z.B. $\sqrt{2}$ darstellt. $\{x\in \mathbb{Q}\mid x^2\leq 2\}$
    \item Vervollständigung:
          $\sqrt{2}= (1.41, 1.414, \dots)$
    \item Vollständigkeitsaxiom:
          Wenn wir eine Menge haben, brauchen wir nur eine obere Schranke, diese muss nicht besonders gut sein.
          Axiom besagt, wenn es eine obere Schranke gibt, wird es eine kleinste obere Schranke geben
\end{enumerate}

\textbf{Vorstellung konvergenter Folgen}

$x: \mathbb{N}\to \mathbb{R}, n\to x_n$ Kann ich mir auch so vorstellen, wie eine Punkte Plot welche für $\sqrt{2}$ wie eine logarithmisches Wachstum nur halt nicht ins unendliche.

\subsection{Grenzwertbegriff}

\know{Fast alle ...}{Bedeutet es gibt nur endlich viele und somit auch ein letztes Element, welches die Bedingung nicht erfüllt und alle danach weren die Bedingung erfüllen}

Reelle Folge: fast alle $x_n$ liegen in $(x-\epsilon, x+\epsilon)$

d.h. alle, bis auf endliche viele

d.h. es gibt ein $n_\epsilon$, sodass $x_n\in (x-\epsilon, x+\epsilon)$ für alle $n\geq n_\epsilon$

d.h. f.a. $n\geq n_\epsilon$ $\mid x-x_n\mid <\epsilon$

d.h. es gibt ein $n_\epsilon\in \mathbb{N}$, sodass für alle $n\geq n_\epsilon$ $\mid x-x_n\mid <\epsilon$. Für welche?

\subsubsection{Definition}

Für alle $\epsilon > 0$ gibt es ein $n_\epsilon\in \mathbb{N}$, sodass für alle $n\geq n_\epsilon$ gilt

\begin{center}
    $\mid x-x_n\mid <\epsilon$
\end{center}

Dann heißt die Folge $(x_n)_{n\in \mathbb{N}}$ gegen x konvergent. Notation:

\begin{align*}
    \lim_{x\to\infty} x_n = x       \\
    x_n \xrightarrow{n\to \infty} x \\
    \forall_{\epsilon>0}\exists_{n_\epsilon\in \mathbb{N}}\forall_{n\geq n_\epsilon} \mid x-x_n\mid<\epsilon
\end{align*}

\textbf{Eindeutigkeit des Grenzwertes (Idee)}

Annahme, es gibt zwei vermutete Grenzwerte. Somit müssten in zwei Umgebungen fast alle Punkte liegen. Da wir zwei Umgebungen annehmen, müssten in beiden "fast alle" Punkte liegen. Dies ist ein Wiederspruch.

\subsubsection{Rechenregeln}

Wir können konvergente Folgen miteinander addieren, subtrahieren, multiplizieren, dividieren, usw. Hierbei können wir auf die Grenzwerte die Gleichen Operatoren anwenden und bekommen den richtigen Grenzwert.

\textbf{Satz (Rechenregeln konvergenter Folgen)}

gegeben $x_n\xrightarrow{n\to\infty}x, y_n\xrightarrow{n\to\infty}y$. Dann gilt:

\begin{enumerate}[i.]
    \item $x_n\pm y_n \xrightarrow{n\to\infty}x\pm y$
    \item $x_n * y_n \xrightarrow{n\to\infty}x * y$
    \item $y\neq 0$: $x_n / y_n \xrightarrow{n\to\infty}x / y$
    \item $x_n\geq 0$: $\sqrt{x_n} \xrightarrow{n\to\infty} \sqrt{x}$
\end{enumerate}

\subsection{Reihen}

\end{document}
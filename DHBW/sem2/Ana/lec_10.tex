\documentclass[a4paper]{article}

%\usepackage{url}

%% Math
\usepackage{mathtools}
%% For Mengen like natural numbers
\usepackage{amsfonts}
%% Für spezielle Symbole
\usepackage{amssymb}

%% Images
\usepackage{import}
\usepackage{xifthen}
\usepackage{pdfpages}
%\usepackage{transparent}

%%% Command for simpler images
\newcommand{\incfig}[1]{%
    \def\svgwidth{\columnwidth}
    \import{./fig/}{#1.pdf_tex}
}

%% Links
\usepackage{hyperref}
\hypersetup{
    colorlinks=true,
    linkcolor=black,
    filecolor=magenta,
    urlcolor=cyan
}

%% Formatting
\usepackage{parskip}

\title{Analysis}
\author{Moritz}
\date{June 10, 2025}

\begin{document}
\maketitle
\tableofcontents

\section{Reelle Funktionen}

\subsection{Stetigkeit}

Erinnerung: Stetigkeit von Funktionen

\begin{equation*}
    x_n\in D_f \xrightarrow{n\to\infty}x\in D_f 
    \implies f(x_n)\xrightarrow{x\to\infty} f(x)
\end{equation*}

kurz: 

\begin{equation}
    \lim_{n\to\infty} f(x_n)
    = f(\lim_{n\to\infty}x_n) 
\end{equation}

\subsubsection{Satz}

Stetigkeit $\iff$

\begin{equation}
    \lim_{x\to x_0^-}f(x)=\lim_{x\to x_0^+}f(x)=f(x_0)
\end{equation}

Der Linksseitige Grenzwert gleich dem Rechtsseitige Grenzwert ist. Nach Sandwichkriterium ist dies der Grenzwert der Funktion.

\subsubsection{Verkettung}

Stetige Funktionen sind: $\sqrt{x}, \frac{1}{x}, x^n, e^x$

Somit ist auch diese stetig:

\begin{equation*}
    \sqrt{\frac{e^x+1}{x^2+5x-2}}
\end{equation*}

\subsubsection{Wichtige Sätze}

\begin{enumerate}
    \item Satz vom Maximum: Wenn eine Funktion stetig ist, gibt es in jedem Intervall $f: [a, b]$ ein Maximum
    \item Zwischenwertsatz: $f:[a, b]\to \mathbb{R}$ stetig. Alle Werte zwischen $f(a)$ und $f(b)$ in diesem Intervall sind Funktionswerte.
\end{enumerate}

\subsection{Differentialrechnung}

Es geht um Steigungen, d.h. wir suchen  eine  Möglichkeit für eine Funktion $f$ an einer Stelle $x$ zu sagen, wie steil ist $f$ in $x$, d.h. im Punkt $[, f(x)]$

Die Steigung der Tangente an der Stelle $x_0$ (genauer $[x_0, f(x_0)]$)

\textbf{Ergebnis}: Die Steilheit von $f$ im Punkt $[x_0, f(x_0)]$, kurz die Steigung von $f$ an der Stelle $x_0$ ist die Steigung der Tangenet an dieser Stelle.

\subsubsection{Tangente}

Damit is der Begriff der Steigung einer Funktion in $x_0$ auf den schon bekannten Begriff "Steigung einer Geraden, nämlich der Tangent in $x_0$ zurückgeführt"

Wir schreiben für Geraden:

\begin{align*}
    g(x) &= 2x - 1\\
    h(x) &= \frac{1}{2}x+5\\
    t(x) &= -x+2
\end{align*}

Nun zum Steigungsbegriff bei Funktionen. Anders als bei Geraden wird es batürlich nicht eine Steigung für die ganze Funktion geben können - die Funktion ist ein verschiedenen Stellen verschieden Steil. Als MAß für Steilheit wählen wir die Steigung ihrere Tangente an der betreffenden Stelle. Nur was ist die Tangente?

\textbf{Tangente} ist diejenige Gerade, die den Fehler $f(x)-t(x)$ minimiert in einer kleinen umgebung von $x_0$.

$t$ ist die bestapproximierende Gerade ub $[x_0, f(x_0)]$.

\textbf{Konsequenzen}: Es gibt Punkte einer Funktion, wo es keine Tangente gibt. An Knicken gibt es keine Tangent. Das is auc hrichtig so, weil der Begriff Steigung an so einer Stelle sinnlos ist.

\textbf{Sekante}: Da wir die Tangente nicht berechnen können, machen wir mit der Sekante eine Fehle, und schauen ob wir diesen klein bekommen.

Sekantensteigung:

\begin{equation*}
    m_s=\frac{\Delta f(x_0)}{\Delta x_0}
    =\frac{f(x_0+h)-f(x_0)}{h}
\end{equation*}

Ist ein Bruch (Quotient) aus zwei Differenzen, nämlcih $\Delta f(x_0)$ und $\Delta x_0$. Daher der Name:

\begin{center}
    Differenzenquotient
\end{center}

Den Fehler, den wir durch den zweiten Punkt $[x_0+h, f(x_0+h)]$ machen, können wir minimieren, indem wir $h\to 0$ gehen lassen und dabei den Differenzenquotienten berechnen. D.h. wir versuchen den Grenzwert:

\begin{equation*}
    \lim_{h\to0}\frac{f(x_0+h)-f(x_0)}{h}
\end{equation*}

auszurechnen. Diese Steigung nennen wir nicht einfach $m$, sondern $f'(x_0)$...

\subsubsection{Notation}

\begin{align*}
    &\lim_{\Delta x_0\to 0} \frac{\Delta f(x_0)}{\Delta x_0} \\
    = &\lim_{h\to0}\frac{f(x_0+h)-f(x_0)}{h} \\
    =: &\frac{df(x_0)}{dx_0}
\end{align*}

gelesen: $df$ nach $dx_0$ (ableiten)

\begin{equation}
    \frac{df(x)}{dx}=\frac{d}{dx}f(x)
\end{equation}

heißt "Differentialquotient"

\subsubsection{Ableiten Beispiele}

\begin{align*}
    \frac{dx^2}{dx}
    &=2x\\
    \frac{d}{dx}(3x^4-\frac{1}{x})
    &=12x^3+\frac{1}{x^2}
\end{align*}

\begin{align*}
    f(x) 
    &= x^2:\\
    \frac{f(x+h)-f(x)}{h}
    &= \frac{(x+h)^2-x^2}{h}
    = \frac{x^2+2xh+h^2-x^2}{h}\\
    &= \frac{h(2x+h)}{h}
    = 2x+h \xrightarrow{h\to0} 2x = f'(x) = \frac{dx^2}{dx}
\end{align*}

\begin{equation*}
    nx^{n-1}=\frac{dx^n}{dx}\qquad\text{Potenzregel}
\end{equation*}

\subsubsection{Ableitungsregeln}

sind aus dem Differentialquotient hergeleitet:

\begin{enumerate}[i)]
    \item $(f\pm g)' = f' \pm g'$ (Summenregel)
    \item $(f*g)' = f'*g + f*g'$ (Produktregel)
    \item $(\frac{f}{g})' = \frac{f'*g - f*g'}{g^2}$ (Quotientenregel)
    \item $(f\circ g)' = (f'\circ g)*g'$ (Kettenregel)
\end{enumerate}

\textbf{Kettenregel}

\begin{align*}
    \frac{d}{dx}f(g(x)) = f'(g(x))*g'(x)
\end{align*}

Knapp: "Äußere Abbleitung mal Innere Ableitung"

\end{document}
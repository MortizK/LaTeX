\documentclass[a4paper]{article}

%\usepackage{url}

%% Math
\usepackage{mathtools}
%% For Mengen like natural numbers
\usepackage{amsfonts}
%% Für spezielle Symbole
\usepackage{amssymb}

%% Images
\usepackage{import}
\usepackage{xifthen}
\usepackage{pdfpages}
%\usepackage{transparent}

%%% Command for simpler images
\newcommand{\incfig}[1]{%
    \def\svgwidth{\columnwidth}
    \import{./fig/}{#1.pdf_tex}
}

%% Links
\usepackage{hyperref}
\hypersetup{
    colorlinks=true,
    linkcolor=black,
    filecolor=magenta,
    urlcolor=cyan
}

%% Formatting
\usepackage{parskip}

\title{Analysis}
\author{Moritz}
\date{July 8, 2025}

\begin{document}
\maketitle
\tableofcontents

\section{Integralrechnung}

\subsection{Substitutionsregel}

\begin{align*}
    \frac{d}{dx}F(u(x))
     & = F'(u(x))*u'(x)                       \\
     & = f(u(x))*u'(x),\quad\mid \int_{a}^{b} \\
    \int_{a}^{b}\frac{d}{dx}F(u(x))dx
     & = \int_{a}^{b} f(u(x))*u'(x) dx        \\
    \left[F(u(x))\right]_a^b = F(u(b)-F(u(a)))
     & = \int_{u(a)}^{u(b)}f(u)du
\end{align*}

Formel:

\begin{equation*}
    \int_{a}^{b}f(u(x))u'(x) dx = \int_{u(a)}^{u(b)}f(u)du
\end{equation*}

Beispiel:

\begin{align*}
    \int \sqrt{u} du = \frac{2}{3}u^{\frac{3}{2}}=\frac{2}{3}u\sqrt{u}
\end{align*}

Rechenverfahren (formal, aber alle machen es so):

\begin{align*}
    \int x\sqrt{1+x^2}dx
       & = \frac{1}{2} \int\sqrt{u} du     \\
       & = \frac{1}{2}\frac{2}{3}u\sqrt{u}
    = \frac{1}{3}\sqrt{u}^3                \\
       & = \frac{1}{3}\sqrt{1+x^2}^3       \\
    u  & := 1+x^2                          \\
    u' & = \frac{du}{dx}=2x                \\
       & \implies du = 2xdx
\end{align*}

Weiter Beispiele sind in meinem Heft unter 08.07 Substitution.

\end{document}
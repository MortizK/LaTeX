\documentclass[a4paper]{article}

%\usepackage{url}

%% Math
\usepackage{mathtools}
%% For Mengen like natural numbers
\usepackage{amsfonts}
%% Für spezielle Symbole
\usepackage{amssymb}

%% Images
\usepackage{import}
\usepackage{xifthen}
\usepackage{pdfpages}
%\usepackage{transparent}

%%% Command for simpler images
\newcommand{\incfig}[1]{%
    \def\svgwidth{\columnwidth}
    \import{./fig/}{#1.pdf_tex}
}

%% Links
\usepackage{hyperref}
\hypersetup{
    colorlinks=true,
    linkcolor=black,
    filecolor=magenta,
    urlcolor=cyan
}

%% Formatting
\usepackage{parskip}

\title{Analysis}
\author{Moritz}
\date{June 4, 2025}

\begin{document}
\maketitle
\tableofcontents

\section{Reelle Funktionen}

\subsection{Satz von BOLZANO WEIERSTRASS I \& II}

\begin{enumerate}[a)]
    \item Jede beschränkte Folge hat mindestens einen Häufungspunkt
    \item Jede beschränkte Folge hat wenigstens eine konvergente Teilfolge
\end{enumerate}

\subsubsection{Beweisidee}

\textbf{i)}

Wir haben eine Intervall $x\in[a, b]$ wo jedes $x$ einen Funktionswert $f(x)$ hat. Somit kann es keine Asymptote geben, da diese an der Grenze nicht definiert ist. Und falls sie dort definiert ist, so ist die Funktion nicht mehr Stetig.

Nun gehen wir aus, das es immer einen Höheren Funktionswert geben wird. Mit diesem Wissen bauen wir eine Indexfolge $x_{n_k}\xrightarrow{k\to\infty}x\in[a, b]$ die auf dieses "Unendliche" konvergiert.

\begin{equation*}
    n_k < f(x_{n_k})\xrightarrow{k\to\infty}f(x)
\end{equation*}

und $f$ ist stetig.

d.h. $(f(x_{n_k}))_{k\in \mathbb{N}}$ ist einerseits unbeschränkt und anderseits konvergent: W! $\implies f$ ist $[a, b]$ nach oben/unten beschränkt.

\textbf{Iii)}

Wir wissen jetzt, dass die Funktionswerte $f(x)$ für $x\in[a, b]$ nach oben beschränkt sind.

Somit gibt es ein Supremum (beste obere Schranke)

\begin{figure}[ht]
    \centering
    \incfig{supremum}
    \caption{Supremum}
    \label{fig:supremum}
\end{figure}

\begin{align*}
    0\leq M_0 - f(x_n) < \frac{1}{n} \\
    \implies \lim_{h\to\infty}f(x_n)=M_0
\end{align*}

Nun konvergiert jede Teilfolge eine konvergenten Folge auch mit demselben Grenzwert.

Falls mir was fehlt hat Rosalie mehr aufgeschrieben!

\subsubsection{NEWTON-Verfahren}

Als Näherungsverfahren für Nullstellen.

\begin{enumerate}
    \item Punkt als Startwert raten
    \item Steigung der Tangent an diesem Punkt bestimmen (Ableitung).
    \item Nullstelle der Tangete als nächsten Startpunkt wählen.
\end{enumerate}

Dies nähert sich sehr schnell einer Nullstelle (Stellenverdoppelung).

\textbf{Tangenten}

\begin{equation*}
    t(x)=f(x_0)+f'(x_0)*(x-x_0)
\end{equation*}

Verschiebung der Paraellen an die Stelle $x_0$ und dann noch $f(x_0)$ nach oben.

\begin{equation*}
    x_q=x_0-\frac{f(x_0)}{f'(x_0)}
\end{equation*}

\textbf{Normale}

\begin{equation*}
    t(x)=f(x_0)-\frac{1}{f'(x_0)}*(x-x_0)
\end{equation*}

\subsection{Zwischenwertsatz}

Für eine stetige Funktion $f$ sind alle Werte $y$ zwischen $f(a)$ und $f(b)$ Funktionswerte, also:

\begin{center}
    $f(a)<y<f(b)\implies \exists_{x\in(a, b)} f(x)=y$
\end{center}

\textbf{Beweis:} Zunächst nur für $f(a)<0, f(b)>0, y=0$

Wir benutzen auf das Intervall $(a, b)$ ein \textit{"Divide and Conquer"} (Intervallhalbierungverfahren) und haben dadurch eine Intervallschachtelung (von dieser Wissen wir, dass dieser auf einem Wert konvergiert)

\begin{equation*}
    \implies \exists!x\in [a_n, b_n]
    \quad\text{und}\quad
    \lim_{n\to\infty}a_n=x=\lim_{n\to\infty}b_n
\end{equation*}

Da $0>f(a_n)\xrightarrow{n\to\infty}f(x)\leq0$\\
und $0<f(b_n)\xrightarrow{n\to\infty}f(x)\geq0$. Somit muss $f(x)=0$ sein.

Hilfsfunktion $g(x):=f(x)-y$

$g(a)=f(a)-y<0$\\
$g(b)=f(b)-y>0$\\
$\implies x\in(a, b): g(x)=0=f(x)-y \implies f(x)=y$

$f(a)<y<f(b)$\\
$y=f(x)$

\subsection{Linksseitige/ Rechtsseitige Stetigkeit}

\textbf{Beispiel:} Funktion mit Sprungstelle. Hier kann eine der beiden Seiten Stetig sein.

$f$ ist in $x_0$ linksseitig stetig, falls für alle zulässigen Folgen $x_n < x_0$ mit Grenzwert $x_0$ gilt:

\begin{equation*}
    \lim_{n\to\infty}f(x_n)=f(x_0)
\end{equation*}

Notation für linksseitige und rechtsseitige Stetigkeit:

\begin{equation}
    \lim_{x\to x_0^-}f(x)=f(x_0)\qquad
    \lim_{x\to x_0^+}f(x)=f(x_0)
\end{equation}

\textbf{Satz:} $f$ ist in $x_0$ stetig $\iff$

\begin{equation*}
    \lim_{x\to x_0^-}f(x)=\lim_{x\to x_0^+}f(x)=f(x_0)
\end{equation*}

\end{document}
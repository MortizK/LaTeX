\documentclass[a4paper]{article}

%\usepackage{url}

%% Math
\usepackage{mathtools}
%% For Mengen like natural numbers
\usepackage{amsfonts}
%% Für spezielle Symbole
\usepackage{amssymb}

%% Images
\usepackage{import}
\usepackage{xifthen}
\usepackage{pdfpages}
%\usepackage{transparent}

%%% Command for simpler images
\newcommand{\incfig}[1]{%
    \def\svgwidth{\columnwidth}
    \import{./fig/}{#1.pdf_tex}
}

%% Links
\usepackage{hyperref}
\hypersetup{
    colorlinks=true,
    linkcolor=black,
    filecolor=magenta,
    urlcolor=cyan
}

%% Formatting
\usepackage{parskip}

\title{Analysis}
\author{Moritz}
\date{June 3, 2025}

\begin{document}
\maketitle
\tableofcontents

\section{Reelle Funktionen}

\subsection{Stetigkeit}

\textbf{Unstetigkeit} können Sprungstellen haben oder eine Funktion, die immer häufiger schwingen.

Beispiele für Unstetige Funktionen:

\begin{align*}
    f(x) & =
    \begin{cases}
        0 & , \quad x < 0   \\
        1 & , \quad x\geq 0
    \end{cases} \\
    f(x) & =
    \begin{cases}
        \sin(\frac{1}{x}) & , \quad x \neq 0 \\
        0                 & , \quad x = 0
    \end{cases}
\end{align*}


\begin{center}
    Eine Funktion ist stetig, wenn alle ihrer Punkte stetig sind.
\end{center}

Somit ist auch die Funktion $f(x)=\frac{1}{x}$, trotz ihrer Polstelle, stetig. Da der Punkt bei $x=0$ nicht definiert ist, interessiert er uns nicht.

Stetig an der Stelle $x_0$:

\begin{equation*}
    x\to x_0\implies f(x)\to f(x_0)
\end{equation*}

Und dies müssen wir für alle Folgen machen:

\begin{equation}
    \forall x_n\xrightarrow{n\to\infty}x_0: f(x_0)\xrightarrow{n\to\infty}f(x_0)
\end{equation}

\subsection{Definition}

Eine Funktion $f:D_f\to \mathbb{R}$ heißt an der Stelle $x_0\in D_f$ stetig, falls für jede zulässige Folge $(x_n)_{n\in \mathbb{N}}$, die gegen $x_0$ konvergiert gilt:

\begin{center}
    $f(x_n)\xrightarrow{n\to\infty}f(x_0)$
\end{center}

(\textit{zulässig} meint: $x_n\in D_f$)

Dafür schreiben wir

\begin{equation*}
    \lim_{x\to x_0}f(x)=f(x_0):\iff \forall_{x_n\to x_0}: \lim_{n\to\infty}f(x_n)=f(x_0)
\end{equation*}

"$x\to x_0$": $\forall x_n\to x_0$

$f$ heißt (global) stetig, falls $f$ an jeder Stelle $x_0\in D_f$ stetig ist im obigen Sinn.

\subsubsection{Beispiel}

$f(x)=\frac{1}{x}; D_f= \mathbb{R}\setminus \{0\}$ ist stetig.

$0\neq x_n \xrightarrow{n\to\infty}x_0$ muss $f(x_n)=\frac{1}{x_n}\xrightarrow{n\to\infty}\frac{1}{x_0}=f(x_0)$ gezeigt werden.

Nun sieht man, das $x_n$ schon zu $x_0$ konvergiert und nach der Quotientenregel somit auch $\frac{1}{x_n}$ zu $\frac{1}{x_0}$.

$\implies f$ ist stetig.

\subsubsection{Satz: Rechenregeln stetiger Funktionen}

$f, g$ stetig (auf denselben Def. bereich)

\begin{enumerate}[i)]
    \item $f\pm g$
    \item $f*g$
    \item $\frac{f}{g}$
    \item $f\circ g$
\end{enumerate}

sind wieder stetige Funktionen.

\textbf{Beweis:} Nur für ii) und iv). Also sei $x_0\in D_f=D_g$ und $x_n\xrightarrow{n\to\infty}x_0$ zulässig.

\begin{equation*}
    (f*g)_{(x_n)}=f(x_n)*g(x_n)\xrightarrow{n\to\infty}f(x_0)*g(x_0)=(f*g)_{(x_0)}
\end{equation*}

Für iv). Also sei $x_0\in D_g$ und $g(x_0)\in D_f$

\begin{align*}
    f\circ g(x_n)
     & =f(g(x_n))
    \xrightarrow{n\to\infty} f(g(x_0))=f\circ g(x_0) \\
    g(x_n)
     & \xrightarrow{n\to\infty}g(x_0)
    \qquad\text{g stetig}                            \\
    f(g(x_n))
     & \xrightarrow{n\to\infty}f(g(x_0))
    \quad\text{da f stetig ist}                      \\
\end{align*}

\subsection{Grenzfunktion}

\begin{align*}
    f(x) & =\sum_{k=0}^{\infty}a_kx^k \quad\text{Potenzreihe} \\
         & =\lim_{n\to\infty}\sum_{k=0}^{\infty}a_kx^k        \\
         & = p_n(x) \quad\text{Polynom n-ten Grades}
\end{align*}

$f(x)=\lim_{n\to\infty}p_n(x)$, d.h. $f$ ist die Grenzfunktion der Funktionsfolge $(p_n)_{n\in \mathbb{N}}$

$f$ ist wieder Stetig d.h. Potenzreihen sind steige Funktionen (als Grenz(wert/funktion) stetiger Funktionen)

Das ist nicht selbstverständlich, d.h. der Grenzwert stetiger Funktionen muss nicht wieder stetig sein - wie das folgende Beispiel zeigt:

\begin{equation}
    p_n(x):=x^n
\end{equation}

\begin{align*}
    p_n(x)=x^n\xrightarrow{n\to\infty}0
    \quad f.a. 0\leq x<1                \\
    p_n(1)=1^n\xrightarrow{n\to\infty}1 \\
    p_n(x)\to
    \begin{cases}
        0, \quad 0\leq x < 1 \\
        1, \quad x = 1
    \end{cases}
\end{align*}

Hier ist die Grenzfunktion nicht stetig.

\subsubsection{Satz vom Maximum}

Eine steige Funktion auf einem \textit{kompaktem Intervall} (meint ein Intervall $[a, b]$, abgeschlossen und beschränkt) nimmt dort sein Maximum und sein Minimum an.

Es gibt ein $x_1$ (Maximum) und ein $x_2$ (Minimum)

\subsubsection{Definition: Teilfolge}

Für eine Teilfolge bilden wir erst eine Indexfolge $n_1, n_2, n_3, \dots$ und setzen diese in unsere Ursprungsfolge ein.

Wir bilden also: $(x_{n_k})_{k\in \mathbb{N}}$ Teilfolge (zur Indexfolge $(n_k)_{k\in \mathbb{N}}$)

Wir lassen somit einfach nur unendlich viele Folgeelemente weg. \textit{Bildlich:} einfach streichen von Elementen in einer Tabelle.

\textbf{Beispiel:}

\begin{align*}
    (x_n)_{n\in \mathbb{N}}
     & : (x_1, x_2, x_3, \dots)      \\
    n_k
     & := 2k                         \\
    (x_{n_k})=(x_{2k})
     & : (x_2, x_4, x_6, x_8, \dots)
\end{align*}

\textbf{Definition: Indexfolge/ Teilfolge}

Eine Indexfolge $(n_k)_{k\in \mathbb{N}}$ ist eine streng monoton wachsende Folge natürlicher Zahlen.

Die zugehörige Teilfolge von $(x_n)_{n\in \mathbb{N}}$ ist $(x_{n_k})_{k\in \mathbb{N}}$

\subsection{Satz von BOLZANO WEIERSTRASS I \& II}

\begin{enumerate}[a)]
    \item Jede beschränkte Folge hat mindestens einen Häufungspunkt
    \item Jede beschränkte Folge hat wenigstens eine konvergente Teilfolge
\end{enumerate}

\textbf{Beweisidee:}

Wir haben eine beschränkte Folge und ich kann in jeder $\epsilon$-Umgebung ein $x_{n_k}$ finden, sodass bei kleiner werdenden $\epsilon$ ich immer einen größeren Index $n_k$ finde. Dies ist eine konvergente Folge die auf den Häufungspunkt konvergiert.

\end{document}
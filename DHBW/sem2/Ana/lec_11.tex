\documentclass[a4paper]{article}

%\usepackage{url}

%% Math
\usepackage{mathtools}
%% For Mengen like natural numbers
\usepackage{amsfonts}
%% Für spezielle Symbole
\usepackage{amssymb}

%% Images
\usepackage{import}
\usepackage{xifthen}
\usepackage{pdfpages}
%\usepackage{transparent}

%%% Command for simpler images
\newcommand{\incfig}[1]{%
    \def\svgwidth{\columnwidth}
    \import{./fig/}{#1.pdf_tex}
}

%% Links
\usepackage{hyperref}
\hypersetup{
    colorlinks=true,
    linkcolor=black,
    filecolor=magenta,
    urlcolor=cyan
}

%% Formatting
\usepackage{parskip}

\title{Analysis}
\author{Moritz}
\date{June 11, 2025}

\begin{document}
\maketitle
\tableofcontents

\section{Differentialrechnung}

\subsection{Mehrdimensionale Funktionen}

\begin{align*}
    f: \mathbb{R}^2\to \mathbb{R}     \\
    \frac{\partial}{\partial x}f(x,y) \\
    \frac{\partial}{\partial y}f(x,y) \\
\end{align*}

Partial Ableitung: Wir suchen eine equivalent Gleichung zum Differentialquotienten ohne Brüche.

\begin{align*}
    \frac{f(x)-f(x_0)}{x_0-x}\xrightarrow{x\to x_0}f'(x_0) \\
    \Theta(x)\xrightarrow{x\to x_0}f'(x_0)=\Theta(x_0)
\end{align*}

\begin{equation}
    \Theta(x):=
    \begin{cases}
        \frac{f(x)-f(x_0)}{x-x_0} & \quad x\neq x_0 \\
        f'(x_0                    & \quad x=x_0)
    \end{cases}
\end{equation}

Falls $f'(x_0)$ existiert, ist die Funktion $\Theta$ an der Stelle $x_0$ stetig. Es gilt für $x\neq x_0$:

\begin{align*}
    \Theta(x)=\frac{f(x)-f(x_0)}{x-x_0}\quad\text{, d.h.} \\
    f(x)-f(x_0)=\Theta(x)(x-x_0)
\end{align*}

$\Theta$ ist an der Stelle $x_0$ stetig und $\Theta(x_0)=f'(x_0)$. Umgekehrt: Falls $\Theta$ an der Stelle $x_0$ stetig ist, dass

\begin{equation}
    \frac{f(x)-f(x_0)}{x-x_0}=\Theta(x)\xrightarrow{x\to x_0}\Theta(x_0)\implies f'(x_0) \text{ ex.}
\end{equation}

und es gilt: $\Theta(x_0)=f'(x_0)$

\subsection{Kettenregeln Herleitung}

\begin{align*}
    f(y)-f(y_0)=\Theta(y)(y-y_0) \\
    f(x)-f(x_0)=\Gamma(x)(x-x_0)
\end{align*}

Gamma ist einfach nur ein andere Griechischer Buchstabe.

\begin{align*}
           & f(g(x))-f(g(x_0))             \\
    =\quad & f(y)-f(y_0)                   \\
    =\quad & \Theta(y)(y-y_0)              \\
    =\quad & \Theta(g(x))(g(x)-g(x_0))     \\
    =\quad & \Theta(g(x))*\Gamma(x)(x-x_0)
\end{align*}

$\implies f(g(x))$ ist ableitbar und die Ableitung ist:

\begin{equation}
    f'(g(x_0))*g'(x_0)
\end{equation}

\subsubsection{Lemma}

$f$ in $x_0$ differenzierbar (diffbar/ Ableitbar) $\implies f$ in $x_0$ stetig.

\subsubsection{Satz}

$f:[a, b]\to \mathbb{R}$ stetig und in $(a, b)$ differenzierbar.

\know{Für Kurvendiskussion}{
    Falls ein lokales Maximum/ Minimum an einer Stelle $x_0\in(a,b)$ existiert, dann muss
    \begin{center}
        $f'(x_0)=0$
    \end{center} sein.
}

\textbf{Beweis}: Annahme: $f'(x_0)\neq 0$, oBdA $f'(x_0)>0$ und $x>x_0$:

\begin{align*}
    f(x)-f(x_0)=\Theta(x)(x-x_0) = (>0)*(>0) >0
\end{align*}

$\implies f(x)>f(x_0)$: W! zu "$f(x_0) lokales Maxima$" Also $f'(x_0)=0$

\subsection{Ableitungen}

\begin{align*}
    \frac{d}{dx}x^n=nx^{n-1}\quad                               & ,n\in \mathbb{Z} \\
    \frac{d}{dx}x^\frac{1}{n}=\frac{1}{n}x^{\frac{1}{n}-1}\quad & ,n\in \mathbb{Q} \\
\end{align*}

Stetigkeit und Ableitungen von Sinus und Kosinus haben Rosalie und Nicklas.

\subsection{Mittelwertsatz}

nächste Woche

\end{document}
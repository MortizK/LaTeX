\documentclass[a4paper]{article}

%\usepackage{url}

%% Math
\usepackage{mathtools}
%% For Mengen like natural numbers
\usepackage{amsfonts}
%% Für spezielle Symbole
\usepackage{amssymb}

%% Images
\usepackage{import}
\usepackage{xifthen}
\usepackage{pdfpages}
%\usepackage{transparent}

%%% Command for simpler images
\newcommand{\incfig}[1]{%
    \def\svgwidth{\columnwidth}
    \import{./fig/}{#1.pdf_tex}
}

%% Links
\usepackage{hyperref}
\hypersetup{
    colorlinks=true,
    linkcolor=black,
    filecolor=magenta,
    urlcolor=cyan
}

%% Formatting
\usepackage{parskip}

\title{Analysis}
\author{Moritz}
\date{May 28, 2025}

\begin{document}
\maketitle
\tableofcontents

\section{Folgen und Reihen}

\subsection{Potenzreihen: Allgemeine Situation}

\begin{align*}
    \sum_{k=0}^{\infty}a_k*x^{pk+q} \\
    R=\frac{1}{\sqrt[p]{\lim_{k\to\infty}\sqrt[k]{|a_k|}}}
\end{align*}

\subsection{Beispiele}

\begin{align*}
     & \sum_{k=0}^{\infty}\frac{x^k}{k!}
    , \quad a_k=\frac{1}{k!}                 \\
     & \frac{a_{k+1}}{a_k}=\frac{k!}{(k+1)!}
    =\frac{1}{k+1}\xrightarrow{k\to\infty}0  \\
     & \implies R=\infty
\end{align*}

\section{Reelle Funktionen}

Dies ist die e-Funktion:

\begin{equation*}
    \sum_{k=0}^{\infty}\frac{x^k}{k!}=:exp(x)
\end{equation*}

\subsection{e-Funktion}

$exp(x)$ stimmt mit $e^x$ für alle $x\in \mathbb{Q}$ überein.

\begin{equation*}
    exp(n)=exp(1+1+\dots+1)=exp(1)*exp(1)*\dots*exp(1)=e*e*\dots*e=e^n
\end{equation*}

Was ist mit $exp(-n)$ Wir gehen über eine Gleichung die wir lösen können.

\begin{align*}
     & exp(-n)*exp(n)=exp(-n+n)=exp(0)=1                      \\
     & \implies exp(-n)=\frac{1}{exp(n)}=\frac{1}{e^n}=e^{-n}
\end{align*}

Was ist mit $exp(\frac{1}{n})$?

\begin{align*}
     & (exp(\frac{1}{n}))^n
    = exp(\frac{1}{n})*exp(\frac{1}{n})*\dots*exp(\frac{1}{n}) \\
     & = exp(       \frac{1}{n}+frac{1}{n}+\dots+\frac{1}{n})
    = exp(1) = e                                               \\
     & \implies exp(\frac{1}{n})=\sqrt[n]{e}=e^{\frac{1}{n}}
\end{align*}

Und nun der allgemeine Fall $exp(\frac{m}{n})$

\begin{align*}
     & exp(\frac{m}{n})
    = exp(\frac{1}{n}_1+\frac{1}{n}_2+\dots+\frac{1}{n}_m)              \\
     & = exp(\frac{1}{n}_1)*exp(\frac{1}{n}_2)*\dots*exp(\frac{1}{n}_m)
    = (e^{\frac{1}{n}})^m = e^{\frac{m}{n}}                             \\
     & \implies exp(x)=e^x
\end{align*}

für alle $x\in \mathbb{Q}$ nachgewiesen. Definiere also:

\begin{equation}
    e^x:=exp(x), \quad\text{für alle }x\in \mathbb{C}
\end{equation}

\subsection{Ableitung}

\begin{align*}
    e^x=\sum_{k=0}^{\infty}\frac{x^k}{k!}
     & =1 + x + \frac{x^2}{2} + \frac{x^3}{3!} + \frac{x^4}{4!} + \frac{x^5}{5!} + \dots \\
    \frac{d}{dx}e^x
     & = 0 + 1 + x + \frac{3x^2}{3*2!} + \frac{4x^3}{4*3!} + \frac{5x^4}{5*4!} + \dots   \\
     & = \sum_{k=0}^{\infty}\frac{x^k}{k!} = e^x
\end{align*}

Diese Reihe darf man nicht abbrechen, da diese sich erst im unendlichem reproduziert.

\subsection{Definition}

Reelle Funktionen sind Abbildungen $f$ die eine Menge $D_f\subseteq \mathbb{R}$ nach $\mathbb{R}$ abbildet:

\begin{center}
    $f: D_f \to \mathbb{R}, x \mapsto f(x)$
\end{center}

Dabei heißt $D_f$ \textit{Definitionsbereich} von $f$.

\textbf{Beispiel} $f:\mathbb{R}\to \mathbb{R}, x\mapsto x^2 (=f(x))$

Wenn wir die Funktion $f$ in einem Graph Darstellen, heißt das:

\begin{center}
    "Graph von $f$: $G_f$"

    $G_f:=\left\{[x, f(x)]\in \mathbb{R}^2|x\in D_f\right\}\subseteq \mathbb{R}^2$
\end{center}

\textbf{Beispiel} $g:\mathbb{R}_0^+\to \mathbb{R}, x\mapsto x^2$

Wir können die Funktion auch einschränken! Wie hier bei $g$

\subsubsection{Umkehrfunktion}

Die Umkehrfunktion $g^{-1}$ von $g$ ausrechnen heißt, die Gleichung:

\begin{center}
    $g(x)=y$
\end{center}

nach $x$ auflösen: $x = g^{-1}(y)$

Am Beispiel: $g(x)=x^2=y \implies x=\sqrt{y}$ hier nur positiv, da der Uhrbereich von $g$ nur die positiven Zahlen beinhaltet.

\know{Umkehren vom Sinus}{Es gibt keine Umkehrfunktion vom normalem Sinus, da es für jeden y-Wert von $[-1, 1]$, $\infty$-viele x-Werte gibt. Lösung:
    \begin{center}
        Wir schränken Sinus auf $[-\frac{\pi}{2}, \frac{\pi}{2}]$ ein.
    \end{center}}

\subsubsection{Verketten}

\textbf{Beispiel} $f\circ g = f(g(x))$

\textbf{Beispiel} $g^{-1}\circ g = g^{-1}(g(x)) = x = id(x)$

\subsection{Rechenoperationen für Funktionen}

Für Operationen sollte die Definitionsbereich $(D_f)$ gleich sein.

$f: D_f\to \mathbb{R}, x\mapsto f(x)$, und $g:D_f\to \mathbb{R}, x\mapsto g(x)$

\begin{enumerate}
    \item $(f \pm g)_{(x)} := f(x) \pm g(x)$
    \item $(f * g)_{(x)} := f(x) * g(x)$
    \item $\left(\frac{f}{g}\right)_{(x)} := \frac{f(x)}{g(x)}$ OHNE $D_f$, wo $g(x)=0$
    \item $f: D_f\to \mathbb{R}; g: D_g\to D_f\subseteq \mathbb{R}$

          $f\circ g(x):=f(g(x)); f\circ g: D_g\to \mathbb{R}$
\end{enumerate}

\textbf{Beispiel von 4.}: $f(x)= \sqrt{x}, g(x)=x^2-9$. Somit ist $f\circ g(x)=\sqrt{x^2-9}$

\know{Ableitung}{Hierfür ist die Kettenregel für verkettete Funktionen wichtig.}



\end{document}
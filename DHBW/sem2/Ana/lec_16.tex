\documentclass[a4paper]{article}

%\usepackage{url}

%% Math
\usepackage{mathtools}
%% For Mengen like natural numbers
\usepackage{amsfonts}
%% Für spezielle Symbole
\usepackage{amssymb}

%% Images
\usepackage{import}
\usepackage{xifthen}
\usepackage{pdfpages}
%\usepackage{transparent}

%%% Command for simpler images
\newcommand{\incfig}[1]{%
    \def\svgwidth{\columnwidth}
    \import{./fig/}{#1.pdf_tex}
}

%% Links
\usepackage{hyperref}
\hypersetup{
    colorlinks=true,
    linkcolor=black,
    filecolor=magenta,
    urlcolor=cyan
}

%% Formatting
\usepackage{parskip}

\title{Analysis}
\author{Moritz}
\date{July 1, 2025}

\begin{document}
\maketitle
\tableofcontents

\section{Differentialrechnung}

\subsection{Übungen}

\begin{align*}
      & \lim_{x\to0^+}x\ln(1-e^{-x})                                 \\
    = & \lim_{x\to0^+} \frac{\ln(1-e^{-x})}{\frac{1}{x}}             \\
    = & \lim_{x\to0^+}\frac{\frac{e^{-x}}{1-e^{-x}}}{-\frac{1}{x^2}} \\
    = & \lim_{x\to0^+} e^{-x}\frac{-x^2}{1-e^{-x}}                   \\
    = & \lim_{x\to0^+} e^{-x} \lim_{x\to0^+} \frac{-x^2}{1-e^{-x}}   \\
    = & \lim_{x\to0^+} \frac{-2x}{e^{-x}} = 0
\end{align*}

Wegen der Stetigkeit der $e$-Funktion und der Rechenregeln Konvergenten Folgen konvergiert diese Folge. WICHTIG für einen Punkt!

Berechnung der TAYLOR-Reihe

\begin{align*}
    x\mapsto \frac{1}{1-x^2}+\frac{1}{1+x^2}
     & = \sum_{k=0}^{\infty}x^{2k} + \sum_{k=0}^{\infty}(-x^2)^k       \\
     & = \sum_{k=0}^{\infty} x^{2k} + \sum_{k=0}^{\infty}(-1)^k x^{2k}
    = 2\sum_{k=0}^{\infty}x^{4k}                                       \\
    \frac{1}{1-q}
     & =\sum_{k=0}^{\infty}q^k,\quad\text{Wissen wir}
\end{align*}

Weitere in meinem Heft unter Ana 1.7. Übungen

\subsection{TAYLOR-Entwicklung}

Tabelle mehrere Entwicklungen im Heft unter Ana 1.7 Übungen

\section{Integralrechnung}

Obersummer ermitteln:

\begin{equation*}
    (x_{k-1}-x_k)*f(x_k)=f(x_k)*\Delta x_k
\end{equation*}

Dieser Flächeninhalt $\approx$

\begin{equation*}
    \sum_{k=0}^{n}f(k)*\Delta x_k
\end{equation*}

\begin{equation*}
    \xrightarrow{"n\to\infty"}\int_{(x=)a}^{(x=)b}f(x)*dx
\end{equation*}

Wir können Integrale auch benutzen, um die Länge einer Funktion zu ermitteln.

\begin{align*}
        & \sum_{k=0}^{n} \sqrt{\Delta f^2(x_k) \Delta x^2_k}                                 \\
        & \sum_{k=0}^{n} \sqrt{\left(\frac{\Delta f(x_k)}{\Delta x_k}\right)^2+1}*\Delta x_k \\
        & \sum_{k=0}^{n} \sqrt{f'^{2}(x_k)+1}*\Delta x_k                                     \\
    \to & \int_{a}^{b} \sqrt{f'^{2}(x_k)+1} dx
\end{align*}

\subsection{Alternative zur Methode der RIEMANN-Summen}

Idee: Flächenfunktion (Eine festes $c$ und eine variable $x$, welche die Fläche zwischen $c$ und $x$ zurückgibt)

Falls wir $F(x)$ kennen, ist das Flächenproblem gelöst:

\begin{equation*}
    \int_{a}^{b}f(x)dx = F(b)-F(a) =: [F(x)]^b_a
\end{equation*}

Ich möchte die Funktion $F(x)$ kennenlernen. Idee: $F'(x)$

\begin{equation*}
    \frac{F(x+h)-F(x)}{h}\approx \frac{f(x)*h}{h}=f(x)
\end{equation*}

Über eine Grafik mit ausrechnen des Flächeninhaltes des Streifens zwischen $x+h$ und $x$, welcher $F(x+h)-F(x)=h*f(x)$

\begin{equation*}
    \implies F'(x)=f(x)
\end{equation*}

\end{document}
\documentclass[a4paper]{article}

%\usepackage{url}

%% Math
\usepackage{mathtools}
%% For Mengen like natural numbers
\usepackage{amsfonts}
%% Für spezielle Symbole
\usepackage{amssymb}

%% Images
\usepackage{import}
\usepackage{xifthen}
\usepackage{pdfpages}
%\usepackage{transparent}

%%% Command for simpler images
\newcommand{\incfig}[1]{%
    \def\svgwidth{\columnwidth}
    \import{./fig/}{#1.pdf_tex}
}

%% Links
\usepackage{hyperref}
\hypersetup{
    colorlinks=true,
    linkcolor=black,
    filecolor=magenta,
    urlcolor=cyan
}

%% Formatting
\usepackage{parskip}

\title{Analysis}
\author{Moritz}
\date{June 18, 2025}

\begin{document}
\maketitle
\tableofcontents

\section{Differentialrechnung}

\subsection{Umkehrfunktionen \& ihre Ableitung}

Bedingung: Funktion muss \textit{injektiv} sein, also jedem x nur ein y und nicht mehrere x das gleiche y.

\begin{align*}
    f(x)=f(\bar{x}) & \implies x=\bar{x}           \\
    x\neq \bar{x}   & \implies f(x)\neq f(\bar{x})
\end{align*}

Und: Umkehrbar nur im Wertebereich (Alle Werte die y annehmen kann)

\begin{equation*}
    W_f=\{y\in Y\mid \exists_{x\in X}y=f(x)\}
\end{equation*}

\begin{align}
    f:      & D_f\to W_f \\
    f^{-1}: & W_f\to D_f
\end{align}

\subsubsection{Beispiele}

\begin{enumerate}
    \item $f(x)=x^2=y\geq 0$ nach $x$ auflösen:
          \begin{equation*}
              x=\pm\sqrt{y}
          \end{equation*}
    \item $g(x)=x^2, D_f\mathbb{R}^+_0, W_f=\mathbb{R}^+_0$
          \begin{equation*}
              y=g(x)=x^2\implies x = \sqrt{y}
          \end{equation*}
          Da der wir uns durch den Definitionsbereich nur im Oberen Rechtem Quadranten befinden.
    \item $sin(x), D_{sin}=[-1, 1], W_{sind}=[-\frac{\pi}{2}, \frac{\pi}{2}]$
\end{enumerate}

\textbf{Satz}: $f$ auf $[a, b]$ stetig und injektiv. Dann ist $f^{-1}$ ebenfalls stetig.

\begin{center}
    Falls $f$ auch diffbar ist, dann auch $f^{-1}$
\end{center}

\textbf{Satz}: $f$ auf $[a, b]$ stetig. Dann ist äquivalent:

\begin{center}
    $f$ umkehrbar und $f$ \textit{entweder} streng monoton wachsend\\
    \textit{oder} streng monoton fallend.
\end{center}

Hinreichend: entweder $f'(x)\geq 0$ oder $f'(x)\leq 0$ f.a. $x\in(a, b)$

\subsubsection{e-Funktion}

$exp(x)=e^x$. Nun um den Wertebereich zu bestimmen, leiten wir die Funktion ab. $exp'(x)=e^x>0$. Somit:

\begin{align*}
    W_{exp} & =\mathbb{R}^+ \\
    D_{exp} & =\mathbb{R}
\end{align*}

Die Umkehrfunktion von $e^x$ ist $\ln(x)$

\begin{align*}
    exp(x)=e^x & =f(x)      \\
    ln(x)      & =f^{-1}(x)
\end{align*}

\begin{align*}
    f^{-1}(y)    & =x \\
    f^{-1}(f(x)) & =x \\
    f(x)         & =y \\
    f(f^{-1}(y)) & =y \\
\end{align*}

\subsubsection{Ableitungen}

\begin{align*}
    y & =e^{ln(y)}\quad\mid \frac{d}{dy} \\
    1 & =e^{ln(y)}*ln'(y)                \\
    \frac{1}{y}=\frac{1}{e^{ln(y)}}
      & =ln(y)
\end{align*}

\begin{center}
    \begin{tabular}{c|c}
        $f$                       & $f'$              \\
        \hline
        $\ln(x)$                  & $\frac{1}{x}$     \\
        $e^{ax+b}$                & $ae^{ax+b}$       \\
        $\frac{1}{\sqrt{1-y^2}}$  & $\sin^{-1'}(y)$   \\
        $\frac{-1}{\sqrt{1-y^2}}$ & $\cos^{-1'}(y)$   \\
        $arctan(y)$               & $\frac{1}{1+y^2}$
    \end{tabular}
    \qquad
    \begin{tabular}{c|c}
        $f$               & $F$                       \\
        \hline
        $\frac{1}{x}$     & $\ln(|x|)$                \\
        $e^{ax+b}$        & $\frac{1}{a}e^{ax+b}$     \\
        $\sin^{-1'}(y)$   & $\frac{1}{\sqrt{1-y^2}}$  \\
        $\cos^{-1'}(y)$   & $\frac{-1}{\sqrt{1-y^2}}$ \\
        $\frac{1}{1+y^2}$ & $arctan(y)$
    \end{tabular}
\end{center}

\subsubsection{ln-Rechengesetze}

\begin{equation}
    e^{\ln(x*y)}=x*y=e^{\ln(x)}*e^{\ln(y)}=e^{\ln(x)+\ln(y)}
\end{equation}

\begin{enumerate}
    \item $\ln(x*y)=\ln(x)+\ln(y)$
    \item $\ln(\frac{x}{y})=\ln(x)-\ln(y)$
    \item $\ln(x^y)=y*\ln(x)$
    \item $\ln(e)=1$
\end{enumerate}

Herleitung von $\ln$ im Intervall $(-1,1)$ mithilfe der Leibnitz Reihen.

\begin{align*}
    \frac{d}{dx}\ln(1+x)
    =\frac{1}{1+x}
    =\frac{1}{1-(-x)}
     & =\sum_{k=0}^{\infty}(-1)^k x^k
    \qquad\mid\text{Aufl.}                             \\
    \ln(1+x)
     & =\sum_{k=0}^{\infty}(-1)^k\frac{x^{k+1}}{k+1}+c
\end{align*}

\begin{align*}
    x=1\quad
     & \ln(2)=\sum_{k=0}^{\infty}(-1)^k\frac{1}{k+1}            \\
    x=-\frac{1}{2}\quad
     & \ln(2)=\sum_{k=0}^{\infty}\frac{1}{2^{k+1}}\frac{1}{k+1}
\end{align*}

\subsection{Ableitungen von Umkehrfunktion}

\begin{equation}
    f^{-1'}(y)=\frac{1}{f'(f^{-1}(y))}
\end{equation}

Am Beispiel von $\sin^{-1}(x)$

\begin{align*}
    \sin^{-1'}(y)
     & =\frac{1}{\cos(\sin^{-1}(y))}              \\
     & =\frac{1}{\sqrt{1-\sin^2(\sin^{-1}(y))}}   \\
     & =\frac{1}{\sqrt{1-(\sin(\sin^{-1}(y)))^2}} \\
     & =\frac{1}{\sqrt{1-y^2}}                    \\
\end{align*}

\end{document}
\documentclass[a4paper]{article}

%\usepackage{url}

%% Math
\usepackage{mathtools}
%% For Mengen like natural numbers
\usepackage{amsfonts}
%% Für spezielle Symbole
\usepackage{amssymb}

%% Images
\usepackage{import}
\usepackage{xifthen}
\usepackage{pdfpages}
%\usepackage{transparent}

%%% Command for simpler images
\newcommand{\incfig}[1]{%
    \def\svgwidth{\columnwidth}
    \import{./fig/}{#1.pdf_tex}
}

%% Links
\usepackage{hyperref}
\hypersetup{
    colorlinks=true,
    linkcolor=black,
    filecolor=magenta,
    urlcolor=cyan
}

%% Formatting
\usepackage{parskip}

\title{Web Engineering}
\author{Moritz}
\date{May 6, 2025}

\begin{document}
\maketitle
\tableofcontents

\section{Konzepte}

\begin{enumerate}
    \item TCP / UDP - Datenübertragungsprotokolle
    \item DNS - Zuordnung von Name zu IP-Adresse
    \item URLs - Eindeutiger Zeiger auf Resource eines Servers
    \item HTTP - Request und Respond als Header und Body
    \item HTTPS - Sichere Kommunikation über HTTP über Zertifikate
    \item Proxy Server - Zwischenelement um selbst nicht sichtbar zu sein
\end{enumerate}

\subsection{HTTP}

\subsubsection{Methoden}

\begin{enumerate}
    \item GET - Anfrage
    \item POST - Erstellen
    \item PUT - Update
    \item HEAD - Metainformation
    \item DELETE - Löschen
    \item OPTIONS - Welche Kommunikation ist möglich
\end{enumerate}

\subsubsection{Codes}

\begin{enumerate}
    \item 200 OK
    \item 201 Created
    \item 301 Moved Permanently
    \item 400 Bad Request
    \item 401 Unauthorized
    \item 404 Not Found
    \item 500  Internal Server Error
    \item 503 Service Unavailable
\end{enumerate}

\subsection{HTML}

Begriffe: XML, XHTML, HTML

XML: Strukturierte Daten. Lesbar für Mensch und Maschine

XHTML: Wird nicht mehr weiterentwickelt

HTML: Darstellungsbeschreibung von Webseiten

\end{document}
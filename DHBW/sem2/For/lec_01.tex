\documentclass[a4paper]{article}

%\usepackage{url}

%% Math
\usepackage{mathtools}
%% For Mengen like natural numbers
\usepackage{amsfonts}
%% Für spezielle Symbole
\usepackage{amssymb}

%% Images
\usepackage{import}
\usepackage{xifthen}
\usepackage{pdfpages}
%\usepackage{transparent}

%%% Command for simpler images
\newcommand{\incfig}[1]{%
    \def\svgwidth{\columnwidth}
    \import{./fig/}{#1.pdf_tex}
}

%% Links
\usepackage{hyperref}
\hypersetup{
    colorlinks=true,
    linkcolor=black,
    filecolor=magenta,
    urlcolor=cyan
}

%% Formatting
\usepackage{parskip}

\title{Forschungsprojekt}
\author{Moritz}
\date{May 5, 2025}

\begin{document}
\maketitle
\tableofcontents

\section{Versuche}

Wir müssen sicherstellen, das wir etwas beeinflussen können (Placebo oder Medikament) und die anderen Störfaktoren rauszuschmeißen.

\subsection{Kontrolltechniken}

between-subjects: Zwei Gruppen dessen Unterschiede untersucht werden.

within-subject: Eine Gruppe die beide Szenarien durchläuft

\subsection{Experiment}

Hypothese: Studierende die einen Mathe-Vorkurs besuchen sind erfolgreicher in den MAthe-Prüfungen des 1. Studienjahres

Unabhängige Variable: Vorkursbesuch ja/nein

Abhängige Variable: Prüfungsergebnis

Störvariablen: Dozenten und deren Prüfungen, Wissenstand

Dieser Versuch ist unethisch, da ich kontrollieren müsste wer an dem Kurs teilnehmen muss oder nicht darf.

\subsubsection{Randomisierung}

Um die Gruppen einzuteilen kann eine Blockrandomisierung genutzt werden.

\subsubsection{Parallelisieren}

Bestimmen einer Störvariablen, welche ich gleichmäßig verteilen möchte (Balancieren)

\end{document}
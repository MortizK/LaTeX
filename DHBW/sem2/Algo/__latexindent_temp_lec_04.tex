\documentclass[a4paper]{article}

%\usepackage{url}

%% Math
\usepackage{mathtools}
%% For Mengen like natural numbers
\usepackage{amsfonts}
%% Für spezielle Symbole
\usepackage{amssymb}

%% Images
\usepackage{import}
\usepackage{xifthen}
\usepackage{pdfpages}
%\usepackage{transparent}

%%% Command for simpler images
\newcommand{\incfig}[1]{%
    \def\svgwidth{\columnwidth}
    \import{./fig/}{#1.pdf_tex}
}

%% Links
\usepackage{hyperref}
\hypersetup{
    colorlinks=true,
    linkcolor=black,
    filecolor=magenta,
    urlcolor=cyan
}

%% Formatting
\usepackage{parskip}

\title{Algorithmen}
\author{Moritz}
\date{May 14, 2025}

\begin{document}
\maketitle
\tableofcontents

\section{Komplexität}

\subsection{Rechenregeln}

Grenzwert. g(n) liegt unter f(n), also $g(n)<f(n)$

\begin{equation*}
    \lim_{n\to\infty}\frac{g(n)}{f(n)}\in \mathbb{R}\implies g\in \mathcal{O}(f)
\end{equation*}

Beispiel $f(n)=5n, g(n)=3n+2$

\begin{equation*}
    \lim_{n\to\infty}\frac{3n+2}{5n}~\lim_{n\to\infty}\frac{3n}{5n}=\frac{3}{5}\in \mathbb{R}
\end{equation*}

\subsection{Regel von l'Hôpital}

\begin{equation*}
    \lim_{x\to\infty}\frac{f(x)}{g(x)}=
    \lim_{x\to\infty}\frac{f'(x)}{g'(x)}
\end{equation*}

Das Verhältnis der Steigungen ist identisch mit dem Eigentlichem Verhältnis der Funktionswerten.

Somit lassen sich Funktionen die entweder $0$ oder $\infty$ sind lösen.

\end{document}
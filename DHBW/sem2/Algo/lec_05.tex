\documentclass[a4paper]{article}

%\usepackage{url}

%% Math
\usepackage{mathtools}
%% For Mengen like natural numbers
\usepackage{amsfonts}
%% Für spezielle Symbole
\usepackage{amssymb}

%% Images
\usepackage{import}
\usepackage{xifthen}
\usepackage{pdfpages}
%\usepackage{transparent}

%%% Command for simpler images
\newcommand{\incfig}[1]{%
    \def\svgwidth{\columnwidth}
    \import{./fig/}{#1.pdf_tex}
}

%% Links
\usepackage{hyperref}
\hypersetup{
    colorlinks=true,
    linkcolor=black,
    filecolor=magenta,
    urlcolor=cyan
}

%% Formatting
\usepackage{parskip}

\title{Algorithmen}
\author{Moritz}
\date{May 28, 2025}

\begin{document}
\maketitle
\tableofcontents

\section{Komplexität}

\subsection{Master Theorem}

Wurde behandelt und ist für \href{https://de.wikipedia.org/wiki/Karazuba-Algorithmus}{Karazuba Multi.} in meinem Heft.

\href{https://de.wikipedia.org/wiki/Master-Theorem}{Wikipedia zu Master Theorem} kommt in die Klausur

\subsubsection{Vorlage}

\begin{align*}
    a < b^d & \implies O(n^d)           \\
    a = b^d & \implies O(\log_b(n)*n^d) \\
    a > b^d & \implies O(n^{\log_b(a)})
\end{align*}

\subsubsection{Beispiele}

\begin{align*}
    f(n)    & = 4 * f(\frac{n}{2}) + n                    \\
    a       & = 4, \quad b = 2, \quad d = 1               \\
    4 > 2^1 & \implies f(n) \in O(n^{\log_2(4)}) = O(n^2) \\\\
    f(n)    & = 4 * f(\frac{n}{2}) + n^2                  \\
    a       & = 4, \quad b = 2, \quad d = 2               \\
    4 = 2^2 & \implies f(n) \in O(n^2\log_2(n))
\end{align*}

\subsection{Limitierung}

\begin{equation*}
    T(n) = a * T\left(\frac{n}{b}\right) + n^d
\end{equation*}

\begin{enumerate}
    \item $b$ oder $a$ sind keine Konstanten
    \item $T(n)$ ist kein Polynom
    \item $T(n)$ is periodisch, wie $\cos()$ oder $\sin()$
\end{enumerate}

\end{document}
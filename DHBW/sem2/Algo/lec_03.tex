\documentclass[a4paper]{article}

%\usepackage{url}

%% Math
\usepackage{mathtools}
%% For Mengen like natural numbers
\usepackage{amsfonts}
%% Für spezielle Symbole
\usepackage{amssymb}

%% Images
\usepackage{import}
\usepackage{xifthen}
\usepackage{pdfpages}
%\usepackage{transparent}

%%% Command for simpler images
\newcommand{\incfig}[1]{%
    \def\svgwidth{\columnwidth}
    \import{./fig/}{#1.pdf_tex}
}

%% Links
\usepackage{hyperref}
\hypersetup{
    colorlinks=true,
    linkcolor=black,
    filecolor=magenta,
    urlcolor=cyan
}

%% Formatting
\usepackage{parskip}

\title{Algorithmen}
\author{Moritz}
\date{May 12, 2025}

\begin{document}
\maketitle
\tableofcontents

\section{Komplexität}

O(f) bezeichnet alle Funktionen, welche ab k kleiner oder gleich c*f(n) sind. Das heißt: Alle Funktionen wachsen langfristig langsamer als f(n).

O(n³) ist eine Menge von mehreren Funktionen, wie 3n³ oder 4n³+2n²-n, aber auch n² ist ein Teil von O(n³)

\subsection{Rechenregeln}

\begin{enumerate}
    \item Konstanter Faktor
    \item Summe
    \item Transitivität
    \item Grenzwert
\end{enumerate}

\end{document}
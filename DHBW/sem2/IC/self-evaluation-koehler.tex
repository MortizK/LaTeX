\documentclass[a4paper]{article}

%\usepackage{url}

%% Math
\usepackage{mathtools}
%% For Mengen like natural numbers
\usepackage{amsfonts}
%% Für spezielle Symbole
\usepackage{amssymb}

%% Images
\usepackage{import}
\usepackage{xifthen}
\usepackage{pdfpages}
%\usepackage{transparent}

%%% Command for simpler images
\newcommand{\incfig}[1]{%
    \def\svgwidth{\columnwidth}
    \import{./fig/}{#1.pdf_tex}
}

%% Links
\usepackage{hyperref}
\hypersetup{
    colorlinks=true,
    linkcolor=black,
    filecolor=magenta,
    urlcolor=cyan
}

%% Formatting
\usepackage{parskip}

\title{Intercultural Communication}
\author{Moritz Köhler, 1645427}
\date{June 21, 2025}

\begin{document}
\maketitle

MatNr: 1645427

Group: 2 Orange

Course Title: \href{https://www.coursera.org/learn/active-listening-enhancing-communication-skills}{Active Listening} - Enhancing Communication Skills

I choose the STAR-Method, because it is goal orientated. From the starting point to what needed to be done and reflection of the results.

\section{Situation}

My Group Members where Rosalie Fischer and Leon Salenbacher. We worked on \textit{"Active Listening"} which included Emotional Intelligence, Nonverbal Communication and Cross-Cultural Considerations.

Our Task was to develop a 30-45min teaching session for our class. We meet up in a teams group, shared documents and created slides, a Kahoot quiz and a case discussion.

\section{Task}

I was responsible to create a summary of the Coursera course, which was our basis for our teaching lesson. In the meantime my group members started with the main structure of our teaching session.

When we got together in the team group, we brainstormed Ideas and agreed on our structure. Then everybody worked on the things that needed to be done.

While creating the PowerPoint, I was mainly responsible to implement my summary into it. This was done by updating the slide and adding useful text into the presenters notes.

Everybody gathered questions for our Kahoot. 

\section{Action}

To complete the summary I listed carefully to the course and made notes in \LaTeX. On some topics I paused to understand the displayed graphs and representations.

\section{Result}

Our slightly longer than 45 min class session was a success. We had a nice dialog with our classmates which helped with understanding and discussing the topic.

Next time we should better think through our time management. But this time we gathered expirience with what time is needed for breakout rooms.

\end{document}
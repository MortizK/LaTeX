\documentclass[a4paper]{article}

%\usepackage{url}

%% Math
\usepackage{mathtools}
%% For Mengen like natural numbers
\usepackage{amsfonts}
%% Für spezielle Symbole
\usepackage{amssymb}

%% Images
\usepackage{import}
\usepackage{xifthen}
\usepackage{pdfpages}
%\usepackage{transparent}

%%% Command for simpler images
\newcommand{\incfig}[1]{%
    \def\svgwidth{\columnwidth}
    \import{./fig/}{#1.pdf_tex}
}

%% Links
\usepackage{hyperref}
\hypersetup{
    colorlinks=true,
    linkcolor=black,
    filecolor=magenta,
    urlcolor=cyan
}

%% Formatting
\usepackage{parskip}

\title{Intercultural Communication}
\author{Moritz}
\date{May 16, 2025}

\begin{document}
\maketitle
\tableofcontents

\section{Active Listening}

Coursera: \href{https://www.coursera.org/learn/active-listening-enhancing-communication-skills}{Active Listening} - Enhancing Communication Skills

\subsection{Foundational Principles}

\textbf{Active listening} involves focused attention, comprehension of emotions and intentions, and thoughtful responses, while basic listening is a surface-level activity with passive reception of words.

\textbf{Empathetic Listening} creates Trust and strengthens connections.

Don't jump to conclusions based on personal biases and give space and time to let each other share their emotions.

When is it really important:

\begin{enumerate}
    \item Conflict Resolution
    \item Feedback and Performance Reviews
    \item Customer Relations
    \item Team Collaboration
    \item Change Management
\end{enumerate}

\textbf{Silence and Pauses} create a space for thinking and nurturing new Ideas, and to reflect the shared information.

Pauses can also be used, so that the other group can regroup and discuss internally.

\textbf{Open-Ended Questions} allow for emotional exploration and in depth insights

Simpel Yes and No Questions are quick, but don't encourage a deeper communication.

So Open-Ended Questions encourage a longer talk in a two-way dialog.

\textbf{Paraphrasing, Summarizing and Perspectives}

Para: in your own words - to clear up misunderstanding

Summ: the broader picture - clear up the main points

Using these can result i a still and unnatural dialog. Use when in need of clarity.

\subsection{Emotional Intelligence}

\textbf{Emotional Awareness}

Manage your own Emotional Reaction like defensiv responses or snap judgments.

And look at our dialog partner and adjust accordingly.

\know{Building Rapport}{Refers to the harmonious connection and understanding between individuals.}

Techniques to foster Trust \& Rapport

\begin{enumerate}
    \item Active Listening
    \item Empathy
    \item Nonverbal Cues
    \item Authenticity
    \item Consistency
\end{enumerate}

Strike a balance between respect and hierarchy (in a business)

\textbf{Self Awareness}

\begin{enumerate}
    \item Emotional Neutrality
    \item Empathic Engagement
    \item Objective Assessment
    \item Empowered Negotiation
    \item Reduced Bias
    \item Clear Thinking
\end{enumerate}

\textbf{Non-Judgmental Listening}

Don't jump to conclusions and reflect the responses.

Don't use stereotypes and don't judge.

\subsection{Nonverbal Communication}

\textbf{Body Language}

Gestures, Postures, Facial Expressions, Mirroring, Eye Contact, Nodding, Leaning Forward, Open Palms, \dots

\textbf{Eye Contact}

The Meaning of Eye Contact is different in each cultures.

\textbf{Cultural Nuances}

They can have different comfort zone (personal space) and different perceptions of punctuality (Respect vs. to rigid)

\subsection{Challenging Situations}

\textbf{Conflict Resolution}

\begin{enumerate}
    \item Empathetic Validation (Acknowledge Emotions)
    \item Clarifying Misunderstandings
    \item Paraphrasing for Agreement (Viewpoints)
    \item Defusing Emotional Tension
    \item Brainstorming Solutions (After Understanding)
\end{enumerate}

\textbf{Maintaining Composure} and help others regain their Composure using active Listening.

\textbf{Reflecting Listening} \dots

\subsection{Cross-Cultural Consideration}

\textbf{Norms and Values} can be different in each culture. So those need be accounted for in active listening.

\textbf{Global Audience}

You need to adapt to create a respecting environment.

Change your perspective to understand the other participants to create an inclusive environment.

\textbf{Takeaways}

Different Names for the Different Content Headings.

\begin{enumerate}
    \item Active Listening as a Foundation
    \item Empathy and Emotional Intelligence
    \item The Silent Language
    \item Active Listening in Challenging Moments
    \item Embracing Diversity
\end{enumerate}

\begin{center}
    \textit{"Listen to Understand"}
\end{center}

\section{Teaching Lesson}

\subsection{What is possible/ needed?}

30min - 45min

This session should reflect a flipped-classroom approach, where the group becomes the instructor team. The format is flexible and may include a short summary of the course content followed by a deep dive into a specific aspect, or a structured walkthrough of key learnings. Groups are encouraged to use creative methods—slides, discussion prompts, video excerpts, live polls, Kahoots, breakout room activities, etc.—to actively engage their peers and maximize learning outcomes

\textbf{Each group member} designs one realistic, open-ended case study that highlights intercultural or negotiation challenges related to their topic. These cases will be used during the teaching session to stimulate discussion and apply theoretical knowledge to practical contexts.
Example: “You are part of a virtual team with different cultural backgrounds. A miscommunication has occurred. How do you resolve it using active listening?”

\subsection{Our Lesson}

Breakout Rooms



Kahoot

\section{Self Evaluation}

Moritz Köhler, 1645427

Group 2 Orange

Coursera: \href{https://www.coursera.org/learn/active-listening-enhancing-communication-skills}{Active Listening} - Enhancing Communication Skills

I choose the STAR-Method, because it is goal orientated. From the starting point to what needed to be done and reflection of the results.

\subsection{Situation}

My Group Members where Rosalie Fische and Leon Salenbacher. We worked on \textit{"Active Listening"} which included Emotional Intelligence, Nonverbal Communication and Cross-Cultural Considerations.

Our Task was to develop a 30-45min teaching session for our class. We meet up in a teams group, shared documents and created slides, a Kahoot quiz and a case discussion.

\subsection{Task}

I was responsible to create a summary of the Coursera course, which was our basis for our teaching lesson. In the meantime my group members started with the main structure of our teaching session.

When we got together in the team group, we brainstormed Ideas and agreed on our structure. Then everybody worked on the things that needed to be done.

\subsection{Action}

To complete the summary I listed carefully to the course and made notes in \LaTeX. On some topics I paused to understand the displayed graphs and representations.

\subsection{Result}

\dots

\end{document}
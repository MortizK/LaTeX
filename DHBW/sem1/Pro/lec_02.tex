\documentclass[a4paper]{article}

%\usepackage{url}

%% Math
\usepackage{mathtools}
%% For Mengen like natural numbers
\usepackage{amsfonts}
%% Für spezielle Symbole
\usepackage{amssymb}

%% Images
\usepackage{import}
\usepackage{xifthen}
\usepackage{pdfpages}
%\usepackage{transparent}

%%% Command for simpler images
\newcommand{\incfig}[1]{%
    \def\svgwidth{\columnwidth}
    \import{./fig/}{#1.pdf_tex}
}

%% Links
\usepackage{hyperref}
\hypersetup{
    colorlinks=true,
    linkcolor=black,
    filecolor=magenta,
    urlcolor=cyan
}

%% Formatting
\usepackage{parskip}

\title{Programmieren}
\author{Moritz}
\date{January 16, 2025}

\begin{document}
\maketitle
\tableofcontents

\section{Übungsaufgaben}

Loader: lädt ein Programm und deren Abhängigkeiten in den Arbeitsspeicher und bereit sie für die Ausführung vor.

Debugger: hilft Entwicklern, nach dem Bauen des Programms, um dieses Schrittweise auszuführen und tieferer Informationen zu dem Programm.

\subsection{Preferences of C}

Most Important Features: Rosalie fragen

Main Preferences and Characteristics: Speicherverwaltung, Prä-Processor macht nur Sachen, die neu sind

Characteristics of C as a procedural programming language: Sequenzielle Abwicklung

\section{GitHub}

Wir können eine Lokal Branch haben, dann den Main Branch pullen und danach mergen. Somit behalten wir unsere Lokalen Änderungen und können die Konflikte Lösen.

\section{Funktionen und Libraries}

Funktionen sind Programmausschnitte, welche für andere Programme zur Verfügung stehen. Diese Funktionen haben einen Namen und können Eingabe- und Ausgabewerte haben.

Jede Funktion Hat einen Return Wert, einen Namen, eingabeWerte und einen Scope.

\subsection{Libraries}

Libraries sind Sammlungen an Funktionen und Definitionen. Diese sind separate Dateien und können von mehreren Programmen benutzt werden.

Wir können auch unsere eigenen Libraries erstellen.

Als header file .h, als Implementation file .c. Diese Custom Library wird mit $\#$include"lib.h" inkludiert.

Wir können ein Library mit gcc main.c lib.c -o myProgram erstellen.

\section{Datentypen}

\subsection{Pointer}

Ein Pointer ist eine Variable, die eine Adresse Speichert. Dieser hat einen Wert und selber auch eine Adresse. Die Eigenen Adresse kann: 0x4712 sein und der Wert des Pointers speichert dann eine Adresse z.B. 0x001A.

Wenn wir noch nicht wissen, wohin der Pointer Zeigen soll, Initialisiere eine Nullpointer: int * pointer = NULL;

Um dem Pointer auf einen Variable Zeigen zu lassen brauchen wir ein $\&$. Kann also so aussehen: int a = 42; int * pointer = $\&$ a;

\subsubsection{Pointer Operations}

Wir können Pointer auch inkrementieren, somit zeigt er auf die nächste Adresse, mit einer Schrittweite von dem aktuellen Datentype. Wir addieren somit die Größe eines Speichers.

\subsection{Arrays}

Ist eine Variable mit mehreren Werten. Auf diese Werte kann über einen Index zugegriffen werden.

Es können auch mehrdimensionale Arrays gebildet werden.

\subsection{struct}

Ein Struct ist ein Datentyp, welche mehrere Andere Datentypen zuordnet.

struct person {int age = 20; char[20] name = "Moritz"}

\subsubsection{Typedef}

Wir können auch ein Struct als "typedef struct {} name" festlegen. Dies ermöglicht uns die Zuweisung zu vereinfachen auf "name test" anstatt "struct name test".

\subsubsection{Zugriff}

Über die Punktschreibweise oder mit einem Pointer auf ein struct gezeigt wird über einen Pfeil: pointer->age oder für den Namen: strcpy(pointer->name "Ben")

\subsubsection{Special Type: Time}

Es gibt ein Paar standard stucts, ein Beispiel ist tm mit der Zeit in verschiedenen Einheiten.

\subsection{union}

Ist dazu da um mehre Zustände eines Wertes zu wechseln. Es speichert zur Zeit nur in einem der Werte von einem union.

Ein Beispiel ist die Speicherung einer Zahl, entweder als Binär oder Decimal Zahl.

\section{assert()}

Ist ein Debugging Werkzeug um Logikfehler zu finden. Es kann mit $\#$include <assert.h> eingebunden werden.

Wenn die Bedingung von assert false ausgibt, dann wird eine Fehlermeldung ausgegeben. Bei einer Rückgabe von true, passiert nichts.

\section{Passing Arguments}

\subsection{Call by Value}

Eine Kopie von dem Wert wird übergeben, damit die Funktion damit rechnen kann.

Veränderungen in der Funktion, ändern das Original nicht.

\subsection{Call by Reference}

Hier wird die Reference weitergegeben, die Funktion rechnet dann mit den eigentlichen Werte über die Reference, wo diese sind.

Veränderungen ändern das Original.

Ist Speichereffizienzen als Call by Value bei Großen Datenmengen.

Es können unvorhergesehene Nebeneffekte entstehen.

\end{document}
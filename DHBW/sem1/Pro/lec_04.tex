\documentclass[a4paper]{article}

%\usepackage{url}

%% Math
\usepackage{mathtools}
%% For Mengen like natural numbers
\usepackage{amsfonts}
%% Für spezielle Symbole
\usepackage{amssymb}

%% Images
\usepackage{import}
\usepackage{xifthen}
\usepackage{pdfpages}
%\usepackage{transparent}

%%% Command for simpler images
\newcommand{\incfig}[1]{%
    \def\svgwidth{\columnwidth}
    \import{./fig/}{#1.pdf_tex}
}

%% Links
\usepackage{hyperref}
\hypersetup{
    colorlinks=true,
    linkcolor=black,
    filecolor=magenta,
    urlcolor=cyan
}

%% Formatting
\usepackage{parskip}
%\usepackage{url}

%% Math
\usepackage{mathtools}
%% For Mengen like natural numbers
\usepackage{amsfonts}
%% Für spezielle Symbole
\usepackage{amssymb}

%% Images
\usepackage{import}
\usepackage{xifthen}
\usepackage{pdfpages}
%\usepackage{transparent}

%%% Command for simpler images
\newcommand{\incfig}[1]{%
    \def\svgwidth{\columnwidth}
    \import{./fig/}{#1.pdf_tex}
}

%% Links
\usepackage{hyperref}
\hypersetup{
    colorlinks=true,
    linkcolor=black,
    filecolor=magenta,
    urlcolor=cyan
}

%% Formatting
\usepackage{parskip}

\title{Programmieren}
\author{Moritz}
\date{January 30, 2025}

\begin{document}
\maketitle
\tableofcontents

\section{Variadic Functions}

Diese können eine Variable Anzahl an Inputs haben. Ein Beispiel ist die printf() Funktion.

Um Funktionen mit Variablen Inputs zu deklarieren und auf diese Inputs zuzugreifen, brauchen wir die macros va\_list, va\_start, va\_arg und va\_end. Zudem werden die Variablen mit ... deklariert.

\subsection{Beispiel Sum}

\begin{lstlisting}[language=C]
    #include <stdio.h>
    #include <stdarg.h>

    // Function for calculating the sum 
    // of a variable number of arguments
    int calcSumVariadic(int count, ...)
    {
        // Deklarieren von den Argumenten, 
        // welche von ... uebergeben werden
        va_list args;
        int tempSum = 0;

        var_start(args, count);

        for (int i = 0; i < count; i++)
        {
            // Hier wird das naechste Element aus der 
            // Argumenten-Liste ausgelesen
            tempSum += va_arg(args, int);
        }

        // Beendet die Nutzung
        va_end(args);

        return tempSum;
    }

    void main()
    {
        int sum = calcSumVariadic(3, 1, 2, 3);
        printf("\nSUM OF 1,2,3 --> %d", sum);
        return 0;
    }
\end{lstlisting}

\section{Debugging}

Wir brauchen einen Debugger und ausführbarer Code. Dieser Code wird getestet um logische Fehler zu finden.

Wie z.B. Unit-tests die fehlschlagen.

Variables: Wir können alle jetzt genutzten Variablen mit ihren Werten einsehen.

Watch: Wir können die werte Variablen im Verlauf des Programmes überwachen.

Call Stack: Einsehen, in welcher reihenfolge calls aufgerufen werden.

\know{Debugging}{Um unsere Debugging Umgebung zu start, muss unsere Aufgabe in einem eigenem VS-Code Fenster geöffnet werden.}

\subsection{Methoden}

Output mit printf oder logging.

Debugging Tools wie VS-Code, GDB mit Breakpoints, variable monitoring und call stacks.

Analyse Tools wie Automatic Error Detection.

\section{Transformation}

Funktionen aus stdlib.h, dies ermöglicht beispielweise eine Umwandlung von strings in long oder unsigned long. Wir können auch in Gleitkommazahlen umwandeln.

atoi: ASCII to int

\section{Sorting}

Ist die Übung und das Assignment.

Merge, Quick, Heap und Bubble-sort.

\section{CS50}

Ist der Computer Science Kurs von Harvard \href{https://cs50.harvard.edu/x/2025/}{CS50} und auf \href{https://www.youtube.com/@cs50}{YouTube}.

\end{document}
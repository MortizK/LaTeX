\documentclass[a4paper]{article}

%\usepackage{url}

%% Math
\usepackage{mathtools}
%% For Mengen like natural numbers
\usepackage{amsfonts}
%% Für spezielle Symbole
\usepackage{amssymb}

%% Images
\usepackage{import}
\usepackage{xifthen}
\usepackage{pdfpages}
%\usepackage{transparent}

%%% Command for simpler images
\newcommand{\incfig}[1]{%
    \def\svgwidth{\columnwidth}
    \import{./fig/}{#1.pdf_tex}
}

%% Links
\usepackage{hyperref}
\hypersetup{
    colorlinks=true,
    linkcolor=black,
    filecolor=magenta,
    urlcolor=cyan
}

%% Formatting
\usepackage{parskip}

\title{Programmieren in C}
\author{Moritz}

\begin{document}
\maketitle
\tableofcontents

\section{Organisatorisch}

\subsection{Materialien}

Onlinequellen wie: \url{de.cppreference.com} und \url{stackoverflow.com}

Clean Code: A Handbock of Agile Software Craftmanship - Robert C. Martin.

The Programming Language C - Kerninghan and Rithcie

\subsection{Hinweis}

Verzichten auf Stackoverflow, GitHub Copilot und ChatGPT für die Übungen. Um die Prinzipien zu Lernen, später im Job kann ich diese dann effektiv einsetzen.

\subsection{GitHub Classroom}

Wir bekommen Aufgaben über GitHub Classroom.

\subsection{Prüfungsleistung}

Die GitHub Aufgaben sind teil der Punkte bei der Prüfungsleistung. Hierdurch können bis zu 10\% Punkte erreichen.

Wir können auch Programme vorstellen und können 5\% zusätzlich erreichen.

Programmieraufgabe für 85\% in 85min vor Ort.

Termin ist auf Anfang bis Mitte Februar angelegt.

\section{Introduction of C}

\subsection{Algorithmus und Programme}

Was ist ein Algorithmus? Eine Abfolge von Befehlen, Eine Blackbox mit Inputs und Outputs. Ein Beschreibung eines Algorithmus heißt Programm und für ein Algorithmus brauchen wir eine formale Sprache.

Ein Algorithmus ist eine Beschreibung von Aufgaben, welche aus Inputs einen Output berechnen.

\subsubsection{Eigenschaften}

Specification (Ein und Ausgabe müssen Konkret sein)

Feasibility (Das Program muss endlich lang sein, Effizient)

Correctness (Alles muss Richtige Ergebnisse liefern und muss Terminieren)

\subsubsection{Pseudocodes und FLowcharts}

Um effiziente Lösungsansätze zu generieren.

Eine Art ist als Flussdiagramm für die Verschiedenen Level eines Projektes/ Algorithmus.

Definition: \url{https://www.edrawsoft.com/de/flowchart-definition.html}

Ich kann auch PapDesigner oder Structurizer benutzen.

Im Grundkonzept gibt es auch noch Process description und UML (Unified Modelling Language)

\subsection{Programming Tools}

\subsubsection{VSCode}

Unsere IDE

\subsubsection{Gnu Compiler Collection}

ist unsere Compiler für C. Kurz auch $gcc$.

\subsubsection{Compilation Process}

Wir haben verschiedenen C Dateien mit einer main.c.

Jede ?.c Datei hat eine ?.h (Header Datei) um auf sich selbst zu verweisen, damit Funktion sich gegenseitig aufrufen können.

Pre-Processor: Der Pre-Processor \#include die header Datei. Dieser macht eine Textersetzung. Alles mit \# ist ein Pre-Processor Schritt.

Der Compiler nimmt sich jeder ?.c Datei und macht ein ?.o Objekt Datei draus

Der Linker löst platzhalter, welche der Compiler frei lässt, da dort Funktionen aus anderen ?.c Dateien aufgerufen werden, und fügt diese ein. Somit wird eine Ausführbare Datei erstellt.

\subsubsection{\#Inlcude}

Es gibt zwei unterschiedlicher Arten: \#include "blah.h" sind eigene Dateien wie zusätzliche blah.c Dateien. Oder \#include <stdio.h> für externe Libraries.

\subsubsection{ASCII}

Ist ein wichtiger Standard, welcher zahlen (Bytes) in unsere Zeichen und Symbole repräsentiert.

\subsection{Variablen und Konstanten}

Charakteristiken: name, data type, value, address

derived data types: struct, union, pointer, arrays

numeric data types: char, short, int, float, double

\subsubsection{Magic Numbers}

Diese sollten vermieden werden, und wo möglich konstanten verwendet werden. Hier ist die Namensgebung wichtig um zu sehen, was diese Konstante macht und dass es eine Konstante ist.

\subsubsection{Konstanten}

entweder mit \#define name value. Hier ist der Datentyp nicht sicher.

es gibt auch const data\_type name = value;

als UPPER\_SNAKE\_CASE

\subsubsection{Variablen}

als camelCase: Beispiele: aValue, radiusCircle

\subsection{Ciding Conventions}

\subsubsection{UPER\_SNAKE\_CASE}

MAX\_BUFFER\_SIZE, PI\_value

\subsubsection{camelCase}

für lokale Variablen und Funktionen

\subsubsection{PascalCase}

für die derived data wie struct und union

\subsection{GitHub}

Ist ein Versions-Kontroll Tool. Für den Workflow

\url{https://github.com/MortizK}

\end{document}
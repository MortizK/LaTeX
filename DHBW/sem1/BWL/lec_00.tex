\documentclass[a4paper]{article}

%\usepackage{url}

%% Math
\usepackage{mathtools}
%% For Mengen like natural numbers
\usepackage{amsfonts}
%% Für spezielle Symbole
\usepackage{amssymb}

%% Images
\usepackage{import}
\usepackage{xifthen}
\usepackage{pdfpages}
%\usepackage{transparent}

%%% Command for simpler images
\newcommand{\incfig}[1]{%
    \def\svgwidth{\columnwidth}
    \import{./fig/}{#1.pdf_tex}
}

%% Links
\usepackage{hyperref}
\hypersetup{
    colorlinks=true,
    linkcolor=black,
    filecolor=magenta,
    urlcolor=cyan
}

%% Formatting
\usepackage{parskip}

\title{Betriebswirtschaftslehre}
\author{Moritz}

\begin{document}
\maketitle
\tableofcontents

\section{Vorbereitung}

\subsection{Wissenspyramide}

Das Management von Wissen hat darüber hinaus in den letzten Jahren einen enormen Stellenwert in Unternehmen und Hochschulen erhalten. Methoden zur Wissensgenerierung, -verwaltung und -verbreitung sind enorm wichtig. Gängige Werkzeuge sind E-Learning-Systeme und Wikis.

\begin{figure}[ht]
    \centering
    \incfig{wissenspyramide}
    \caption{wissenspyramide}
    \label{fig:wissenspyramide}
\end{figure}

\subsection{Staatsformen}

Die Betriebswirtschaft ist immer vom Umfeld Abhängig. Die Staatsform.

\subsubsection{Monokratier}

Herrschaft eines Einzelnen (z.B. Monarchie, Diktatur)

\subsubsection{Aristokratie}

Herrscher mehrerer ("der Besten")

\subsubsection{Demokratie}

Herrschaft aller ("des Volkes")

\subsubsection{Unterschiede in einer Demokratie}

Unmittelbare Demokratie. Entscheidungen werden durch einen großteil der GEsamtbevölkerung getroffen.

Mittelbare Demokratie. (auch repräsentative Demokratie) Repräsentanten werden vom Volk gewählt, um Entscheidungen zu entscheiden.

Plebiszitäre Demokratie: Eine Mischung aus Repräsentanten und gesamt-volksentscheide.

\subsection{Wirtschaftsformen}

Freie Marktwirtschaft, Soziale Marktwirtschaft und Zentral-verwaltungswirtschaft

Wir beschränken uns auf die Freie und Soziale Marktwirtschaft.

\subsection{Ökonomische Grundbegriffe}

Bedürfnisse, Existenzbedürfnisse, Kultur- und Luxusbedürfnisse, Individual- und Kollektivbedürfnisse, Bedarf, Nutzen, Wirtschaft, Ökonomisches Prinzip

\subsubsection{Güter}

Konsumgüter: Gebrauchsgüter (mehrmalig) oder Verbrauchsgüter (einmalig).

Nutzungsdauer: Kurzfristig oder langfristig

Verwendungsart: Konsumgut oder Investitionsgut

Gegenständigkeit: Materiell oder Immateriell.

Beweglichkeit: Mobilien oder Immobilien

\subsubsection{Abhängikeit der Güter Untereinander}

Komplementär: Sie Ergänzen Sich (Benzin - Auto)

Substitutions: Sie ersetzen Sich

Autonome: Sind unabhängig

\subsubsection{Elastizitäten}

Preiselastizität der Nachfrage

Einkommenselastizität der Nachfrage

\subsection{Wirtschaftskreislauf}

Wunderschöne Grafik in dem Script.

\subsection{Marktformen}

Wunderschöne Grafik in dem Script.

\subsection{Bedingungen}

Vollständige Marktübersicht (Markttransparenz)

Gleichartigkeit des Marktgutes (Homogenität)

Nichtvorhandensein von Bevorzugungen (Präferenzen)

Keine zeitliche Verzögerung (time-log)

\subsection{Marktmechanismus}

Grafiken von einem Preis-Mengen Diagram mit Nachfragegerade und Anbietergerade. Deren Schnittpunkt ist der Gleichgewichtspreis.

Funktion des Gleichgewichtspreises.

\subsubsection{Ausschaltungsfunktion}

Ausschaltung von Anbietern die nicht mehr konkurrenzfähig sind, da sie mit überhöhten Kosten anbieten.

"gesunder" Wettbewerb

\subsubsection{Lenkungsfunktion}

Unternehmen setzen ihre Produktionsfaktoren dort ein, wo am meisten Gewinn zu erwarten ist.

\section{Vorlesung \& Einführung}

Jetzt bei einem Neuen Dozenten Uwe Osterrieder wie Projektmanagement.

\end{document}
\documentclass[a4paper]{article}

%\usepackage{url}

%% Math
\usepackage{mathtools}
%% For Mengen like natural numbers
\usepackage{amsfonts}
%% Für spezielle Symbole
\usepackage{amssymb}

%% Images
\usepackage{import}
\usepackage{xifthen}
\usepackage{pdfpages}
%\usepackage{transparent}

%%% Command for simpler images
\newcommand{\incfig}[1]{%
    \def\svgwidth{\columnwidth}
    \import{./fig/}{#1.pdf_tex}
}

%% Links
\usepackage{hyperref}
\hypersetup{
    colorlinks=true,
    linkcolor=black,
    filecolor=magenta,
    urlcolor=cyan
}

%% Formatting
\usepackage{parskip}

\title{Betriebswirtschaftslehre}
\author{Moritz}
\date{February 3, 2025}

\begin{document}
\maketitle
\tableofcontents

\section{Porter Analyse}

\subsection{Umweltanalyse}

Welche Wettbewerbe gibt es in meiner Branche schon. Welche Konkurrenz gibt es und wie hoch ist die einstiegsbarriere? Gibt es Ersatzprodukte? Verhandlungsmacht mit den Abnehmern und Lieferanten.

Auch das 5 Kräfte Modell genannt.

\subsection{Unternehmensanalyse}

Primäraktivitäten: Interne Logistik, Produktion, Externe Logistik, Marketing und Sales, Service

Unterstützungsaktivitäten: Human Resource Management, Unternehmens Infrastruktur, Technologie Entwicklung

Auch Wertschöpfungskette genannt.

\subsection{Typologie}

\begin{center}
    \begin{tabular}{c|c c}
                        & Niedrigere Kosten                & Differenzierung \\
        \hline
        Gesamte Branche & Kostenführerschaft               & Differenzierung \\
        Segment         & \multicolumn{2}{c}{Fokussierung}                   \\
    \end{tabular}
\end{center}

Generische Strategietypen sind auf S.20 im Foliensatz.

\section{Prüfungsrelevant}

\know{Wichtig!}{Immer Ausformulieren und Beispiele Nennen wie z.B. Konkurenten. Fragen wie: Warum technisch vordernd? Auch ausformulieren.}

\end{document}
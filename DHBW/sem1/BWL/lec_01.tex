\documentclass[a4paper]{article}

%\usepackage{url}

%% Math
\usepackage{mathtools}
%% For Mengen like natural numbers
\usepackage{amsfonts}
%% Für spezielle Symbole
\usepackage{amssymb}

%% Images
\usepackage{import}
\usepackage{xifthen}
\usepackage{pdfpages}
%\usepackage{transparent}

%%% Command for simpler images
\newcommand{\incfig}[1]{%
    \def\svgwidth{\columnwidth}
    \import{./fig/}{#1.pdf_tex}
}

%% Links
\usepackage{hyperref}
\hypersetup{
    colorlinks=true,
    linkcolor=black,
    filecolor=magenta,
    urlcolor=cyan
}

%% Formatting
\usepackage{parskip}

\title{Betriebswirtschaftslehre}
\author{Moritz}
\date{December 16, 2024}

\begin{document}
\maketitle
\tableofcontents

\section{Einordnung}

Wissenschaften, Nichtmetaphysichen Wissenschaften, Realwissenschaften, Geisteswissenschaften, Sozialwissenschaften, Wirtschaftswissenschaften, Betriebswirtschaftslehre.

Betriebswirtschaftslehre bezieht sich nur auf unseren Betrieb, während die Volkswirtschaftslehre bezieht sich auf einen Staat.

\subsection{Realwissenschaften}

haben reale Sachverhalte zum Forschungsgegenstand... auf der Folie.

\subsection{Betriebswirtschaftslehre}

\subsubsection{Merkmale}

Orientierung an Vorgänger´ innerhalb der Wirtschaftssubjekte und Betriebe.

\subsubsection{Aufgaben}

Reale Sachverhalte beschreiben

Theoretische Erklärungen für Ursache-Wirkungs-...

\subsection{Ziele}

langfristige Gewinnmaximierung und die Wirtschaftlichkeit.

\subsection{Empirisch}

Empirisch ist nicht in allen Fällen dominierende Zielsetzung.

BWL leistet einseitig dem Gewinnstreben Hilfestellung.

\subsection{Güter}

Güter sind Knapp.

Unbegrenzte Bedürfnisse.

Und Güter befriedigen Bedürfnisse.

\subsubsection{Digitale Güter}

Hier können Güter nicht mehr unbedingt Knapp sein. z. B. Softwarelizenzen.

\section{Ökonomisches Prinzip}

Die Mittel sind gegeben und ich versuche maximalen ertrag zu erzielen. Das Maximalprinzip.

Und das Minimalprinzip andersherum. Wir haben ein Ziel und wollen das mit wenig Mittel erreichen.

Optimumprinzip, so handeln, dass Input und Output optimal aufeinander abgestimmt sind.

\subsection{Homo Oeconomicus}

völlig zweckrationales Handeln.

Vollständige Marktinformation und Markttransparenz.

Gewinn bzw. Nutzenmaximierung.

sofortige Reaktion auf Datenänderung

\subsection{Wirtschaften}

Wirtschaften bedeutet, knappe Güter geplant so einzusetzen, dass die Bedürfnisbefriedigungen in möglichst...

\begin{equation}
    \begin{array}{c|c|c}
        \text{Input}         & \text{Transformation} & \text{Output}                 \\
        \text{Knappe Mittel} & \text{Wirtschaften}   & \text{Unendliche Bedürfnisse}
    \end{array}
\end{equation}

\section{Wirtschaftskreislauf}

Grafik auf der Folie.

Haushalte, Kapitalsammelstellen, Unternehmen und den Staat.

Uns reicht das 4-Sektoren-Modell

\subsection{Bedarf}

ist der Teil der Bedürfnisse, den der Mensch mit den Ihm zur Verfügung stehenden Mittel befriedigen will und kann.

Bedürfnispyramide

\begin{equation}
    \begin{array}{c}
        \text{Selbstverwirklichung}        \\
        \text{Wertschätzung}               \\
        \text{Soziale Bedürfnisse}         \\
        \text{Sicherheit und Geborgenheit} \\
        \text{Grundbedürfnisse zum Leben}
    \end{array}
\end{equation}

\subsubsection{Existenzbedürfnisse}

Unterschlupf, Trinken, Essen, Kleidung

\subsubsection{Grundbedürfnisse}

KUlturelles und Soziales Leben, sowie dem allgemeinen Lebensstandard einer Gesellschaft.

Haushaltsgegenstände, Reisen, Sport, Kultur, Weiterbildung

\subsubsection{Luxusbedürfnisse}

ISt nach oben offen.

\subsection{Nachfrage}

ISt durch die Kaufkraft gedeckte Anforderung von Gütern bei den Marktpartnern, die diese Güter bereithalten.

\section{Werturteilsfrage}

\subsection{Positive Aussage}

Sind Objektiv, also Messbar

\subsection{Normative Aussage}

Sind Subjektiv

\section{Entscheidungsorientierte BWL}

\subsection{Produktivität}

Mengenmäßiger Output / Mengenmäßiger Input

\subsection{Wirtschaftlichkeit}

wertmäßiger Output / Wertmäßiger Input

\subsection{Rentabilität}

Erfolgsgröße / Basisgröße

\section{Was sind Güter?}

\subsection{Frei und Knappe Güter}

\begin{equation}
    \begin{array}{c|c}
        \text{Freie}       & \text{Knappe}           \\
        \hline
        \text{Luft}        & \text{Wirtschaftsgüter} \\
        \text{Wasser}      &                         \\
        \text{Sonnenlicht} &                         \\
    \end{array}
\end{equation}

\subsection{Unterscheidung von Gütern}

Auf Folie Seite 72.

\subsection{Nominalgut \& Realgut}

Nominalgut der in Gelt bezifferte Wert.

Realgut der Wert des Gegenstand.

\subsection{Verbrauchsgut \& Gebrauchsgut}

Ist es Wiederverwendbar oder nicht.

\subsection{Investitionsgut \& Konsumgut}

Investitionsgut um damit Geld zu verdienen.

Konsumgut ist für das eigene Bedürfnis.

\section{Was ist heute anders?}

Was ist heutzutage grundlegend anders als noch vor 20 Jahren?

Die größten Unternehmen wie Taxis besitzen selbst keine Autos (Uber). Oder Amazon hat selbst kein Inventar.

\section{Digitale Güter}

Sind Immaterielle Mittel zur Befriedigung von Bedürfnissen.

Produktion, Logistik, Änder- und Reproduzierbarkeit. Verschleißfrei, Systemwettbewerb, Unsichere Zahlungsbereitschaft.

Hohe Fixkosten (entwicklungskosten). Geringe variable Kosten.

Keine (kaum) Lagerkosten.

Keine (kaum) Transportkosten.

Wir brauchen ein Device auf dem unsere Software läuft.

\subsection{Netzwerkeffekt}

Der Nutzen ist abhängig, wie viele Endgeräte im System sind.

Z.B. Das Telefonieren ist immer besser je mehr Telefonieren können. Der Nutzen steigt exponential mit der Anzahl an Nutzern.

Weil jeder WhatsApp benutzt jeder WhatsApp!

Direkt: Der Wert der Netzleistung steigt an, wenn andere Nachfrager das gleiche Gut verwenden.

\subsection{Switching Costs}

Oder auch Lock in Effect.

Wie schwierig ist es das System zu verlassen?

Am Beispiel Apple. Das Verlassen des Systems ist so schwierig, das man lieber drinnen bleibt als sich nach alternativen suchen.

\section{Erzeugnisse}

\subsection{Unfertige Erzeugnisse}

Hat schon gewisse Prozesse durchlaufen. Und ist ein Zwischenerzeugnis.

\subsection{Fertige Erzeugnisse}

Diese sind verkaufsfähig.

\section{Kosten}

Materialkosten + Fertigungskosten = Herstellungskosten

\subsection{Kostenträger}

aus Folie Seite ca.110.

\end{document}
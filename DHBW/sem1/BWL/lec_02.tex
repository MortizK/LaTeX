\documentclass[a4paper]{article}

%\usepackage{url}

%% Math
\usepackage{mathtools}
%% For Mengen like natural numbers
\usepackage{amsfonts}
%% Für spezielle Symbole
\usepackage{amssymb}

%% Images
\usepackage{import}
\usepackage{xifthen}
\usepackage{pdfpages}
%\usepackage{transparent}

%%% Command for simpler images
\newcommand{\incfig}[1]{%
    \def\svgwidth{\columnwidth}
    \import{./fig/}{#1.pdf_tex}
}

%% Links
\usepackage{hyperref}
\hypersetup{
    colorlinks=true,
    linkcolor=black,
    filecolor=magenta,
    urlcolor=cyan
}

%% Formatting
\usepackage{parskip}

\title{Betriebswirtschaftslehre}
\author{Moritz}

\begin{document}
\maketitle
\tableofcontents

\section{Markt}

\subsection{Vollkommener Markt}

Dieser Hat verschieden Bedingungen und ist nur ein Theoretisches Konstrukt.

\subsection{Preisbildung}

Diese entstehen durch Angebot und Nachfrage.

\subsection{Marktformen Matrix}

Auf S. 10

\section{Wirtschaftspolitik}

\subsection{Angebotspolitik}

"Wir stellen alles her was die Leute brauchen."

\subsubsection{Trickle Down-Effekt}

Wenn oben Geld reinkommt, wird es schon irgendwie über die mittelschicht zu allen umverteilt.

\subsection{Nachfragepolitik}

"Wir stellen her was geht, irgendwer kauft es schon."

\section{Ansoff Matrix}

Oder auch Produkt-Markt-Matrix genannt.

Es wird im Markt und Produkt zwischen Bestehend und Neu unterschieden. Dadurch entsteht eine Matrix.

\begin{equation}
    \begin{split}
        \begin{array}{cc|cc}
                    &           & Markt                  &                  \\
                    &           & Bestehend              & Neu              \\
            \hline
            Produkt & Bestehend & Marktdurchdringung     & Markterweiterung \\
                    & Neu       & Produktdifferenzierung & Diversifikation
        \end{array}
    \end{split}
\end{equation}

Details sind auf den Folien ab S. 10

\section{SWOT Analyse}

Steht für Strength, Weakness, Opportunities, Threads. Wo

Es ist immer wichtig zu unterscheiden ob wir einfluss haben und dieses verändern haben.

Es soll eine nüchterne Analyse sein, was deutlich zeigt, wo zu verbessern ist.

\subsection{Positionierung}

Wo sind wir im Markt? Kann auf einer Graf mit mehreren Achsen dargestellt werden. z.B. Marke, Luxus, Preis, Service, Problemlösung, Erlebnis, Erfahrung.

\subsubsection{Corporate Identity}

Corporate Design und Communication sind untergeordnet zu der Identity. In all diesen Bereichen, können wir beeinflussen.

Es gibt aber auch viele Faktoren, welche nicht steuerbar sind.

\subsection{Ressourcenanalyse}

In der Darstellung von einem Spinnennetz für interne und externe Faktoren. Sind schön um dies mit anderen zu vergleichen.

\subsection{Ablauf}

1. Externe Einflüsse

2. Chancen-Risiken Analyse

3. Interne Einflüsse

4. Verknüpfen der Stärken und Schwächen mit den Chancen und Risiken.

5. Definition der zentralen Marketingproblemstellung.

\subsubsection{Matrix}

Dies kann wieder in einer Matrix dargestellt werden. Hierbei muss die Stärke mit den Chance und Risiken vermengt werden, sowie die Schwächen mit diesen beiden.

Welche Chancen bieten sich durch unsere Stärken?

Welche Chancen verpassen wir evtl. durch unsere Schwächen

Wie können wir unsere Stärken nutzen  um den Herausforderungen zu begegnen?

Welche Risiken sind wir durch unseren Schwäche ausgesetzt

\subsubsection{Analyse/ Strategie}

Hier wird nun eine Strategie aus dieser Matrix erstellt. Also ein Lösungsansatz.

\end{document}
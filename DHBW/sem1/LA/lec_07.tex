\documentclass[a4paper]{article}

%\usepackage{url}

%% Math
\usepackage{mathtools}
%% For Mengen like natural numbers
\usepackage{amsfonts}
%% Für spezielle Symbole
\usepackage{amssymb}

%% Images
\usepackage{import}
\usepackage{xifthen}
\usepackage{pdfpages}
%\usepackage{transparent}

%%% Command for simpler images
\newcommand{\incfig}[1]{%
    \def\svgwidth{\columnwidth}
    \import{./fig/}{#1.pdf_tex}
}

%% Links
\usepackage{hyperref}
\hypersetup{
    colorlinks=true,
    linkcolor=black,
    filecolor=magenta,
    urlcolor=cyan
}

%% Formatting
\usepackage{parskip}

\title{Lineare Algebra}
\author{Moritz}
\date{January 7, 2025}

\begin{document}
\maketitle
\tableofcontents

\section{Wiederholung}

\subsection{Teilen mit Rest}

$p$ wird von $q$ geteilt mit dem Rest $r$.

\begin{equation}
    p\in \mathbb{Z}, q\in \mathbb{N}: p = tq + r, 0\leq r < q, t\in \mathbb{Z}
\end{equation}

\subsection{ggT}

Satz nach Euklid haben wir bekommen durch den erweiterten Euklidischen Algorithmus.

\begin{equation}
    \begin{split}
        ggT(p,q) =          & tp + sq, t, s\in \mathbb{Z} \\
        ggT(p_1, p_2): p_1= & t_2p_2+p_3, 0\leq p_3 < p_2 \\
        p_2=                & t_3p_3+p_4, 0\leq p_4 < p_3 \\
        p_{n-1}=            & t_np_n+0, 0\leq p_n < p_3   \\
    \end{split}
\end{equation}

\subsection{Teilbarkeilsätze}

\begin{equation}
    \begin{split}
        p|ab\land ggT(p,a)=1      & \implies p|b          \\
        p|ab\land p\in \mathbb{P} & \implies p|a \lor p|b
    \end{split}
\end{equation}

\subsubsection{Beweis von i)}

Mit Hilfe von dem Satz von Euklid.

\begin{equation}
    \begin{split}
        1=tp+sa \mid *b \\
        b=tb*p+sab
    \end{split}
\end{equation}

Da $ab=k*p$ ist und somit ein Vielfaches von $p$ ist. Somit zeigt die Gleichung $b=p*(\dots)\implies p|b$

\subsection{Rechnen Modulo p}

\begin{equation}
    \begin{split}
        p                         & \in \mathbb{N}                               \\
        a                         & = b \text{ mod } p                           \\
        a                         & =_{p} b \iff \exists_{t\in \mathbb{Z}a=b+tp} \\
        \mathbb{Z} \text{ mod } 7 & : \{0, 1, 2, 3, 4, 5, 6\}=\mathbb{Z}_7
    \end{split}
\end{equation}

\subsubsection{Addition}

\begin{equation}
    \begin{split}
        a            & =a'+tp                   \\
        b            & =a'+sp                   \\
        \implies a+b & = a'+b'+(s+t)*p=_p a'+b'
    \end{split}
\end{equation}

\subsubsection{Multiplikation}

\begin{equation}
    a*b=(a'+tp)(b'+sp)=a'b'+(tb'+sa'+tsp)p=_p a'b'
\end{equation}

\subsubsection{Rechentabelle}

\begin{equation}
    \begin{matrix}
        \begin{array}{c|ccccccc}
            +/7 & 0 & 1 & 2 & 3 & 4 & 5 & 6 \\
            \hline
            0   & 0 & 1 & 2 & 3 & 4 & 5 & 6 \\
            1   & 1 & 2 & 3 & 4 & 5 & 6 & 0 \\
            2   & 2 & 3 & 4 & 5 & 6 & 0 & 1 \\
            3   & 3 & 4 & 5 & 6 & 0 & 1 & 2 \\
            4   & 4 & 5 & 6 & 0 & 1 & 2 & 3 \\
            5   & 5 & 6 & 0 & 1 & 2 & 3 & 4 \\
            6   & 6 & 0 & 1 & 2 & 3 & 4 & 5 \\
        \end{array} &
        \begin{array}{c|cccccc}
            */7 & 1 & 2 & 3 & 4 & 5 & 6 \\
            \hline
            1   & 1 & 2 & 3 & 4 & 5 & 6 \\
            2   & 2 & 4 & 6 & 1 & 3 & 5 \\
            3   & 3 & 6 & 2 & 5 & 1 & 4 \\
            4   & 4 & 1 & 5 & 2 & 6 & 3 \\
            5   & 5 & 3 & 1 & 6 & 4 & 2 \\
            6   & 6 & 5 & 4 & 3 & 2 & 1 \\
        \end{array}
    \end{matrix}
\end{equation}

Aus der Multiplikationstabelle lässt sich auch eine Inverse herauslesen, denn die Inverse ist eine Zahl, welche bei der Multiplikation 1 ergibt. Hier gibt es in jeder spalte genau eine Zahl.

Alle Grundrechenarten $\implies$ Körper. bisherige Körper die wir kennen sind $\mathbb{Q}, \mathbb{R}, \mathbb{C}$ und nun auch $\mathbb{Z}_p$

Das liegt folgender Vermutung nahe:

\begin{equation}
    p\in \mathbb{N}, a\in \mathbb{N}\land ggT(a,p)=1\land 0<a<p\implies \exists_{b\in \mathbb{N}}^{!}0<b<p \land a*b=_p1
\end{equation}

\subsubsection{Lemma}

1.

\begin{equation}
    \begin{split}
        ggT(a,p)=1 \implies 1 = ta+so =_p ta; \\
        \text{F.N.}: 0<t<p: t=b               \\
        \text{andernfalls: Teilen mit Rest}   \\
        t = kp*r, 0\leq r < p                 \\
        r=0?: \implies p|1: \text{W! } 0<r<p  \\
        1=p ta= (r+kp)*a=ra+kap=_p ra         \\
        \text r=b.
    \end{split}
\end{equation}

2. Eindeutig durch W!: Annahme es gibt ein weiteres $b'$ mit $0<b<p\land a*b'=_p 1$

\begin{equation}
    \begin{split}
        d.h. \substack{ab=_p 1                     \\
        ab'=_p 1} & \implies ab=_P 1 =_p ab'       \\
                  & \implies ab-ab' = a(b-b')=_p 0 \\
                  & d.h. p|a(b-b')                 \\
                  & \substack{ggT(p,a)= 1          \\
        \implies } p|b-b'                          \\
                  & b-b'<p: \text{ W!}
    \end{split}
\end{equation}

\subsubsection{Beispiel}

\begin{equation}
    \begin{split}
        ggT(5610, 637) & = 1 = -1277*637 + 145*5610 =_{5610} -1277*637 \\
                       & =(5610-1277)*637=4333*637
    \end{split}
\end{equation}

Definition: $a,p\in \mathbb{N}, a<p \land ggT(a,p)=1$. Dann gibt es genau ein $b\in \mathbb{N}: b<p\land a*b=1$. $b$ heißt Inverse von $a \text{ mod } b$. $a^{-1} \text{ mod } p$.

\begin{equation}
    a,p\in \mathbb{N}, a<p \land ggT(a,p)=1.
\end{equation}

\subsubsection{Satz: kleiner Satz von FERMAT}

$a\in \mathbb{N}, p\in \mathbb{P}$. Dann gilt $a^p=_p a$.

Falls zusätzlich $ggT(a, p)=1$. Dann gilt $a^{p-1}=_p 1$.

\begin{equation}
    \begin{split}
        a=1                                   & : \text{trivial}                                      \\
        a\to a+1                              & : a^p=_p a (IV), z.z. (a+1)^p=_p a+1                  \\
        \text{Also}                           & : (a+1)^p=\sum_{k=1}^{p}\binom{p}{k}a^k               \\
                                              & = 1+a^p+\sum_{k=1}^{p-1}\binom{p}{k}a^k               \\
                                              & =_p 1+a + \sum_{k=1}^{p-1}\binom{p}{k}a^k             \\
        k!\binom{p}{k}=k!*\frac{p!}{k!(p-k)!} & =k!\frac{(p-k)!(p-k+1)(p-k+2)\dots(p-1)*p}{k!*(p-k)!}
        \\
        d.h. p|k!\binom{p}{k}                 & \substack{\implies                                    \\
            ggT(p,k!)=1}p|\binom{p}{k}
    \end{split}
\end{equation}

Falls $ggT(a,p)=1: \exists_b ab=_p 1$.

\begin{equation}
    \begin{split}
        a^p        & =_p a \mid *b \\
        a^{p-1}*ab & =_p a*b =_p 1 \\
    \end{split}
\end{equation}

\subsubsection{Lemma}

$p\neq q\in \mathbb{P}, a=_p b\land a=_q b \implies a =_{pq} b$

\begin{equation}
    \begin{split}
        a=_p b:      & a=b+tp                                                         \\
        a=_q b:      & a=b+sq                                                         \\
                     & 0= tp-sq, tp= sq \implies p|sq \implies p|s,\text{ d.h. }s=r*p \\
        \text{Also } & a=b+sq=b+rpq=_{pq} b
    \end{split}
\end{equation}

Kleine Version des Satzes von EULER

$p\neq q\in \mathbb{P}$ und $ggT(a, pq)=1 \implies a^{(p-1)(q-1)=_{pq} 1}$.

Beweis: $ggT(a, pq)=1$, d.h. $ggT(a,p)=1\land ggT(a,q)=1$

\begin{equation}
    \begin{split}
        \begin{array}{cc}
            ggT(a,p)=1                    & ggT(a,q)=1                    \\
            \implies a^{p-1}=_p 1         & \implies a^{q-1}=_q 1         \\
            (a^{p-1})^{q-1}=_p 1^{q-1}= 1 & (a^{q-1})^{p-1}=_q 1^{p-1}= 1 \\
        \end{array} \\
        \text{d.h. } a^{(p-1)(q-1)}=_p 1 \land a^{(p-1)(q-1)}=_q 1    \\
        \substack{Lemma                                               \\
            \implies} a^{(p-1)(q-1)}=_{pq} 1
    \end{split}
\end{equation}

\section{RSA Verfahren}



\end{document}
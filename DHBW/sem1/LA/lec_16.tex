\documentclass[a4paper]{article}

%\usepackage{url}

%% Math
\usepackage{mathtools}
%% For Mengen like natural numbers
\usepackage{amsfonts}
%% Für spezielle Symbole
\usepackage{amssymb}

%% Images
\usepackage{import}
\usepackage{xifthen}
\usepackage{pdfpages}
%\usepackage{transparent}

%%% Command for simpler images
\newcommand{\incfig}[1]{%
    \def\svgwidth{\columnwidth}
    \import{./fig/}{#1.pdf_tex}
}

%% Links
\usepackage{hyperref}
\hypersetup{
    colorlinks=true,
    linkcolor=black,
    filecolor=magenta,
    urlcolor=cyan
}

%% Formatting
\usepackage{parskip}

\title{Lineare Algebra}
\author{Moritz}
\date{February 5, 2025}

\begin{document}
\maketitle
\tableofcontents

\section{Matrizen}

\subsection{Transponieren}

\subsubsection{Notation}

\begin{equation}
    x=\begin{bmatrix}
        x_1 \\x_2\\\vdots\\x_n
    \end{bmatrix}=[x_1, x_2, \dots, x_n]^t
\end{equation}

Das t steht für transponieren.

Für Matrizen gilt dann:

\begin{equation}
    A^t=[a_1, a_2, \dots,a_n]^t=\begin{bmatrix}
        a_1^t \\a_2^t\\\vdots\\a_n^t
    \end{bmatrix}
\end{equation}

\subsubsection{Beispiel}

Bei Quadratischen Matrizen, ist das transponieren eine Spiegelung um die Hauptachse.

\begin{equation}
    \begin{split}
        A^t & =
        \begin{bmatrix}
            2   & 4i & 5  \\
            6   & 0  & -3 \\
            1+i & 2  & 3
        \end{bmatrix}^t=
        \begin{bmatrix}
            2  & 6  & 1+i \\
            4i & 0  & 2   \\
            5  & -3 & 3
        \end{bmatrix} \\
        B^t & =
        \begin{bmatrix}
            2 & 5  \\
            0 & 1  \\
            3 & -i \\
            1 & 9
        \end{bmatrix}^t=
        \begin{bmatrix}
            2 & 0 & 3  & 1 \\
            5 & 1 & -i & 9
        \end{bmatrix} \\
        c^t & =
        \begin{bmatrix}
            0 & 1  & 4  \\
            1 & 3  & -6 \\
            4 & -6 & 5
        \end{bmatrix} = C
    \end{split}
\end{equation}

Bei einer Adjungierten Matrix, welche im Komplexen liegt, hat tauscht bei dem transponieren z.B. $i$ mit $-i$, welche durch das Konjungierkomplex das vorzeichen wieder tauscht.

\subsection{Adjungierung}

\begin{equation}
    \begin{split}
        A^*        & =[a_1, a_2, \dots, a_n]^*=
        \begin{bmatrix}
            a_1^* \\a_2^*\\\vdots\\a_n^*
        \end{bmatrix}             \\
        A          & =
        \begin{bmatrix}
            a_{11} & a_{12} & \dots & a_{1n} \\
            a_{21} & a_{22} & \dots & a_{2n} \\
            \vdots &        &       & \vdots \\
            a_{m1} & a_{m2} & \dots & a_{mn} \\
        \end{bmatrix} =:
        [a_{ij}]_{\substack{i=1,\dots,m         \\j=1,\dots, n}}\\
        A_{ij}     & =a_{ij}                    \\
        (A^*)_{ij} & =\bar{a_{ji}}
    \end{split}
\end{equation}

\begin{equation}
    \begin{split}
        \langle A^*x\mid y\rangle    & = \langle x\mid Ay\rangle   \\
        \langle A^*x\mid e_1\rangle  & = \langle x\mid Ae_1\rangle \\
        A^*x=
        \begin{bmatrix}
            a_1^* \\a_2^*\\\vdots\\a_n^*
        \end{bmatrix}*x & =
        \begin{bmatrix}
            \bar{x_11} & \bar{x_21} & \dots & \bar{x_m1} \\
                       & a_2        &       &            \\
                       & \vdots     &       &            \\
                       & a_n        &       &            \\
        \end{bmatrix} *
        \begin{bmatrix}
            x_1 \\x_2\\\vdots\\x_n
        \end{bmatrix} =
        \begin{bmatrix}
            \langle a_1\mid x\rangle \\
            \langle a_2\mid x\rangle \\
            \vdots                   \\
            \langle a_m\mid x\rangle \\
        \end{bmatrix}
    \end{split}
\end{equation}

Den Rest für die allgemeine Form lässt sich durch Rechnen mit der Summe aller Basisvektoren $\langle A^*x\mid\sum_{j=1}^{m}y_je_j\rangle$. Diese Formen wir dann einfach um, bis wir die allgemeine Form erhalten $\langle A^*x\mid y\rangle = \langle x\mid Ay\rangle$.

\subsubsection{Rechenregeln}

\begin{enumerate}[i.]
    \item $\langle A^*x\mid y\rangle = \langle x\mid Ay\rangle$
    \item $(tA+sB)^* = \bar{t}A^*+\bar{s}B^*$
    \item $(AB)^* = B^*A^*$
\end{enumerate}

\begin{equation}
    \begin{split}
        \langle x\mid ABy\rangle   & = \langle (AB)^*x\mid y\rangle                              \\
        \langle x\mid A(By)\rangle & = \langle A^*x\mid By\rangle = \langle B^*A^*x\mid y\rangle \\
                                   & \implies (AB)^*x=B^*A^*x                                    \\
                                   & \implies (AB)^*=B^*A^*
    \end{split}
\end{equation}

\subsubsection{Folgerungen}

$A^*x$ sind Skalarprodukt in jeder Zeile. Daraus können wir auch was für $A^*B$ folgern

\begin{equation}
    \begin{split}
        A^*B & =A^*[b_1, b_2, \dots, b_m]=[A^*b_1, A^*b_2, \dots, A^*b_m] \\
             & =
        \begin{bmatrix}
            \langle a_1\mid b_1\rangle & \langle a_1\mid b_2\rangle & \dots & \langle a_1\mid b_m\rangle \\
            \langle a_2\mid b_1\rangle & \langle a_2\mid b_2\rangle & \dots & \langle a_2\mid b_m\rangle \\
            \vdots                     & \vdots                     &       & \vdots                     \\
            \langle a_n\mid b_1\rangle & \langle a_n\mid b_2\rangle & \dots & \langle a_n\mid b_m\rangle \\
        \end{bmatrix}
    \end{split}
\end{equation}

\subsubsection{Beispiel mit Inversen}

$B=[b_1, b_2, b_3]$ ist eine Orthogonalbasis mit Normierten Vektoren.

\begin{equation}
    \begin{split}
        B^*B         & =
        \begin{bmatrix}
            \langle b_1\mid b_1\rangle & \langle b_1\mid b_2\rangle & \langle b_1\mid b_3\rangle \\
            \langle b_2\mid b_1\rangle & \langle b_2\mid b_2\rangle & \langle b_2\mid b_3\rangle \\
            \langle b_3\mid b_1\rangle & \langle b_3\mid b_2\rangle & \langle b_3\mid b_3\rangle \\
        \end{bmatrix} \\
                     & =
        \begin{bmatrix}
            1 & 0 & 0 \\
            0 & 1 & 0 \\
            0 & 0 & 1
        \end{bmatrix} = \text{Eins Matrix}                                                   \\
        \implies B^* & =B^{-1}
    \end{split}
\end{equation}

Unter dieser Bedingung ist die Adjungierte Matrix gleich der Inversen Matrix.

\subsection{Beispiele}

\begin{equation}
    \begin{split}
        \begin{bmatrix}
            2 & 1 \\
            6 & 3
        \end{bmatrix} *
        \begin{bmatrix}
            -1 \\ 2
        \end{bmatrix} =
        \begin{bmatrix}
            0 \\ 0
        \end{bmatrix}
    \end{split}
\end{equation}

Die Lösungsmenge der homogenen Gleichungen $Ax=0$ heißt Kern von $A$.

\begin{center}
    $ker A =\{x\in V\mid Ax=0\}\subseteq V$
\end{center}

Das Bild von $A$,

\begin{center}
    $im A:=\{y\in W\mid \exists_{x\in V} y=Ax\}\subseteq W$
\end{center}

\subsubsection{Vektorraum}

Die "richtige" Kategorie von Teilmengen eines Vektorraums sind Teilräume. $T\subseteq V$ ist Teilraum

\begin{center}
    $x, y\in T\implies \forall_{t, s\in \mathbb{K}}tx+sy\in T$
\end{center}

d.h. anschaulich:

\begin{center}
    $T$ ist ein kleiner Vektorraum im großem Vektorraum $V$.
\end{center}

Im $\mathbb{R}^3$ sind das Gerade, Ebenen, die Null und der gesamte $\mathbb{R}^3$. Die Geraden und Ebenen müssen durch den Nullpunkt gehen.

\subsubsection{Lemma}

Kern von A $ker A$ und Bild von A $im A$ sind Teilräume.

$ker A$ heißt $x, y \in ker A$, d.h. $Ax=0\land Ay=0$.

Zu zeigen $tx+sy\in ker A$ dann

\begin{equation}
    \begin{split}
         & A(tx+sy)=tAx+sAy= t*0+s*0=0 \\
         & \implies tx+sy\in ker A
    \end{split}
\end{equation}

Somit ist $ker A$ Teilraum von $W$.

$im A$. heißt $y_1, y_2\in im A$, d.h. $\exists_{x_1, x_2\in V}: y_1=Ax_1, y_2=Ax_2$

\begin{equation}
    \begin{split}
        ty_1+sy_2=tAx_1+sAx_2=A(tx_1+sx_2) \\
        \implies ty_1+sy_2\in im A
    \end{split}
\end{equation}

Somit ist $im A$ Teilraum von $W$.

\subsection{Dimensionsformel für lineare Abbildung}

Ist in meinem Heft auf LA 05.02 unter Dimensionsformel mit Skizze

\subsection{Injektivität für lin. Abbildung}

\begin{equation}
    \begin{split}
        Ax=Ax'\implies A(x-x')=0 \\
        \implies x-x'\in ker A   \\
        \implies x = x'
    \end{split}
\end{equation}

d.h. $ker A =\{0\}\implies A$ ist injektiv.

\section{Relationen}

Eine $f$ Funktion/ Abbildung heißt injektive: $\iff x\neq x'\implies f(x)\neq f(x')\iff f(x)=f(x')\implies x=x'$

$f$ heißt surjektiv, falls $im f = y$. Und bijektiv, wenn injektive $\land$ surjektiv.

\end{document}
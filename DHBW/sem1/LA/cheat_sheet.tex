\documentclass[a4paper]{article}

%\usepackage{url}

%% Math
\usepackage{mathtools}
%% For Mengen like natural numbers
\usepackage{amsfonts}
%% Für spezielle Symbole
\usepackage{amssymb}

%% Images
\usepackage{import}
\usepackage{xifthen}
\usepackage{pdfpages}
%\usepackage{transparent}

%%% Command for simpler images
\newcommand{\incfig}[1]{%
    \def\svgwidth{\columnwidth}
    \import{./fig/}{#1.pdf_tex}
}

%% Links
\usepackage{hyperref}
\hypersetup{
    colorlinks=true,
    linkcolor=black,
    filecolor=magenta,
    urlcolor=cyan
}

%% Formatting
\usepackage{parskip}

\title{Lineare Algebra}
\author{Moritz}

\setcounter{tocdepth}{2}

\begin{document}
\maketitle
\tableofcontents

\section{Ziele}

Diese Dokument soll nicht nochmal alles im Detail aufschrieben, was wir gemacht haben, sondern eine Sammlung sein um sich auf die Prüfung vorzubereiten.

Dies dient auch als Vorarbeit für meinen A4 Spickzettel.

\subsection{Altklausuren}

Ich werde Altklausuren analysieren um die Hauptaufgaben zu filtern und in mehr Detail zu analysieren und üben.

\subsection{Formelsammlung/ Tools}

Die Idee ist es eine Sortierte Formelsammlung zu jedem Unten aufgelisteten Themengebiet zu erstellen.

\subsection{Begriffsverzeichnis}

Zudem werde ich wie Formel auch Begriffe sammel und lernen, um nicht mehr darüber nachdenken zu müssen.

\subsection{Bildverzeichnis}

Von hilfsreichen Grafischen Darstellungen, wie z.B. Alle Formel im Dreieck.

\subsection{Rechnungen}

Vorgerechnete Aufgaben in dem Korrekten Klausuraufschrieb.

\section{Grundlagen}

\subsection{Aussagenlogik}

\know{Aussagen}{Wir können Aussagen als Sprachliches Konstrukt bauen. Und diese dan durch Nachdenken als entweder wahr oder falsch bestimmen. Wichtig: Manche Aussagen sind bis heute nicht bewiesen worden und es gibt keinen wiederspruch.}

\begin{center}
    \begin{tabular}{c c|c|c|c|c|c}
        $A$ & $B$ & $\neg A$ & $A\land B$ & $A\lor B$ & $A\implies B$ & $A\iff B$ \\
        \hline
        0   & 0   & 1        & 0          & 0         & 1             & 1         \\
        0   & 1   & 1        & 0          & 1         & 1             & 0         \\
        1   & 0   & 0        & 0          & 1         & 0             & 0         \\
        1   & 1   & 0        & 1          & 1         & 1             & 1         \\
    \end{tabular}
\end{center}

\know{Tautologie}{Sind Aussagen, die immer wahr sind. Es gibt auch das Gegenstück.}

\subsubsection{De Morgansche Regeln}

\begin{enumerate}[i.]
    \item $\overline{A\land B}\iff\overline{A}\lor\overline{B}$
    \item $\overline{A\lor B}\iff\overline{A}\land\overline{B}$
    \item $A\land(B\lor C)\iff(A\land B)\lor(A\land C)$
    \item $A\lor(B\land C)\iff(A\lor B)\land(A\lor C)$
\end{enumerate}

\subsubsection{Aussageformen}

\begin{center}
    \begin{tabular}{r l}
        $\exists_{x\in X}{P(x)}$ & Es existiert ein $x$ aus $X$, für das $P(x)$ gilt. \\
        $\forall_{x\in X}{P(x)}$ & Für alle $x$ aus $X$ gilt $P(x)$.
    \end{tabular}
\end{center}

Somit können wir Mathematische Aussagen treffen, wie:

\begin{equation}
    A:= \forall_{n\in \mathbb{N}}{D(n)}= \forall_{n\in \mathbb{N}}{1+2+3\dots n = \frac{1}{2}n*(n+1)}
\end{equation}

Quantoren lassen sich auch negieren. Hierzu muss nur der Quantor umgetauscht werden und die Aussage negiert werden.

\subsection{Mengen}

Mengen lassen sich durch das Angeben ihrer Elemente Darstellen:

\begin{center}
    $M_1=\{1, 2, 3, 4, 5, 6\}$
\end{center}

\begin{center}
    \begin{tabular}{c l}
        $m\in M$       & Das Element $m$ gehört zur Menge $M$         \\
        $\in$          & gehört zu                                    \\
        $\notin$       & gehört nicht zu                              \\
        $A\subseteq B$ & $A$ ist eine Teilmenge von $B$               \\
        $A\subset B$   & $A$ ist eine Teilmenge von $B$ und kleiner   \\
        $A=B$          & dann gilt: $A\subseteq B$ und $B\subseteq A$ \\
    \end{tabular}
\end{center}

\know{Abhängige Menge}{Wir können Menge auch beschreiben, indem wir von bekannten Mengen aussgehen, wie z.B. von der Menge der Natürlcihen Zahlen $\mathbb{N}$
    \begin{center}
        $A=\{a\in \mathbb{N}\mid P(a)\}$
    \end{center}}

Mit diesem Wissen lassen sich alle Geraden Zahlen darstellen:

\begin{center}
    $G:=\{n\in \mathbb{N}\mid \exists_{k\in \mathbb{N}}{n=2k}\}$
\end{center}

\subsubsection{Intervalle}

Intervalle sind eine Bestimmte Art von Mengen.

\begin{center}
    $I:=\{x\in \mathbb{R}\mid a\leq x\leq b\}$
\end{center}

\begin{equation}
    \begin{split}
        [a, b]       & := \{x\in \mathbb{R}\mid a\leq x\leq b\} \\
        [a, b)       & := \{x\in \mathbb{R}\mid a\leq x < b\}   \\
        (-\infty, b] & := \{x\in \mathbb{R}\mid x\leq b\}       \\
        (-\infty, b) & := \{x\in \mathbb{R}\mid x < b\}
    \end{split}
    \quad
    \begin{split}
        (a, b)      & := \{x\in \mathbb{R}\mid a < x < b\}   \\
        (a, b]      & := \{x\in \mathbb{R}\mid a < x\leq b\} \\
        [a, \infty) & := \{x\in \mathbb{R}\mid a\leq x\}     \\
        (a, \infty) & := \{x\in \mathbb{R}\mid a < x\}
    \end{split}
\end{equation}

Abgeschlossene Intervalle haben Grenzen $[a, b]$, offene nicht. Zudem sind diese beschränkt, anders als $[a, \infty)$, welches unbeschränkt ist.

\know{Konstruktionsvorschrift}{Bei Aufwändigeren Menge, wie die Mersenne Zahlen: $M:=\{x\in \mathbb{R}\mid \exists_{k\in \mathbb{R}{n=2^k-1}}\}$. Diesen Schreibaufwand können wir uns nun sparen:
    \begin{center}
        $M:=\{2^k-1\mid k\in \mathbb{R}\}$
    \end{center}}

\subsubsection{Produktmenge}

Die Produktmenge beschreibt eine Menge, $A\times B$ wo alle Kombinationen von $a\in A$ und $b\in B$ gelten. Wir schreiben:

\begin{center}
    $A\times B:={[a, b]\mid a\in A\land b\in B}$
\end{center}

Zur Vorstellung hilft es sich zwei Intervalle vorzustellen und diese im Koordinatensystem zu markieren:

\begin{figure}[ht]
    \centering
    \incfig{produktmenge}
    \caption{produktmenge}
    \label{fig:Prdouktmenge}
\end{figure}

\subsubsection{Mengenlogik}

\begin{center}
    \begin{tabular}{r l l l}
        $A\cap B$ & $:=\{x\in X\mid x\in A \lor x\in B\}$     & Durchschnitt    \\
        $A\cup B$ & $:=\{x\in X\mid x\in A \land x\in B\}$    & Vereinigung     \\
        $A^c$     & $:=\{x\in X\mid \neg x\in A\}$            & Komplement      \\
        $A/ B$    & $:=\{x\in X\mid x\in A \land x\notin B\}$ & Mengendifferenz \\
    \end{tabular}
\end{center}

Der Durchschnitt sind die Gemeinsamkeiten der beiden Mengen. Die Vereinigung sind alle Elemente aus $A$ und $B$. Das Komplement sind alle Elemente außer in der Menge. Die Mengendifferenz sind alle Elemente aus $A$, ohne die gemeinsamen Elemente mit $B$.

\know{De Morgansche Regeln}{Es gelten die Gleichen Rechenregeln wie in der Aussagenlogik, so zeigt sich, dass das Komplement $A^c$ gleichwertig ist mit dem $\overline{A}$.
    \begin{center}
        $(A\cup B)^c=A^c\cap B^c$
    \end{center}}

\subsubsection{Die Natürlichen Zahlen $\mathbb{N}$}

Die Natürlichen Zahlen bauen vollständig auf der leeren Menge auf. So ist die $0:=\emptyset$ und die $1:=\{\emptyset\}$ und die $2:=\{\emptyset, \{\emptyset\}\}$ usw. Das allgemeine Verfahren:

\begin{center}
    $n'=n\cup \{n\}$
\end{center}

\subsubsection{Vollständige Induktion}

Wir können mit zwei Eigenschaften einer Menge zeigen, dass diese eine Teilmenge von $\mathbb{N}$ ist.

\begin{enumerate}[I]
    \item $1\in M$
    \item $n\in M \implies n + 1 \in M$
\end{enumerate}

Aus diesen Beiden Eigenschaften können wir folgern, dass $M=\mathbb{N}$ gilt. Nun gibt es für jedes $n\in \mathbb{N}$ eine Aussage $A_n$. Dies können wir beweisen, indem wir die beiden Eigenschaften $I$ und $II$ beweisen.

\textbf{Beweis:} $1+2+\dots+n=\frac{1}{2}n(n+1)$

\begin{center}
    \begin{tabular}{r l}
        $n=1$:      & $1=\frac{1}{2}*1(1+1)$ was off. wahr ist       \\
        $n\to n+1$: & z.z. $1+2+\dots+n+(n+1)=\frac{1}{2}(n+1)(n+2)$ \\
    \end{tabular}
\end{center}

\begin{equation}
    \begin{split}
        1+2+\dots+n+(n+1) & _{=}^{IV} \frac{1}{2}n(n+1)+(n+1)       \\
                          & = \frac{1}{2}n(n+1)+1*(n+1)             \\
                          & = \frac{1}{2}n(n+1)+2*\frac{1}{2}*(n+1) \\
                          & = \frac{1}{2}(n+1)(n+2)
    \end{split}
\end{equation}

\subsubsection{Binomischer Lehrsatz}

\begin{center}
    $\binom{n}{k}:=\frac{n!}{k!(n-k)!}$
\end{center}

Hierbei muss beachtet werden, dass $n!$ für $0!=1$ definiert ist. Für $1\leq k\leq n$ gilt zusätzlich die Konstruktionsvorschrift.

\begin{center}
    $\binom{n+1}{k}=\binom{n}{k-1}+\binom{n}{k}$
\end{center}

Diese lässt sich visuell am Pascalschem Dreieck erkennen. Denn dort entsteht Jede folgende Zahl aus den beiden diagonal darüberlegenden Zahlen.

\begin{center}
    \begin{tabular}{c c c c c c c c c c c}
          &   &   &   &    & 1 &    &   &   &   &   \\
          &   &   &   & 1  &   & 1  &   &   &   &   \\
          &   &   & 1 &    & 2 &    & 1 &   &   &   \\
          &   & 1 &   & 3  &   & 3  &   & 1 &   &   \\
          & 1 &   & 4 &    & 6 &    & 4 &   & 1 &   \\
        1 &   & 5 &   & 10 &   & 10 &   & 5 &   & 1 \\
    \end{tabular}
\end{center}

Es wird auch deutlich, dass die Summe alle Binomialkoeffizienten $2^n$ sein muss, da sich mit jeder Ebene der Gesamtwert verdoppelt.

\section{Relationen und Funktionen}

\subsection{Allgemeine Eigenschaften von Relationen}

\know{Relation}{Eine zweistellige Relation $R$ von $A\times B$:
    \begin{center}
        $R\subseteq A\times B$
    \end{center}
    Man sagt jedes $a\in A$ steht somit in Relation zu jedem $b\in B$}

Ich kann mir Relationen also so verstehen, wie man sich das vorstellt. Somit den gleichen Ansatz wie Relationen in einer Datenbank.

\subsection{Klassifikation von Relationen}

\begin{tabular}{r l}
    homogen         & $A=B$, alle hier müssen homogen sein!            \\
    reflexiv        & Für alle $a\in A$ ist $[a, a]\in R$              \\
    symmetrisch     & $[a, b]\in R$, gibt es auch $[b, a]\in R$        \\
    antisymmetrisch & $[a, b]\in R \land [b, a]\in R$ folgt $a=b$      \\
    transitiv       & $[a, b]\in R$, $[b, c]\in R$ folgt $[a, c]\in R$
\end{tabular}

Die weiteren müssen nicht unbedingt homogen sein:

\begin{tabular}{r l}
    linkstotal  & Für alle $a\in A$ gibt es ein $b\in B$, so dass $[a, b]\in R$       \\
    rechtstotal & Für alle $b\in B$ gibt es ein $a\in A$, so dass $[a, b]\in R$       \\
    funktional  & aus $[a, b]\in R$ und $[a, c]\in R$ folgt $b=c$                     \\
    injektiv    & aus $[a, b]\in R$ und $[c, b]\in R$ folt $a=c$                      \\
    bijektiv    & für alle $b\in B$ gibt es genau ein $a\in A$, so dass $[a, b]\in R$
\end{tabular}

Visualisieren kann man diese in Tabellen:

\begin{center}
    \begin{tabular}{c c c}
        \begin{tabular}{c|c c c c}
              & 1 & 2 & 3 & 4 \\
            \hline
            1 & 1 &   &   &   \\
            2 &   & 1 &   &   \\
            3 &   &   & 1 &   \\
            4 &   &   &   & 1 \\
        \end{tabular} &
        \begin{tabular}{c|c c c c}
              & 1 & 2 & 3 & 4 \\
            \hline
            1 &   & 1 &   &   \\
            2 & 1 &   & 1 &   \\
            3 &   & 1 &   & 1 \\
            4 &   &   & 1 &   \\
        \end{tabular} &
        \begin{tabular}{c|c c c c}
              & 1 & 2 & 3 & 4 \\
            \hline
            1 &   &   &   &   \\
            2 & 1 &   &   &   \\
            3 & 1 & 1 &   &   \\
            4 &   &   &   &   \\
        \end{tabular}    \\
        reflexiv                   &
        symmetrisch                &
        transitiv                    \\
        Diagonalen                 &
        Spiegelbild                &
        ZUgnetz                      \\\\
        \begin{tabular}{c|c c c c}
              & 1 & 2 & 3 & 4 \\
            \hline
            1 &   & 1 & 1 & 1 \\
            2 & 1 &   &   &   \\
            3 & 1 &   &   &   \\
            4 &   &   & 1 &   \\
        \end{tabular} &
        \begin{tabular}{c|c c c c}
              & 1 & 2 & 3 & 4 \\
            \hline
            1 &   &   &   & 1 \\
            2 &   & 1 &   &   \\
            3 & 1 &   & 1 &   \\
            4 &   &   &   &   \\
        \end{tabular} &
        \begin{tabular}{c|c c c c}
              & 1 & 2 & 3 & 4 \\
            \hline
            1 &   &   &   & 1 \\
            2 &   & 1 &   &   \\
            3 & 1 &   &   &   \\
            4 &   &   &   & 1 \\
        \end{tabular}    \\
        linkstotal                 &
        funktional                 &
        injektiv                     \\
        Jedes a                    &
        Zeilen                     &
        Spalten
    \end{tabular}
\end{center}

\subsection{Äquivalenzrelationen}

Eine Äquivalenzrelationen ist eine Relation, welche Reflexiv, Transitiv und Symmetrisch ist.

\subsection{Verkettung von Relation und die inverse Relation}

\subsection{Funktionen}

\section{Zahlentheorie}

% \subsection{Teilbarkeitstheorie ganzer Zahlen}

% Wir können die Teilbarkeit auch durch die Umstellung ausdrücken mit $a, b, c\in \mathbb{Z}$

% \begin{center}
%     $b=c*a$
% \end{center}

% Wir schreiben: $a|b$, $a$ teilt $b$.

% Wir können diese Regeln auch erweitern, denn jede division, kann in eine eindeutige Schreibweise mit Rest formuliert werden.

% \begin{center}
%     \begin{tabular}{c c}
%         $p=tq+r$, & $0\leq r < q$
%     \end{tabular}
% \end{center}

% \subsubsection{Teilbarkeitssätze}

% \begin{enumerate}[i.]
%     \item $p|ab\land ggT(p, a)=1 \implies p|b$
%     \item $p|ab\land p\in \mathbb{P} \implies p|a\lor p|b$
% \end{enumerate}

% \subsubsection{ggT und euklidischer Algo.}

% Der größte gemeinsame Teiler beschreibt die größte $t$, für mehrere Zahlen $p$. Wenn $ggT(p_1, p_2)=1$ sind $p_1$ und $p_2$ teilerfremd.

% ggT kann durch wiederholtes teilen mit Rest bestimmt werden.

% \begin{center}
%     \begin{tabular}{r c l c r c l}
%         $84$ & $=$ & $4*18 + 12$ & $\implies$ & $12$ & $=$ & $84 - 4*18$ \\
%         $18$ & $=$ & $1*12 + 6$  & $\implies$ & $6$  & $=$ & $18 - 1*12$ \\
%         $12$ & $=$ & $2*6 + 0$   &            &      &     &             \\
%     \end{tabular}
% \end{center}

% Somit ist der $ggT(84, 18)=6$. Der $ggT$ ist auch für alle Variationen von $84, -84$ mit $18, -18$ gleich $6$.

% \know{Euklid}{Es gibt zu jedem $ggT$ auch die Form
%     \begin{center}\begin{tabular}{c c}
%             $ggT(p, q)=xp+yq$, $x, y\in \mathbb{Z}$
%         \end{tabular}\end{center}}

% Um in diese Form zu kommen, können wir den erweiterten euklidischen Algorithmus anwenden. Dieser stellt wie ob, die Gleichung einfach um und setzt die oberen Gleichung in die unteren ein.

% Wenn $ggT(p, q)=1$ ist die gesuchte Form $1=xp+yq$

% \begin{center}
%     $6=18-1*(84-4*18)=3*18-1*84$
% \end{center}

% \subsection{Rechnen Modulo p}

% Zwei Zahlen Modulo $p$ sind genau dann gleich, wenn Sie sich um ein vielfaches von $p$ unterscheiden:

% \begin{center}
%     $x=_{p}y \iff \exists t\in \mathbb{Z}: x=y+tp$
% \end{center}

% Die Rechnung modulo $p$ ist eine Äquivalenzrelationen, welche eine unendlich Große Menge auch eine endliche Menge reduziert.

% \begin{center}
%     $\mathbb{Z}_p=\{0,1,2,3,\dots,p-1\}$
% \end{center}

% \subsubsection{Addition}

% Wir können Zahlen modulo $p$ mit einander addieren und subtrahieren. $x_1=_px_2$ und $y_1=_py_2$. Beide haben die Form $x_1=x_2+tp$, wenn wir diese Addieren oder Subtrahieren, bekommen wir:

% \begin{center}
%     $x_1\pm y_1=x_2+tp\pm y_2\pm sp=_px_2\pm y_2$
% \end{center}

% \subsubsection{Multiplikation}

% Die Multiplikation funktioniert in modulo $p$ auch:

% \begin{center}
%     $x_1*y_1=(x_2+tp)(y_2+sp)=x_2y_2+tpy_2+spx_2+tsp^2=_px_2y_2$
% \end{center}

% \subsubsection{Inverse modulo p}

% Die Inverse modulo $p$ ist definiert als $ab=_p1$, dabei muss $p>1$ und für jedes $a\in \mathbb{Z}_p^{*}$ muss $ggT(a, p)=1$ gelten.

% \begin{center}
%     \begin{tabular}{c c}
%         \begin{tabular}{c|c c c c c c c}
%             $*/7$ & 1 & 2 & 3 & 4 & 5 & 6 \\
%             \hline
%             1     & 1 & 2 & 3 & 4 & 5 & 6 \\
%             2     & 2 & 4 & 6 & 1 & 3 & 5 \\
%             3     & 3 & 6 & 2 & 5 & 1 & 4 \\
%             4     & 4 & 1 & 5 & 2 & 6 & 3 \\
%             5     & 5 & 3 & 1 & 6 & 4 & 2 \\
%             6     & 6 & 5 & 4 & 3 & 2 & 1 \\
%         \end{tabular} &
%         \begin{tabular}{c|c c c c}
%             $*/12$ & 1  & 5  & 7  & 11 \\
%             \hline
%             1      & 1  & 5  & 7  & 11 \\
%             5      & 5  & 1  & 11 & 7  \\
%             7      & 7  & 11 & 1  & 5  \\
%             11     & 11 & 7  & 5  & 1  \\
%         \end{tabular}
%     \end{tabular}
% \end{center}

% Aus diesen Tabellen lassen sich die inversen leicht auslesen.

% \know{Kleiner Satz von FERMAT}{Dieser zeigt, dass für jedes $p\in \mathbb{P}$, $a^p=_pa$ gilt. Wenn zudem noch $ggT(a, p)=1$, dann gilt zusätzlich $a^{p-1}=_p1$. Dies erhalten wir durch erweitern mit der inversen $b$, wo $ab=1$ gilt:
%     \begin{center}
%         $a^{p-1}=_pa^{p-1}*ab=a^pb=_pab=_p1$
%     \end{center}}

% Für zwei unterschiedliche Primzahl $p,q$ gilt $a=_{pq}b$, wenn $a=_pb$, $a=_qb$ gelten. Mit dem kleinem Satz von FERMAT gilt unter der Bedingung, dass $ggT(a, pq)=1$ zusätzlich:

% \begin{center}
%     $a^{(p-1)(q-1)}=_{pq}1$
% \end{center}

% \subsection{RSA-Verschlüsselung}

% \begin{enumerate}[i.]
%     \item Wähle zwei verschiedene Primzahlen $p, q$ und bilde $n:= p*q$.
%     \item Wähle $1<e<\varphi:=(p-1)(q-1)$ mit $ggT(\varphi, e)=1$. Dann ist $S_o:=[e, n]$
%     \item Bilde die Inverse $d$ von $e$ modulo $\varphi$ (mit erw. euklid. Alg.) $d=_{\varphi}e^{-1}$. Dann ist $S_p:=[d, n]$

%           Danach ist die Schlüsselgeneration abgeschlossen. $p$ und $q$ können gelöscht werden. $S_p$ muss verschlüsselt auf dem Rechner gespeichert werden.
%     \item Eine Nachricht $0<M<n$ wird durch den öffentlichen Schlüssel ($S_o$) verschlüsselt:
%           \begin{center}
%               $C:=_n M^{e}$
%           \end{center}
%           $C$ wird mit $S_p$ entschlüsselt:
%           \begin{center}
%               $M=_n C^{d}$
%           \end{center}
% \end{enumerate}

% \subsubsection{Rechnungen}

% Es gibt einen online Rechner zur Korrektur: \href{https://rsa-calculator.netlify.app/}{RSA-Calculator}

% $p=101, q=17$ und $e=411$ sind gegeben und die beiden Schlüssel sollen bestimmt werden.

% Für den Öffentlichen Schlüssel muss nur noch $n=p*q$ bestimmt werden. Somit ist $n=101*17=1717$ und falls $ggT(\varphi, e)=1$, gilt $S_o=[e, n]$. $\varphi=(p-1)(q-1)=100*16=1600$.

% \begin{equation}
%     \begin{split}
%         1600 & = 3*411 + 367 \\
%         411  & = 1*367 + 44  \\
%         367  & = 8*44 + 15   \\
%         44   & = 2*15 + 14   \\
%         15   & = 1*14 + 1    \\
%     \end{split}
%     \qquad\qquad
%     \begin{split}
%         367 & = 1600 - 3*411 \\
%         44  & = 411 - 1*367  \\
%         15  & = 367 - 8*44   \\
%         14  & = 44 - 2*15    \\
%         1   & = 15 - 1*14    \\
%     \end{split}
% \end{equation}

% $S_o=[411, 1717]$. Für den Privaten Schlüssel brauchen wir die Darstellung nach EUKLID:

% \begin{equation}
%     \begin{split}
%         1 & = 15 - 1*14 = 15 - 1*(44 - 2*15)= 3*15 - 1*44 \\
%           & =  3*(367 - 8*44) - 1*44 = 3*367 - 25*44      \\
%           & =  3*367 - 25*(411 - 1*367) = 28*367 - 25*411 \\
%           & =  28*(1600 - 3*411) - 25*411                 \\
%           & = 28*1600 - 109*411
%     \end{split}
% \end{equation}

% $d=1600-109=1491$. Daraus ergibt sich $S_p=[1491, 1717]$

% Zum Verschlüsseln und Entschlüsseln gilt der gleiche Ansatz, wir brauchen nur noch ein $M=405$:

% \begin{equation}
%     \begin{split}
%         C & =_n M^e                                                    \\
%           & =_{1717} 405^{411} = 405*405^{410} = 405*(405^2)^{205}     \\
%           & =_{1717} 405*910^{205} \text{ Es gilt: } 405^2=_{1717} 910 \\
%           & =_{1717} 409*910*910^{204} =409*910*(910^2)^{102}          \\
%           & =_{1717} 1112*506^102 = 1112 * (506^2)^{51}                \\
%           & =_{1717} 1112 * 203^{51} = 1112*203 * (203^2)^{25}         \\
%           & =_{1717} 809 * 1^{25} = 809
%     \end{split}
% \end{equation}

% Ist identisch für das Entschlüsseln nur mit anderem Ansatz

% \begin{equation}
%     \begin{split}
%         M & =_n C^d                                 \\
%           & =_{1717} 809^{1491} = 809*(809^2)^{745} \\
%           & =_{1717} 809*304*(304^2)^{372}          \\
%           & =_{1717} 405 * (1415^2)^{186}           \\
%           & =_{1717} 405 * (203^2)^{93}             \\
%           & =_{1717} 405 * 1^{93} = 405
%     \end{split}
% \end{equation}

% \subsection{Chinesische Restsatz}

\section{Kombinatorik}

\subsection{Die Urnenmodelle}

\section{Vektorräume}

\subsection{$\mathbb{R}^2$ und $\mathbb{R}^3$ als Vektorraum}

Sind Ursprünglich Definiert als die Produktmenge mehrere Mengen.

\begin{center}
    $\mathbb{R}^n=\{[x_1, x_2, \dots, x_n]\mid x_1, x_2, \dots, x_n\in \mathbb{R}\}$
\end{center}

Wir können Vektoren miteinander addieren und sie auch mit einem einfachen Skalar, multiplizieren:

\begin{center}
    $x+y$ oder halt $t*x$
\end{center}

Es gibt zudem noch den Nullvektor, welche einfach $0:=[0, 0, \dots, 0]$ definiert wird.

\know{Linearkombination}{Ist im allgemeinsten eine addition von gestreckten Vektoren
    \begin{center}
        $t_1x_1+t_2x_2+\dots t_nx_n$
    \end{center}}

\subsubsection{Geraden und Ebenen}

Im allgemeinsten können wir Geraden und Ebenen als Menge Beschreiben, welche einen Stützvektor $q$ haben und eine Anzahl an Richtungsvektoren. Für Gerade ist es ein Richtungsvektor $u$ und bei der Ebene sind es zwei $u$ und $v$.

\begin{center}
    $g:=\{q+tu\mid t\in \mathbb{R}\}$

    $e:=\{q+tu+sv\mid t, s\in \mathbb{R}\}$
\end{center}

\subsubsection{Norm}

Die Norm: Ist die Länge eines Vektors und lässt sich mit dem Satz von Pythagoras bestimmen.

\begin{center}
    $\|x\|:=\sqrt{x_1^2+x_2^2+\dots+x_n^2}$
\end{center}

\subsubsection{Skalarprodukt}

Skalarprodukt: Dies ermöglicht den Winkel zwischen zwei Vektoren zu bestimmen.

\begin{center}
    $\langle x\mid y\rangle:=\|x\|*\|y\|*\cos(\alpha)$
\end{center}

Mithilfe des Kosinussatzes $c^2=a^2+b^2-2ab\cos(\gamma)$ und etwas umformen erhalten wir die einfache Form vom Skalarprodukt:

\begin{equation}
    \langle x\mid y\rangle = x_1y_1+x_2+y_2+x_3y_3
\end{equation}

Eigenschaften des Skalarproduktes:

\begin{enumerate}[\qquad i)]
    \item $\langle x\mid x\rangle\geq 0, =0\iff x=0$ \qquad\qquad  Definitheit
    \item $\langle x\mid y\rangle=\langle y\mid x\rangle$ \qquad\qquad\qquad\qquad\qquad Symmetrie
    \item $\langle tx + sy\mid z\rangle=t*\langle x\mid z\rangle+s*\langle y\mid z\rangle$ \qquad Linearität
\end{enumerate}

Um nun den Winkel $\alpha$ zu bestimmen, stellen wir nach $\cos(\alpha)$ um:

\begin{equation}
    \cos(\alpha)=\frac{\langle x\mid y\rangle}{\|x\|*\|y\|}
\end{equation}

\know{Senkrecht: $\langle x\mid y\rangle=0$}{Wenn das Skalarprodukt gleich $0$ ist, so stehen die beiden Vektoren seknrecht aufeinander. Dies wird auch Orthogonal genannt:
    \begin{center}
        $x\bot y:\iff \langle x\mid y\rangle=0$
    \end{center}}

\subsubsection{Kreuzprodukt}

Das Kreuzprodukt zweier Vektoren $x$ und $y$ resultiert in einem Vektor, welche Senkrecht zur $x$ und $y$ ist.

\begin{equation}
    x\times y:=
    \begin{bmatrix}
        x_2y_3 - x_3y_2 \\
        x_3y_1 - x_1y_3 \\
        x_1y_2 - x_2y_1 \\
    \end{bmatrix}
\end{equation}

Eigenschaften

\begin{enumerate}[\qquad i)]
    \item $(tx + sy)\times z=tx\times z+sy\times z$
    \item $x\times y = -y\times x$
    \item $x\times y=0\iff x\| y$
    \item $\|x\times y\|=\|x\|*\|y\|\cos(\alpha)$ Die Länge des Skalarproduktes ist der Flächeninhalt des Parallelogram zwischen $x$ und $y$, da $h=\|y\|\cos(\alpha)$
\end{enumerate}

\subsubsection{Normalform der Ebene}

Wir so gebaut: Wir haben einen Punkt in der Ebene $p\in E$ und bilden alle Vektoren $x-p$. Diese Vektoren sind nur dann $\in E$, wenn diese Senkrecht zu dem Normalvektor $n$ sind. Dieser Normalvektor ist auch Normiert $n_0=n / \|n\|$. Also:

\begin{center}
    $E:=\{x\in \mathbb{R}^3\mid \langle x-p\mid n_0\rangle=0\}$

    $E:=\{x\in \mathbb{R}^3\mid \langle x\mid n_0\rangle=\langle p\mid n_0\rangle\}$
\end{center}

\subsubsection{Koordinatenform}

In der Koordinatenform ist der Normalvektor kodiert.

\begin{center}
    $E=\{x_1, x_2, x_3\in \mathbb{R}\mid n_1x_1+n_2x_2+n_3x_3=d\}$
\end{center}

Der Normalvektor setzt sich aus $n=[n_1, n_2, n_3]$ zusammen.

Und unser $d=\langle p\mid n_0\rangle$, wo $p$ unser Punkt in der Ebene ist.

\subsubsection{Abstand: Punkt zur Ebene}

Da das Skalarprodukt eine Projektion von einem Vektor $y$ auf $x$ ist, können wir wenn wir einen Punkt (Vektor) auf den Normalvektor projektieren, den Abstandsvektor bestimmen und dessen Norm (Länge) ist die Lösung:

\begin{equation}
    d(y, E):=\frac{1}{\|n\|}*|\langle x\mid n_0\rangle-\langle p\mid n_0\rangle|
\end{equation}

\subsubsection{Abstand: Punkt zur Geraden}

Wir haben einen Punkt $q$ und eine Gerade $g:=\{p+tu\mid t\in \mathbb{R}\}$. Wenn $\|u\|=1$ und somit $u$ die Länge 1 hat, ist die Fläche von dem Parallelograms gleich der Fläche von dem Abstand $d(q, g) * \|u\|$. Da $\|u\|=1$ ist die Fläche gleich dem Wert von $d(q, g)$.

\begin{equation}
    d(q, g)=\|u\times(q-p)\|
\end{equation}

Siehe \href{https://elearning.dhbw-stuttgart.de/moodle/pluginfile.php/812513/mod_folder/content/0/Vorlesungsskript/stgt-math_20250130.pdf#page=91}{Abbildung 5.9} im Skriptum

\subsubsection{Spatprodukt und Determinante}

Das Spatprodukt kombiniert die geometrischen Interpretationen von dem Kreuzprodukt $x\times y$ und dem Skalarprodukt $\langle x\mid y\rangle$.

So bildet das Kreuzprodukt die Grundfläche des Spats und das Skalarprodukt wird auf den Normalvektor dieser Grundfläche projektiert und ist somit die Höhe.

\begin{center}
    $\langle x\times y\mid z\rangle$
\end{center}

Diese Spatprodukt kann positiv und negativ sein. Wenn es negativ ist, benutzen wir zum korrigieren einfach $-z$ statt $z$.

\know{Positives Spatprodukt}{Wenn das Spatprodukt positiv ist, sprechen wir bei den drei Vektoren von einem Rechtssystem oder auch mathematisch positiv orientiert
    \begin{center}
        Rechte-Hand-Regel
    \end{center}}

\subsubsection{Determinante}

Die Determinante ist die Verallgemeinerung des Spatproduktes auch für höhere Dimensionen.

\begin{center}
    $det(x, y, z)=\langle x\times y\mid z\rangle$

    $det(x, y, z + tx + sy)=det(x, y, z)$
\end{center}

Die Determinante hat zudem noch die Eigenschaft, dass wenn $det(x, y, z)=0$ gilt, sind die Vektoren linear abhängig. Das heißt sie bilden kein Volumen und einer der Vektoren ist "unnötig".

\subsection{Die komplexen Zahlen}

Immer mal wieder taucht etwas in der Form $x^2=-1$ auf, welche keine Reellen Lösungen hat. Somit entsteht die Idee von $i$:

\begin{center}
    $i^2=-1$
\end{center}

Nun bestehen Zahlen in $\mathbb{C}$ aus einem Reellen Part und einem Imaginärem Part. z.B. $5+3i$.

\subsubsection{Rechnregeln}

\begin{align}
    (z_1+iz_2)\pm(w_1+iw_2) & =z_1\pm w_1 + i(z_2\pm w_2)        \\
    (z_1+iz_2)(w_1+iw_2)    & =z_1w_1 -z_2w_2 + i(z_1w_2+z_2w_1)
\end{align}

\subsubsection{Gaussche Zahlenebene}

Da alle Imaginären Zahlen aus genau zwei werte bestehen, wovon einer imaginär ist, können wir diese auch als ein Zahlenpaar im $\mathbb{R}^2$ darstellen.

Somit ganz geschickt in einer Zahlenebene mit den Reellen Zahlen auf der x-Achse und den Imaginären auf der y-Achse. Somit befinden wir uns wieder unter Vektoren mit Zusätzlichen Definitionen:

\begin{center}
    \begin{tabular}{c c}
        Betrag                    & Konjugiert Komplex  \\\\
        $|z|:=\sqrt{z_1^2+z_2^2}$ & $\bar{z}:=z_1-iz_2$ \\
    \end{tabular}
\end{center}

\subsubsection{Polardastellung}

Wir können Komplexe Zahlen auch in der Polardarstellung darstellen:

\begin{center}
    $z=|z|*(\cos(\varphi)+i\sin(\varphi))$
\end{center}

\subsubsection{EULER-Formel}

\begin{center}
    $e^{i\varphi}=\cos(\varphi)+i\sin(\varphi)$

    $z=|z|e^{i\varphi}$
\end{center}

Dies gilt und formmal gelten somit auch die Additionssätze der Exponentialfunktionen. Somit ist $e^{i\varphi}e^{i\Psi}=e^{i(\varphi+\Psi)}$ und auch das gegenpaar:

\begin{center}
    $(\cos(\varphi)+i\sin(\varphi)*\cos(\Psi)+i\sin(\Psi))=\cos(\varphi+\Psi)+i\sin(\varphi+\Psi)$
\end{center}

Geometrisch ist somit eine Multiplikation/ Division einfach eine Addition/ Subtraktion der Winkel. Und Potenzierung einfach die Multiplikation des Winkels.

Außerhalb des Einheitskreis muss die Normale Operation auf die Länge noch beachtet werden!

\subsubsection{Beispiele}

\begin{enumerate}[\qquad i.]
    \item $e^{i\frac{\pi}{2}}=i$
    \item $e^{i\pi}=-1$
    \item $e^{2\pi i}=1$
    \item $\bar{e^{i\varphi}}=e^{-i\varphi}$
\end{enumerate}

\subsubsection{n-ten Wurzeln ziehen}

Wir können uns leicht davon überzeugen, dass es für die n-te Wurzel n Lösungen gibt, denn die Gleichung $e^{p*2\pi i}=1$ ist periodisch und hat mehrere Lösungen für $p\in \mathbb{Z}$.

Für die Einheitswurzeln, also $=1$ lassen sich nach:

\begin{equation}
    u_p:=e^{ip*\frac{2\pi}{n}}
\end{equation}

Wenn wir nun $u_p^n$ bilden wird schnell klar, dass alle Lösungen $=1$ sein müssen.

Für den Allgemeinfall der Form: $z=|z|e^{i\varphi}$ gibt es die Grundlösung:

\begin{center}
    $w_0:=\sqrt[n]{|z|}e^{i\frac{\varphi}{n}}$
\end{center}

Diese können wir mit den Einheitswurzeln multiplizieren um alle Lösungen zu bekommen.

\know{Ablauf: $n$-ten Wurzel ziehen}{Winkel $\varphi$ durch $n$ teilen und für alle $n$-Lösungen den Winkel der Einheitswurzeln $\frac{\varphi}{n}*n$ dazu-Addieren.
    \begin{center}
        $w_p=e^{\frac{1}{n}(\varphi+p*2\pi)}$
    \end{center}
    Und nicht vergessen die $n$-te Wurzel von der Länge auszurechnen}

\subsubsection{Kubische Gleichungen/ Kardanische Formeln}

Um die Nullstellen von Kubische Gleichungen zu finden braucht man die Kardanischen Formel. Für Näherungswerte reicht auch Newtons-Methode.

Die Kardanische Formel bestehen aus mehreren Schritten und funktionieren in der Form $x^3+bx^2+cx+d=0$:

\begin{enumerate}
    \item Gleichung in die Normalform $y^3+py+q=0$ umwandeln, indem man mit $x=y-\frac{b}{3}$ substituiert.
    \item $p$ und $q$ bestimme mit: $p=c-\frac{1}{3}b^2$ und $q=\frac{2}{27}b^3-\frac{1}{3}bc+d$
    \item Die Diskriminante bestimme: $\Delta=(\frac{p}{3})^3+(\frac{q}{2})^2$
    \item Wenn $\Delta>0$ ist, so gibt es eine Reelle Lösung:
          \begin{equation}
              x_1:=\sqrt[3]{\sqrt{\Delta}-\frac{q}{2}}-\sqrt[3]{\sqrt{\Delta}+\frac{q}{2}}-\frac{b}{3}
          \end{equation}
    \item Die anderen Fallunterscheidungen der Diskriminante sind im \href{https://elearning.dhbw-stuttgart.de/moodle/pluginfile.php/812513/mod_folder/content/0/Vorlesungsskript/stgt-math_20250130.pdf#page=100}{Skriptum}
\end{enumerate}

\subsection{$\mathbb{C}^n$ als Vektorraum}

In dem Komplexen Vektorraum müssen wir unsere Norm und das Skalarprodukt leicht ändern:

\begin{align*}
    \|x\|                  & :=\sqrt{\sum_{k-1}^{n}|x_k|^2}     \\
    \langle x\mid y\rangle & := \sum_{k-1}^{n}\overline{x_k}y_k
\end{align*}

Diese Änderungen behalten die gewünschten Eigenschaften $\langle x\mid x\rangle=\|x\|^2$.

Aber: Durch unsere Änderung in dem Skalarprodukt, ist dies nur noch in der zweiten Komponente $y$ linear, in der ersten ist Sie antilinear, da $\overline{x}$ benutzt wird, selbts wenn $x$ die Eingabe ist.

\begin{center}
    $\langle \lambda x_1 + \gamma x_2 \mid y\rangle = \overline{\lambda}\langle x_1 \mid y\rangle + \overline{\gamma}\langle x_2 \mid y\rangle$
\end{center}

\subsection{Der allgemeine Vektorraumbegriff$^*$}

Die Kurzform ist: Ein Vektorraum ist eine Menge $V$, wessen Elemente $x, y, z$ sich in einem Körper $\mathbb{K}$ befinden.

Wir können Vektoren miteinander Addieren und diese mit Skalaren $t, s, r$ strecken, ohne dass diese den Vektorraum verlassen. So sind $x+y\in V$ und $t*x\in V$.

\begin{enumerate}
    \item Es gibt immer einen Nullvektor, $x+0=x$
    \item Es gibt immer eine Inverse, $x+(-x)=0$
    \item Es gibt das Einselement $1\in \mathbb{K}$, $1*x=x$
\end{enumerate}

\subsubsection{Teilmengen des Vektorraums}

Eine Teilraum $W\subset V$ besteht dann, wenn wir durch Addition oder Multiplikation mit Skalaren, diesen Teilraum nicht verlassen.

Dafür kennen wir die Linearkombination:

\begin{center}
    $s*x+t*y\in W$
\end{center}

\subsection{Vektorräume mit Norm und Skalarprodukt}

Hier wird im Skript der Zusammenhang zwischen Norm und Skalarprodukt im allgemeinem Vektorraum gezeigt.

Mit vielen Beispielen und Beweisen, welche ich hier nicht übernehmen werde.

\section{Matrizen}

\subsection{Lineare Gleichungssysteme und das GAUSS-Verfahren}

Das unten Aufgeführte GAUSS-Verfahren ist mit Korrekturspalte, für die Korrekturfaktoren der Determinante einer Matrix.

Hier wird sich zudem die Ergebnisspalte von $2x_1+5x_3+7x_4=0$ gedacht.

\begin{center}
    \begin{tabular}{l|cccc|l|c}
              & $x_1$ & $x_2$ & $x_3$ & $x_4$ & Aufgaben        &                \\
        \hline
        $I$   & 2     & 0     & 5     & 7     &                 &                \\
        $II$  & 4     & 1     & 3     & 6     & $II-2*I$        &                \\
        $III$ & -3    & 3     & 6     & 2     & $2*III+3*I$     & $\frac{1}{2}$  \\
        $IV$  & 6     & 0     & 4     & 4     & $IV-3*I$        &                \\
        \hline
              & 2     & 0     & 5     & 7     &                 &                \\
              & 0     & 1     & -8    & -8    &                 &                \\
              & 0     & 6     & 27    & 23    & $III - 6*II$    &                \\
              & 0     & 0     & -12   & -17   &                 &                \\
        \hline
              & 2     & 0     & 5     & 7     &                 &                \\
              & 0     & 1     & -8    & -8    &                 &                \\
              & 0     & 0     & 75    & 71    &                 &                \\
              & 0     & 0     & -12   & -17   & $25*IV + 4*III$ & $\frac{1}{25}$ \\
        \hline
              & 2     & 0     & 5     & 7     &                 &                \\
              & 0     & 1     & -8    & -8    &                 &                \\
              & 0     & 0     & 75    & 71    &                 &                \\
              & 0     & 0     & 0     & -141  &                 &                \\
        \hline
    \end{tabular}
\end{center}

Das allgemein GAUSS-Verfahren wird nicht nur bis zur oberen Dreiecksform gerechnet, aber an dieser kann man etwas über die Lösungen aussagen:

\begin{center}
    \begin{tabular}{ccc}
        \begin{tabular}{|ccccc|c|}
            \hline
            . & . & . & . & . & . \\
              & . & . & . & . & . \\
              &   & . & . & . & . \\
              &   &   & . & . & . \\
              &   &   &   & . & . \\
              &   & 0 &   &   & 0 \\
            \hline
        \end{tabular} &
        \begin{tabular}{|ccccc|c|}
            \hline
            . & . & . & . & . & . \\
              & . & . & . & . & . \\
              &   & . & . & . & . \\
              &   &   & . & . & . \\
              &   & 0 &   &   & 0 \\
              &   & 0 &   &   & 0 \\
            \hline
        \end{tabular} &
        \begin{tabular}{|ccccc|c|}
            \hline
            . & . & . & . & . & . \\
              & . & . & . & . & . \\
              &   & . & . & . & . \\
              &   &   & . & . & . \\
              &   & 0 &   &   & . \\
              &   & 0 &   &   & 0 \\
            \hline
        \end{tabular}    \\\\
        genau eine Lösung          &
        $\infty$ viele Lösungen    &
        keine Lösung
    \end{tabular}
\end{center}

\know{Inhomogene \& Homogene Lösungen}{Die Inhomogene Lösung ist die, die nur aus Zahlen besteht.

    Die Homogenen Lösungen sind die, die an eine Variable gebunden sind. Diese gibt es nur bei $\infty$ viele Lösungen.}

\subsection{Die Matrix zum LGS}

Eine $n\times m$-Matrix lässt sich mit einem $n$ großem Vektor multiplizieren:

\begin{center}
    $\begin{bmatrix}
            a_{11} & a_{12} & \dots  & a_{1n} \\
            a_{21} & a_{22} & \dots  & a_{2n} \\
            \vdots & \vdots & \ddots & \vdots \\
            a_{m1} & a_{m2} & \dots  & a_{mn}
        \end{bmatrix}*
        \begin{bmatrix}
            x_1 \\x_2\\\vdots\\x_n
        \end{bmatrix}$
\end{center}

Die Multiplikation kann man sich vorstellen, indem wir den Vektor auf die MAtrix raufkippen und dann Ziele für Zeile Reinmultiplizieren. Das Ergebnis ist wieder eine Matrix.

\subsection{Das Schema zum GAUSS-Verfahren}

\begin{center}
    \begin{tabular}{|l|cccc|l|}
        \hline
           & $x_1$ & $x_2$ & $x_3$ &    &        \\
        \hline
        I  & 1     & 2     & 1     & 2  & I-2*II \\
        II & 0     & 1     & 3     & -4 &        \\
        \hline
        I  & 1     & 0     & -5    & 10 &        \\
        II & 0     & 1     & 3     & -4 &        \\
        \hline
    \end{tabular}
\end{center}

Mit diesem Ergebnis können wir nun einen Lösungsvektor erstellen:

\begin{equation*}
    \begin{bmatrix}
        x_1 \\x_2\\x_3
    \end{bmatrix}=
    \begin{bmatrix}
        10+5x_3 \\
        -4-3x_3 \\
        x_3
    \end{bmatrix}=
    \begin{bmatrix}
        10 \\-4\\0
    \end{bmatrix}+x_3
    \begin{bmatrix}
        5 \\-3\\1
    \end{bmatrix}
\end{equation*}

Hier ist der erste Vektor die Inhomogene Lösung und der zweiter, welche an $x_3$ gebunden ist der einzige Homogen.

Somit ist die Lösungsmenge:

\begin{equation*}
    L=\left\{
    \begin{bmatrix}
        10 \\-4\\0
    \end{bmatrix}+t
    \begin{bmatrix}
        5 \\-3\\1
    \end{bmatrix}\mid t\in \mathbb{R}
    \right\}
\end{equation*}

\subsection{Lineare Unabhängigkeit}

Eine Nullkombination $\lambda_1b_1+\lambda_2b_2+\dots+\lambda_nb_n=0$ heißt linear unabhängig, wenn diese nur durch die trivial-Lösung $\lambda_1=\lambda_2=\dots=\lambda_n=0$ zu erfüllen ist.

% Hier muss ich weiter arbeiten!

\subsection{Basis und Dimension}

\subsection{Matrizen als lineare Abbildung}

\subsection{Direkte Zerlegung eines Vektorraum}

\subsection{Die Dimensionsformel}

\subsection{Die inverse Matrix}

\subsection{Die adjungierte Matrix}

\subsection{Koordinatentransformation}

\subsection{Determinanten}

\section{Eigenwerttheorie}

\subsection{Spektrum und Eigenvektoren}

\subsection{Selbtsadjungierte lineare Abbildungen}

\subsection{Funktionalkalkül}

\subsection{Normale Abbildungen}

\end{document}
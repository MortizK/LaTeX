\documentclass[a4paper]{article}

%\usepackage{url}

%% Math
\usepackage{mathtools}
%% For Mengen like natural numbers
\usepackage{amsfonts}
%% Für spezielle Symbole
\usepackage{amssymb}

%% Images
\usepackage{import}
\usepackage{xifthen}
\usepackage{pdfpages}
%\usepackage{transparent}

%%% Command for simpler images
\newcommand{\incfig}[1]{%
    \def\svgwidth{\columnwidth}
    \import{./fig/}{#1.pdf_tex}
}

%% Links
\usepackage{hyperref}
\hypersetup{
    colorlinks=true,
    linkcolor=black,
    filecolor=magenta,
    urlcolor=cyan
}

%% Formatting
\usepackage{parskip}

\title{Lineare Algebra}
\author{Moritz}

\setcounter{tocdepth}{2}

\begin{document}
\maketitle
\tableofcontents

\section{Ziele}

Diese Dokument soll nicht nochmal alles im Detail aufschrieben, was wir gemacht haben, sondern eine Sammlung sein um sich auf die Prüfung vorzubereiten.

Dies dient auch als Vorarbeit für meinen A4 Spickzettel.

\subsection{Altklausuren}

Ich werde Altklausuren analysieren um die Hauptaufgaben zu filtern und in mehr Detail zu analysieren und üben.

\subsection{Formelsammlung/ Tools}

Die Idee ist es eine Sortierte Formelsammlung zu jedem Unten aufgelisteten Themengebiet zu erstellen.

\subsection{Begriffsverzeichnis}

Zudem werde ich wie Formel auch Begriffe sammel und lernen, um nicht mehr darüber nachdenken zu müssen.

\subsection{Bildverzeichnis}

Von hilfsreichen Grafischen Darstellungen, wie z.B. Alle Formel im Dreieck.

\subsection{Rechnungen}

Vorgerechnete Aufgaben in dem Korrekten Klausuraufschrieb.

\section{Grundlagen}

\subsection{Aussagenlogik}

\know{Aussagen}{Wir können Aussagen als Sprachliches Konstrukt bauen. Und diese dan durch Nachdenken als entweder wahr oder falsch bestimmen. Wichtig: Manche Aussagen sind bis heute nicht bewiesen worden und es gibt keinen wiederspruch.}

\begin{center}
    \begin{tabular}{c c|c|c|c|c|c}
        $A$ & $B$ & $\neg A$ & $A\land B$ & $A\lor B$ & $A\implies B$ & $A\iff B$ \\
        \hline
        0   & 0   & 1        & 0          & 0         & 1             & 1         \\
        0   & 1   & 1        & 0          & 1         & 1             & 0         \\
        1   & 0   & 0        & 0          & 1         & 0             & 0         \\
        1   & 1   & 0        & 1          & 1         & 1             & 1         \\
    \end{tabular}
\end{center}

\know{Tautologie}{Sind Aussagen, die immer wahr sind. Es gibt auch das Gegenstück.}

\subsubsection{De Morgansche Regeln}

\begin{enumerate}[i.]
    \item $\overline{A\land B}\iff\overline{A}\lor\overline{B}$
    \item $\overline{A\lor B}\iff\overline{A}\land\overline{B}$
    \item $A\land(B\lor C)\iff(A\land B)\lor(A\land C)$
    \item $A\lor(B\land C)\iff(A\lor B)\land(A\lor C)$
\end{enumerate}

\subsubsection{Aussageformen}

\begin{center}
    \begin{tabular}{r l}
        $\exists_{x\in X}{P(x)}$ & Es existiert ein $x$ aus $X$, für das $P(x)$ gilt. \\
        $\forall_{x\in X}{P(x)}$ & Für alle $x$ aus $X$ gilt $P(x)$.
    \end{tabular}
\end{center}

Somit können wir Mathematische Aussagen treffen, wie:

\begin{equation}
    A:= \forall_{n\in \mathbb{N}}{D(n)}= \forall_{n\in \mathbb{N}}{1+2+3\dots n = \frac{1}{2}n*(n+1)}
\end{equation}

Quantoren lassen sich auch negieren. Hierzu muss nur der Quantor umgetauscht werden und die Aussage negiert werden.

\subsection{Mengen}

Mengen lassen sich durch das Angeben ihrer Elemente Darstellen:

\begin{center}
    $M_1=\{1, 2, 3, 4, 5, 6\}$
\end{center}

\begin{center}
    \begin{tabular}{c l}
        $m\in M$       & Das Element $m$ gehört zur Menge $M$         \\
        $\in$          & gehört zu                                    \\
        $\notin$       & gehört nicht zu                              \\
        $A\subseteq B$ & $A$ ist eine Teilmenge von $B$               \\
        $A\subset B$   & $A$ ist eine Teilmenge von $B$ und kleiner   \\
        $A=B$          & dann gilt: $A\subseteq B$ und $B\subseteq A$ \\
    \end{tabular}
\end{center}

\know{Abhängige Menge}{Wir können Menge auch beschreiben, indem wir von bekannten Mengen aussgehen, wie z.B. von der Menge der Natürlcihen Zahlen $\mathbb{N}$
    \begin{center}
        $A=\{a\in \mathbb{N}\mid P(a)\}$
    \end{center}}

Mit diesem Wissen lassen sich alle Geraden Zahlen darstellen:

\begin{center}
    $G:=\{n\in \mathbb{N}\mid \exists_{k\in \mathbb{N}}{n=2k}\}$
\end{center}

\subsubsection{Intervalle}

Intervalle sind eine Bestimmte Art von Mengen.

\begin{center}
    $I:=\{x\in \mathbb{R}\mid a\leq x\leq b\}$
\end{center}

\begin{equation}
    \begin{split}
        [a, b]       & := \{x\in \mathbb{R}\mid a\leq x\leq b\} \\
        [a, b)       & := \{x\in \mathbb{R}\mid a\leq x < b\}   \\
        (-\infty, b] & := \{x\in \mathbb{R}\mid x\leq b\}       \\
        (-\infty, b) & := \{x\in \mathbb{R}\mid x < b\}
    \end{split}
    \quad
    \begin{split}
        (a, b)      & := \{x\in \mathbb{R}\mid a < x < b\}   \\
        (a, b]      & := \{x\in \mathbb{R}\mid a < x\leq b\} \\
        [a, \infty) & := \{x\in \mathbb{R}\mid a\leq x\}     \\
        (a, \infty) & := \{x\in \mathbb{R}\mid a < x\}
    \end{split}
\end{equation}

Abgeschlossene Intervalle haben Grenzen $[a, b]$, offene nicht. Zudem sind diese beschränkt, anders als $[a, \infty)$, welches unbeschränkt ist.

\know{Konstruktionsvorschrift}{Bei Aufwändigeren Menge, wie die Mersenne Zahlen: $M:=\{x\in \mathbb{R}\mid \exists_{k\in \mathbb{R}{n=2^k-1}}\}$. Diesen Schreibaufwand können wir uns nun sparen:
    \begin{center}
        $M:=\{2^k-1\mid k\in \mathbb{R}\}$
    \end{center}}

\subsubsection{Produktmenge}

Die Produktmenge beschreibt eine Menge, $A\times B$ wo alle Kombinationen von $a\in A$ und $b\in B$ gelten. Wir schreiben:

\begin{center}
    $A\times B:={[a, b]\mid a\in A\land b\in B}$
\end{center}

Zur Vorstellung hilft es sich zwei Intervalle vorzustellen und diese im Koordinatensystem zu markieren:

\begin{figure}[ht]
    \centering
    \incfig{produktmenge}
    \caption{produktmenge}
    \label{fig:Prdouktmenge}
\end{figure}

\subsubsection{Mengenlogik}

\begin{center}
    \begin{tabular}{r l l l}
        $A\cap B$ & $:=\{x\in X\mid x\in A \lor x\in B\}$     & Durchschnitt    \\
        $A\cup B$ & $:=\{x\in X\mid x\in A \land x\in B\}$    & Vereinigung     \\
        $A^c$     & $:=\{x\in X\mid \neg x\in A\}$            & Komplement      \\
        $A/ B$    & $:=\{x\in X\mid x\in A \land x\notin B\}$ & Mengendifferenz \\
    \end{tabular}
\end{center}

Der Durchschnitt sind die Gemeinsamkeiten der beiden Mengen. Die Vereinigung sind alle Elemente aus $A$ und $B$. Das Komplement sind alle Elemente außer in der Menge. Die Mengendifferenz sind alle Elemente aus $A$, ohne die gemeinsamen Elemente mit $B$.

\know{De Morgansche Regeln}{Es gelten die Gleichen Rechenregeln wie in der Aussagenlogik, so zeigt sich, dass das Komplement $A^c$ gleichwertig ist mit dem $\overline{A}$.
    \begin{center}
        $(A\cup B)^c=A^c\cap B^c$
    \end{center}}

\subsubsection{Die Natürlichen Zahlen $\mathbb{N}$}

Die Natürlichen Zahlen bauen vollständig auf der leeren Menge auf. So ist die $0:=\emptyset$ und die $1:=\{\emptyset\}$ und die $2:=\{\emptyset, \{\emptyset\}\}$ usw. Das allgemeine Verfahren:

\begin{center}
    $n'=n\cup \{n\}$
\end{center}

\subsubsection{Vollständige Induktion}

Wir können mit zwei Eigenschaften einer Menge zeigen, dass diese eine Teilmenge von $\mathbb{N}$ ist.

\begin{enumerate}[I]
    \item $1\in M$
    \item $n\in M \implies n + 1 \in M$
\end{enumerate}

Aus diesen Beiden Eigenschaften können wir folgern, dass $M=\mathbb{N}$ gilt. Nun gibt es für jedes $n\in \mathbb{N}$ eine Aussage $A_n$. Dies können wir beweisen, indem wir die beiden Eigenschaften $I$ und $II$ beweisen.

\textbf{Beweis:} $1+2+\dots+n=\frac{1}{2}n(n+1)$

\begin{center}
    \begin{tabular}{r l}
        $n=1$:      & $1=\frac{1}{2}*1(1+1)$ was off. wahr ist       \\
        $n\to n+1$: & z.z. $1+2+\dots+n+(n+1)=\frac{1}{2}(n+1)(n+2)$ \\
    \end{tabular}
\end{center}

\begin{equation}
    \begin{split}
        1+2+\dots+n+(n+1) & _{=}^{IV} \frac{1}{2}n(n+1)+(n+1)       \\
                          & = \frac{1}{2}n(n+1)+1*(n+1)             \\
                          & = \frac{1}{2}n(n+1)+2*\frac{1}{2}*(n+1) \\
                          & = \frac{1}{2}(n+1)(n+2)
    \end{split}
\end{equation}

\subsubsection{Binomischer Lehrsatz}

\begin{center}
    $\binom{n}{k}:=\frac{n!}{k!(n-k)!}$
\end{center}

Hierbei muss beachtet werden, dass $n!$ für $0!=1$ definiert ist. Für $1\leq k\leq n$ gilt zusätzlich die Konstruktionsvorschrift.

\begin{center}
    $\binom{n+1}{k}=\binom{n}{k-1}+\binom{n}{k}$
\end{center}

Diese lässt sich visuell am Pascalschem Dreieck erkennen. Denn dort entsteht Jede folgende Zahl aus den beiden diagonal darüberlegenden Zahlen.

\begin{center}
    \begin{tabular}{c c c c c c c c c c c}
          &   &   &   &    & 1 &    &   &   &   &   \\
          &   &   &   & 1  &   & 1  &   &   &   &   \\
          &   &   & 1 &    & 2 &    & 1 &   &   &   \\
          &   & 1 &   & 3  &   & 3  &   & 1 &   &   \\
          & 1 &   & 4 &    & 6 &    & 4 &   & 1 &   \\
        1 &   & 5 &   & 10 &   & 10 &   & 5 &   & 1 \\
    \end{tabular}
\end{center}

Es wird auch deutlich, dass die Summe alle Binomialkoeffizienten $2^n$ sein muss, da sich mit jeder Ebene der Gesamtwert verdoppelt.

\section{Relationen und Funktionen}

\subsection{Allgemeine Eigenschaften von Relationen}

\know{Relation}{Eine zweistellige Relation $R$ von $A\times B$:
    \begin{center}
        $R\subseteq A\times B$
    \end{center}
    Man sagt jedes $a\in A$ steht somit in Relation zu jedem $b\in B$}

Ich kann mir Relationen also so verstehen, wie man sich das vorstellt. Somit den gleichen Ansatz wie Relationen in einer Datenbank.

\subsection{Klassifikation von Relationen}

\begin{tabular}{r l}
    homogen         & $A=B$, alle hier müssen homogen sein!            \\
    reflexiv        & Für alle $a\in A$ ist $[a, a]\in R$              \\
    symmetrisch     & $[a, b]\in R$, gibt es auch $[b, a]\in R$        \\
    antisymmetrisch & $[a, b]\in R \land [b, a]\in R$ folgt $a=b$      \\
    transitiv       & $[a, b]\in R$, $[b, c]\in R$ folgt $[a, c]\in R$
\end{tabular}

Die weiteren müssen nicht unbedingt homogen sein:

\begin{tabular}{r l}
    linkstotal  & Für alle $a\in A$ gibt es ein $b\in B$, so dass $[a, b]\in R$       \\
    rechtstotal & Für alle $b\in B$ gibt es ein $a\in A$, so dass $[a, b]\in R$       \\
    funktional  & aus $[a, b]\in R$ und $[a, c]\in R$ folgt $b=c$                     \\
    injektiv    & aus $[a, b]\in R$ und $[c, b]\in R$ folt $a=c$                      \\
    bijektiv    & für alle $b\in B$ gibt es genau ein $a\in A$, so dass $[a, b]\in R$
\end{tabular}

Visualisieren kann man diese in Tabellen:

\begin{center}
    \begin{tabular}{c c c}
        \begin{tabular}{c|c c c c}
              & 1 & 2 & 3 & 4 \\
            \hline
            1 & 1 &   &   &   \\
            2 &   & 1 &   &   \\
            3 &   &   & 1 &   \\
            4 &   &   &   & 1 \\
        \end{tabular} &
        \begin{tabular}{c|c c c c}
              & 1 & 2 & 3 & 4 \\
            \hline
            1 &   & 1 &   &   \\
            2 & 1 &   & 1 &   \\
            3 &   & 1 &   & 1 \\
            4 &   &   & 1 &   \\
        \end{tabular} &
        \begin{tabular}{c|c c c c}
              & 1 & 2 & 3 & 4 \\
            \hline
            1 &   &   &   &   \\
            2 & 1 &   &   &   \\
            3 & 1 & 1 &   &   \\
            4 &   &   &   &   \\
        \end{tabular}    \\
        reflexiv                   &
        symmetrisch                &
        transitiv                    \\
        Diagonalen                 &
        Spiegelbild                &
        ZUgnetz                      \\\\
        \begin{tabular}{c|c c c c}
              & 1 & 2 & 3 & 4 \\
            \hline
            1 &   & 1 & 1 & 1 \\
            2 & 1 &   &   &   \\
            3 & 1 &   &   &   \\
            4 &   &   & 1 &   \\
        \end{tabular} &
        \begin{tabular}{c|c c c c}
              & 1 & 2 & 3 & 4 \\
            \hline
            1 &   &   &   & 1 \\
            2 &   & 1 &   &   \\
            3 & 1 &   & 1 &   \\
            4 &   &   &   &   \\
        \end{tabular} &
        \begin{tabular}{c|c c c c}
              & 1 & 2 & 3 & 4 \\
            \hline
            1 &   &   &   & 1 \\
            2 &   & 1 &   &   \\
            3 & 1 &   &   &   \\
            4 &   &   &   & 1 \\
        \end{tabular}    \\
        linkstotal                 &
        funktional                 &
        injektiv                     \\
        Jedes a                    &
        Zeilen                     &
        Spalten
    \end{tabular}
\end{center}

\subsection{Äquivalenzrelationen}

Eine Äquivalenzrelationen ist eine Relation, welche Reflexiv, Transitiv und Symmetrisch ist.

\subsection{Verkettung von Relation und die inverse Relation}

\subsection{Funktionen}

\section{Zahlentheorie}

\subsection{Teilbarkeitstheorie ganzer Zahlen}

Wir können die Teilbarkeit auch durch die Umstellung ausdrücken mit $a, b, c\in \mathbb{Z}$

\begin{center}
    $b=c*a$
\end{center}

Wir schreiben: $a|b$, $a$ teilt $b$.

Wir können diese Regeln auch erweitern, denn jede division, kann in eine eindeutige Schreibweise mit Rest formuliert werden.

\begin{center}
    \begin{tabular}{c c}
        $p=tq+r$, & $0\leq r < q$
    \end{tabular}
\end{center}

\subsubsection{Teilbarkeitssätze}

\begin{enumerate}[i.]
    \item $p|ab\land ggT(p, a)=1 \implies p|b$
    \item $p|ab\land p\in \mathbb{P} \implies p|a\lor p|b$
\end{enumerate}

\subsubsection{ggT und euklidischer Algo.}

Der größte gemeinsame Teiler beschreibt die größte $t$, für mehrere Zahlen $p$. Wenn $ggT(p_1, p_2)=1$ sind $p_1$ und $p_2$ teilerfremd.

ggT kann durch wiederholtes teilen mit Rest bestimmt werden.

\begin{center}
    \begin{tabular}{r c l c r c l}
        $84$ & $=$ & $4*18 + 12$ & $\implies$ & $12$ & $=$ & $84 - 4*18$ \\
        $18$ & $=$ & $1*12 + 6$  & $\implies$ & $6$  & $=$ & $18 - 1*12$ \\
        $12$ & $=$ & $2*6 + 0$   &            &      &     &             \\
    \end{tabular}
\end{center}

Somit ist der $ggT(84, 18)=6$. Der $ggT$ ist auch für alle Variationen von $84, -84$ mit $18, -18$ gleich $6$.

\know{Euklid}{Es gibt zu jedem $ggT$ auch die Form
    \begin{center}\begin{tabular}{c c}
            $ggT(p, q)=xp+yq$, $x, y\in \mathbb{Z}$
        \end{tabular}\end{center}}

Um in diese Form zu kommen, können wir den erweiterten euklidischen Algorithmus anwenden. Dieser stellt wie ob, die Gleichung einfach um und setzt die oberen Gleichung in die unteren ein.

Wenn $ggT(p, q)=1$ ist die gesuchte Form $1=xp+yq$

\begin{center}
    $6=18-1*(84-4*18)=3*18-1*84$
\end{center}

\subsection{Rechnen Modulo p}

Zwei Zahlen Modulo $p$ sind genau dann gleich, wenn Sie sich um ein vielfaches von $p$ unterscheiden:

\begin{center}
    $x=_{p}y \iff \exists t\in \mathbb{Z}: x=y+tp$
\end{center}

Die Rechnung modulo $p$ ist eine Äquivalenzrelationen, welche eine unendlich Große Menge auch eine endliche Menge reduziert.

\begin{center}
    $\mathbb{Z}_p=\{0,1,2,3,\dots,p-1\}$
\end{center}

\subsubsection{Addition}

Wir können Zahlen modulo $p$ mit einander addieren und subtrahieren. $x_1=_px_2$ und $y_1=_py_2$. Beide haben die Form $x_1=x_2+tp$, wenn wir diese Addieren oder Subtrahieren, bekommen wir:

\begin{center}
    $x_1\pm y_1=x_2+tp\pm y_2\pm sp=_px_2\pm y_2$
\end{center}

\subsubsection{Multiplikation}

Die Multiplikation funktioniert in modulo $p$ auch:

\begin{center}
    $x_1*y_1=(x_2+tp)(y_2+sp)=x_2y_2+tpy_2+spx_2+tsp^2=_px_2y_2$
\end{center}

\subsubsection{Inverse modulo p}

Die Inverse modulo $p$ ist definiert als $ab=_p1$, dabei muss $p>1$ und für jedes $a\in \mathbb{Z}_p^{*}$ muss $ggT(a, p)=1$ gelten.

\begin{center}
    \begin{tabular}{c c}
        \begin{tabular}{c|c c c c c c c}
            $*/7$ & 1 & 2 & 3 & 4 & 5 & 6 \\
            \hline
            1     & 1 & 2 & 3 & 4 & 5 & 6 \\
            2     & 2 & 4 & 6 & 1 & 3 & 5 \\
            3     & 3 & 6 & 2 & 5 & 1 & 4 \\
            4     & 4 & 1 & 5 & 2 & 6 & 3 \\
            5     & 5 & 3 & 1 & 6 & 4 & 2 \\
            6     & 6 & 5 & 4 & 3 & 2 & 1 \\
        \end{tabular} &
        \begin{tabular}{c|c c c c}
            $*/12$ & 1  & 5  & 7  & 11 \\
            \hline
            1      & 1  & 5  & 7  & 11 \\
            5      & 5  & 1  & 11 & 7  \\
            7      & 7  & 11 & 1  & 5  \\
            11     & 11 & 7  & 5  & 1  \\
        \end{tabular}
    \end{tabular}
\end{center}

Aus diesen Tabellen lassen sich die inversen leicht auslesen.

\know{Kleiner Satz von FERMAT}{Dieser zeigt, dass für jedes $p\in \mathbb{P}$, $a^p=_pa$ gilt. Wenn zudem noch $ggT(a, p)=1$, dann gilt zusätzlich $a^{p-1}=_p1$. Dies erhalten wir durch erweitern mit der inversen $b$, wo $ab=1$ gilt:
    \begin{center}
        $a^{p-1}=_pa^{p-1}*ab=a^pb=_pab=_p1$
    \end{center}}

Für zwei unterschiedliche Primzahl $p,q$ gilt $a=_{pq}b$, wenn $a=_pb$, $a=_qb$ gelten. Mit dem kleinem Satz von FERMAT gilt unter der Bedingung, dass $ggT(a, pq)=1$ zusätzlich:

\begin{center}
    $a^{(p-1)(q-1)}=_{pq}1$
\end{center}

\subsection{RSA-Verschlüsselung}

\begin{enumerate}[i.]
    \item Wähle zwei verschiedene Primzahlen $p, q$ und bilde $n:= p*q$.
    \item Wähle $1<e<\varphi:=(p-1)(q-1)$ mit $ggT(\varphi, e)=1$. Dann ist $S_o:=[e, n]$
    \item Bilde die Inverse $d$ von $e$ modulo $\varphi$ (mit erw. euklid. Alg.) $d=_{\varphi}e^{-1}$. Dann ist $S_p:=[d, n]$

          Danach ist die Schlüsselgeneration abgeschlossen. $p$ und $q$ können gelöscht werden. $S_p$ muss verschlüsselt auf dem Rechner gespeichert werden.
    \item Eine Nachricht $0<M<n$ wird durch den öffentlichen Schlüssel ($S_o$) verschlüsselt:
          \begin{center}
              $C:=_n M^{e}$
          \end{center}
          $C$ wird mit $S_p$ entschlüsselt:
          \begin{center}
              $M=_n C^{d}$
          \end{center}
\end{enumerate}

\subsubsection{Rechnungen}

Es gibt einen online Rechner zur Korrektur: \href{https://rsa-calculator.netlify.app/}{RSA-Calculator}

$p=101, q=17$ und $e=411$ sind gegeben und die beiden Schlüssel sollen bestimmt werden.

Für den Öffentlichen Schlüssel muss nur noch $n=p*q$ bestimmt werden. Somit ist $n=101*17=1717$ und falls $ggT(\varphi, e)=1$, gilt $S_o=[e, n]$. $\varphi=(p-1)(q-1)=100*16=1600$.

\begin{equation}
    \begin{split}
        1600 & = 3*411 + 367 \\
        411  & = 1*367 + 44  \\
        367  & = 8*44 + 15   \\
        44   & = 2*15 + 14   \\
        15   & = 1*14 + 1    \\
    \end{split}
    \qquad\qquad
    \begin{split}
        367 & = 1600 - 3*411 \\
        44  & = 411 - 1*367  \\
        15  & = 367 - 8*44   \\
        14  & = 44 - 2*15    \\
        1   & = 15 - 1*14    \\
    \end{split}
\end{equation}

$S_o=[411, 1717]$. Für den Privaten Schlüssel brauchen wir die Darstellung nach EUKLID:

\begin{equation}
    \begin{split}
        1 & = 15 - 1*14 = 15 - 1*(44 - 2*15)= 3*15 - 1*44 \\
          & =  3*(367 - 8*44) - 1*44 = 3*367 - 25*44      \\
          & =  3*367 - 25*(411 - 1*367) = 28*367 - 25*411 \\
          & =  28*(1600 - 3*411) - 25*411                 \\
          & = 28*1600 - 109*411
    \end{split}
\end{equation}

$d=1600-109=1491$. Daraus ergibt sich $S_p=[1491, 1717]$

Zum Verschlüsseln und Entschlüsseln gilt der gleiche Ansatz, wir brauchen nur noch ein $M=405$:

\begin{equation}
    \begin{split}
        C & =_n M^e                                                    \\
          & =_{1717} 405^{411} = 405*405^{410} = 405*(405^2)^{205}     \\
          & =_{1717} 405*910^{205} \text{ Es gilt: } 405^2=_{1717} 910 \\
          & =_{1717} 409*910*910^{204} =409*910*(910^2)^{102}          \\
          & =_{1717} 1112*506^102 = 1112 * (506^2)^{51}                \\
          & =_{1717} 1112 * 203^{51} = 1112*203 * (203^2)^{25}         \\
          & =_{1717} 809 * 1^{25} = 809
    \end{split}
\end{equation}

Ist identisch für das Entschlüsseln nur mit anderem Ansatz

\begin{equation}
    \begin{split}
        M & =_n C^d                                 \\
          & =_{1717} 809^{1491} = 809*(809^2)^{745} \\
          & =_{1717} 809*304*(304^2)^{372}          \\
          & =_{1717} 405 * (1415^2)^{186}           \\
          & =_{1717} 405 * (203^2)^{93}             \\
          & =_{1717} 405 * 1^{93} = 405
    \end{split}
\end{equation}

\subsection{Chinesische Restsatz}



\section{Kombinatorik}

\subsection{Die Urnenmodelle}

\section{Vektorräume}

\subsection{$\mathbb{R}^2$ und $\mathbb{R}^3$ als Vektorraum}

\subsection{Die komplexen Zahlen}

\subsection{$\mathbb{C}^n$ als Vektorraum}

\subsection{Der allgemeine Vektorraumbegriff$^*$}

\subsection{Vektorräume mit Norm und Skalarprodukt}

\section{Matrizen}

\subsection{Lineare Gleichungssysteme und das GAUSS-Verfahren}

\subsection{Die Matrix zum LGS}

\subsection{Das Schema zum GAUSS-Verfahren}

\subsection{Lineare Unabhängigkeit}

\subsection{Basis und Dimension}

\subsection{Matrizen als lineare Abbildung}

\subsection{Direkte Zerlegung eines Vektorraum}

\subsection{Die Dimensionsformel}

\subsection{Die inverse Matrix}

\subsection{Die adjungierte Matrix}

\subsection{Koordinatentransformation}

\subsection{Determinanten}

\end{document}
\documentclass[a4paper]{article}

%\usepackage{url}

%% Math
\usepackage{mathtools}
%% For Mengen like natural numbers
\usepackage{amsfonts}
%% Für spezielle Symbole
\usepackage{amssymb}

%% Images
\usepackage{import}
\usepackage{xifthen}
\usepackage{pdfpages}
%\usepackage{transparent}

%%% Command for simpler images
\newcommand{\incfig}[1]{%
    \def\svgwidth{\columnwidth}
    \import{./fig/}{#1.pdf_tex}
}

%% Links
\usepackage{hyperref}
\hypersetup{
    colorlinks=true,
    linkcolor=black,
    filecolor=magenta,
    urlcolor=cyan
}

%% Formatting
\usepackage{parskip}

\title{Lineare Algebra}
\author{Moritz}
\date{January 8, 2025}

\begin{document}
\maketitle
\tableofcontents

\section{Wiederholung}

\subsection{Lemma}

\begin{equation}
    \begin{split}
        b\neq q\in \mathbb{P} \\
        a=_pb\land a=_qb\implies a=_{pq}b
    \end{split}
\end{equation}

\begin{equation}
    \begin{split}
        ggT(a, pq)=1\implies a^{(p-1)(q-1)}=_{pq}1
    \end{split}
\end{equation}

\section{RSA Verfahren}

Name kommt von den Entwicklern: Rivest, Shamir, Adleman.

1. Wähle zwei verschiedene Primzahlen $p, q$ und bilde $n:= p*q$.

2. Wähle $1<e<\varphi:=(p-1)(q-1)$ mit $ggT(\varphi, e)=1$. Dann ist $S_o:=[e, n]$

3. Bilde die Inverse $d$ von $e$ modulo $\varphi$ (mit erw. euklid. Alg.) $d=_{\varphi}e^{-1}$. Dann ist $S_p:=[d, n]$

Danach ist die Schlüsselgeneration abgeschlossen. $p$ und $q$ können gelöscht werden. $S_p$ muss verschlüsselt auf dem Rechner gespeichert werden.

4. Eine Nachricht $0<M<n$ wird durch den öffentlichen Schlüssel ($S_o$) verschlüsselt: $C:=_n M^{e}$

5. $C$ wird mit $S_p$ entschlüsselt: $M=_n C^{d}$

\subsection{Beweis}

Nur 5.: 1. $ggT(M, pq)=1$, d.h. wir wissen $M^{(p-1)(q-1)}=_n 1$

$C^{d}=_n (M^{e})^{d}=M^{ed}$

Wir wissen $ed=_{\varphi}1$, d.h. $ed = 1 + t\varphi= 1+t(p-1)(q-1)$.

Also: $M^{ed}=M^{1+t(p-1)(q-1)}=M*M^{(p-1)(q-1)t}=_n M*1=M$

2. OBdA $ggT(M, pq)=\substack{\*\\p}\implies ggT(M, q)=1\implies M^{q-1}=_q 1\implies (m^{q-1})^{p-1}=M^{(p-1)(q-1)}=_q 1\implies M^{(p-1)(q-1)t}=_q 1$.

d.h. $M^{(p-1)(q-1)t}=1+rq$

Also: $M^{ed}=M*M^{(p-1)(q-1)t}=M(1+rq)=M+rMq\substack{\*\\=}M+rspq=_n M$

\subsection{Rechnung}

$p=101, q=17, M=405, e=411$

$e=-109*411+28*1600=_{\varphi}-109*411\implies d=1600-109=1491$

\subsubsection{Grundrechnung}

Immer wieder einzelne Produkte modulo $n$ hier $1717$ Rechnen.

\begin{equation}
    C=_{1717}405^{411}=405*(405^2)^{205}=_{1717}405*910^{205}
\end{equation}

\section{Aufräumen}

\subsection{Winkel}

Winkeleinheiten 360°$=2\pi$ Bogenmaß ist einfach die Strecke des Bogens, welcher der Winkel in einem Einheitskreis hat.

1°=60' (Minuten)

1'=60'' (Sekunden)

\begin{equation}
    \begin{array}{c|c}
        deg^\circ & rad                        \\
        \hline
        0^\circ   & 0                          \\
        360^\circ & 2\pi                       \\
        180^\circ & \pi                        \\
        \alpha    & b = \frac{\pi}{180}*\alpha \\
    \end{array}
\end{equation}

\subsection{Sinus und Cosinus}

Trigonometrische Funktionen:

\begin{figure}[ht]
    \centering
    \incfig{rechtwinkliges_dreieck}
    \caption{Rechtwinkliges Dreieck}
    \label{fig:rechtwinkliges_dreieck}
\end{figure}

Die Verhältnisse zweier Seiten hängen nicht vom Maßstab ab.

\begin{equation}
    \begin{split}
        \frac{b}{c} & =\sin{\alpha}                                                                         \\
        \frac{a}{c} & =\cos{\alpha}                                                                         \\
        \frac{b}{a} & =\tan{\alpha} = \frac{b*\frac{1}{c}}{a*\frac{1}{c}}=\frac{\sin{\alpha}}{\cos{\alpha}} \\
    \end{split}
\end{equation}

Wir wählen den Maßstab im Einheitskreis, wo die Hypotenuse $c = 1$ ist. Hier ergibt sich $\frac{b}{c} =\sin{\alpha}\implies b=\sin{\alpha}$. Und Satz des Pythagoras ergibt $1=a^2+b^2\implies 1=\sin^2{\alpha}+\cos^2{\alpha}$

Um eine Strecke für den Tangens zu bekommen, wählen wir die Ankathete $a=1$ und können das Dreieck vergrößern. Nun ist die Gegenkathete unser Tangens.

\section{Vektoren}

\end{document}
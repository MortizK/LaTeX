\documentclass[a4paper]{article}

%\usepackage{url}

%% Math
\usepackage{mathtools}
%% For Mengen like natural numbers
\usepackage{amsfonts}
%% Für spezielle Symbole
\usepackage{amssymb}

%% Images
\usepackage{import}
\usepackage{xifthen}
\usepackage{pdfpages}
%\usepackage{transparent}

%%% Command for simpler images
\newcommand{\incfig}[1]{%
    \def\svgwidth{\columnwidth}
    \import{./fig/}{#1.pdf_tex}
}

%% Links
\usepackage{hyperref}
\hypersetup{
    colorlinks=true,
    linkcolor=black,
    filecolor=magenta,
    urlcolor=cyan
}

%% Formatting
\usepackage{parskip}

\title{Lineare Algebra}
\author{Moritz}
\date{January 29, 2025}

\begin{document}
\maketitle
\tableofcontents

\section{Komplexen Zahlen $\mathbb{C}$}

\subsection{Darstellung}

\begin{figure}[ht]
    \centering
    \incfig{komplexe}
    \caption{komplexe}
    \label{fig:komplexe}
\end{figure}

Polardarstellung von $z=|z|*(\cos(\alpha)+i\sin(\alpha))$

\subsubsection{Additionssätze der Trigonometrie}

Regeln Hergeleitet auf LA 29.01 in meinem Buch.

\begin{equation}
    \begin{split}
        \cos(\alpha +\beta) & =\sin(\alpha)\sin(\beta)+\cos(\beta)\cos(\alpha) \\
        \sin(\alpha +\beta) & =\sin(\alpha)\cos(\beta)+\sin(\beta)\cos(\alpha) \\
        \sin(2\alpha)       & =2\sin(\alpha)\cos(\alpha)                       \\
        \cos(2\alpha)       & =2\cos^2(\alpha) - 1
    \end{split}
\end{equation}

Daraus folgt:

\begin{center}
    $z*w=|z|*|w|*(\cos(\alpha+\beta)+i\sin(\alpha+\beta))$
\end{center}

\subsection{EULER-Formel}

\begin{equation}
    \begin{split}
        e^{i\alpha}         & =\cos(\alpha)+i\sin(\alpha) \\
        e^{i(\alpha+\beta)} & =e^{i\alpha}*e^{i\beta}
    \end{split}
\end{equation}

\subsubsection{Eigenschaften}

Eigenschaften von $e^{i\alpha}$:

\begin{enumerate}[i.]
    \item $|e^{i\alpha}|=1$
    \item $\bar{e^{i\alpha}}=\cos(\alpha)-i\sin(\alpha)=\cos(-\alpha)+i\sin(-\alpha)=e^{-i\alpha}$
    \item $e^{i\alpha}/e^{i\beta}=e^{i(\alpha-\beta)}$
    \item $\frac{z}{w}=\frac{|z|}{|w|}e^{i(\alpha-\beta)}$
\end{enumerate}

\begin{center}
    \begin{tabular}{r c l}
        $e^{i0}$              & $=$ & $1$                        \\
        $e^{i\pi}$            & $=$ & $-1$                       \\
        $e^{i\frac{\pi}{2}}$  & $=$ & $i$                        \\
        $e^{i\frac{3\pi}{2}}$ & $=$ & $-i$                       \\
        $e^{i2\pi}$           & $=$ & $1$                        \\
        $e^{i\frac{\pi}{4}}$  & $=$ & $\frac{1}{2}\sqrt{2}(1+i)$ \\
        $e^{i\frac{\pi}{3}}$  & $=$ & $\frac{1}{2}(1+i\sqrt{3})$ \\
    \end{tabular}
\end{center}

\subsection{Wurzeln}

Die n-ten Einheitswurzeln: $u^n=1$.

Die 3-ten Einheitswurzeln

\begin{figure}[ht]
    \centering
    \incfig{einheitswurzeln}
    \caption{3-ten Einheitswurzeln}
    \label{fig:3-ten Einheitswurzeln}
\end{figure}

Allgemein für beliebige $n\in \mathbb{N}$ gilt:

\begin{equation}
    u_k:=e^{ik*\frac{2\pi}{n}}, k=0, 1,\dots, n-1
\end{equation}

\subsubsection{Für alle Wurzeln}

n-ten Wurzeln aus $z$: Suche alle Zahlen $w$ mit $w^n=z$: $z=|z|e^{i\alpha}$

\begin{figure}[ht]
    \centering
    \incfig{wurzel}
    \caption{Beliebige Wurzeln}
    \label{fig:Beliebige Wurzeln}
\end{figure}

\begin{equation}
    \begin{split}
        w_0:=\sqrt[n]{|z|}*e^{i\frac{\alpha}{2}} \\
        w_k:=n_k*w_0                             \\
        n_k: \text{n-te Einheitswurzel}          \\
    \end{split}
\end{equation}

Denn: $W_k^n=W_0^n*u_k^n=z*1=z$

\subsubsection{Polardarstellung}

Wir haben a und b als die Koordinaten und wollen aus diese in die Polardarstellung umwandeln. Hierzu können wir $\alpha$ so bestimmen:

\begin{equation}
    \alpha = \begin{cases}
        \cos^{-1}(\frac{a}{|z|})  & \quad, b\geq 0 \\
        -\cos^{-1}(\frac{a}{|z|}) & \quad, b < 0
    \end{cases}
\end{equation}

\subsection{Matrixmultiplikation - Interpretation}

In der ersten Anschauung ist $a_1=[a_{11}, a_{21}, \dots, a_{n1}]$.

\begin{equation}
    \begin{split}
        A   & = [a_1, a_2, \dots, a_n]     \\
        A*x & = A *
        \begin{bmatrix}
            x_1 \\ x_2\\ \vdots\\ x_n
        \end{bmatrix} = [a_1, a_2, \dots, a_n] *
        \begin{bmatrix}
            x_1 \\ x_2\\ \vdots\\ x_n
        \end{bmatrix}           \\
            & = x_1a_1+x_2a_2+\dots+x_na_n \\
    \end{split}
\end{equation}

Aktive Betrachtung: Matrix als lineare Abbildung. $\mathbb{R}^n$  oder $\mathbb{C}^n$ nun auch $\mathbb{K}^n$ genannt.

\begin{equation}
    \mathbb{K}^n\substack{A\\\to} \mathbb{K}^m
\end{equation}

Problem: Zu einer gegebenen linearen Abbildung $A$ die zugehörige Matrix finden, d.h. finde $a_1, a_2, \dots a_n$.

Universelles Konstruktionsprinzip:

\begin{equation}
    a_1=A*e_1, a_2=Ae_2, \dots, a_n=Ae_n
\end{equation}

d.h. die Spaltenvektoren von $A$ sind die Bilder der kanonischen Basisvektoren $e_1, e_2, \dots, e_n = [0, 0, \dots, 0, 1]$

\subsubsection{Drehmatrix}

In einem zweidimensionales Koordinatensystem sind die Basisvektoren die Achsen und es ergiben sich bei der Drehung um $\alpha$ zwei Rechtwinklinge Dreiecke im Einheitskreis.

\begin{equation}
    D_{\alpha}=\begin{bmatrix}
        \cos(\alpha) & -\sin(\alpha) \\
        \sin(\alpha) & \cos(\alpha)  \\
    \end{bmatrix}
\end{equation}

\end{document}
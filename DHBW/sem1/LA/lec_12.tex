\documentclass[a4paper]{article}

%\usepackage{url}

%% Math
\usepackage{mathtools}
%% For Mengen like natural numbers
\usepackage{amsfonts}
%% Für spezielle Symbole
\usepackage{amssymb}

%% Images
\usepackage{import}
\usepackage{xifthen}
\usepackage{pdfpages}
%\usepackage{transparent}

%%% Command for simpler images
\newcommand{\incfig}[1]{%
    \def\svgwidth{\columnwidth}
    \import{./fig/}{#1.pdf_tex}
}

%% Links
\usepackage{hyperref}
\hypersetup{
    colorlinks=true,
    linkcolor=black,
    filecolor=magenta,
    urlcolor=cyan
}

%% Formatting
\usepackage{parskip}

\title{Lineare Algebra}
\author{Moritz}
\date{January 22, 2025}

\begin{document}
\maketitle
\tableofcontents

\section{Vektoren}

\subsection{Determinante}

\subsubsection{GAUSS-Verfahren}

Wir Arbeiten im GAUSS-Verfahren mit einer spalte für Arbeitsaufträge und eine für Korrekturfaktoren.

Jede Arbeitszeile kann nur einmal benutzt werden, sonst werden unsere 0 wieder verändert.

Wenn wir wie in erstem Tempo bei $III$ mit 2 multiplizieren, müssen wir den Korrekturfaktor aufschreiben und später die Determinante korrigieren.

WICHTIG: Wenn wir Zeilen oder Spalten tauschen, müssen wir einen Korrekturfaktor von $-1$ aufschreiben.

\begin{equation}
    \begin{array}{c|cccc|c|c}
            & s_1 & s_2 & s_3 & s_4  &               &              \\
        \hline
        I   & 2   & 0   & 5   & 7    &               &              \\
        II  & 4   & 1   & 2   & 6    & II - 2*I      &              \\
        III & -3  & 3   & 6   & 1    & 2*III + 3*II  & \frac{1}{2}  \\
        IV  & 6   & 0   & 3   & 4    & IV - 3*I      &              \\
        \hline
            & 2   & 0   & 5   & 7    &               &              \\
            & 0   & 1   & -8  & -8   &               &              \\
            & 0   & 6   & 27  & 23   & III - 6*II    &              \\
            & 0   & 0   & -12 & -17  &               &              \\
        \hline
            & 2   & 0   & 5   & 7    &               &              \\
            & 0   & 1   & -8  & -8   &               &              \\
            & 0   & 0   & 75  & 71   &               &              \\
            & 0   & 0   & -12 & -17  & 25*IV + 4*III & \frac{1}{25} \\
        \hline
            & 2   & 0   & 5   & 7    &               &              \\
            & 0   & 1   & -8  & -8   &               &              \\
            & 0   & 0   & 75  & 71   &               &              \\
            & 0   & 0   & 0   & -141 &               &              \\
        \hline
    \end{array}
\end{equation}

Also ist die Determinante der Matrix A, das Produkt der Diagonal mal allen Korrekturfaktoren.

\begin{equation}
    \begin{split}
        det(A) & =2*1*75*-141*\frac{1}{2}*\frac{1}{25} \\
        det(A) & =-141*\frac{2}{2}*\frac{75}{25}       \\
               & =-141*3                               \\
               & = -423
    \end{split}
\end{equation}

\subsection{Gleichungssysteme}

\begin{equation}
    \begin{split}
        \begin{split}
            2x_1+0x_2+5x_3+2x_4  & = 3  \\
            4x_1+1x_2+2x_3+6x_4  & = 9  \\
            -3x_1+3x_2+6x_3+1x_4 & = 0  \\
            6x_1+0x_2+3x_3+4x_4  & = 12 \\
        \end{split} \\
        \begin{array}{ccccc|c}
            x_1 & x_2 & x_3 & x_4 & y  \\
            \hline
            2   & 0   & 5   & 7   & 3  \\
            4   & 1   & 2   & 6   & 9  \\
            -3  & 3   & 6   & 1   & 0  \\
            6   & 0   & 3   & 4   & 12 \\
        \end{array}
    \end{split}
\end{equation}

Die 4x4 Matrix des Schema nennen wir A

\begin{equation}A:=\begin{bmatrix}
        2  & 0 & 5 & 7 \\
        4  & 1 & 2 & 6 \\
        -3 & 3 & 6 & 1 \\
        6  & 0 & 3 & 4 \\
    \end{bmatrix}
\end{equation}

Unsere Neu Multiplikation ist somit $A*x=y$. Bedingungen für die Größen sind: $A$ hat $m$ Zeilen und $n$ Spalten und wird mit $x$ (Vektor der Länge $n$) und ergibt y (Vektor der Länge $m$)

\subsubsection{Rechenregeln}

\begin{equation}
    \begin{split}
        A*x             & = y \\
        \begin{bmatrix}
            a_{11} & a_{12} & a_{13} & \dots  & a_{1n} \\
            a_{21} & a_{22} & a_{23} & \dots  & a_{2n} \\
            \vdots & \vdots & \vdots & \ddots & \vdots \\
            a_{m1} & a_{m2} & a_{m3} & \dots  & a_{mn} \\
        \end{bmatrix} *
        \begin{bmatrix}
            x_1    \\
            x_2    \\
            x_3    \\
            \vdots \\
            x_n
        \end{bmatrix} & =
        \begin{bmatrix}
            y_1    \\
            y_2    \\
            \vdots \\
            y_m
        \end{bmatrix}
    \end{split}
\end{equation}

Diese Rechenregeln sind linear.

\begin{equation}
    \begin{split}
        A*(tx)    & =t*A*x       \\
        A*(x+z)   & =A*x+A*z     \\
        A*(tx+sz) & =t*A*x+s*A*z
    \end{split}
\end{equation}

\subsubsection{Ablauf}

GAUSS-Verfahren $\to$ LGS lösen $\to$ Matrix-Schreibweise $\to$ allgemeine Lösungstheorie lineare Gleichung

\begin{equation}
    \begin{split}
        A: m\times n\text{-Matrix} & = [a_1, a_2, a_3, \dots, a_n] \\
        a_1=\begin{bmatrix}
                a_{11} \\
                a_{21} \\
                a_{31} \\
                \vdots \\
                a_{m1} \\
            \end{bmatrix},        &
        a_2=\begin{bmatrix}
                a_{12} \\
                a_{22} \\
                a_{32} \\
                \vdots \\
                a_{m2} \\
            \end{bmatrix}                                         \\
    \end{split}
\end{equation}

A: $m\times n$-Matrix

Also steht: $A*x=y$ und $A: m\times n$, $x: n\times 1$ und $y: m\times 1$ Matrizen.

\subsection{Lösungstheorien linearer Gleichungen}

$Ax=y$ mit $y\neq 0$: inhomogene Gleichung; eine Lösung $x$ heißt inhomogene Lösung.

$Ax=0$: homogene Gleichung; eine Lösung $x$ heißt homogene Lösung.

JEDE inhomogene Lösung $x$, setzt sich zusammen aus EINER FESTEN inhomogene Lösung $x_1$ und EINER homogene Lösung $x_0$.

\begin{equation}
    x = x_1 + x_0
\end{equation}

Denn: \begin{equation}
    \begin{split}
        Ax_1       & = y \\
        Ax_0       & = 0 \\
        Ax_1+Ax_0  & = y \\
        A(x_1+x_0) & = y
    \end{split}
\end{equation}

Umkehrung: \begin{equation}
    \begin{split}
        Ax   & = y                        \\
        Ax_1 & = y                        \\
        x    & = x_1+ (x-x_1) = x_1 + x_0
    \end{split}
\end{equation}

\begin{equation}
    \begin{split}
        Ax_0=A(x-x_1)=Ax-Ax_1=y-y=0
    \end{split}
\end{equation}

\subsection{Lineare (Un)Abhängigkeit}

Im $\mathbb{R}^3$

Linear Unabhängig (l.u.), wenn die drei Vektoren $a, b, c$ eine "Frittentüte" bilden.

Linear Abhängig (l.a.), wenn die drei Vektoren in einer Ebene liegen. Hier gilt $b = tc+sa$. Allgemein: $a, b$ oder $c$ lässt sich aus den restlichen kombinieren: z.B. $b=t*c+s*a$ oder $a = tb + sc$.

\begin{equation}
    \begin{split}
        b=t*c+s*a   & \implies 0=sa-1b+tc  \\
        a = tb + sc & \implies 0=-1a+tb+sc \\
    \end{split}
\end{equation}

Wenn die einzige Lösung von $t*a+s*b+r*c=0$ nur $t=s=r=0$. Wenn die Lösung eindeutig ist, sind die Vektoren linear Unabhängig (l.u.).

Hier lohnt es sich, GAUSS-Verfahren, nur bis zur oberen Dreiecksform zu lösen. Wenn es diese gibt, ist das Gleichungssysteme eindeutig Lösbar und es wird unsere triviale Lösung $t=s=r=0$ sein.

Nach GAUSS Form: \begin{equation}
    \begin{array}{cc}
        \begin{bmatrix}
            . & . & . & . & . \\
              & . & . & . & . \\
              &   & . & . & . \\
              &   &   & . & . \\
              &   &   &   & . \\
        \end{bmatrix}     &
        \begin{bmatrix}
            . & . & . & . & . \\
              & . & . & . & . \\
              &   & . & . & . \\
              &   &   & . & . \\
              &   &   &   &   \\
        \end{bmatrix}          \\
        \text{Linear Unabhängig} &
        \text{Linear Abhängig}
    \end{array}
\end{equation}

\subsubsection{Definition}

Im $\mathbb{R}^n$ sind $k$ Vektoren ${a_1, a_2, \dots, a_k}$ mit $k\leq n$ heißen linear unabhängig, falls jede Nullkombination trivial ist.

$t_1a_1+t_2a_2+\dots+t_ka_k=0$ trivial ist, d.h. es muss $t_1=t_2=\dots=t_k=0$ gelten.

Praktisch: Bilde eine Nullkombination $t_1a_1+t_2a_2+\dots+t_ka_k=0$ und versuche sie mit GAUSS zu lösen. Wenn dabei zwingend $t_1=t_2=\dots=t_k=0$ rauskommt, sind sie l.u., andernfalls l.a.

i) GAUSS in obere Dreiecksform mit allen Werten der Diagonalen $\neq 0$.

ii) GAUSS in obere Dreiecksform mit mindesten einem Wert der Diagonal $= 0$

\section{Besprechen}

Termin der Übungsklausur!

\end{document}
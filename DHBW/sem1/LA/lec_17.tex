\documentclass[a4paper]{article}

%\usepackage{url}

%% Math
\usepackage{mathtools}
%% For Mengen like natural numbers
\usepackage{amsfonts}
%% Für spezielle Symbole
\usepackage{amssymb}

%% Images
\usepackage{import}
\usepackage{xifthen}
\usepackage{pdfpages}
%\usepackage{transparent}

%%% Command for simpler images
\newcommand{\incfig}[1]{%
    \def\svgwidth{\columnwidth}
    \import{./fig/}{#1.pdf_tex}
}

%% Links
\usepackage{hyperref}
\hypersetup{
    colorlinks=true,
    linkcolor=black,
    filecolor=magenta,
    urlcolor=cyan
}

%% Formatting
\usepackage{parskip}

\title{Lineare Algebra}
\author{Moritz}
\date{February 11, 2025}

\begin{document}
\maketitle
\tableofcontents

\section{Matrizen}

\subsection{Dimensionsformel}

\begin{equation}
    dim V = dim ker A + dim im A, A:V\to W
\end{equation}


\know{A lineare Abbildung: $m\times n$-Matrix}{
    \begin{center}
        A injektiv $\iff ker A =\{0\}$

        $\iff A=[a_1, \dots, a_n]\land \{a_1, \dots,a_n\} l.u.$

        $\underset{*)n=m}{\iff}det(a)\neq 0$
    \end{center}}

\begin{enumerate}
    \item A inj: $x\in kern A: Ax=0=A0\overset{A inj}{\implies} x=0 \implies ker A=\{=\}$
    \item $ker A =\{0\}: Ax=Ax'\implies A(\underset{\in ker A}{x-x'})=0\implies x=x'$
\end{enumerate}

\subsubsection{Quadratische Matrizen}

Wenn wir eine $n\times n$-Matrix haben, gelten noch weiteres:

A inj. $\implies ker A=\{0\}\iff dim ker A = 0$, nach der Dimensionsformel gilt somit, dass die Dimension von A gleich der Dimension vom Bild von A ist, da die Dimension vom Kern = 0 ist.

\begin{center}
    $dim V = dim ker A + dim im A = 0 + dim im A \iff im A = \mathbb{K}^n\iff$ A surj.
\end{center}

\know{Bijektivität}{Eine Quadratische Matrix ist bijektiv, wenn sie entweder surjektiv oder injektiv ist, da wir aus dem einem immer das andere Schließen können.
    \begin{center}
        A inj $\iff$ A surj. $\iff$ A bij.
    \end{center}
    In dieser Situation gibt es eine inverse Abbildung $A^{-1}$ von $A$}

\subsubsection{Inverse Matrix}

A: $n\times n$-Matrix. Dann gilt:

\begin{equation}
    A^{-1}ex. \iff \exists_B BA=1\lor AB=1
\end{equation}

In diesem Fall ist $B=A^{-1}$

Beweis:

\begin{enumerate}
    \item $AB=!\implies A\underset{0}{Bx}=x$ f.a. $x\in \mathbb{K}^n\implies im A =\mathbb{K}^n\implies$ A surj. $\iff A^{-1} ex.$
    \item $BA=!\land x\in ker A$ dann $x=B\underset{=0}{Ax}=B0=0\implies ker A=\{0\}$
\end{enumerate}

Also AB=1 oder BA=1 führt dazu, dass $A^{-1}$ existiert.

\begin{equation}
    \implies A^{-1} / AB=1 \implies A^{-1}A=B=A^{-1}*1=A^{-1}
\end{equation}

\subsubsection{Ausrechnen von der Inverse}

Also praktisches Verfahren zum Test auf Existenz von $A^{-1}$ und gegebenenfalls zur Berechnung von $A^{-1}$:

\begin{center}
    $AB=1=[e_1, e_2, \dots, e_n]$

    $B=[b_1, b_2, b_n]$
\end{center}

Daraus können wir mehrere GAUSS-Verfahren durchführen der Form $Ab=e$

\begin{center}
    Ab = e und mit GAUSS wird es zu der Form 1*b=b
\end{center}

mit diesem Wissen können wir jür jeden Spaltenvektor GAUSS machen. Hier wird aber jedes mal das gleiche GAUSS-Verfahren gemacht nur mit einem anderen e. Somit können wir auch ein "großes" GAUSS machen.

\begin{center}
    \begin{tabular}{|c|c|}
        \hline
            &   \\
        $A$ & 1 \\
            &   \\
        \hline
    \end{tabular} $\underset{GAUSS}{\to}$
    \begin{tabular}{|c|c|}
        \hline
          &          \\
        1 & $A^{-1}$ \\
          &          \\
        \hline
    \end{tabular}
\end{center}

\subsubsection{Rechenregeln der Inversenbildung}

A, B seien invertierbar

\begin{enumerate}
    \item $\implies (AB)^{-1}=B^{-1}A^{-1}$, denn $AB*B^{-1}*A^{-1}=A*A^{-1}=1$
    \item $(A^*)^{-1}=(A^{-1})^*$, denn $(A^{-1})^*A^*=(AA^{-1})^*=1^*=1$
\end{enumerate}

\subsubsection{Determinanten-Entwicklungssatz}

Wir nutzen einfach die Entwicklung der Determinante in einer Spalte oder einer Zeile.

So können wir in eine Schwachen Spalte nach den einzelnen kleineren Determinanten auflösen.

\begin{equation}
    det
    \begin{bmatrix}
        2 & 5 & -1 \\
        3 & 2 & 1  \\
        4 & 0 & 5
    \end{bmatrix} = 5*det
    \begin{bmatrix}
        3 & 1 \\
        4 & 5
    \end{bmatrix} - 2*det
    \begin{bmatrix}
        2 & -1 \\
        4 & 5
    \end{bmatrix}
\end{equation}

Genauere Skizze mit zwischenschritten und Erklärung in meinem Heft.


Wenn wir nicht überschneidende Quadrate haben, welche Kante and Kante ind er Determinate liegen, gilt noch eine zusätzliche Regel:

\begin{equation}
    det\begin{bmatrix}
        1 & 2 & 0 & 4 & 3  & 0 & 1  \\
        3 & 4 & 3 & 4 & -3 & 7 & 9  \\
          &   & 5 & 0 & 1  & 0 & 7  \\
          &   & 0 & 2 & 7  & 5 & 9  \\
          &   & 3 & 3 & 1  & 0 & 1  \\
          &   &   &   &    & 1 & -1 \\
          &   &   &   &    & 3 & 8  \\
    \end{bmatrix}=det
    \begin{bmatrix}
        1 & 2 \\
        3 & 4
    \end{bmatrix}*det
    \begin{bmatrix}
        5 & 0 & 1 \\
        0 & 2 & 7 \\
        3 & 5 & 1
    \end{bmatrix}*det
    \begin{bmatrix}
        1 & -1 \\
        3 & 8
    \end{bmatrix}
\end{equation}

\subsubsection{Determinanten-Produktsatz}

\begin{center}
    $det(A*B) = det(A)*det(B)$

    $1 = det(1) = det(A*A^{-1}) = det(A) * det(A^{-1})$
\end{center}

Dies lässt sich auch Umstellen

\know{Berechne die Inverse mit ihrer Determinante}{HIerfür kann mit GAUSS und Korrekturspalte die Inverse und die Determinante der Normalen bestimmt werden. Nach Der Produktregel:
    \begin{center}
        $det(A^{-1})=1 / det(A)$
    \end{center}}

Zuletzt gilt auch noch:

\begin{equation}
    \begin{split}
        A      & =B*\tilde{A}*B^{-1}                                      \\
        det(A) & =det(B*\tilde{A}*B^{-1})=det(B)det(\tilde{A})det(B^{-1}) \\
               & = det(B)det(\tilde{A})*\frac{1}{det(B)}=det(\tilde{A})
    \end{split}
\end{equation}

\end{document}
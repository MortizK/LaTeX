\documentclass[a4paper]{article}

%\usepackage{url}

%% Math
\usepackage{mathtools}
%% For Mengen like natural numbers
\usepackage{amsfonts}
%% Für spezielle Symbole
\usepackage{amssymb}

%% Images
\usepackage{import}
\usepackage{xifthen}
\usepackage{pdfpages}
%\usepackage{transparent}

%%% Command for simpler images
\newcommand{\incfig}[1]{%
    \def\svgwidth{\columnwidth}
    \import{./fig/}{#1.pdf_tex}
}

%% Links
\usepackage{hyperref}
\hypersetup{
    colorlinks=true,
    linkcolor=black,
    filecolor=magenta,
    urlcolor=cyan
}

%% Formatting
\usepackage{parskip}

\title{Lineare Algebra}
\author{Moritz}
\date{February 18, 2025}

\begin{document}
\maketitle
\tableofcontents

\section{Matrizen}

\subsection{Eigenwerttheorie}

Eigenwertgleichung

\begin{equation}
    \exists_{x\neq 0}\quad Ax=\lambda x
\end{equation}

Lambda $\lambda$ ist der Eigenwert (EW) zum Eigenvektor (EV) $x$

Die Menge aller Eigenwerte von $A$ heißt Spektrum von $A$; $sp(A)$

\subsubsection{Lösen der EW-Gleichung}

Mit mehreren Schlüssen, können wir die EW-Gleichung umformen:

\begin{equation}
    \begin{split}
        \exists_{x\neq 0, \lambda\in \mathbb{C}} Ax=\lambda x & \iff \exists_{x\neq 0, \lambda\in \mathbb{C}} (A-\lambda 1)x = 0 \\
                                                              & \iff \exists_{\lambda\in \mathbb{C}} ker(A-\lambda 1)\neq\{0\}   \\
                                                              & \iff A-\lambda 1 \text{nicht invertierbar}                       \\
                                                              & \iff det(A-\lambda 1)=0                                          \\
                                                              & =:\chi_a(\lambda)
    \end{split}
\end{equation}

Chi $\chi$ auch charakteristisch Polynom genannt.

Das Spektrum besteht also aus allen Nullstellen des charakt. Polynoms.

Beispiel:

\begin{equation}
    \begin{split}
        A               & =
        \begin{bmatrix}
            1 & 2 \\
            4 & 3
        \end{bmatrix}                                                \\
        A-\lambda 1     & =
        \begin{bmatrix}
            1 & 2 \\
            4 & 3
        \end{bmatrix}-
        \begin{bmatrix}
            \lambda & 0       \\
            0       & \lambda
        \end{bmatrix}=
        \begin{bmatrix}
            1-\lambda & 2         \\
            4         & 3-\lambda
        \end{bmatrix}                                         \\
        \chi_A(\lambda) & =det
        \begin{bmatrix}
            1-\lambda & 2         \\
            4         & 3-\lambda
        \end{bmatrix}=(1-\lambda)(3-\lambda)-8=\lambda^2-4\lambda+3-8 \\
                        & = \lambda^2-4\lambda-5=0                    \\
        \lambda_{1/2}   & =2\pm\sqrt{4+5}=5;-1
    \end{split}
\end{equation}

EV zum EW 5: $\begin{bmatrix}
        -4 & 2  \\
        4  & -2
    \end{bmatrix}x=0$ ergibt $b_1\sim\begin{bmatrix}
        2 \\4
    \end{bmatrix}\sim\frac{1}{\sqrt{5}}\begin{bmatrix}
        1 \\2
    \end{bmatrix}$

EV zum EW -1:$\begin{bmatrix}
        2 & 2 \\
        4 & 4
    \end{bmatrix}x=0$ ergibt $b_2\sim\begin{bmatrix}
        -1 \\1
    \end{bmatrix}\sim\frac{1}{\sqrt{2}}\begin{bmatrix}
        -1 \\1
    \end{bmatrix}$

Somit ergibt sich die Basis der EV:

\begin{equation}
    B = \begin{bmatrix}
        \frac{1}{\sqrt{5}} & -\frac{1}{\sqrt{2}} \\
        \frac{2}{\sqrt{5}} & \frac{1}{\sqrt{2}}
    \end{bmatrix}=\frac{1}{\sqrt{10}}\begin{bmatrix}
        \sqrt{2}  & -\sqrt{5} \\
        2\sqrt{2} & \sqrt{5}
    \end{bmatrix}
\end{equation}

Inverse von A in $2\times 2$

\begin{equation}
    A^{-1}=\frac{1}{det(A)}*\begin{bmatrix}
        d  & -b \\
        -c & a
    \end{bmatrix}
\end{equation}

\know{Ergebnis}{Nach dem bestimmen der Inversen können wir in dieser Basis mit den Eigenvektoren Rechnen:
    \begin{center}
        $A=B\begin{bmatrix}
                5 & 0  \\
                0 & -1
            \end{bmatrix}B^{-1}$
    \end{center}}

\subsubsection{Selbstadjungierte Abbildungen}

\begin{center}
    $\langle x\mid Ay\rangle=\langle Ax\mid y\rangle \implies A^*=\overline{A}^{t}=A$
\end{center}

\begin{enumerate}
    \item Alle EWe sind reell: $Ax=\lambda x$, mit $x\neq 0$
          \begin{center}
              $\langle x\mid Ax\rangle=\langle x\mid \lambda x\rangle=\lambda \langle x\mid x\rangle = \lambda\|x\|^2$

              $\langle Ax\mid x\rangle=\langle \lambda x\mid  x\rangle=\overline{\lambda} \langle x\mid x\rangle = \overline{\lambda}\|x\|^2$

              $\implies (\overline{\lambda} -\lambda)\|x\|^2=0\implies \overline{\lambda} = \lambda\implies \lambda\in \mathbb{R}$
          \end{center}
    \item Die EVen zu verschieden EWen sind Orthogonal:
          \begin{center}
              $Ax=\lambda x, Ay=\gamma y, \lambda\neq\gamma$

              $\langle x\mid Ay\rangle=\langle x\mid \gamma y\rangle=\gamma\langle x\mid y\rangle$


              $\langle Ax\mid y\rangle=\langle \lambda x\mid  y\rangle=\lambda\langle x\mid y\rangle$

              $(\lambda - \gamma)\langle x\mid y\rangle\implies \langle x\mid y\rangle = 0 \iff x \bot y$
          \end{center}
          Da $\gamma$ und $\lambda$ nicht gleich sind, können wir die orhtogonalität finden.
\end{enumerate}

\end{document}
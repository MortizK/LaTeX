\documentclass[a4paper]{article}

%\usepackage{url}

%% Math
\usepackage{mathtools}
%% For Mengen like natural numbers
\usepackage{amsfonts}
%% Für spezielle Symbole
\usepackage{amssymb}

%% Images
\usepackage{import}
\usepackage{xifthen}
\usepackage{pdfpages}
%\usepackage{transparent}

%%% Command for simpler images
\newcommand{\incfig}[1]{%
    \def\svgwidth{\columnwidth}
    \import{./fig/}{#1.pdf_tex}
}

%% Links
\usepackage{hyperref}
\hypersetup{
    colorlinks=true,
    linkcolor=black,
    filecolor=magenta,
    urlcolor=cyan
}

%% Formatting
\usepackage{parskip}

\title{Lineare Algebra}
\author{Moritz}
\date{December 18, 2024}

\begin{document}
\maketitle
\tableofcontents

\section{Elemente der Zahlentheorie}

\subsection{Übung: Erweiterter Euklidischer Algorithmus}

Der $ggT(5610, 637)$ ist $1$ und die Darstellung ist $1=145*5610-1277*637$

Es gibt immer $ggT(p, q)=tp+sq$, $t,s\in \mathbb{Z}$

\subsection{Teilbarkeitssatz}

$p\in \mathbb{N}$, $a,b\in \mathbb{N}$

i) $p|a*b\land ggT(a,p)=1\implies p|b$

ii) $p|ab\land p\in \mathbb{P}\implies p|a \lor p|b$


$\exists_kab=k*p$

Beweis: i) $ggT(p,a)=1\implies 1=ta+sp\mid *b\implies b=t*ab+sb*ptkp+sbp=(tk+sb)*p\implies p|b$

ii) $p|ab\land p\in \mathbb{P}$: Zwei Fälle $\substack{ggT(a,p)=p\implies p|a\\ggT(a,p)=1\substack{\text{i)}\\ \implies}p|b}$

\subsection{Satz}

Satz: $p\in \mathbb{P}\implies \sqrt{p}\notin \mathbb{Q}$

Beweis durch Widerspruch: Annahme $\sqrt{p}=\frac{a}{b}$, oBdA $ggT(a,b)=1$

Weitere auf La 18.12

\subsection{Rechnen modulo p}

Definition: $a,b\in \mathbb{Z}, p\in \mathbb{N}, a=b\mod p:\iff p|a-b\iff a=_{p}b$, wenn $p$ die Differenz teilt.

Das bedeutet: $a=b\mod p$, d.h. $p|a-b$, d.h. $\exists_{t\in \mathbb{Z}}a-b=t*p\iff \exists_{t\in \mathbb{Z}}a=b+t*p$ d.h. $a=b\mod p$ genau dann, wenn sich $a$ von $b$ nur um ganzzahlige vielfachen von $p$ unterscheiden.

\subsubsection{Übungsaufgabe}

die ersten beiden Stellen von $7^{1000}$

$=7^{2*500}=49^{500}=49^{2*250}=2401^250=_{100}1^{250}=1$

Somit sind die ersten beiden Ziffern 1 (Einer) und 0 (Zehner).

\subsubsection{Beweise +,-,*}

$a=_pa'$ und $b=_pb'\implies a+b=_pa'+b'$.

d.h. $a=a'+t*p$ und $b=b'+s*p\implies a\pm b=a'\pm b'+(t\pm s)p=_pa'\pm b'$

$a*b=(a'+tp)(b'+sp)=a'b'+a'sp+tb'p+tsp^2=a'b'+(a's+tb'+tsp)p=_pa'b'$

\end{document}
\section{Relationen und Funktionen}

\subsection{Allgemeine Eigenschaften von Relationen}

\know{Relation}{Eine zweistellige Relation $R$ von $A\times B$:
    \begin{center}
        $R\subseteq A\times B$
    \end{center}
    Man sagt jedes $a\in A$ steht somit in Relation zu jedem $b\in B$}

Ich kann mir Relationen also so verstehen, wie man sich das vorstellt. Somit den gleichen Ansatz wie Relationen in einer Datenbank.

\subsection{Klassifikation von Relationen}

\begin{tabular}{r l}
    homogen         & $A=B$, alle hier müssen homogen sein!            \\
    reflexiv        & Für alle $a\in A$ ist $[a, a]\in R$              \\
    symmetrisch     & $[a, b]\in R$, gibt es auch $[b, a]\in R$        \\
    antisymmetrisch & $[a, b]\in R \land [b, a]\in R$ folgt $a=b$      \\
    transitiv       & $[a, b]\in R$, $[b, c]\in R$ folgt $[a, c]\in R$
\end{tabular}

Die weiteren müssen nicht unbedingt homogen sein:

\begin{tabular}{r l}
    linkstotal  & Für alle $a\in A$ gibt es ein $b\in B$, so dass $[a, b]\in R$       \\
    rechtstotal & Für alle $b\in B$ gibt es ein $a\in A$, so dass $[a, b]\in R$       \\
    funktional  & aus $[a, b]\in R$ und $[a, c]\in R$ folgt $b=c$                     \\
    injektiv    & aus $[a, b]\in R$ und $[c, b]\in R$ folt $a=c$                      \\
    bijektiv    & für alle $b\in B$ gibt es genau ein $a\in A$, so dass $[a, b]\in R$
\end{tabular}

Visualisieren kann man diese in Tabellen:

\begin{center}
    \begin{tabular}{c c c}
        \begin{tabular}{c|c c c c}
              & 1 & 2 & 3 & 4 \\
            \hline
            1 & 1 &   &   &   \\
            2 &   & 1 &   &   \\
            3 &   &   & 1 &   \\
            4 &   &   &   & 1 \\
        \end{tabular} &
        \begin{tabular}{c|c c c c}
              & 1 & 2 & 3 & 4 \\
            \hline
            1 &   & 1 &   &   \\
            2 & 1 &   & 1 &   \\
            3 &   & 1 &   & 1 \\
            4 &   &   & 1 &   \\
        \end{tabular} &
        \begin{tabular}{c|c c c c}
              & 1 & 2 & 3 & 4 \\
            \hline
            1 &   &   &   &   \\
            2 & 1 &   &   &   \\
            3 & 1 & 1 &   &   \\
            4 &   &   &   &   \\
        \end{tabular}    \\
        reflexiv                   &
        symmetrisch                &
        transitiv                    \\
        Diagonalen                 &
        Spiegelbild                &
        ZUgnetz                      \\\\
        \begin{tabular}{c|c c c c}
              & 1 & 2 & 3 & 4 \\
            \hline
            1 &   & 1 & 1 & 1 \\
            2 & 1 &   &   &   \\
            3 & 1 &   &   &   \\
            4 &   &   & 1 &   \\
        \end{tabular} &
        \begin{tabular}{c|c c c c}
              & 1 & 2 & 3 & 4 \\
            \hline
            1 &   &   &   & 1 \\
            2 &   & 1 &   &   \\
            3 & 1 &   & 1 &   \\
            4 &   &   &   &   \\
        \end{tabular} &
        \begin{tabular}{c|c c c c}
              & 1 & 2 & 3 & 4 \\
            \hline
            1 &   &   &   & 1 \\
            2 &   & 1 &   &   \\
            3 & 1 &   &   &   \\
            4 &   &   &   & 1 \\
        \end{tabular}    \\
        linkstotal                 &
        funktional                 &
        injektiv                     \\
        Jedes a                    &
        Zeilen                     &
        Spalten
    \end{tabular}
\end{center}

\subsection{Äquivalenzrelationen}

Eine Äquivalenzrelationen ist eine Relation, welche Reflexiv, Transitiv und Symmetrisch ist.

\subsection{Verkettung von Relation und die inverse Relation}

\subsection{Funktionen}


\section{Formelsammlung}

\subsection{Induktionen}

Induktionsvoraussetzung: (IV)

Induktionsanfang $n=1$ behauptet:

Induktionsschritt $n\to n+1$: z.z.:

Also:

\subsubsection{Typische Hilfen}

\begin{enumerate}
    \item Kleiner Satz von FERMAT mit $p$ einer Primzahl
          \begin{align*}
              n^p=_p n, p\in \mathbb{P}
          \end{align*}
    \item Kleiner Satz von EULER
          \begin{align*}
              a^{(p-1)(q-1)}=_{pq} \\
              n^{p-1}=_p 1
          \end{align*}
    \item Erweiterung mit $*1$ wie $\frac{x-2}{x-2}$
    \item Erweiterung mit $+0$ wie $x^2-x^2$
\end{enumerate}

\subsubsection{Euklidischer Algorithmus und Darstellung}

\begin{center}
    \begin{tabular}{r l r l}
        $r_0=$ & $q_0r_1+r_2$ & \qquad $r_2=$ & $r_0-q_0r_1$ \\
        $r_1=$ & $q_1r_2+r_3$ & \qquad $r_3=$ & $r_1-q_1r_2$ \\
        $r_2=$ & $q_2r_3+0$   & \qquad $0=$   & $r_2-q_2r_3$ \\
    \end{tabular}
\end{center}

Bei einer $+0$ ist hier $r_3$ der $ggT(r_0, r_1)$

Für die Darstellung brauchen wir die Form: $ggT(r_0, r_1)=xr_0+yr_1$ mit $y>0$. Dies wird durch einsetzen der Rechten Seite von unten nach oben.

\subsubsection{RSA}

p, q sind Primzahlen und e wird gewählt. $n=p*q$ und $\phi = (p-1)(q-1)$. Falls $ggT(\phi, n)=1$ ist $S_o=[e, n]$

$S_p=[d, n]$ mit $d=y$ von der Darstellung von Euklid.

Nachricht verschlüsseln: $C=_n M^e$

Nachricht entschlüsseln: $M=_n C^d$

\subsubsection{Komplexe Wurzeln Ziehen}

\begin{align*}
    z   & =te^{i\alpha}                                      \\
    w_k & =\sqrt[n]{t}*e^{\frac{i\alpha}{n}+\frac{k2\pi}{n}}
\end{align*}

Winkel durch $n$ teilen und die $n$-te Wurzel der Länge Ziehen.

\subsubsection{Cardanischen Formeln}

Normalform $y^3+py+q=0$ wird mit Substitution $x=y-1$ erreicht.

\begin{align*}
    p      & =c-\frac{1}{3}b^2                                                                    \\
    q      & =\frac{2}{27}b^3-\frac{1}{3}bc+d                                                     \\
    \Delta & =\left(\frac{p}{3}\right)^3+\left(\frac{q}{2}\right)^2                               \\
    x_1    & =\sqrt[3]{\sqrt{\Delta}-\frac{q}{2}}-\sqrt[3]{\sqrt{\Delta}+\frac{q}{2}}-\frac{b}{3}
\end{align*}

\subsection{Drehen und Basis}

\subsubsection{Ebenengleichungen}

Schnittgerade

\subsubsection{ONB}

Rechtssystem, Darstellung in einer Basis,

\subsubsection{Kern und Bild}

\begin{align*}
    ker A & =\{x\in V\mid Ax=0\}                    \\
    im A  & = \{y\in W \mid \exists_{x\in V} y=Ax\}
\end{align*}

\subsubsection{Drehung}

Winkel zwischen Vektoren und Ebenen.

\subsection{Determinante und Inverse}


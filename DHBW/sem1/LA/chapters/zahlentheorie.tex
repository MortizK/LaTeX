\section{Zahlentheorie}

\subsection{Teilbarkeitstheorie ganzer Zahlen}

Wir können die Teilbarkeit auch durch die Umstellung ausdrücken mit $a, b, c\in \mathbb{Z}$

\begin{center}
    $b=c*a$
\end{center}

Wir schreiben: $a|b$, $a$ teilt $b$.

Wir können diese Regeln auch erweitern, denn jede division, kann in eine eindeutige Schreibweise mit Rest formuliert werden.

\begin{center}
    \begin{tabular}{c c}
        $p=tq+r$, & $0\leq r < q$
    \end{tabular}
\end{center}

\subsubsection{Teilbarkeitssätze}

\begin{enumerate}[i.]
    \item $p|ab\land ggT(p, a)=1 \implies p|b$
    \item $p|ab\land p\in \mathbb{P} \implies p|a\lor p|b$
\end{enumerate}

\subsubsection{ggT und euklidischer Algo.}

Der größte gemeinsame Teiler beschreibt die größte $t$, für mehrere Zahlen $p$. Wenn $ggT(p_1, p_2)=1$ sind $p_1$ und $p_2$ teilerfremd.

ggT kann durch wiederholtes teilen mit Rest bestimmt werden.

\begin{center}
    \begin{tabular}{r c l c r c l}
        $84$ & $=$ & $4*18 + 12$ & $\implies$ & $12$ & $=$ & $84 - 4*18$ \\
        $18$ & $=$ & $1*12 + 6$  & $\implies$ & $6$  & $=$ & $18 - 1*12$ \\
        $12$ & $=$ & $2*6 + 0$   &            &      &     &             \\
    \end{tabular}
\end{center}

Somit ist der $ggT(84, 18)=6$. Der $ggT$ ist auch für alle Variationen von $84, -84$ mit $18, -18$ gleich $6$.

\know{Euklid}{Es gibt zu jedem $ggT$ auch die Form
    \begin{center}\begin{tabular}{c c}
            $ggT(p, q)=xp+yq$, $x, y\in \mathbb{Z}$
        \end{tabular}\end{center}}

Um in diese Form zu kommen, können wir den erweiterten euklidischen Algorithmus anwenden. Dieser stellt wie ob, die Gleichung einfach um und setzt die oberen Gleichung in die unteren ein.

Wenn $ggT(p, q)=1$ ist die gesuchte Form $1=xp+yq$

\begin{center}
    $6=18-1*(84-4*18)=3*18-1*84$
\end{center}

\subsection{Rechnen Modulo p}

Zwei Zahlen Modulo $p$ sind genau dann gleich, wenn Sie sich um ein vielfaches von $p$ unterscheiden:

\begin{center}
    $x=_{p}y \iff \exists t\in \mathbb{Z}: x=y+tp$
\end{center}

Die Rechnung modulo $p$ ist eine Äquivalenzrelationen, welche eine unendlich Große Menge auch eine endliche Menge reduziert.

\begin{center}
    $\mathbb{Z}_p=\{0,1,2,3,\dots,p-1\}$
\end{center}

\subsubsection{Addition}

Wir können Zahlen modulo $p$ mit einander addieren und subtrahieren. $x_1=_px_2$ und $y_1=_py_2$. Beide haben die Form $x_1=x_2+tp$, wenn wir diese Addieren oder Subtrahieren, bekommen wir:

\begin{center}
    $x_1\pm y_1=x_2+tp\pm y_2\pm sp=_px_2\pm y_2$
\end{center}

\subsubsection{Multiplikation}

Die Multiplikation funktioniert in modulo $p$ auch:

\begin{center}
    $x_1*y_1=(x_2+tp)(y_2+sp)=x_2y_2+tpy_2+spx_2+tsp^2=_px_2y_2$
\end{center}

\subsubsection{Inverse modulo p}

Die Inverse modulo $p$ ist definiert als $ab=_p1$, dabei muss $p>1$ und für jedes $a\in \mathbb{Z}_p^{*}$ muss $ggT(a, p)=1$ gelten.

\begin{center}
    \begin{tabular}{c c}
        \begin{tabular}{c|c c c c c c c}
            $*/7$ & 1 & 2 & 3 & 4 & 5 & 6 \\
            \hline
            1     & 1 & 2 & 3 & 4 & 5 & 6 \\
            2     & 2 & 4 & 6 & 1 & 3 & 5 \\
            3     & 3 & 6 & 2 & 5 & 1 & 4 \\
            4     & 4 & 1 & 5 & 2 & 6 & 3 \\
            5     & 5 & 3 & 1 & 6 & 4 & 2 \\
            6     & 6 & 5 & 4 & 3 & 2 & 1 \\
        \end{tabular} &
        \begin{tabular}{c|c c c c}
            $*/12$ & 1  & 5  & 7  & 11 \\
            \hline
            1      & 1  & 5  & 7  & 11 \\
            5      & 5  & 1  & 11 & 7  \\
            7      & 7  & 11 & 1  & 5  \\
            11     & 11 & 7  & 5  & 1  \\
        \end{tabular}
    \end{tabular}
\end{center}

Aus diesen Tabellen lassen sich die inversen leicht auslesen.

\know{Kleiner Satz von FERMAT}{Dieser zeigt, dass für jedes $p\in \mathbb{P}$, $a^p=_pa$ gilt. Wenn zudem noch $ggT(a, p)=1$, dann gilt zusätzlich $a^{p-1}=_p1$. Dies erhalten wir durch erweitern mit der inversen $b$, wo $ab=1$ gilt:
    \begin{center}
        $a^{p-1}=_pa^{p-1}*ab=a^pb=_pab=_p1$
    \end{center}}

Für zwei unterschiedliche Primzahl $p,q$ gilt $a=_{pq}b$, wenn $a=_pb$, $a=_qb$ gelten. Mit dem kleinem Satz von FERMAT gilt unter der Bedingung, dass $ggT(a, pq)=1$ zusätzlich:

\begin{center}
    $a^{(p-1)(q-1)}=_{pq}1$
\end{center}

\subsection{RSA-Verschlüsselung}

\begin{enumerate}[i.]
    \item Wähle zwei verschiedene Primzahlen $p, q$ und bilde $n:= p*q$.
    \item Wähle $1<e<\varphi:=(p-1)(q-1)$ mit $ggT(\varphi, e)=1$. Dann ist $S_o:=[e, n]$
    \item Bilde die Inverse $d$ von $e$ modulo $\varphi$ (mit erw. euklid. Alg.) $d=_{\varphi}e^{-1}$. Dann ist $S_p:=[d, n]$

          Danach ist die Schlüsselgeneration abgeschlossen. $p$ und $q$ können gelöscht werden. $S_p$ muss verschlüsselt auf dem Rechner gespeichert werden.
    \item Eine Nachricht $0<M<n$ wird durch den öffentlichen Schlüssel ($S_o$) verschlüsselt:
          \begin{center}
              $C:=_n M^{e}$
          \end{center}
          $C$ wird mit $S_p$ entschlüsselt:
          \begin{center}
              $M=_n C^{d}$
          \end{center}
\end{enumerate}

\subsubsection{Rechnungen}

Es gibt einen online Rechner zur Korrektur: \href{https://rsa-calculator.netlify.app/}{RSA-Calculator}

$p=101, q=17$ und $e=411$ sind gegeben und die beiden Schlüssel sollen bestimmt werden.

Für den Öffentlichen Schlüssel muss nur noch $n=p*q$ bestimmt werden. Somit ist $n=101*17=1717$ und falls $ggT(\varphi, e)=1$, gilt $S_o=[e, n]$. $\varphi=(p-1)(q-1)=100*16=1600$.

\begin{equation}
    \begin{split}
        1600 & = 3*411 + 367 \\
        411  & = 1*367 + 44  \\
        367  & = 8*44 + 15   \\
        44   & = 2*15 + 14   \\
        15   & = 1*14 + 1    \\
    \end{split}
    \qquad\qquad
    \begin{split}
        367 & = 1600 - 3*411 \\
        44  & = 411 - 1*367  \\
        15  & = 367 - 8*44   \\
        14  & = 44 - 2*15    \\
        1   & = 15 - 1*14    \\
    \end{split}
\end{equation}

$S_o=[411, 1717]$. Für den Privaten Schlüssel brauchen wir die Darstellung nach EUKLID:

\begin{equation}
    \begin{split}
        1 & = 15 - 1*14 = 15 - 1*(44 - 2*15)= 3*15 - 1*44 \\
          & =  3*(367 - 8*44) - 1*44 = 3*367 - 25*44      \\
          & =  3*367 - 25*(411 - 1*367) = 28*367 - 25*411 \\
          & =  28*(1600 - 3*411) - 25*411                 \\
          & = 28*1600 - 109*411
    \end{split}
\end{equation}

$d=1600-109=1491$. Daraus ergibt sich $S_p=[1491, 1717]$

Zum Verschlüsseln und Entschlüsseln gilt der gleiche Ansatz, wir brauchen nur noch ein $M=405$:

\begin{equation}
    \begin{split}
        C & =_n M^e                                                    \\
          & =_{1717} 405^{411} = 405*405^{410} = 405*(405^2)^{205}     \\
          & =_{1717} 405*910^{205} \text{ Es gilt: } 405^2=_{1717} 910 \\
          & =_{1717} 409*910*910^{204} =409*910*(910^2)^{102}          \\
          & =_{1717} 1112*506^{102} = 1112 * (506^2)^{51}              \\
          & =_{1717} 1112 * 203^{51} = 1112*203 * (203^2)^{25}         \\
          & =_{1717} 809 * 1^{25} = 809
    \end{split}
\end{equation}

Ist identisch für das Entschlüsseln nur mit anderem Ansatz

\begin{equation}
    \begin{split}
        M & =_n C^d                                 \\
          & =_{1717} 809^{1491} = 809*(809^2)^{745} \\
          & =_{1717} 809*304*(304^2)^{372}          \\
          & =_{1717} 405 * (1415^2)^{186}           \\
          & =_{1717} 405 * (203^2)^{93}             \\
          & =_{1717} 405 * 1^{93} = 405
    \end{split}
\end{equation}

\subsection{Chinesische Restsatz}

\section{Vektorräume}

\subsection{$\mathbb{R}^2$ und $\mathbb{R}^3$ als Vektorraum}

Sind Ursprünglich Definiert als die Produktmenge mehrere Mengen.

\begin{center}
    $\mathbb{R}^n=\{[x_1, x_2, \dots, x_n]\mid x_1, x_2, \dots, x_n\in \mathbb{R}\}$
\end{center}

Wir können Vektoren miteinander addieren und sie auch mit einem einfachen Skalar, multiplizieren:

\begin{center}
    $x+y$ oder halt $t*x$
\end{center}

Es gibt zudem noch den Nullvektor, welche einfach $0:=[0, 0, \dots, 0]$ definiert wird.

\know{Linearkombination}{Ist im allgemeinsten eine addition von gestreckten Vektoren
    \begin{center}
        $t_1x_1+t_2x_2+\dots t_nx_n$
    \end{center}}

\subsubsection{Geraden und Ebenen}

Im allgemeinsten können wir Geraden und Ebenen als Menge Beschreiben, welche einen Stützvektor $q$ haben und eine Anzahl an Richtungsvektoren. Für Gerade ist es ein Richtungsvektor $u$ und bei der Ebene sind es zwei $u$ und $v$.

\begin{center}
    $g:=\{q+tu\mid t\in \mathbb{R}\}$

    $e:=\{q+tu+sv\mid t, s\in \mathbb{R}\}$
\end{center}

\subsubsection{Norm}

Die Norm: Ist die Länge eines Vektors und lässt sich mit dem Satz von Pythagoras bestimmen.

\begin{center}
    $\|x\|:=\sqrt{x_1^2+x_2^2+\dots+x_n^2}$
\end{center}

\subsubsection{Skalarprodukt}

Skalarprodukt: Dies ermöglicht den Winkel zwischen zwei Vektoren zu bestimmen.

\begin{center}
    $\langle x\mid y\rangle:=\|x\|*\|y\|*\cos(\alpha)$
\end{center}

Mithilfe des Kosinussatzes $c^2=a^2+b^2-2ab\cos(\gamma)$ und etwas umformen erhalten wir die einfache Form vom Skalarprodukt:

\begin{equation}
    \langle x\mid y\rangle = x_1y_1+x_2+y_2+x_3y_3
\end{equation}

Eigenschaften des Skalarproduktes:

\begin{enumerate}[\qquad i)]
    \item $\langle x\mid x\rangle\geq 0, =0\iff x=0$ \qquad\qquad  Definitheit
    \item $\langle x\mid y\rangle=\langle y\mid x\rangle$ \qquad\qquad\qquad\qquad\qquad Symmetrie
    \item $\langle tx + sy\mid z\rangle=t*\langle x\mid z\rangle+s*\langle y\mid z\rangle$ \qquad Linearität
\end{enumerate}

Um nun den Winkel $\alpha$ zu bestimmen, stellen wir nach $\cos(\alpha)$ um:

\begin{equation}
    \cos(\alpha)=\frac{\langle x\mid y\rangle}{\|x\|*\|y\|}
\end{equation}

\know{Senkrecht: $\langle x\mid y\rangle=0$}{Wenn das Skalarprodukt gleich $0$ ist, so stehen die beiden Vektoren seknrecht aufeinander. Dies wird auch Orthogonal genannt:
    \begin{center}
        $x\bot y:\iff \langle x\mid y\rangle=0$
    \end{center}}

\subsubsection{Kreuzprodukt}

Das Kreuzprodukt zweier Vektoren $x$ und $y$ resultiert in einem Vektor, welche Senkrecht zur $x$ und $y$ ist.

\begin{equation}
    x\times y:=
    \begin{bmatrix}
        x_2y_3 - x_3y_2 \\
        x_3y_1 - x_1y_3 \\
        x_1y_2 - x_2y_1 \\
    \end{bmatrix}
\end{equation}

Eigenschaften

\begin{enumerate}[\qquad i)]
    \item $(tx + sy)\times z=tx\times z+sy\times z$
    \item $x\times y = -y\times x$
    \item $x\times y=0\iff x\| y$
    \item $\|x\times y\|=\|x\|*\|y\|\cos(\alpha)$ Die Länge des Skalarproduktes ist der Flächeninhalt des Parallelogram zwischen $x$ und $y$, da $h=\|y\|\cos(\alpha)$
\end{enumerate}

\subsubsection{Normalform der Ebene}

Wir so gebaut: Wir haben einen Punkt in der Ebene $p\in E$ und bilden alle Vektoren $x-p$. Diese Vektoren sind nur dann $\in E$, wenn diese Senkrecht zu dem Normalvektor $n$ sind. Dieser Normalvektor ist auch Normiert $n_0=n / \|n\|$. Also:

\begin{center}
    $E:=\{x\in \mathbb{R}^3\mid \langle x-p\mid n_0\rangle=0\}$

    $E:=\{x\in \mathbb{R}^3\mid \langle x\mid n_0\rangle=\langle p\mid n_0\rangle\}$
\end{center}

\subsubsection{Koordinatenform}

In der Koordinatenform ist der Normalvektor kodiert.

\begin{center}
    $E=\{x_1, x_2, x_3\in \mathbb{R}\mid n_1x_1+n_2x_2+n_3x_3=d\}$
\end{center}

Der Normalvektor setzt sich aus $n=[n_1, n_2, n_3]$ zusammen.

Und unser $d=\langle p\mid n_0\rangle$, wo $p$ unser Punkt in der Ebene ist.

\subsubsection{Abstand: Punkt zur Ebene}

Da das Skalarprodukt eine Projektion von einem Vektor $y$ auf $x$ ist, können wir wenn wir einen Punkt (Vektor) auf den Normalvektor projektieren, den Abstandsvektor bestimmen und dessen Norm (Länge) ist die Lösung:

\begin{equation}
    d(y, E):=\frac{1}{\|n\|}*|\langle x\mid n_0\rangle-\langle p\mid n_0\rangle|
\end{equation}

\subsubsection{Abstand: Punkt zur Geraden}

Wir haben einen Punkt $q$ und eine Gerade $g:=\{p+tu\mid t\in \mathbb{R}\}$. Wenn $\|u\|=1$ und somit $u$ die Länge 1 hat, ist die Fläche von dem Parallelograms gleich der Fläche von dem Abstand $d(q, g) * \|u\|$. Da $\|u\|=1$ ist die Fläche gleich dem Wert von $d(q, g)$.

\begin{equation}
    d(q, g)=\|u\times(q-p)\|
\end{equation}

Siehe \href{https://elearning.dhbw-stuttgart.de/moodle/pluginfile.php/812513/mod_folder/content/0/Vorlesungsskript/stgt-math_20250130.pdf#page=91}{Abbildung 5.9} im Skriptum

\subsubsection{Spatprodukt und Determinante}

Das Spatprodukt kombiniert die geometrischen Interpretationen von dem Kreuzprodukt $x\times y$ und dem Skalarprodukt $\langle x\mid y\rangle$.

So bildet das Kreuzprodukt die Grundfläche des Spats und das Skalarprodukt wird auf den Normalvektor dieser Grundfläche projektiert und ist somit die Höhe.

\begin{center}
    $\langle x\times y\mid z\rangle$
\end{center}

Diese Spatprodukt kann positiv und negativ sein. Wenn es negativ ist, benutzen wir zum korrigieren einfach $-z$ statt $z$.

\know{Positives Spatprodukt}{Wenn das Spatprodukt positiv ist, sprechen wir bei den drei Vektoren von einem Rechtssystem oder auch mathematisch positiv orientiert
    \begin{center}
        Rechte-Hand-Regel
    \end{center}}

\subsubsection{Determinante}

Die Determinante ist die Verallgemeinerung des Spatproduktes auch für höhere Dimensionen.

\begin{center}
    $det(x, y, z)=\langle x\times y\mid z\rangle$

    $det(x, y, z + tx + sy)=det(x, y, z)$
\end{center}

Die Determinante hat zudem noch die Eigenschaft, dass wenn $det(x, y, z)=0$ gilt, sind die Vektoren linear abhängig. Das heißt sie bilden kein Volumen und einer der Vektoren ist "unnötig".

\subsection{Die komplexen Zahlen}

Immer mal wieder taucht etwas in der Form $x^2=-1$ auf, welche keine Reellen Lösungen hat. Somit entsteht die Idee von $i$:

\begin{center}
    $i^2=-1$
\end{center}

Nun bestehen Zahlen in $\mathbb{C}$ aus einem Reellen Part und einem Imaginärem Part. z.B. $5+3i$.

\subsubsection{Rechnregeln}

\begin{align}
    (z_1+iz_2)\pm(w_1+iw_2) & =z_1\pm w_1 + i(z_2\pm w_2)        \\
    (z_1+iz_2)(w_1+iw_2)    & =z_1w_1 -z_2w_2 + i(z_1w_2+z_2w_1)
\end{align}

\subsubsection{Gaussche Zahlenebene}

Da alle Imaginären Zahlen aus genau zwei werte bestehen, wovon einer imaginär ist, können wir diese auch als ein Zahlenpaar im $\mathbb{R}^2$ darstellen.

Somit ganz geschickt in einer Zahlenebene mit den Reellen Zahlen auf der x-Achse und den Imaginären auf der y-Achse. Somit befinden wir uns wieder unter Vektoren mit Zusätzlichen Definitionen:

\begin{center}
    \begin{tabular}{c c}
        Betrag                    & Konjugiert Komplex  \\\\
        $|z|:=\sqrt{z_1^2+z_2^2}$ & $\bar{z}:=z_1-iz_2$ \\
    \end{tabular}
\end{center}

\subsubsection{Polardastellung}

Wir können Komplexe Zahlen auch in der Polardarstellung darstellen:

\begin{center}
    $z=|z|*(\cos(\varphi)+i\sin(\varphi))$
\end{center}

\subsubsection{EULER-Formel}

\begin{center}
    $e^{i\varphi}=\cos(\varphi)+i\sin(\varphi)$

    $z=|z|e^{i\varphi}$
\end{center}

Dies gilt und formmal gelten somit auch die Additionssätze der Exponentialfunktionen. Somit ist $e^{i\varphi}e^{i\Psi}=e^{i(\varphi+\Psi)}$ und auch das gegenpaar:

\begin{center}
    $(\cos(\varphi)+i\sin(\varphi)*\cos(\Psi)+i\sin(\Psi))=\cos(\varphi+\Psi)+i\sin(\varphi+\Psi)$
\end{center}

Geometrisch ist somit eine Multiplikation/ Division einfach eine Addition/ Subtraktion der Winkel. Und Potenzierung einfach die Multiplikation des Winkels.

Außerhalb des Einheitskreis muss die Normale Operation auf die Länge noch beachtet werden!

\subsubsection{Beispiele}

\begin{enumerate}[\qquad i.]
    \item $e^{i\frac{\pi}{2}}=i$
    \item $e^{i\pi}=-1$
    \item $e^{2\pi i}=1$
    \item $\bar{e^{i\varphi}}=e^{-i\varphi}$
\end{enumerate}

\subsubsection{n-ten Wurzeln ziehen}

Wir können uns leicht davon überzeugen, dass es für die n-te Wurzel n Lösungen gibt, denn die Gleichung $e^{p*2\pi i}=1$ ist periodisch und hat mehrere Lösungen für $p\in \mathbb{Z}$.

Für die Einheitswurzeln, also $=1$ lassen sich nach:

\begin{equation}
    u_p:=e^{ip*\frac{2\pi}{n}}
\end{equation}

Wenn wir nun $u_p^n$ bilden wird schnell klar, dass alle Lösungen $=1$ sein müssen.

Für den Allgemeinfall der Form: $z=|z|e^{i\varphi}$ gibt es die Grundlösung:

\begin{center}
    $w_0:=\sqrt[n]{|z|}e^{i\frac{\varphi}{n}}$
\end{center}

Diese können wir mit den Einheitswurzeln multiplizieren um alle Lösungen zu bekommen.

\know{Ablauf: $n$-ten Wurzel ziehen}{Winkel $\varphi$ durch $n$ teilen und für alle $n$-Lösungen den Winkel der Einheitswurzeln $\frac{\varphi}{n}*n$ dazu-Addieren.
    \begin{center}
        $w_p=e^{\frac{1}{n}(\varphi+p*2\pi)}$
    \end{center}
    Und nicht vergessen die $n$-te Wurzel von der Länge auszurechnen}

\subsubsection{Kubische Gleichungen/ Kardanische Formeln}

Um die Nullstellen von Kubische Gleichungen zu finden braucht man die Kardanischen Formel. Für Näherungswerte reicht auch Newtons-Methode.

Die Kardanische Formel bestehen aus mehreren Schritten und funktionieren in der Form $x^3+bx^2+cx+d=0$:

\begin{enumerate}
    \item Gleichung in die Normalform $y^3+py+q=0$ umwandeln, indem man mit $x=y-\frac{b}{3}$ substituiert.
    \item $p$ und $q$ bestimme mit: $p=c-\frac{1}{3}b^2$ und $q=\frac{2}{27}b^3-\frac{1}{3}bc+d$
    \item Die Diskriminante bestimme: $\Delta=(\frac{p}{3})^3+(\frac{q}{2})^2$
    \item Wenn $\Delta>0$ ist, so gibt es eine Reelle Lösung:
          \begin{equation}
              x_1:=\sqrt[3]{\sqrt{\Delta}-\frac{q}{2}}-\sqrt[3]{\sqrt{\Delta}+\frac{q}{2}}-\frac{b}{3}
          \end{equation}
    \item Die anderen Fallunterscheidungen der Diskriminante sind im \href{https://elearning.dhbw-stuttgart.de/moodle/pluginfile.php/812513/mod_folder/content/0/Vorlesungsskript/stgt-math_20250130.pdf#page=100}{Skriptum}
\end{enumerate}

\subsection{$\mathbb{C}^n$ als Vektorraum}

In dem Komplexen Vektorraum müssen wir unsere Norm und das Skalarprodukt leicht ändern:

\begin{align*}
    \|x\|                  & :=\sqrt{\sum_{k-1}^{n}|x_k|^2}     \\
    \langle x\mid y\rangle & := \sum_{k-1}^{n}\overline{x_k}y_k
\end{align*}

Diese Änderungen behalten die gewünschten Eigenschaften $\langle x\mid x\rangle=\|x\|^2$.

Aber: Durch unsere Änderung in dem Skalarprodukt, ist dies nur noch in der zweiten Komponente $y$ linear, in der ersten ist Sie antilinear, da $\overline{x}$ benutzt wird, selbts wenn $x$ die Eingabe ist.

\begin{center}
    $\langle \lambda x_1 + \gamma x_2 \mid y\rangle = \overline{\lambda}\langle x_1 \mid y\rangle + \overline{\gamma}\langle x_2 \mid y\rangle$
\end{center}

\subsection{Der allgemeine Vektorraumbegriff$^*$}

Die Kurzform ist: Ein Vektorraum ist eine Menge $V$, wessen Elemente $x, y, z$ sich in einem Körper $\mathbb{K}$ befinden.

Wir können Vektoren miteinander Addieren und diese mit Skalaren $t, s, r$ strecken, ohne dass diese den Vektorraum verlassen. So sind $x+y\in V$ und $t*x\in V$.

\begin{enumerate}
    \item Es gibt immer einen Nullvektor, $x+0=x$
    \item Es gibt immer eine Inverse, $x+(-x)=0$
    \item Es gibt das Einselement $1\in \mathbb{K}$, $1*x=x$
\end{enumerate}

\subsubsection{Teilmengen des Vektorraums}

Eine Teilraum $W\subset V$ besteht dann, wenn wir durch Addition oder Multiplikation mit Skalaren, diesen Teilraum nicht verlassen.

Dafür kennen wir die Linearkombination:

\begin{center}
    $s*x+t*y\in W$
\end{center}

\subsection{Vektorräume mit Norm und Skalarprodukt}

Hier wird im Skript der Zusammenhang zwischen Norm und Skalarprodukt im allgemeinem Vektorraum gezeigt.

Mit vielen Beispielen und Beweisen, welche ich hier nicht übernehmen werde.

\documentclass[a4paper]{article}

%\usepackage{url}

%% Math
\usepackage{mathtools}
%% For Mengen like natural numbers
\usepackage{amsfonts}
%% Für spezielle Symbole
\usepackage{amssymb}

%% Images
\usepackage{import}
\usepackage{xifthen}
\usepackage{pdfpages}
%\usepackage{transparent}

%%% Command for simpler images
\newcommand{\incfig}[1]{%
    \def\svgwidth{\columnwidth}
    \import{./fig/}{#1.pdf_tex}
}

%% Links
\usepackage{hyperref}
\hypersetup{
    colorlinks=true,
    linkcolor=black,
    filecolor=magenta,
    urlcolor=cyan
}

%% Formatting
\usepackage{parskip}

\title{Lineare Algebra}
\author{Moritz}
\date{February 19, 2025}

\begin{document}
\maketitle
\tableofcontents

\section{Exkurs: Hamming Code}

Ist in meinem Heft unter LA 19.02 Hamming Code.

Grundlegend Codieren wir unseren Binären Vektor $w$ und multiplizieren diesen mit dem Generator $G$. Dieser ist die 1-Matrix verlängert mit Redundanz, sodas $H*G=0$, mit $H$ die Kontrollmatrix Binär Hochzählt.

Wenn $HGw\neq 0$, so ist ein Fehler aufgetreten, welche an der Binär zu lesenden Stelle der Multiplikation.

\section{Matrizen}

\subsection{Selbstadjungierte Abbildungen}

Satz: Jede selbstadjungierte lineare Abbildung hat eine ONB aus Eigenvektoren.

Beweis: Induktion nach der Dimension $n$ des Raumes.

$n=1$: trivial

$n\to n+1$: $dim V = n: \langle x\mid Ay\rangle = \langle Ax\mid y\rangle$ f.a. $x, y\in V$

$\implies \exists$ ONB aus EVen $\{b_1, \dots, b_2\} (IV)$ $(Ab_1=\lambda_1b_1, Ab_2=\lambda_2b_2, \dots)$

\begin{equation}
    \begin{split}
         & dim V = n+1: \langle x\mid Ay\rangle = \langle Ax\mid y\rangle\dots \\
         & \text{Es gibt wenigsten einen EV } b_0\quad (\neq 0)                \\
         & \text{Bilde } V_1:= b_0^{\bot}:=\{x\in V\mid x\bot b_0\}            \\
         & \text{Teilraum von } V \text{, der Dimension }n.
    \end{split}
\end{equation}

Zeige: $AV_1\subseteq V_1$, d.h. zu zeigen: $x\in b_0^{\bot}\implies Ax\in b_0^{\bot}$

Also: $\langle Ax\mid b_0\rangle=\langle x\mid Ab_0\rangle=\lambda \langle x\mid b_0\rangle=0$

Nach (IV) folgt: $\exists$ ONB aus EVen für $V_1:\{b_1, b_2, \dots, b_n\}$. Dann ist $B=\{b_0, b_1, b_2, \dots, b_n\}$ eine ONB aus EVen für $V$.

\subsubsection{Umgangsprachlich}

Wir haben gezeigt, dass wir den neuen Vektor, welche zu allen anderen Eigenvektoren Orthogonal ist, nicht diese Eigenschaft durch Matrizenmultiplikation verliert.

Oder. Dieser neue Eigenvektor wird nicht aus dem Raum, wo dieser Orthogonal zu allen anderen Eigenvektoren ist, herausgebogen wird.

\subsection{Unitäre Abbildungen}

Wenn diese Reell sind, so sind diese Orthogonal.

Eigenschaft: $U^*U=1$ (d.h. $U^{-1}=U^*$) $\implies UU^*=1$

hat ebenfalls eine ONB aus Eigenvektoren; die Eigenwerte $\lambda_k$ haben den Betrag $|\lambda_k|=1$ (daher der Name)

Zur Hilfe: $Ux=\lambda x$
\begin{equation}
    \begin{split}
        \|Ux\|^2                 & =\|\lambda x\|^2=|\lambda|^2\|x\|^2                        \\
        \langle Ux\mid Ux\rangle & =\langle x\mid U^*Ux\rangle=\langle x\mid x\rangle=\|x\|^2
    \end{split}
\end{equation}

Beispiel in meinem Heft LA 19.02 Unitär.

\subsection{Was können wir mit den selbstadjungierten machen?}

$A=A*, \mathfrak{B}=\{b_1, \dots, b_n\}\implies B=[b_1, \dots, b_n]\implies B^*=B^{-1}$

Koordinatentransformation von A

\begin{equation}
    \begin{split}
        A=B
        \begin{bmatrix}
            \lambda_1 &        &           \\
                      & \ddots &           \\
                      &        & \lambda_n
        \end{bmatrix}B^*\implies B^*AB=
        \begin{bmatrix}
            \lambda_1 &        &           \\
                      & \ddots &           \\
                      &        & \lambda_n
        \end{bmatrix} \\
        \begin{bmatrix}
            b_1^* \\b_2^*\\\vdots\\b_n^*
        \end{bmatrix}[Ab_1, Ab_2, \dots, Ab_n]=
        \begin{bmatrix}
            \langle b_1\mid \lambda_1b_1\rangle &                                     \\
            \langle b_2\mid \lambda_2b_1\rangle & \langle b_2\mid \lambda_2b_2\rangle
        \end{bmatrix} =
        \begin{bmatrix}
            \lambda_1 & 0         \\
            0         & \lambda_2
        \end{bmatrix}
    \end{split}
\end{equation}

\subsubsection{Quatriken}

Kegelschnitte: Ellipse $\frac{x^2}{a^2}+\frac{y^2}{b^2}=1$ und Hyperbel $\frac{x^2}{a^2}-\frac{y^2}{b^2}=1$.

Dazwischen liegen die Parabeln.

Allgemein können wir dies auch als Matrizen und Vektoren schreiben:

\begin{equation}
    \{x\in \mathbb{R}^2\mid \langle x\mid Ax\rangle+\langle b\mid x\rangle + c = 0\}=Q
\end{equation}

Diese kann umgewandelt werden unter der Annahme, dass $A=A^*$ gilt, zu: $a_1x^2+2a_2xy+a_4y^2+b_1x+b_2y+c=0$

Zahlenbeispiel im Heft LA 19.02 Quatriken

Mit der Eigenwerttheorie können wir in der neuen Basis $\mathfrak{B}$ und den Eigenwerten $\lambda$ die Gleichung entkoppeln.

\begin{equation}
    \lambda_1\tilde{x}^2+\lambda_2\tilde{y}^2+b1_B\tilde{x}+b2_b\tilde{y}+c
\end{equation}

\end{document}
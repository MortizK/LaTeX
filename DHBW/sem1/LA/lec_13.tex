\documentclass[a4paper]{article}

%\usepackage{url}

%% Math
\usepackage{mathtools}
%% For Mengen like natural numbers
\usepackage{amsfonts}
%% Für spezielle Symbole
\usepackage{amssymb}

%% Images
\usepackage{import}
\usepackage{xifthen}
\usepackage{pdfpages}
%\usepackage{transparent}

%%% Command for simpler images
\newcommand{\incfig}[1]{%
    \def\svgwidth{\columnwidth}
    \import{./fig/}{#1.pdf_tex}
}

%% Links
\usepackage{hyperref}
\hypersetup{
    colorlinks=true,
    linkcolor=black,
    filecolor=magenta,
    urlcolor=cyan
}

%% Formatting
\usepackage{parskip}

\title{Lineare Algebra}
\author{Moritz}
\date{January 28, 2025}

\begin{document}
\maketitle
\tableofcontents

\section{Vektoren}

\subsection{Lineare (Un)Abhängigkeit}

Lineare (Un)Abhängigkeit einer Teilmenge $\{b_1, b_2, \dots, b_n\}\subset V$ (V: Vektorraum) heißen l.u., falls

\begin{center}
    jede Nullkombination der Vektoren trivial ist!

    (d.h.: $t_1b_1+t_2b_2+\dots+t_nb_n=0\implies t_1=t_2=\dots=t_n=0$)
\end{center}

andernfalls l.a.

Nach GAUSS Form: \begin{equation}
    \begin{array}{cc}
        \begin{bmatrix}
            . & . & . & . & . \\
              & . & . & . & . \\
              &   & . & . & . \\
              &   &   & . & . \\
              &   &   &   & . \\
        \end{bmatrix}     &
        \begin{bmatrix}
            . & . & . & . & . \\
              & . & . & . & . \\
              &   & . & . & . \\
              &   &   & . & . \\
              &   &   &   &   \\
        \end{bmatrix}          \\
        \text{Linear Unabhängig} &
        \text{Linear Abhängig}
    \end{array}
\end{equation}

\know{Linear Abhängig}{Hier bleiben Parameter Übrig und wir müssen die anderen PAramter in abhängigkeit angeben.}

\subsubsection{Dimension}

Definition: Die Maximale Anzahl $n$ von einer linear unabhängigen Menge $\mathcal{B}:=\{b_1, \dots, b_n\}\subset V$ heißt Dimension von $V$,

\begin{center}
    dim $V=n$
\end{center}

Falls es keine maximale Anzahl gibt, heißt V $\infty$-Dimensional: dim $V=\infty$. z.B. in dem Vektorraum der Polynome (Alle Funktion $p^n$).

Einordnung von l.a. in unsere Vorstellung:

\begin{center}
    $t_1b_1+t_2b_2+\dots+t_nb_n=0$ mit $[t_1, t_2, \dots, t_n]\neq 0$
\end{center}

Dies wäre eine nicht triviale Nullkombination. oBdA: $t_1\neq 0$ und $t_1=1\implies b_1+t_2b_2+\dots+t_nb_n$. Da $t_1$ nicht Null ist, können wir alle anderen $t$ durch $t_1$ teilen und neuere Faktoren definieren. Somit können wir sagen, dass $t_1=1$ sein muss.

\begin{equation}
    b_1=-t_2b_2-\dots-t_2b_n
\end{equation}

d.h.: $b_1$ lässt sch au den restlichen linear kombinieren.

\know{Basis}{Eine solche Menge $\mathcal{B}$ heißt Basis von V. Dann lässt sich jeder Vektor $x\in V$ in dieser Basis entwickeln}

d.h. $x=\tilde{x}_1b_1+\tilde{x}_2b_2+\dots+\tilde{x}_nb_n$.

Koordinatenvektor bzg. der Basis $\mathcal{B}$.

\begin{equation}
    x_{\mathcal{B}}:=\begin{bmatrix}
        \tilde{x}_1 \\\tilde{x}_2\\\vdots\\\tilde{x}_n
    \end{bmatrix}_{\mathcal{B}}
\end{equation}

Diese Darstellung ist eindeutig. Annahme:

\begin{equation}
    \begin{split}
        x & = \tilde{x}_1b_1+\dots+\tilde{x}_nb_n                                                          \\
        x & = \tilde{y}_1b_1+\dots+\tilde{y}_nb_n                                                          \\
        0 & = (\tilde{x}_1-\tilde{y}_1)b_1+(\tilde{x}_2-\tilde{y}_2)b_2+\dots+(\tilde{x}_n-\tilde{y}_n)b_n \\
    \end{split}
\end{equation}

muss eine triviale Nullkombinationsein $\implies \tilde{x}_1=\tilde{y}_1,\dots,tilde{x}_n=\tilde{y}_n$.

\section{Komplexen Zahlen $\mathbb{C}$}

\know{Komplexen Zahlen $\mathbb{C}$}{\begin{center}
        $i^2=-1$
    \end{center}}

\subsection{Mitternachtsformel}

\begin{equation}
    \begin{split}
        x^2+px+q & =0                                      \\
        x_{1/2}  & =-\frac{p}{2}\pm \sqrt{\frac{p^2}{4}-q}
    \end{split}
\end{equation}

Um zu abc-Formel zu kommen, einfach $p=\frac{b}{a}$ und $q=\frac{c}{a}$

Wo kommt sie her?

\begin{equation}
    \begin{split}
        (x+r)^2                         & =t                          \\
        x^2+2rx+r^2                     & =t                          \\
        x^2+2px+q                       & =0                          \\
        x^2+2*\frac{p}{2}               & =-q                         \\
        x^2+2*\frac{p}{2}+\frac{p}{2}^2 & =-q+\frac{p}{2}^2           \\
        (x+\frac{p}{2})^2               & =\frac{p^2}{4}-q            \\
        x_{1/2}+\frac{p}{2}             & = \pm\sqrt{\frac{p^2}{4}-q}
    \end{split}
\end{equation}

\subsubsection{Nullstellen Raten}

\begin{equation}
    \begin{split}
        3x^3-x^2+5x-7     & =0             \\
        \text{Annahme } x & \in \mathbb{Z} \\
        3x^3-x^2+5      x & =7             \\
        x(3x^2-x+5)       & =7
    \end{split}
\end{equation}

Somit ist $x$ ein Teiler von 7 (auch negativ).

\subsubsection{Kubische Parabeln}

$ax^3+bx^2+cx+d=0$ CARDANO-Formeln. Kann transformiert werden zu $y^3+py+q=0$.

\begin{equation}
    \begin{split}
        y      & =n-v                                                   \\
        u      & =\sqrt[3]{\sqrt{\Delta}-\frac{q}{2}}                   \\
        v      & =\sqrt[3]{\sqrt{\Delta}+\frac{q}{2}}                   \\
        \Delta & =\left(\frac{p}{3}\right)^3+\left(\frac{q}{2}\right)^2
    \end{split}
\end{equation}

\subsection{Definition}

\begin{equation}
    \mathbb{C}:=\{a+ib\mid a, b\in \mathbb{R}\}
\end{equation}

\begin{equation}
    \begin{split}
        z            & =a+ib                                     \\
        z            & =2+3i                                     \\
        w            & =5-8i                                     \\
        z+w          & =7.5i                                     \\
        z*w          & =(2+3i)*(5-8i)                            \\
                     & = 10 - 16i + 15i - 24i^2                  \\
                     & = 10 - 16i + 15i - 24*(-1)                \\
                     & = 34 - i                                  \\
        (a+ib)(c+id) & = ac-bd+i(bc+ad)                          \\
        \frac{1}{z}  & = \frac{1}{a+ib}*\frac{a-ib}{a-ib}        \\
                     & = \frac{a-ib}{a^2+b^2}=-i\frac{b}{a^2b^2}
    \end{split}
\end{equation}

\know{(Rechner) Mathematik}{$z=[a, b]$. Im Beispiel ist $z=[2, 3], w=[5, -8], z+w=[2+5, 3-8]$. Dies erinnert an die Vektorrechnung im $\mathbb{R}^2$ mit ein paar extra Rechenregeln.

\begin{center}
    \begin{tabular}{c c c}
        $[a, b]+[c, d]$ & := & $[a+c, b+d]$                           \\
        $[a, b]*[c, d]$ & := & $[ac-bd, bc+ad]$                       \\
        $[0, 1]*[0, 1]$ & =  & $[-1, 0]$                              \\
        $[a, b]^{-1}$   & := & $[\frac{a}{a^2b^2},-\frac{b}{a^2b^2}]$
    \end{tabular}
\end{center}}

\begin{center}
    $\mathbb{C}=\mathbb{R}^2+$Rechengesetze
\end{center}

\subsubsection{Darstellung}

\begin{figure}[ht]
    \centering
    \incfig{komplexe}
    \caption{komplexe}
    \label{fig:komplexe}
\end{figure}

\begin{equation}
    \begin{split}
        z       & =a+ib  \\
        \bar{z} & :=a-ib
    \end{split}
\end{equation}

$b$ ist der imaginäre Teil von $z$, kurz $Im(z)$. Das gleiche für den Realen Teil: $Re(z)$. SOmit hat jede imaginäre Zahl die Form:

\begin{equation}
    \begin{split}
        z         & =Re(z)+i*Im(z)                                                             \\
        z*\bar{z} & =a^2+b^2 = |z|^2                                                           \\
        z^{-1}    & = \frac{a-ib}{a^2+b^2} = \frac{\bar{z}}{z*\bar{z}} = \frac{\bar{z}}{|z|^2} \\
        z*z^{-1}  & = z *\frac{\bar{z}}{|z|^2} = \frac{z*\bar{z}}{|z|^2}                       \\
                  & = \frac{|z|^2}{|z|^2} = 1
    \end{split}
\end{equation}

Nennt man auch die zu $z$ konjungiert komplexe Zahl.

\section{Besprechen}

Termin der Übungsklausur!

\end{document}
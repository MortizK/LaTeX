\documentclass[a4paper]{article}

%\usepackage{url}

%% Math
\usepackage{mathtools}
%% For Mengen like natural numbers
\usepackage{amsfonts}
%% Für spezielle Symbole
\usepackage{amssymb}

%% Images
\usepackage{import}
\usepackage{xifthen}
\usepackage{pdfpages}
%\usepackage{transparent}

%%% Command for simpler images
\newcommand{\incfig}[1]{%
    \def\svgwidth{\columnwidth}
    \import{./fig/}{#1.pdf_tex}
}

%% Links
\usepackage{hyperref}
\hypersetup{
    colorlinks=true,
    linkcolor=black,
    filecolor=magenta,
    urlcolor=cyan
}

%% Formatting
\usepackage{parskip}

\title{Lineare Algebra}
\author{Moritz}
\date{December 10, 2024}

\begin{document}
\maketitle
\tableofcontents

\section{Aussagenlogik}

Wir müssen von tatsachen folgern um etwas zu beweisen. Ich kann nicht aus dem Ziel Schlüsse ziehen, da das Ziel ja falsch sein kann.

\subsection{Aussageformen}

\begin{equation}
    \begin{split}
        x\in X:          & A(x) \text{ eine Aussage}                 \\
        \forall_{x\in X} & A(x) \text{ "für alle"}                   \\
        \exists_{x\in X} & A(x) \text{ "es gibt wenigstens eins..."}
    \end{split}
\end{equation}

Beispiel für Aussage mit Primzahlen und dem Versuch eines Widerspruchsbeweis.

\begin{equation}
    \begin{split}
        X            & =\mathbb{N}, P(n): n \text{ ist Primzahl} \\
        U            & := \forall_{n\in \mathbb{N}}\exists_{
            \begin{array}{c}
                m\in \mathbb{N} \\
                m\geq n
            \end{array}
        }P(m) \text{ Es gibt unendlich viele Primzahlen}         \\
        \overline{U} & =\exists_{n\in \mathbb{N}}\forall_{
            \begin{array}{c}
                m\in \mathbb{N} \\
                m\geq n
            \end{array}
        }\overline{P(x)} \text{ Es eine größte Primzahl}
    \end{split}
\end{equation}

\subsection{Mengen}

Definition (Cantor 1895)

Unter einer Menge verstehen wir die Zusammenfassung von bestimmten, wohlunterschiedenen Objekten unserer Anschauungen, oder unseres Denkens zu einem Ganzen.

wohlunterschiedenen bedeutet, dass die $2$ nicht mehrmals in einer Menge gibt. Also $\{2, 4\}\cap\{1,2,3\}=\{1,2,3,4\}$

\subsubsection{Beispiele}

\begin{equation}
    \begin{split}
        W_1:= & \{1,2,3,4,5,6\}\text{ Würfelergebnisse mit 1 Würfel}      \\
        W_2:= & \{                                                        \\
              & (1,1),(1,2),\dots,(1,6)                                   \\
              & (2,1),(2,2),\dots,(2,6)                                   \\
              & \vdots                                                    \\
              & (6,1),(6,2),\dots,(6,6)
        \}                                                                \\
        W_2:= & \{(1,1),(1,2),(1,3),(1,4),(1,5),(1,6),(2,2),(2,3),\dots\} \\
    \end{split}
\end{equation}

Bei dem zweitem $W_2$ ergibt sich ein Dreieck, wo die Reihenfolge der Tuple nicht wichtig ist.

\begin{equation}
    \begin{split}
        G:= & \{2,4,6,7,10,\dots\}         \\
        U:= & \{1,3,5,7,\dots\}            \\
        F:= & \{1,2,3,5,8,13,21,34,\dots\} \\
    \end{split}
\end{equation}

\subsubsection{Andere Logik, andere Statistik}

Bei Quantencomputern und Polarisation von Licht, sind die Wahrscheinlichkeiten im gegensatz zum zweitem $W_2$ alle gleich.

Es gibt keine Unterscheidbarkeit.

\subsubsection{Schreibweise von Allgemeinen Mengen}

Die Pünktchen haben bei komischen Formeln für eine Menge keinen bestand, also eine Neue Schreibweise.

Methode 1: $\{f(x)\mid x\in X\}$

\begin{equation}
    \begin{split}
        \{4,5,8,\dots\} & =\{n^2-2n+5\mid n\in \mathbb{N}\}
    \end{split}
\end{equation}

Die Formel vor dem $\mid$ ist der Formelteil "Form der Elemente der Menge und nach dem $\mid$ der Deklarationsteil der Variablen.

Methode 2: $\{x\in X\mid A(x)\}$

Diese Methode kann nur teilmengen einer anderen Menge beschreiben.

\begin{equation}
    \begin{split}
        G=\{n\in \mathbb{N}\mid \exists_{k\in \mathbb{N}}n=2k\}\subset \mathbb{N}
    \end{split}
\end{equation}

Die Formel nach dem $\mid$ ist eine Aussageform über $\mathbb{N}$ und vor dem $\mid$ steht die Menge, auf die die Aussageform geprüft wird. Und die neue Menge ist nur die Menge, wo die Aussage war ist.

\subsubsection{Notationen}

\begin{equation}
    \begin{split}
        \in\dots                     & \text{ Element von ...}             \\
        x\in X                       & \text{ x ist (ist) Element von X}   \\
                                     & \text{ x gehört zu X}               \\
        \overline{x\in X}=:x\notin X & \text{ x (ist) nicht Element von X} \\
                                     & \text{ x gehört nicht zu X}         \\
    \end{split}
\end{equation}

\begin{equation}
    \begin{split}
        A\subseteq B & :\iff x\in A \implies x\in B                                  \\
        A\subset B   & :\iff (x\in A \implies x\in B)\land\exists_{x\in B} x\notin A \\
    \end{split}
\end{equation}

\subsubsection{Lemma}

Ein Lemma ist ein Hilfsmittel, welcher zum Beweis eines Satzes genutzt wird, selber aber kein Satz ist.

\begin{equation}
    \begin{split}
        A=B & \iff A\subseteq B \land B\subseteq A \\
        A=B & \to \dots \text{trivial}
    \end{split}
\end{equation}

Ich darf im Studium niemals etwas als trivial beschreiben.

\subsection{Widerspruchsbeweis}

Bezogen auf das Lemma.

"$\implies$" trivial.

"$\impliedby$": Durch Widerspruch.

Wir gehen von $A\subseteq B \cap B\subseteq A$ aus und nehmen $\overline{A=B}=:A\neq B$ an.

Das heißt (D.h.) es gibt wenigstens ein Element $x\in A$ und $x\notin B$, oder ein $x\in B$ und $x\notin A$.

% ist wichtig
O.B.d.A (Ohne Beschränkung der Allgemeinheit)

Sei $x\in A \land x \notin B$, d.h $x\in A$ und daher $x\in B$, denn $A\subseteq B$. D.h $x\in B \land x\notin B$: W!

Also gilt $A=B$

\subsection{Natürliche Zahlen $\mathbb{N}$}

Konstruktion der Natürlichen Zahlen mit der Variante von Neumann.

\begin{equation}
    \begin{split}
        \emptyset=\{\}           & =: 0                             \\
        \{\emptyset\}=\{0\}      & =: 1                             \\
        \{0,1\}                  & =: 2 = \{0\}\cup\{1\}=1\cup\{1\} \\
        \{0,1,2\}                & =: 3 = 2\cup \{2\}               \\
        1\subset 2               & \subset 3                        \\
                                 & \vdots                           \\
        \text{Nachfolger} \to n' & := n\cup\{n\}                    \\
                                 & \vdots                           \\
        n\leq m                  & :\iff n\subseteq m
    \end{split}
\end{equation}

$\mathbb{N}$: Menge aller Nachfolger von 0.

\subsubsection{Definition}

$\mathbb{N}_0:=\mathbb{N}\cup\{0\}.$

\subsection{Induktionsprinzip}

Für eine Menge $M\subseteq \mathbb{N}$ soll gelten.

\begin{equation}
    \begin{split}
        \text{i) }  & 1\in M                   \\
        \text{ii) } & n\in M \implies n+1\in M \\
    \end{split}
\end{equation}

$\implies M = \mathbb{N}$

\subsection{Vollständige Induktion}

Problem: $\infty$-viele Aussagen, zeige dass alle wahr sind.

\begin{equation}
    \begin{split}
        A_n & : \frac{1}{2}n(n+1)=1+2+\dots+n-1+n \\
    \end{split}
\end{equation}

$M   =\{n\in \mathbb{N}\mid A_n\}$. Zu zeigen: $M= \mathbb{N}$ (dann sind nämlich alle Aussagen wahr). $A_n$ ist die Menge der wahren Indizes.

\begin{equation}
    \begin{split}
        \text{i) }  & A_1 ist wahr                                                                  \\
        \text{ii) } & \text{falls }n\in M \text{, d.h. falls }A_n\text{ wahr,}                      \\
                    & \text{dann ist zu zeigen, dass auch }A_{n+1}\text{ wahr ist -- d.h } n+1\in M
    \end{split}
\end{equation}

Wir nehmen an, $A_n$ wäre wahr und überprüfen, ob eine Schlussfolgerung auf $A_{n+1}$ auch wahr ist.

Kurz: $A_n$: für alle $n\in \mathbb{N}$

\begin{equation}
    \begin{split}
        \text{i) }  & A_1 \text{ ist wahr}                          \\
        \text{ii) } & A_n \text{ wahr}\implies A_{n+1} \text{ wahr}
    \end{split}
\end{equation}

B $A_n: 1+2+\dots+n=\frac{1}{2}n(n+1)$ (IV = Induktionsvoraussetzung)

\begin{equation}
    \begin{split}
        \text{i) }  & n=1 \text{ Zu zeigen ist } 1= \frac{1}{2}*1*2 \text{, was offensichtlich wahr ist} \\
        \text{ii) } & n\to n+1                                                                           \\
                    & 1+2+\dots+n+n+1 \begin{array}{cc}
                                          (IV) &                                  \\
                                          =    & \frac{1}{2}n(n+1)+(n+1)          \\
                                          =    & (n+1)(\frac{1}{2}n+\frac{2}{2}1) \\
                                          =    & \frac{1}{2}(n+1)(n+2)
                                      \end{array}
    \end{split}
\end{equation}

Aufpassen mit falscher beweisrichtung $1= \frac{1}{2}*1*2 \implies 1=1$. Dies wird als Fehler gewertet. $1= \frac{1}{2}*1*2 \iff 1=1$ würde ok sein. Aber Benutzung von "was offensichtlich wahr ist" ist besser.

\section{Für Nächste Vorlesung}

Wir müssen noch Intervalle einführen

\end{document}
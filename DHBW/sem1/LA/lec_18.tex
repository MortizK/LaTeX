\documentclass[a4paper]{article}

%\usepackage{url}

%% Math
\usepackage{mathtools}
%% For Mengen like natural numbers
\usepackage{amsfonts}
%% Für spezielle Symbole
\usepackage{amssymb}

%% Images
\usepackage{import}
\usepackage{xifthen}
\usepackage{pdfpages}
%\usepackage{transparent}

%%% Command for simpler images
\newcommand{\incfig}[1]{%
    \def\svgwidth{\columnwidth}
    \import{./fig/}{#1.pdf_tex}
}

%% Links
\usepackage{hyperref}
\hypersetup{
    colorlinks=true,
    linkcolor=black,
    filecolor=magenta,
    urlcolor=cyan
}

%% Formatting
\usepackage{parskip}

\title{Lineare Algebra}
\author{Moritz}
\date{February 12, 2025}

\begin{document}
\maketitle
\tableofcontents

\section{Matrizen}

\subsection{Dimensionsformel}

\subsubsection{Dimension vom Kern}

Dimension vom Kern lässt sich durch das Abzählen der Elemente der Basis vom Kern.

\begin{equation}
    ker A=\{x\mid Ax=0\}
\end{equation}

Beispielweise haben wir diese Lösungen nach GAUSS:

\begin{equation}
    \begin{matrix}
        x_1 &     &     & + 2x_4 & +4x_5 & +3x_6  & =0 \\
            & x_2 &     &        & + x_5 & + 4x_6 & =0 \\
            &     & x_3 & - 3x_4 & +5x_5 &        & =0
    \end{matrix}
\end{equation}

Nach dem Umstellen nach $x_1, \dots x_6$

\begin{equation}
    \begin{matrix}
        x_1=  & -2x_4 & -4x_5 & -3x_6 \\
        x_2 = &       & -x_5  & -4x_6 \\
        x_3 = & 3x_4  & -5x_5 &       \\
        x_4 = & 1x_4  &       &       \\
        x_5 = &       & 1x_5  &       \\
        x_6 = &       &       & 1x_6  \\
    \end{matrix}
\end{equation}

Somit ist unser Kern:

\begin{equation}
    ker A=\left\{x_4
    \begin{bmatrix}
        -2 \\0\\3\\1\\0\\0
    \end{bmatrix}+x_5
    \begin{bmatrix}
        -4 \\-1\\-5\\0\\1\\0
    \end{bmatrix}+x_6
    \begin{bmatrix}
        -3 \\-4\\0\\0\\0\\1
    \end{bmatrix}\mid x_4, x_5, x_6\in \mathbb{C}\right\}
\end{equation}

Und die Basis hat nur diese drei Vektoren $B_{ker A}=\{b_1, b_2, b_3\}$ und die Dimension vom Kern is $=3$.

\know{Klausuraufgabe}{Berechne die Dimenstion vom Kern, bedeuted: GAUSS-Verfahren machen und somit den gesammten Kern berechnen. }

\subsection{Koordinatentransformation}

\begin{center}
    $\mathfrak{B}:=\{b_1, b_2, b_3\}$ Basis

    $\mathcal{B}:=[b_1, b_2, b_2]$

    $x=Bx_B$

    $x_B=B^{-1}x$
\end{center}

In der Kanonischen Basis gilt $x=x_1e_1+x_2e_2+x_3e_3$. Dies Können wir auch in eine anderen Basis machen. Somit haben:

\begin{equation}
    x=\tilde{x}_1b_1+\tilde{x}_2b_2+\tilde{x}_3b_3 \hat{=}
    \begin{bmatrix}
        \tilde{x}_1 \\\tilde{x}_2\\\tilde{x}_3
    \end{bmatrix}_B = x_B
\end{equation}

Also: $x=Bx_B$ ist die Gleichung der Koordinatentransformation für $x$. Und wir können $x_B=B^{-1}x$ mit GAUSS bestimmen.

\know{Orthogonalbasis}{Falls $\mathfrak{B}$ eine Orthogonalbasis ist, so ist\begin{center}
        $\mathcal{B}^{-1}=\mathcal{B}^*$

        $x_{\mathcal{B}}=B^*x$
    \end{center}}

\subsection{Drehung um eine Beliebige Achse}

Drehung um eine Beliebige Achse mittels Koordinatentransformation.

Wir Drehen im Allgemeinen nur in einer der Ebenen unsere Basis:

\begin{equation}
    \begin{bmatrix}
        \cos(\alpha) & -\sin(\alpha) & 0 \\
        \sin(\alpha) & \cos(\alpha)  & 0 \\
        0            & 0             & 1
    \end{bmatrix}_B=:D_3(\alpha)
\end{equation}

Wir haben unsere Achse $n$ als Normalvektor; $x$ wird gedreht.

\begin{enumerate}
    \item $b_3:= n$, zwei weitere Basisvektoren  $b_1$ und $b_2$ zu wählen, dass
          \begin{center}
              $\mathfrak{B}:=\{b_1, b_2, b_3\}$
          \end{center} eine positiv orientierte ONB ergibt.
    \item $B=[b_1, b_2, b_3]$ Matrix der Koordinatentransformation
    \item $B^{-1}=B^*$; $x_B=B^*x$
    \item Der gedrehte Vektor (in der Basis B dargestellt)
          \begin{center}
              $x_B'=D_3(\alpha)*x_B$
          \end{center}
          Zurücktransformieren:
          \begin{center}
              $x'=Bx_B'$
          \end{center}
\end{enumerate}

\know{Drehformel}{Wir können mit volgerne Formel eine Drehung durchführen:
    \begin{center}
        $x'=Bx_B'=BD_3(\alpha)x_B=BD_3(\alpha)B^*x=D_n(\alpha)x$

        $D_n(\alpha):=BD_3(\alpha)B^*x$
    \end{center}wo $n$ der Drehvektor ist}

\subsubsection{Einfache Basis}

\begin{equation}
    \left\{\frac{1}{3}
    \begin{bmatrix}
        1 \\2\\2
    \end{bmatrix},\frac{1}{3}
    \begin{bmatrix}
        2 \\-2\\1
    \end{bmatrix},\frac{1}{3}
    \begin{bmatrix}
        2 \\1\\-2
    \end{bmatrix},\right\}=B
\end{equation}

\begin{equation}
    \frac{1}{3}\begin{bmatrix}
        1 & 2  & 2  \\
        2 & -2 & 1  \\
        2 & 1  & -2
    \end{bmatrix}=B
\end{equation}

Beispielrechnung in dieser Basis ist in meinem Heft unter LA 12.02 Drehung. Es gibt aber eine Fehler, da irgendwo ein Faktor 2 verloren gegangen ist.

\subsubsection{Allgemeine Dreh-formel/-matrix}

$n$ Drehvektor mit $\|n\|=1$.

\begin{equation}
    D_n(\alpha)=P_n+\cos(\alpha)(1-P_n)+\sin(\alpha)R_n
\end{equation}

Herleitung in meinem Heft unter LA 12.02 Drehformel/matrix

\subsection{Ausblick: Eigenwerttheorie}



\end{document}
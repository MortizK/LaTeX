\documentclass[a4paper]{article}

%\usepackage{url}

%% Math
\usepackage{mathtools}
%% For Mengen like natural numbers
\usepackage{amsfonts}
%% Für spezielle Symbole
\usepackage{amssymb}

%% Images
\usepackage{import}
\usepackage{xifthen}
\usepackage{pdfpages}
%\usepackage{transparent}

%%% Command for simpler images
\newcommand{\incfig}[1]{%
    \def\svgwidth{\columnwidth}
    \import{./fig/}{#1.pdf_tex}
}

%% Links
\usepackage{hyperref}
\hypersetup{
    colorlinks=true,
    linkcolor=black,
    filecolor=magenta,
    urlcolor=cyan
}

%% Formatting
\usepackage{parskip}

\title{Lineare Algebra}
\author{Moritz}

\begin{document}
\maketitle
\tableofcontents

\section{Aussagenlogik}

\subsection{Binomische Formeln}

\begin{equation}
    \begin{split}
        (a+b)^1 & =a+b                                                                                                   \\
        (a+b)^2 & =a^2+2ab+b^2                                                                                           \\
        (a+b)^3 & =a^3+3a^2b+3ab^2+b^3                                                                                   \\
        (a+b)^4 & =a^4+4a^3b+6a^2b^2+4ab^3+b^4                                                                           \\
        (a+b)^n & =\binom{n}{0}a^n+na^{n-1}b^1+\binom{n}{2}a^{n-2}b^2+\dots+\binom{n}{k}a^{n-k}b^k+\dots+\binom{n}{n}b^n \\
                & =\sum_{k=0}^{n}\binom{n}{k}a^{n-k}b^{k}=\sum_{k=0}^{n}\binom{n}{k}b^{n-k}a^{k}
    \end{split}
\end{equation}

\subsection{Binominialkoeffizient}

\begin{equation}
    \begin{split}
        \binom{n}{k} & :=\frac{n!}{k!(n-k)!}   \\
        n!           & :=1*2*3*4...*n, n\geq 1 \\
        0!           & :=1
    \end{split}
\end{equation}

\begin{equation}
    \binom{n+1}{k}=\binom{n}{k-1}+\binom{n}{k}
\end{equation}

\subsection{Rechenregeln}

\begin{equation}
    \begin{split}
        a*\sum_{k=1}^{n}b_k & =a*(b_1+b_2+b_3\dots b_n)     \\
                            & =a*b_1+a*b_2+a*b_3\dots a*b_n \\
                            & =\sum_{k=1}^{n}a*b_k          \\
        % funktioniert für + und -
        % \sum_{k=1}^{n}a_k\sum_{k=1}^{n}b_k&=a_1+a_2\dots a_n+b_1+b_2\dots b_n \\
        %                     & a_1+b_1+a_2+b_2\dots a_n+b_n                   \\
        %                     & \sum_{k=1}^{n}a_k+b_k                          \\
        % \sum_{k=1}^{n}a_k*\sum_{k=1}^{n}b_k&=(a_1+a_2)=\sum_{k=1}^{n}\sum_{k=1}^{n}\sum_{l=1}^{n}a_k*b_l
    \end{split}
\end{equation}

\subsection{Beweis der Binomischen Formel}

Ist auf dem Blatt La 17.12

\subsection{Aufgaben}

\subsubsection{Benoullie Ungleichung}

Zeigt, dass die Tangente einer normalen Quadratfunktion, unterhalb der Funktion liegt.

Beweise: $\substack{\text{für alle }x\geq -1\text{ gilt}\\
        (1+x)^n\geq 1+xn}$

$n=1$ Behauptet: $(1+x)^1\geq 1+1x$ was off. wahr ist.

$n\to n+1$ z.z. $(1+x)^(n+1)\geq 1+x(n+1)$

Also: \begin{equation}
    \begin{split}
        (1+x)^(n+1)=(1+x)(1+x)^n & \substack{\text{(IV)}               \\ \geq}(1+x)(1+xn)\\
                                 & \geq 1+x+nx+nx^2                    \\
        \geq 1+x(1+n)+nx^2       & \geq 1+x(1+n), \text{da }nx^2\geq 0
    \end{split}
\end{equation}

\section{Elemente der Zahlentheorie}

\subsection{Teilbarkeit}

$n\in \mathbb{Z}, p\in \mathbb{N}$. Dann heißt $p$ Teiler von $n$, oder wir sagen "$p$ teilt $n$", $p|n$ falls es ein $k\in \mathbb{Z}$ mit $n=k*p$.

\begin{equation}
    \begin{split}
        k                         & = \frac{n}{p}\in \mathbb{Z} \\
        \exists_{k\in \mathbb{Z}} & : \frac{n}{p}=k             \\
        \exists_{k\in \mathbb{Z}} & : n=k*p
    \end{split}
\end{equation}

\subsubsection{Größter gemeinsamer Teiler}

Größter gemeinsamer Teiler von $m$ und $n$ : $ggT(m,n)$. $m$ und $n$ heißen teilerfremd, falls der $ggt(m, n)=1$.

\subsubsection{Teilen mit Rest}

Für alle $n\in \mathbb{Z}$ und $n\in \mathbb{N}$ gibt es genau ein $k\in \mathbb{Z}$ und genau ein $r\in \mathbb{N}$, mit

Diese Darstellung ist eindeutig.

\begin{equation}
    \begin{split}
        n=k*p+r, 0\leq r < p
    \end{split}
\end{equation}

Beispiele. Unser p muss positiv sein, somit brauchen wir ein k, welches die gesamte Zahl abdecken.

\begin{equation}
    \begin{matrix}
        \begin{aligned}
            n  & =87, p=17 \\
            87 & =5*17+2   \\
        \end{aligned} &
        \begin{aligned}
            n  & =82, p=87 \\
            82 & =0*87+82  \\
        \end{aligned} &
        \begin{aligned}
            n   & =-87,p=17 \\
            -87 & =-6*17+15
        \end{aligned}
    \end{matrix}
\end{equation}

\subsubsection{Perioden}

Alle Brüche sind periodisch, selbst eine $0,25$, da $0,25\overline{0}$ oder $0,24\overline{9}$.

\subsubsection{Euklidischer Algorithmus}

$ggT(84,18)$. Als erstes ausprobieren, ob 18 selbst der gemeinsame Teiler ist. Hier kommt ein Rest von 12. Der nächste Schritt ist also zu gucken, ob die 12 in die 18 Reinpasst. Das Klappt hier auch nicht, somit den nächsten Schritt, ob die 6 in die 12 Reinpasst. Das Klappt $\to$ ist die 6 der größte gemeinsame Teiler.

$ggT(p_1,p_2)$, $p_1>p_2$ und aus $\mathbb{N}$.

\begin{equation}
    \begin{split}
        p_1     & =t_2p_2+p_3, 0\leq p_3<p_2 \\
        p_2     & =t_3p_3+p_4, 0\leq p_4<p_3 \\
        p_3     & =t_4p_4+p_5, 0\leq p_5<p_4 \\
                & \vdots                     \\
        p_{n-1} & = t_np_n+p_{n+1}           \\
        p_n     & = t_{n+1}p_{n+1}           \\
        p_n     & = ggT(p_1,p_2)
    \end{split}
\end{equation}

\subsubsection{Darstellung nach Euklid (nach Bêzout)}

\begin{equation}
    \begin{split}
        \exists_{t,s\in \mathbb{Z}} ggT(p,q)=tp+sq
    \end{split}
\end{equation}

Beispiel für es gibt mehrere Möglichkeiten für diese Darstellung

\begin{equation}
    \begin{split}
        p=5, q=7   \\
        1=-4*5+3*7 \\
        1=3*5-2*7
    \end{split}
\end{equation}

Um diese zahlen auszurechnen. Kann man den Euklidischer Algorithmus nach dem Rest umstellen und die Zahlen von unten nach oben einsetzen. Bis man auf die Richtige Form kommt.

\begin{equation}
    \begin{split}
        126=1314-3*396        \\
        18=396-3*126          \\
        18=396-3*(1314-3*396) \\
        18=10*396-3*1314
    \end{split}
\end{equation}

\subsubsection{Teilbarkeitssätze}

$p|a*b\land ggT(a,p)=1\implies p|b$

Beweis: Nach Euklid:

\begin{equation}
    \begin{split}
        1=ta+sp\mid *b \\
        b=tab+sbp      \\
    \end{split}
\end{equation}

$ab$ ist ein vielfaches von $p$. $\implies p|b$.

\end{document}
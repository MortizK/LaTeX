\documentclass[a4paper]{article}

%\usepackage{url}

%% Math
\usepackage{mathtools}
%% For Mengen like natural numbers
\usepackage{amsfonts}
%% Für spezielle Symbole
\usepackage{amssymb}

%% Images
\usepackage{import}
\usepackage{xifthen}
\usepackage{pdfpages}
%\usepackage{transparent}

%%% Command for simpler images
\newcommand{\incfig}[1]{%
    \def\svgwidth{\columnwidth}
    \import{./fig/}{#1.pdf_tex}
}

%% Links
\usepackage{hyperref}
\hypersetup{
    colorlinks=true,
    linkcolor=black,
    filecolor=magenta,
    urlcolor=cyan
}

%% Formatting
\usepackage{parskip}

\title{Lineare Algebra}
\author{Moritz}
\date{February 4, 2025}

\begin{document}
\maketitle
\tableofcontents

\section{Komplexen Zahlen $\mathbb{C}$}

\subsection{Skalarprodukt}

\begin{enumerate}[i.]
    \item $\langle x\mid x\rangle\geq 0, =0\iff x=0$ Definitheit
    \item $\langle x\mid y\rangle=\bar{\langle y\mid x\rangle}$ Antisymmetrie
    \item $\langle ty+sz\mid x\rangle=\bar{t}\langle y\mid x\rangle+\bar{s}\langle z\mid x\rangle$ Sesquilinearität
\end{enumerate}

\begin{equation}
    \langle x\mid y\rangle = \sum_{i=1}^{n}\bar{x_i}y_i
\end{equation}

zugehörige Norm: $\|x\|:=\sqrt{\langle x\mid x\rangle}$

Metrik: $d(x, y):=\|x-y\|$

\subsubsection{CAUCHY-SCHWARZsche Ungleichung}

ist auf meinem Notizblock unter LA 04.02 Skalarprodukt in $\mathbb{C}$.

\subsubsection{Cardanischen Formeln}

$ax^3+bx^2+cx+d=0$ Wir können annehmen, dass $a=1$ ist, da $a\neg 0$ gilt und wir somit durch $a$ teilen können.

Durch Substitution mit $x=y-\frac{b}{3}$ kommen wir auf die Standardform.

Standardform: $y^3+py+q=0$ mit $p=c-\frac{1}{3}b^2$ und $q=\frac{2}{27}b^3-\frac{1}{3}bc+d$

Ansatz:

\begin{equation}
    \begin{split}
        y           & =u-v                           \\
        0           & =(u-v)^3+p(u-v)+q              \\
                    & =u^3-v^3+3(uv^2-u^2v)+p(u-v)+q \\
        0           & =u^3-v^3-3uv(u-v)+p(u-v)+q     \\
        0           & =u^3-v^3+(p-3uv)(u-v)+q        \\
        \frac{p}{3} & =uv                            \\
        0           & =u^3-v^3+q                     \\
        \implies 0  & = u^3+qu^3-(uv)^3              \\
                    & =(u^3)^2+qu^3-(\frac{p}{3})^3
    \end{split}
\end{equation}

Dies ist an sich eine Quadratische Formel, welche wir mit der Mitternachtsformel lösen können.

\begin{equation}
    \begin{split}
        u^3         & =-\frac{p}{2}+\sqrt{(\frac{p}{2})^2+(\frac{p}{2})^3}                                                                                                                              \\
                    & =\Delta - \frac{p}{2}                                                                                                                                                             \\
        u           & = \sqrt[3]{\sqrt{\Delta}-\frac{p}{2}}                                                                                                                                             \\
        \frac{p}{3} & \implies v =\frac{p}{3u}                                                                                                                                                          \\
        v           & = \frac{p}{3\sqrt[3]{\sqrt{\Delta}-\frac{p}{2}}} = \frac{p}{3\sqrt[3]{\sqrt{\Delta}-\frac{p}{2}}}*\frac{\sqrt[3]{\sqrt{\Delta}+\frac{p}{2}}}{\sqrt[3]{\sqrt{\Delta}+\frac{p}{2}}} \\
                    & = \sqrt[3]{\sqrt{\Delta}+\frac{p}{2}}
    \end{split}
\end{equation}

Daraus ergibt sich:

\begin{equation}
    y=\sqrt[3]{\sqrt{\Delta}-\frac{p}{2}}-\sqrt[3]{\sqrt{\Delta}+\frac{p}{2}}
\end{equation}

Bei $\Delta>0$, genau eine Reelle Lösung.

Bei $\Delta=0$, genau 2 Lösungen eine doppelt.

Die Lösungen sind:

\begin{equation}
    \begin{split}
        y_1     & =\frac{3q}{p}   \\
        y_{2/3} & =-\frac{3q}{2p}
    \end{split}
\end{equation}

Für die Lösungen, wenn $\Delta<0$ gibt es einen Neuen Ansatz: $y=t*\cos(phi)$. Diesen Ansatz einsetzen und nach t lösen:

\begin{equation}
    t=2\sqrt{\frac{|p|}{3}}
\end{equation}

Hier sind die Lösungen:

\begin{equation}
    \begin{split}
        \varphi_0 & = \frac{1}{3}\cos^{-1}\left(-\frac{q}{2}\sqrt{\frac{27}{|p|^3}}\right) \\
        y_1       & =2\sqrt{\frac{|p|}{3}}\cos(\varphi_0)                                  \\
        y_{2/3}   & =-2\sqrt{\frac{|p|}{3}}\cos(\varphi_0\pm\frac{\pi}{3})
    \end{split}
\end{equation}

\section{Matrizen}

\subsection{Lineare Abbildung und ihre Matrix}

\begin{equation}
    \begin{split}
        A           & =[a_1, a_2, \dots, a_n]     \\
        x           & =\begin{bmatrix}
                           x_1 \\x_2\\\vdots\\x_n
                       \end{bmatrix}      \\
        \implies Ax & =x_1a_1+x_2a_2+\dots+x_na_n
    \end{split}
\end{equation}

Wir können nun mit den Basisvektoren die einzelnen a rausbekommen:

\begin{center}
    $a_1=Ae_1, a_2=Ae_2,\dots, a_n=Ae_n$
\end{center}

\subsubsection{Universelles Konstruktionsprinzip von Matrizen}

\know{Das heißt\:}{Die Spaltenvektoren $a_1, a_2, ..., a_n$ sind die Bilder der kanonischen Basisvektoren unter der Abbildung $A$.}

\subsection{Drehungen}

Im $\mathbb{R}^2$ können wir einfach Drehen:

\begin{equation}
    D_{\alpha}=\begin{bmatrix}
        \cos(\alpha) & -\sin(\alpha) \\
        \sin(\alpha) & \cos(\alpha)  \\
    \end{bmatrix}
\end{equation}

Im $\mathbb{R}^3$ gibt es einfach Fälle der Drehung, wenn wir um die Achsen drehen.

\begin{equation}
    D_{\alpha}=\begin{bmatrix}
        \cos(\alpha) & -\sin(\alpha) & 0 \\
        \sin(\alpha) & \cos(\alpha)  & 0 \\
        0            & 0             & 1
    \end{bmatrix}
\end{equation}

\subsubsection{Matrixmultiplikation}

Ist eine Hintereinanderausführung von Abbildung. z.B. Drehen und Bewegen

\begin{equation}
    \begin{split}
        \mathcal{B}=[b_1,\dots, b_n]: \mathbb{K}^n         & \to \mathbb{K}^m                       \\
        \mathbb{K}^n\overset{\mathcal{B}}{\to}\mathbb{K}^m & \overset{\mathcal{A}}{\to}\mathbb{K}^r
    \end{split}
\end{equation}

Skizze zur Multiplikation ist auf LA 04.02.

Die Rechenvorschrift verlangt, das nur gewisse Matrizen miteinander Multiplizierbar sind. So muss die Höhe von A = der Breite von B sein.

\begin{equation}
    A*B=[Ab_1, Ab_2, \dots, Ab_n]
\end{equation}

Das bedeutet: Wir Rechnen Spaltenweise Die Multiplikation und wenn wir mehrere Matrizen miteinander multiplizieren, können wir zuerst die Symbole aufschreiben.

Rechneregeln:

\begin{equation}
    \begin{split}
        tA        & =[ta_1, ta_2, \dots, ta_n]          \\
        A+B       & =[a_1+b_1, a_2+b_2, \dots, a_n+b_n] \\
        A*(tB+sC) & =tAB+sAC                            \\
    \end{split}
\end{equation}

Aber meistens:

\begin{equation}
    AB\neq BA
\end{equation}

\end{document}
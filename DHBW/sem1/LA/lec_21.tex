\documentclass[a4paper]{article}

%\usepackage{url}

%% Math
\usepackage{mathtools}
%% For Mengen like natural numbers
\usepackage{amsfonts}
%% Für spezielle Symbole
\usepackage{amssymb}

%% Images
\usepackage{import}
\usepackage{xifthen}
\usepackage{pdfpages}
%\usepackage{transparent}

%%% Command for simpler images
\newcommand{\incfig}[1]{%
    \def\svgwidth{\columnwidth}
    \import{./fig/}{#1.pdf_tex}
}

%% Links
\usepackage{hyperref}
\hypersetup{
    colorlinks=true,
    linkcolor=black,
    filecolor=magenta,
    urlcolor=cyan
}

%% Formatting
\usepackage{parskip}

\title{Lineare Algebra}
\author{Moritz}
\date{February 25, 2025}

\begin{document}
\maketitle
\tableofcontents

\section{Matrizen/ Spektrum}

\subsection{Normale Matrix}

Def.: $A$ heißt normal, falls: $A^*A=AA^*$

Beispiel: Unitäre Abbildungen: $U^*=U^{-1}\iff U^*U=1 (=UU^{-1}=UU^*)$

\begin{equation*}
    f(A):=B
    \begin{bmatrix}
        f(\lambda_1) &              &        &              \\
                     & f(\lambda_2) &        &              \\
                     &              & \ddots &              \\
                     &              &        & f(\lambda_n)
    \end{bmatrix}B^*
\end{equation*}

\subsubsection{Theorem}

Jede normale Lineare Abbildung hat eine ONB aus Eigenvektoren.

Lemma: $A=A^* \land B=B^* \land AB=BA \implies \exists $ gemeinsame ONB aus Eigenvektoren. Für zwei selbstadjungierte lin. Abbildungen $A$ und $B$, die miteinander vertauschen gibt es einen Gemeinsame ONB aus Eigenvektoren.

\subsubsection{Idee}

spezielle Situation: $A\times x = \lambda x$, somit $x$ ist einfacher EV zu $\lambda$

Was ist mit $Bx$?

\begin{equation*}
    ABx=BAx=B\lambda x=\lambda Bx\implies Bx=\gamma x
\end{equation*}

Was ist, wenn wir nicht nur einen Eigenvektor haben? Wir müssen unsere Folgerung ändern: $\implies Bx $ ist EV con $A$ zu $\lambda$

\subsubsection{Eigenraum}

$H_\lambda:=\{x\in V \mid Ax=\lambda x\}$ Teilraum von $H$ $(A: H \to H)$ mit typischerweise $\mathbb{C}^n$

Dann: $\lambda\neq\gamma\implies H_\lambda \bot H_\gamma$

$sp(A)=\{\lambda_1, \dots, \lambda_k\}$ $H=H_{\lambda_1}\oplus H_{\lambda_2}\oplus \dots\oplus H_{\lambda_k}$

\begin{equation*}
    x\in H_{\lambda_1}\implies Bx\in H_{\lambda_1}
\end{equation*}

Das heißt $B:H_{\lambda_1}\to H_{\lambda_1}\implies$ es gibt ONB $\mathcal{B}_1$ aus EVen von $B$ für $H_{\lambda_1}$

Es gibt ONB aus EVen von $B$ für $H_{\lambda_k}: \mathcal{B}_k$

\subsubsection{Beweis von Theorem}

Rechnung auf Heft II LA 25.02. unter Theorem.

$A_1A_2=A_2A_1\implies$ es gibt gemeinsame Basis aus EVen $\mathcal{B}=\{b_1, ..., b_n\}$

\begin{equation*}
    A_1b_1=\lambda_1b_1, A_2b_1=\gamma_1b_1\implies Ab_1 = A_1b_1+iA_2b_1=\lambda_1b_1+i\gamma_1b_1=(\lambda_1+i\gamma_1)b_1
\end{equation*}

Dies gilt für alle Indexe und wir schließen: $\implies \mathcal{B}$ ist ONB aus EVen von A.

\subsubsection{Wichtigstes Beispiel}

Unitäre Abbildung: $U^*U=1\implies U^*=U^{-1}\implies U^*U=1=UU^{-1}=UU^*$. Es existiert ONB aus EVen

Eigenwerte haben den Betrag 1:

\begin{align*}
    Ux= \lambda x:
     & \|Ux\|^2=\|\lambda x\|^2=|\lambda|^2\|x\|^2                                        \\
     & \langle Ux\mid Ux\rangle=\langle U^*Ux\mid x\rangle=\langle x\mid x\rangle=\|x\|^2
\end{align*}

\end{document}
\documentclass[a4paper]{article}

%\usepackage{url}

%% Math
\usepackage{mathtools}
%% For Mengen like natural numbers
\usepackage{amsfonts}
%% Für spezielle Symbole
\usepackage{amssymb}

%% Images
\usepackage{import}
\usepackage{xifthen}
\usepackage{pdfpages}
%\usepackage{transparent}

%%% Command for simpler images
\newcommand{\incfig}[1]{%
    \def\svgwidth{\columnwidth}
    \import{./fig/}{#1.pdf_tex}
}

%% Links
\usepackage{hyperref}
\hypersetup{
    colorlinks=true,
    linkcolor=black,
    filecolor=magenta,
    urlcolor=cyan
}

%% Formatting
\usepackage{parskip}

\title{Lineare Algebra}
\author{Moritz}
\date{January 14, 2025}

\begin{document}
\maketitle
\tableofcontents

\section{Vektoren}

im $\mathbb{R}^2$

\begin{figure}[ht]
    \centering
    \incfig{vektoren}
    \caption{Vektoren mit Addition}
    \label{fig:vektoren}
\end{figure}

Vektoren = Handlungsanweisung

Zuerst b dann c ist eine Handlungsanweisung, also muß dazu ein Vektor gehören.

\begin{equation}
    \begin{split}
        d      & =\begin{bmatrix}
                      -1+4 \\
                      3+4
                  \end{bmatrix}=\begin{bmatrix}
                                    3 \\
                                    7
                                \end{bmatrix} \\
        d      & =: b+c=c+b                    \\
        "-b"   & :=\begin{bmatrix}
                       -b_1 \\
                       -b_2
                   \end{bmatrix}              \\
        b+(-b) & =0                            \\
        c+(-b) & =:c-b=\begin{bmatrix}
                           c_1-b_1 \\
                           c_2-b_2
                       \end{bmatrix}
    \end{split}
\end{equation}

\begin{equation}
    \begin{split}
        t*b:=\begin{bmatrix}
                 tb_1 \\
                 tb_2
             \end{bmatrix}
    \end{split}
\end{equation}

\subsection{Linearkombination}

Allgemein: $ta+sb$

$a$ und $b$ sind Vektoren und $t$ und $s$ sind Faktoren

\subsection{Geraden}

Geraden und Geraden Stück

\begin{equation}
    \begin{split}
        g   & =\{p+tu\mid t\in \mathbb{R}\} \\
        g_1 & =\{p+tu\mid t\in[-2, 3]\}
    \end{split}
\end{equation}

\subsection{Ebenen}

\begin{figure}[ht]
    \centering
    \incfig{ebenen}
    \caption{Ebene im $\mathbb{R}^3$}
    \label{fig:ebenen}
\end{figure}

\begin{equation}
    \begin{split}
        E & =\{p+tu+sv\mid t,s\in \mathbb{R}\}                                \\
        E & =\left\{\begin{bmatrix}
                        x_1 \\
                        x_2 \\
                        x_3
                    \end{bmatrix}\in \mathbb{E}^3\mid 2x_1-x_2+3x_3=5\right\}
    \end{split}
\end{equation}

\begin{equation}
    \begin{split}
        x_1 & =0+1*x_1+0*x_3             \\
        x_2 & =-5+2x_1+1*x_3             \\
        x_3 & =0+0*x_1+1*x_3             \\
        E   & =\left\{\begin{bmatrix}
                          0 \\-5\\0
                      \end{bmatrix}+x_1*
        \begin{bmatrix}
            1 \\2\\0
        \end{bmatrix}+x_3
        \begin{bmatrix}
            0 \\3\\1
        \end{bmatrix}\mid x_1,x_3\in \mathbb{R}\right\}
    \end{split}
\end{equation}

\subsection{Länge = Norm}

Der Norm eines Vektor ist $\|x\|$

Im $\mathbb{R}^2$ ist $\|x\|=x_1^2+x_2^2$ oder $\|x\|=\sqrt{x_1^2+x_2^2}$

Im $\mathbb{R}^3$ ist $\|x\|=d^2+x_3^2=x_1^2+x_2^2+x_3^2$ oder $\|x\|=\sqrt{x_1^2+x_2^2+x_3^2}$

Im $\mathbb{R}^4$ ist $\|x\|:=\sqrt{x_1^2+x_2^2+x_3^2+x_4^2}$

\begin{equation}
    \begin{split}
        \|tx\|=\|\begin{bmatrix}
                     tx_1 \\
                     tx_2
                 \end{bmatrix}\|
         & =\sqrt{t^2x_1^2+t^2x_2^2}=\sqrt{t^2*(x_1^2+x_2^2)} \\
         & =\sqrt{t^2}*\sqrt{x_1^2+x_2^2}=|t|*\|x\|
    \end{split}
\end{equation}

i) $\|x\|\geq 0$, $=0\iff x=0$ Nur Null, wenn es der Nullvektor ist (Definitheit)

ii) $\|tx\|=|t|*\|x\|$ (Homogenität)

iii) $\|x+y\|\leq \|x\|+\|y\|$ (Dreiecks-Ungleichung)

\subsection{Abstand: Metrik}

Eine Grafik mit zwei Vektoren und dem Abstandsvektor zwischen den beiden Punkten: $\|x-y\|=\|y-x\|=:d(x,y)$

i) $d(x,y)\geq 0$, wenn $=0\iff x=y$

ii) $d(x, y)= d(y, x)$

iii) $d(x, y)\leq d(x, z) + d(z, y)$ (Dreiecks-Ungleichung)

\subsection{Skalarprodukt}

Winkel $\to$ Skalarprodukt

\subsubsection{Kosinus-Satz}

$c^2=a^2+b^2-2ab\cos{\gamma}$

Auf LA 14.01

\subsection{Differenzen}

Immer Endpunkt minus Startpunkt - Können auch Negativ sein

\end{document}
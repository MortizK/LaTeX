\documentclass[a4paper]{article}

%\usepackage{url}

%% Math
\usepackage{mathtools}
%% For Mengen like natural numbers
\usepackage{amsfonts}
%% Für spezielle Symbole
\usepackage{amssymb}

%% Images
\usepackage{import}
\usepackage{xifthen}
\usepackage{pdfpages}
%\usepackage{transparent}

%%% Command for simpler images
\newcommand{\incfig}[1]{%
    \def\svgwidth{\columnwidth}
    \import{./fig/}{#1.pdf_tex}
}

%% Links
\usepackage{hyperref}
\hypersetup{
    colorlinks=true,
    linkcolor=black,
    filecolor=magenta,
    urlcolor=cyan
}

%% Formatting
\usepackage{parskip}

\title{Lineare Algebra}
\author{Moritz}
\date{January 15, 2025}

\begin{document}
\maketitle
\tableofcontents

\section{Vektoren}

\subsection{Skalarprodukt}

\begin{equation}
    \begin{split}
        a_0                       & : \text{ Einheitsvektor in Richtung von a} \\
        a_0                       & =\frac{1}{\|a\|}*a                         \\
        \langle a \mid b \rangle: & =\|a\|*\|b\|*\cos(\alpha)                  \\
        \text{Skalarprodukt von a und b}
    \end{split}
\end{equation}

Um das für Winkelberechnungen nutzen zu können, brauchen wir eine alternative Berechnungsmöglichkeit von $\langle a \mid b \rangle$, die nur die Koordination von $a$ und $b$ verwenden.

\begin{figure}[ht]
    \centering
    \incfig{skalarprodukt}
    \caption{Skalarprodukt}
    \label{fig:skalarprodukt}
\end{figure}

Herleitung durch den Kosinussatz

\begin{equation}
    \begin{split}
        \|a\|^2+\|b\|^2-2\|a\|\|b\|*\cos(\alpha)  & =\|a-b\|^2      \\
        \|a\|^2+\|b\|^2-2\langle a \mid b \rangle & =\|a-b\|^2      \\
        \text{Ausmultiplizieren im }\mathbb{R}^2                    \\
        \langle a \mid b \rangle                  & = a_1b_1+a_2b_2
    \end{split}
\end{equation}

\subsubsection{Eigenschaften ermitteln}

Also $\langle a \mid b \rangle=\|a\| \|b\|*\cos(\alpha)=a_1b_1+a_2b_2\implies \cos(\alpha)=\frac{\langle a \mid b \rangle}{\|a\|*\|b\|}$

Für uns sind nur $0\leq\alpha\leq\pi$ relevant. Bis $90^{\circ}$ ist der Kosinus Positiv und wir haben eine Spitzen Winkel, Wenn er Negativ ist, haben wir einen Stumpfen Winkel $\langle a \mid b \rangle<0: \alpha>\frac{\pi}{2}$

Und bei $\langle a \mid b \rangle=0\iff\alpha=\frac{\pi}{2}\iff a\perp b$

Vektorraum mit Skalarprodukt heißt HILBERT-Raum

Vektorraum mit Norm heißt BANACH-Raum

$\langle a \mid a \rangle=a_1a_1+a_2a_2=a_1^2+a_2^2=\|a\|^2$, $\|a\|=\sqrt{\langle a \mid a \rangle}$. Heißt, dass ein Skalarprodukt eine Norm mit sich bringt.

\begin{equation}
    \begin{split}
        \langle a \mid b+c \rangle & =a_1(b_1+c_1)+a_2(b_2+c_2)                         \\
                                   & =a_1b_1+a_1c_1+a_2b_2+a_2c_2                       \\
                                   & =a_1b_1+a_2b_2+a_1c_1+a_2c_2                       \\
                                   & =\langle a \mid b \rangle+\langle a \mid c \rangle
    \end{split}
\end{equation}

$\langle a \mid b \rangle=\langle b \mid a \rangle$, kann umgedreht werden

$\langle ta \mid a \rangle=ta_1b_1+ta_2b_2=t(a_1b_1+a_2b_2)=t\langle a \mid b \rangle$, mit einem Streckungsfaktor.

Also: $\langle a \mid t*b+s*c \rangle=t\langle a \mid b \rangle+s*\langle a \mid c \rangle$. Dies nennt man auch Linearität in der zweiten Komponente. Also: Bilinear, da die Linearität in beiden Komponenten gilt.

\subsubsection{Eigenschaften}

Ein Skalarprodukt $\langle . \mid . \rangle: \mathbb{R}^n \times \mathbb{R}^n\to \mathbb{R}$

\begin{enumerate}
    \item[i)] $\langle a \mid a \rangle\geq0, =0\iff a=0$ (definitheit)
    \item[ii)] $\langle a \mid b \rangle = \langle b \mid a \rangle$ (Symmetrie)
    \item[iii)] $\langle a \mid t*b+s*c \rangle=t\langle a \mid b \rangle+s*\langle a \mid c \rangle$ (Bilinearität)
\end{enumerate}

Folgerung: CAUCHY-SCHWARZ-sche Ungleichung

$|\langle a \mid b \rangle|\leq \|a\|*\|b\|$, denn $|\langle a \mid b \rangle|=|\|a\|*\|b\|*\cos(\alpha)|=\|a\|*\|b\|*|\cos(\alpha)|\leq\|a\|*\|b\|$, da $|\cos(alpha)|\leq 1$

Generiert über $\|a\|:=\sqrt{\langle a \mid a \rangle}$

$\implies$ damit läßt sich die Dreiecksungleichung nachrechnen: $\|a+b\|\leq \|a\|+\|b\|$ ist zu zeigen.

\begin{equation}
    \begin{split}
        \|a+b\|^2 & =\langle a+b \mid a+b \rangle                                                                        \\
                  & =\langle a \mid a \rangle+\langle a \mid b \rangle+\langle b \mid a \rangle+\langle b \mid b \rangle \\
                  & =\|a\|^2+\|b\|^2+2*\langle a \mid b \rangle                                                          \\
                  & \leq\|a\|^2+\|b\|^2+2*|\langle a \mid b \rangle|                                                     \\
                  & \leq\|a\|^2+\|b\|^2+2*\|a\|*\|b\|                                                                    \\
                  & =(\|a\|+\|b\|)^2
    \end{split}
\end{equation}

\subsection{Projektionen}

Projektionseigenschaft: $P(Px)=Px; P^2=P$

$\langle n \mid \langle n \mid x \rangle*n \rangle=\langle n \mid x \rangle*\langle n \mid n \rangle*n$, da $\langle n \mid n \rangle=\|n\|^2=1$ ergibt $\langle n \mid x \rangle*n$

Eine Ebene als HESSE-Normalform, wenn $\|n\|=1$

\begin{figure}[ht]
    \centering
    \incfig{normalform}
    \caption{Ebene in Normalform}
    \label{fig:normalform}
\end{figure}

\begin{equation}
    \{x\in \mathbb{R}^3\mid \langle n \mid x \rangle=c\}
\end{equation}

Dies ist eine Ebene in der HESSE-Form: Abstand eines Punktes x von einer Ebene E. $d(x, E)=|\langle x\mid n\rangle-c|$, wenn $\|n\|=1$

\begin{equation}
    \begin{split}
        E=\{x\mid 2x_1+2x_2-x_3=5\}                  \\
        n=\begin{bmatrix}
              2 \\2\\-1
          \end{bmatrix}                             \\
        \|n\|=3                                      \\
        \{x\mid\frac{2x_1+2x_2-x_3}{3}=\frac{5}{3}\} \\
        y=\begin{bmatrix}
              7 \\10\\2
          \end{bmatrix}                             \\
        d(y, E)= \frac{1}{3}*(2*7+2*10-1*2-5)=\frac{27}{3}=9
    \end{split}
\end{equation}

\subsection{Kreuzprodukt}

gibt es nur im $\mathbb{R}^3$

\begin{equation}
    a\times b=\begin{bmatrix}
        a_1 \\a_2\\a_3
    \end{bmatrix}\times
    \begin{bmatrix}
        b_1 \\b_2\\b_3
    \end{bmatrix}:=
    \begin{bmatrix}
        a_2b_3-a_3b_2 \\
        a_3b_1-a_1b_3 \\
        a_1b_2-a_2b_1 \\
    \end{bmatrix}
\end{equation}

\begin{enumerate}
    \item[i)] $a\times b=-b\times a$
    \item[ii)] $a\times (b+c)=a\times b+a\times c$ und $a\times tb= t*a\times b$
    \item[iii)] $a\times b \perp a \land a\times b \perp b$
    \item[iv)] $\|a\times b\|=$ Fläche des Parallelogramms welches von a und b aufgespannt wird = $\|a\|* \|b\|*\sin(\alpha)$
    \item[v)] $a, b, a\times b$: Rechte-Hand-Regel
\end{enumerate}

\end{document}
\documentclass[a4paper]{article}

%\usepackage{url}

%% Math
\usepackage{mathtools}
%% For Mengen like natural numbers
\usepackage{amsfonts}
%% Für spezielle Symbole
\usepackage{amssymb}

%% Images
\usepackage{import}
\usepackage{xifthen}
\usepackage{pdfpages}
%\usepackage{transparent}

%%% Command for simpler images
\newcommand{\incfig}[1]{%
    \def\svgwidth{\columnwidth}
    \import{./fig/}{#1.pdf_tex}
}

%% Links
\usepackage{hyperref}
\hypersetup{
    colorlinks=true,
    linkcolor=black,
    filecolor=magenta,
    urlcolor=cyan
}

%% Formatting
\usepackage{parskip}

\title{Projektmanagement}
\author{Moritz}
\date{February 10, 2025}

\begin{document}
\maketitle
\tableofcontents

\section{Netzplantechnik}

\subsection{Schritte}

\begin{enumerate}
    \item Projektstrukturplan
    \item Vorgangsliste
    \item Ablaufplan
    \item Zeitplan
\end{enumerate}

\subsubsection{Projektstrukturplan}

Alle Vorgänge werden aufgelistet. Es hilft eine Einteilung in: Vorbereitung, Tätigkeiten, Durchführung und Abschlussarbeiten.

Alles wird mit einer Vorgangsnummer nummeriert.

\subsubsection{Vorgangsliste}

Abhängigkeiten der einzelnen Vorgänge werden bestimmt (Wirtschaftlich und Technisch). Welche können Gleichzeitig abgearbeitet werden und welche Nacheinander.

\subsubsection{Netzknoten}

Aus jedem Punkt in unserem Produktionsplan können wir einen Knoten bauen.

Diese Knoten beinhalten:

\begin{enumerate}
    \item Frühester Anfangszeitpunkt (FAZ)
    \item Spätester Anfangszeitpunkt (SAZ)
    \item Nr.
    \item Bezeichnung
    \item Dauer
    \item Gesamtpuffer
    \item Freier Puffer
    \item Frühestes Endzeitpunkt (FEZ)
    \item Spätestes Endzeitpunkt (SEZ)
\end{enumerate}

Hieraus kann eine Netzplan gebaut werden.

\subsubsection{Netzplantechnik}

Ist eine aneinander Reihung von Netzknoten, welche hintereinander und parallel sein können.

Um die Werte für die Zeitpunkte zu errechnen muss im Netzplan einmal nach vorne Durchgerechnet werden für die Frühesten Start- und Endzeitpunkte und Rückwerts für die Spätesten Start- und Endzeitpunkte.

\section{Mantra}

\know{Kommunikation und Dokumentation}{Hieran darf nicht gesparrt werden. Zudem Profitiert dies von Ehrlichkeit.}

\section{RASIC Modell}

Responsible macht die Arbeit

Accountable muss es Absegnen

Supported: Gruppen oder Personen, die das Projekt unterstützen oder Helfen

Consulted haben Meinungen, die für uns interessant sind

Informed muss nur informiert werden

\section{Briefing \& Meeting}

Wer, Wann, Wo und Was?

In einer PowerPoint auf Moodle.

\end{document}
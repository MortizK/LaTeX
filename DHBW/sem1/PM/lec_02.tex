\documentclass[a4paper]{article}

%\usepackage{url}

%% Math
\usepackage{mathtools}
%% For Mengen like natural numbers
\usepackage{amsfonts}
%% Für spezielle Symbole
\usepackage{amssymb}

%% Images
\usepackage{import}
\usepackage{xifthen}
\usepackage{pdfpages}
%\usepackage{transparent}

%%% Command for simpler images
\newcommand{\incfig}[1]{%
    \def\svgwidth{\columnwidth}
    \import{./fig/}{#1.pdf_tex}
}

%% Links
\usepackage{hyperref}
\hypersetup{
    colorlinks=true,
    linkcolor=black,
    filecolor=magenta,
    urlcolor=cyan
}

%% Formatting
\usepackage{parskip}

\title{Projektmanagement}
\author{Moritz}
\date{January 13, 2025}

\begin{document}
\maketitle
\tableofcontents

\section{Was ist ...}

\subsection{ein Projekt?}

Es gibt eine DIN Norm, die DIN 69900

die Einmaligkeit der Bedingungen in ihrer Gesamtheit,

eine Zielvorgabe,

Begrenzungen zeitlicher, finanzieller oder anderer Art,

Abgrenzung gegenüber anderen Vorhaben, und

eine projektspezifische Organisation gekennzeichnet ist.

\subsection{Projektmanagement}

is: Causing a Planned Undertaking to Happen.

Es ist ein Führungskonzept

\subsubsection{Führungsaufgaben}

Zielsetzung

Planung

Steuerung

Überwachung

Führungsaufbaus (Projektorganisation)

Führungstechnik

Führungsmethoden

\subsubsection{Vorteile}

Geringere GEsamtkosten

Effektive Nutzung von Ressourcen (Ressourcen und Human Ressourcen)

Termingerechtere Projektabwicklung

Höhere Qualität des Endprodukts

\section{Project Management}

\subsection{SOSTAC}

Situation Analysis

Objectives

Strategy

Tactics

Action

Control

\subsection{Project Management Philosophy}

von Folie S. 16

\subsection{Roles and Responsibilities}

Project Execution

Maintain Customer Relations

Create a Positive Environment

\subsection{Principles and Practices}

von Folie S. 18

\subsubsection{Execution Plan}

Project Execution Strategy

Project Management

Quality

Safety

Risk Management

Design/ Develop/ Program

Implementation

Documentation

Training (Die Menschen müssen wissen wie Sie ein neues System benutzen können, sowie Feedback einsammeln um Abläufe weiter zu optimieren)

\subsubsection{Customer Relations}

Develop Communication (Schnittstelle, Dokumente, Fachsprache, usw.)

Ensure Timely Participation

Include the Customer on the Project Team

Develop Trust and Confidence

\subsubsection{Define Project Objectives}

Expected deliverables (Milestones)

Required resources

Required timing

Safety and Environmental

Total Quality

Define Targets

Implement measures (Messsysteme)

\subsection{Zielsetzung}

Funktionale Ziele

Operationale Ziele

Lösungsneutral, Wirkung orientieren, die beeinflussbar sind, positive und negative Wirkungen.

Muss-Ziele und Wunsch-Ziele

Zielkonflikte müssen gelöst werden.

\subsection{Beurteilungskriterien zur Messung}

Sachziele, Terminziele, Kostenziele

Sonstige finanzielle Ziele wie z.B. Liquidität

Einhaltung weitere Rahmenbedingungen

Entscheidungen bei Störungen, unvorhergesehenen Ereignissen

\subsection{S.M.A.R.T}

Specific/ Simple

Measurable/ Manageable

Assignable/ Accepted

Realistic/ Relevant

Time-bound/ Time-related

\subsection{Team Anforderungen}

Benötigt Fähigkeiten?, Welche Personen?, Wann werden Sie gebraucht?, Wo sind Sie, Benötigen Sie eine Schulung, Zwischenmenschliche Kompatibilität

\subsection{Eskalationsprozess}

Macht es ein Problem von einer Höheren Stufe der Hierarchie.

\section{Unterlagen}

Lastenheft

Pflichtenheft

ERS

IRS

Projektplan

\section{WBS - PSP}

Oder in Englisch: Work Breakdown Structure

\subsection{ProjektStrukturPlan}

\url{https://www.theprojectgroup.com/blog/projektstrukturplan/#prettyPhoto/2/}

Hauptziele, Teilung Aktivitäten, Teilung Unteraktivitäten, Wiederholung bis diese klein genug sind.

Niedrigste Teilaktivitäten sind die Grundlage von Arbeitspaketen

\section{Wichtig}

Kommunikation

Dokumentation

Projekt in Phasen einteilen

Zeitplanung mit Puffer planen

\end{document}
\documentclass[a4paper]{article}

%\usepackage{url}

%% Math
\usepackage{mathtools}
%% For Mengen like natural numbers
\usepackage{amsfonts}
%% Für spezielle Symbole
\usepackage{amssymb}

%% Images
\usepackage{import}
\usepackage{xifthen}
\usepackage{pdfpages}
%\usepackage{transparent}

%%% Command for simpler images
\newcommand{\incfig}[1]{%
    \def\svgwidth{\columnwidth}
    \import{./fig/}{#1.pdf_tex}
}

%% Links
\usepackage{hyperref}
\hypersetup{
    colorlinks=true,
    linkcolor=black,
    filecolor=magenta,
    urlcolor=cyan
}

%% Formatting
\usepackage{parskip}

\title{Projektmanagement}
\author{Moritz}
\date{December 9, 2024}

\begin{document}
\maketitle
\tableofcontents

\section{Einleitung}

Wir werden ein Projekt machen, daran scheitern, vollständig dokumentieren und daraus lernen.

\section{Projekt - Der Turm}

\subsection{Planung}

Zeit: 1 Stunde

\subsection{Bauplan}

\subsection{Schriftlich Dokumentieren}

\subsubsection{Risikobewertung}

\subsection{Bau}

Zeit: 1 Stunde

\subsubsection{Materialien}

20 Papier

1x Schere

1m Tesafilm

\subsection{Abschlusspräsentation \& Test}

\subsection{Reflexion}

Arbeitsteilung kann verbessert werden. Unsere Schritte brauchte viel die Schere, von der hatten wir nur eine.

Die Einteilung der Schere, hat für einiges an Lehrlauf gesorgt.

\subsection{Magisches Dreieck}

Zeit, Qualität, Budge

\section{Prüfung}

ist eine Präsentation über eine Projektplanung, wie Abiball oder Großprojekt. Mit Kriterien, welche noch erarbeitet werden.

\end{document}
\documentclass[a4paper]{article}

%\usepackage{url}

%% Math
\usepackage{mathtools}
%% For Mengen like natural numbers
\usepackage{amsfonts}
%% Für spezielle Symbole
\usepackage{amssymb}

%% Images
\usepackage{import}
\usepackage{xifthen}
\usepackage{pdfpages}
%\usepackage{transparent}

%%% Command for simpler images
\newcommand{\incfig}[1]{%
    \def\svgwidth{\columnwidth}
    \import{./fig/}{#1.pdf_tex}
}

%% Links
\usepackage{hyperref}
\hypersetup{
    colorlinks=true,
    linkcolor=black,
    filecolor=magenta,
    urlcolor=cyan
}

%% Formatting
\usepackage{parskip}

\title{Wissenschaftliches Arbeiten}
\author{Moritz}
\date{March 10, 2025}

\begin{document}
\maketitle
\tableofcontents

\section{Sender - Empfänger}

Mein Bericht ist für jemand, somit muss ich wissen, was ich schreibe und wie ich das schreibe.

\subsection{Empfänger}

Die Hochschule weiß:

\begin{enumerate}
    \item IT (Informatiker)
    \item Wissenschaftliches
\end{enumerate}

..., weiß nicht:

\begin{enumerate}
    \item Aufgabe/ Details
    \item Unternehmen
    \item Programmiersprache
    \item technischer Hintergrund
\end{enumerate}

\section{Technischer Bericht}

Das ist das Überthema zu unserer T1000 oder BA. \href{https://www.dhbw-stuttgart.de/studierendenportal/informatik/studienbetrieb/praxisphasen-berichte/}{Unterlagen und Vorgaben}

\subsection{Fragen zu dem Bericht}

Sind auf einer Folie im Skript S. 11

\subsection{Planung}

\subsection{Schreibregeln}

Kurze Sätze und schritt für schritt das Niveau in der Dokumentation ansteigen, um den Leser an das Thema und dem Fachwissen heranzuführen.

\subsubsection{Logische Gliederung}

Den Bericht parallele zur Arbeit mit schreiben, daraus folgt dann auch das Inhaltsverzeichnis, welches aufeinander aufbaut.

\section{Gliederung}

\begin{enumerate}
    \item Einleitung
    \item Umgebung
    \item Ist-Zustand
    \item Problem
    \item Soll-Zustand
    \item Aus dem Delta $\Delta$ vom Ist zum Soll kommt meine Aufgabe
\end{enumerate}

\subsection{Beispieleinleitung}

Thema: Entwickeln eines Systems zur Demonstration eines T-Systems Produkts

Kurzbeschreibung: Der Bereich PLM Solutions erstellt Lösungen für verschiedene Kunden auf Basis eines bestehendes Datenmanagementsystems.

Derzeit erfolgt eine Kundenpräsentation nur über die Vorführung des Produkts, was zum einen die Zeit eines Mitarbeiters beansprucht und zum anderen das Interesse des Kunden nicht immer im gewünschten Umfang weckt.

Ziel ist es daher dem Kunden, innerhalb von kurzer Zeit, eine Umgebung zur Verfügung zu stellen in der er, mit schriftlicher Anleitung, die Möglichkeiten des Produkts selbst erleben und ausprobieren kann.

Hierfür wird, zunächst für das Produkt "Aras Innovator", eine Demoumgebung erstellt werden, welche alle Funktionen des Produkts beinhaltet, aber lediglich Demo-Datensätze enthält.

Um die Nutzung zu erleichtern und die Bereitstellungszeit zu verkürzen, sollen sich die Interessenten direkt über das Internet registrieren können und erhalten anschließend automatisch eine E-Mail mit einer Anleitung wie sie auf ihre Demoumgebung zugreifen können.

Aufgabe der Praxisarbeit ist der Entwurf und die Implementierung folgender Bestandteile:

\begin{enumerate}
    \item Website zur Registrierung
    \item Automatisches Neuanlegen eines Demosystems
    \item Nutzerverwaltung
    \item Versenden von E-Mails
\end{enumerate}

\section{T1000}

Darf auch aus Tätigkeitsschwerpunkten bestehen. Weitere Praxisarbeiten T2000, ... sind genau ein Projekt.



\end{document}
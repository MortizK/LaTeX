\documentclass[a4paper]{article}

%\usepackage{url}

%% Math
\usepackage{mathtools}
%% For Mengen like natural numbers
\usepackage{amsfonts}
%% Für spezielle Symbole
\usepackage{amssymb}

%% Images
\usepackage{import}
\usepackage{xifthen}
\usepackage{pdfpages}
%\usepackage{transparent}

%%% Command for simpler images
\newcommand{\incfig}[1]{%
    \def\svgwidth{\columnwidth}
    \import{./fig/}{#1.pdf_tex}
}

%% Links
\usepackage{hyperref}
\hypersetup{
    colorlinks=true,
    linkcolor=black,
    filecolor=magenta,
    urlcolor=cyan
}

%% Formatting
\usepackage{parskip}

\title{Intercultural Comunication}
\author{Moritz}

\begin{document}
\maketitle
\tableofcontents


\section{Einführung}

Fast alles wird auf OneNote Gemacht.

\subsection{Dozent}

Wir können gerne Dutzen. Marcus A. Schuerstedt.

Er kommt aus Ludwigshafen.

Es gibt in Teams ein OneNote.

\subsection{Noten}

Wir geteilt mit BWL

\subsection{Kommunikation ist mehrdeutig}

Eine Grafik von einer Jungen Dame und Alten Frau - Illusion

\subsection{Orga}

Termin am 13.12 wieder alle.

Termin am 20.12 fällt weg.

Dannach ist Gruppe 1 und Grupp 2 immer im Wechsel.

Am 10.01 ist Gruppe 1 und 17.01 ist Gruppe 2.

Es wird wohl ziemlich locker sein mit ganz viel Eigenverantwortung.

\subsection{Lernpyramide}

Lernpyramdie und Phasen der (IN) Kompetenz

\section{Interview}

Eric ist 23 Jahre Alt

Eric hat mit 3 Freunden mal kanpp 300 CHicken Nuggets gegessen.

Eric ist schonmal Fallschirm gesprungen

\section{Klausur}

Wir eine Vorstellung auf English über ein Land. Sollte gerne alleine gemacht werden.

Mit Interaktionen wie z.B. Kahoot. Und zeit für Disskussionen.

Dürfen unser Land und Thema auch noch ändern.

Wir dürfen selber entscheiden, wann wir Vorstellen, muss nur geplant werden.

Dürfen ChatGPT nutzen

\subsection{Details}

Bezihungsweise Agenda sind auf Moodle.

\subsection{Abgabe}

Hochladen mit Matrikelnummer. Diese sollte auch auf der Präsentation sichtbar sein.

Hochladen bis spätestenz 01.03 und erst nach meiner Präsentation.

\subsection{Gamification}

Sinvolle Fragen um das Wissen zu vertiefen.

Las keine: Wie Viele $km^2$ hat das Land.

\section{Hofstede Comparision}

details ausarbeiten oder auf nächste Woche warten.

\end{document}
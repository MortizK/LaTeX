\documentclass[a4paper]{article}

%\usepackage{url}

%% Math
\usepackage{mathtools}
%% For Mengen like natural numbers
\usepackage{amsfonts}
%% Für spezielle Symbole
\usepackage{amssymb}

%% Images
\usepackage{import}
\usepackage{xifthen}
\usepackage{pdfpages}
%\usepackage{transparent}

%%% Command for simpler images
\newcommand{\incfig}[1]{%
    \def\svgwidth{\columnwidth}
    \import{./fig/}{#1.pdf_tex}
}

%% Links
\usepackage{hyperref}
\hypersetup{
    colorlinks=true,
    linkcolor=black,
    filecolor=magenta,
    urlcolor=cyan
}

%% Formatting
\usepackage{parskip}

\title{Intercultural Comunication}
\author{Moritz}

\begin{document}
\maketitle
\tableofcontents

\section{Was ist Kultur?}

Musik und Tanz gehören zur Kultur. Sprache, Essen. Traditionen.

\subsection{Cultural Iceberg}

20\% sind das was man Sieht und 80\% sind versteckt oder versteht, man erst wenn man sich genauer anguckt.

Die 20\% ist das, was man direkt wahrnehmen kann.

Die 80\% können: Values, Priorities, Assumptions

\subsubsection{Visual Perceptions}

Language, Clothes, Stores

\subsection{Umfragen}

Verschiedene Umfragen in unserem Kurs. Z.b. Wie lange wir Schlafen. In China gelten wir als Faule Studenten, da wir ja noch Zeit zum Schlafen haben. Dort Schlafen Studenten villeicht 4h.

\section{Hall \& Hofstede}

\subsection{Statistik}

Wir sollen nicht auf den Durchschnitts Deutschen schauen, sondern auf die Individuen. Somit kann eine Blaue Person, Oranger sein als die Durchschnittliche Orange Person.

\subsection{Hall}

Hat 4 Dimensionen oder Boxen aufgebaut.

\subsubsection{Zeit/ Time}

Monochronous Culture: Punctual and work is seperable from personal life.

Polychronous Culture: They do things simultaeously. So they are always on time in their own thinking. If you prepare something for guests on 18:00, the guests should only arrive earliest at 18:30, since the you only start preparing at 18:00.

\subsubsection{Context}

Low Context: Ich gebe dir die Infos

High Context: Der Context ergibt sich. Somit ist die Kommunikation Indirekt. Wie würde ein Britte sagen, dass die Nachbarn zu laut sind. "Die Wände sind hier sehr dünn".

\subsubsection{Space}

How close is to close. The (Private) Space.

We as Germans usually need more (private) area. People have distinctive places of what they call "mine".

The Mediterranean countries have a small bubble, need less (private) area.

\subsubsection{Information}

Speech patterns. We mostly talk, when the other one is not talking. In some cultures, these talking patters are overlapping more.

\subsection{Hofstede}

Die 6 Dimensionen von Hofstede mit einer Weltkarte \url{https://geerthofstede.com/culture-geert-hofstede-gert-jan-hofstede/6d-model-of-national-culture/}

Hier ist noch ein Link, um Länder zu vergleichen. \url{https://www.theculturefactor.com/country-comparison-tool}. Es wird empfohlen, von Deutschland auszugehen.

\subsubsection{Individualism}

Denken wir eher als ein "WIR" oder als ein "ICH".

Ein Land, welches Reich ist, desto mehr neigt es dazu Individueller zu sein. Also "ICH".

\subsubsection{Power Distance}

Wie sehr akzeptire ich, das es sehr viel mächtigere Personen gibt.

\subsubsection{Masculinity}

Ich Lebe um zu Arbeiten.

Eine mehre Feminine Welt, ist mehr um das Wir, um einander kümmern.

\subsubsection{Uncertainty Avoidance}

Ich Plane meinen Urlaub und gehe nicht zum Flughafen und schaue, welche FLug denn Richtung Süden geht.

Andere Kulturen sind da "spontaner" und "freier".

\subsubsection{Long Term Orientation}

Hauptsächlich im Asiatischem Raum.

\subsubsection{Indulagence}

Wie sehr genieße ich das Leben?

\section{G2 Orange}

Wir sollen als Hausaufgabe und/oder im Meeting, eine Hofstede Dimension ausarbeiten und der Gruppe Präsentieren.

\subsection{Power Distance}

\section{Präsentation}

Nur auf die ersten 4 Hofsteder Dimension eingehen und wenn nur auf die Relevantesten 2.

Zudem ist eine Art wie Kahoot pflicht. Und spannende wissenswerte Fragen stellen.

Und 10min bis max 15min.

\subsection{Finland}

als Idee

\subsection{Lettland}

Meine Geschichte mit dem Deutsch Russischen Jugendaustausch und dem Problem mit der Einreise. Dazu der Kontrast zu dem Dort sein, wegen Sprache und Kultur.

Ich kann ganz viel von meiner Beobachtung von Jaroslaw (Slava)

\end{document}
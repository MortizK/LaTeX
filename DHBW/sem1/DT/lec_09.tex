\documentclass[a4paper]{article}

%\usepackage{url}

%% Math
\usepackage{mathtools}
%% For Mengen like natural numbers
\usepackage{amsfonts}
%% Für spezielle Symbole
\usepackage{amssymb}

%% Images
\usepackage{import}
\usepackage{xifthen}
\usepackage{pdfpages}
%\usepackage{transparent}

%%% Command for simpler images
\newcommand{\incfig}[1]{%
    \def\svgwidth{\columnwidth}
    \import{./fig/}{#1.pdf_tex}
}

%% Links
\usepackage{hyperref}
\hypersetup{
    colorlinks=true,
    linkcolor=black,
    filecolor=magenta,
    urlcolor=cyan
}

%% Formatting
\usepackage{parskip}

\title{Digitaltechnik}
\author{Moritz}
\date{January 22, 2025}

\begin{document}
\maketitle
\tableofcontents

\section{Kombinatorische Logik}

\subsection{Gatter}

\subsubsection{Zeitverhalten}

Um diesen Strukturhazard zu umgehen, können wir die Schaltung so zu ändern, dass alle Pfade dieselbe Pfadlänge haben.

Dies können wir tun, indem wir unnötige Gatter verwenden. So verändert wir ein $x_1$ zu $x_1\land x_1$, welches die gleiche Laufzeit wie $\overline{x_2}$ hat.

Wir können auch zusätzliche Logikgatter hinzufügen, um das Hazard zu unterdrücken.

Diese können wir aus dem KV-Diagramm ablesen

\begin{equation}
    \begin{array}{c|cccc}
            &   & x_0 & x_0 &     \\
        \hline
            & 0 & 0   & 1   & 1   \\
        x_1 & 0 & 1   & 1   & 0   \\
        \hline
            &   &     & x_2 & x_2 \\
    \end{array}
\end{equation}

Nun kombinieren wir alle Paare in dieser Schaltung um den Sprung zwischen den beiden Werte mit zu berücksichtigen. SOmit landen wir auf einem Term von $Y=x_1*x_0+x_2*\overline{x_1}+x_2*x_0$

\subsection{Logikbausteine}

Logikgatter sind zu klein um diese Einzeln zu kaufen. Die Kleinsten Bausteine sind TTL-ICs. Diese haben mehrere Pins welche mehrere der selben Art von Gattern anliegen und diese Verarbeiten.

Es gibt auch noch programmierbare Logikbausteine und hochintegrierte Bausteine.

\subsubsection{PROM}

Programmable Read Only Memory. In diesem sind alle UND GATTER für die Anzahl der Eingänge fest verdrahte und wir können die Verbindungen zu den ODER GATTERN programmieren.

Einen PROM können wir Programmieren, indem wir die Ergebnisse (y) aus unser Wahrheitstabelle kopieren.

\subsubsection{PAL/ GAL}

Programmable/ Generic Array Logic. Hier ist es genau anders herum. Hier ist die ODER Matrix fest. So können wir. Die feste ODER Matrix erlaubt nur 4 Produktterme.

\subsubsection{PLA}

Programmable Logic Array. Hier sind beide Matrizen Programmierbar.

\section{Sequentielle Logik}

Lernziele in Kapitel 4 auf S.2.

Systematische Entwurf eines synchrones Schaltnetzes Entwerfern aus einer Problembeschreibung. Als Moore- oder Mealy-Automat

\subsection{Schaltwerke}

Ein Schaltnetz hat n Eingänge und m Ausgänge.

Ein Schaltwerk hat zusätzlich einen Speicher, von dem die Verarbeitung abhält.

Was passiert wenn wir eine Rückkopplung haben. Kann eine unerwünschte Schwingung entstehen, welche eine unbestimmte Frequenz hat.

\subsubsection{RS FLip FLop}

Wenn wir zwei NOR Gatter mit einander Rückkoppeln, indem wir den Ausgang von Gatter 1 als Eingang von Gatter 2 und anders herum nutzen, können wir einen FLip FLop Bauen.

\begin{equation}
    \begin{array}{c|cc|cc|c}
            & A & B & X & Y &               \\
        \hline
        t_1 & 1 & 1 & 0 & 0 &               \\
        t_2 & 1 & 0 & 0 & 1 & \text{ Reset} \\
        t_3 & 0 & 0 & 0 & 1 & \text{ Save}  \\
        t_4 & 0 & 1 & 1 & 0 & \text{ Set X} \\
        t_5 & 0 & 0 & 1 & 0 & \text{ Save}  \\
    \end{array}
\end{equation}

Da wir $Y=\overline{X}$ fordern und $A=B=1$ nicht erlauben um Schwingungen zu verhindern. Können wir unsere Name auch zu dem Buchstaben ihrer Beschreibung benennen. Reset und Save - auch RS Flip FLop.

\begin{equation}
    \begin{array}{cc|c}
        R & S & Q' \\
        \hline
        0 & 0 & Q  \\
        0 & 1 & 1  \\
        1 & 0 & 0  \\
        1 & 1 & ?  \\
    \end{array}
\end{equation}

Dieser Flip FLop hat einen Zeitverzögerung von $2*\Delta t$.

\end{document}
\documentclass[a4paper]{article}

%\usepackage{url}

%% Math
\usepackage{mathtools}
%% For Mengen like natural numbers
\usepackage{amsfonts}
%% Für spezielle Symbole
\usepackage{amssymb}

%% Images
\usepackage{import}
\usepackage{xifthen}
\usepackage{pdfpages}
%\usepackage{transparent}

%%% Command for simpler images
\newcommand{\incfig}[1]{%
    \def\svgwidth{\columnwidth}
    \import{./fig/}{#1.pdf_tex}
}

%% Links
\usepackage{hyperref}
\hypersetup{
    colorlinks=true,
    linkcolor=black,
    filecolor=magenta,
    urlcolor=cyan
}

%% Formatting
\usepackage{parskip}

\title{Digitaltechnik}
\author{Moritz}
\date{February 5, 2025}

\begin{document}
\maketitle
\tableofcontents

\section{Sequentielle Logik}

\subsection{Zähler}

\subsubsection{Aufwärt- und Abwärtszählen}

Zum Hochzählen benutzen wir ein negativ flankengetriggerten Flipflop, Da hier die Zustände sich nur ändern, wenn der vorherige Flipflop den Zustand auf 0 ändern. Dies sorgt für die zusätzliche Zeitverzögerung.

Zum Runterzählen benutzen wir ein positiv flankengetriggerten Flipflop, da hier im ersten Schritte alle Flipflops etwas Zeitverzögert auf 1 Schaltet.

\subsection{Frequenzteiler}

Ein Zähler kann auch als Frequenzteiler betrachtet werden, da dieser jedes - vorher betrachtetes Bit - um die halbe Frequenz teilt.

\subsubsection{Teilerverhältnis}

Wenn das Teilerverhältnis m:1 mit m eine Zweierpotenz, bleibt das Tastverhältnis bei 50\% konstant.

Bei Teilerverhältnisen ungleich einer Zweierpotenz, ergeben sich oft Tastverhältnis von ungleich 50\%.

Das Tastverhältnis ist der Anteil der Impulsdauer innerhalb einer Periode.

\subsection{Realisierung von Schaltwerken}



\end{document}
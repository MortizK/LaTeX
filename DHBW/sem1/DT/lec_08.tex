\documentclass[a4paper]{article}

%\usepackage{url}

%% Math
\usepackage{mathtools}
%% For Mengen like natural numbers
\usepackage{amsfonts}
%% Für spezielle Symbole
\usepackage{amssymb}

%% Images
\usepackage{import}
\usepackage{xifthen}
\usepackage{pdfpages}
%\usepackage{transparent}

%%% Command for simpler images
\newcommand{\incfig}[1]{%
    \def\svgwidth{\columnwidth}
    \import{./fig/}{#1.pdf_tex}
}

%% Links
\usepackage{hyperref}
\hypersetup{
    colorlinks=true,
    linkcolor=black,
    filecolor=magenta,
    urlcolor=cyan
}

%% Formatting
\usepackage{parskip}

\title{Digitaltechnik}
\author{Moritz}
\date{January 21, 2025}

\begin{document}
\maketitle
\tableofcontents

\section{Kombinatorische Logik}

\subsection{KV Diagramm}

Wenn wir "don't care" Werte in einem KV Diagramm haben, markieren wir die 1 und die d für "don't care". Wir bilden also immer die größten Rechtecke um möglichst wenig Terme zu benutzen.

\subsubsection{4 Stellig}

Das Modell finde ich am ende der letzten Vorlesung. Hier nochmal eine leicht vereinfachte Version mit Indexen.

\begin{equation}
    \begin{array}{cc|cccc|cc}
          &     & 0  & 0   & 1   & 1   &     &   \\
          &     &    &     & x_3 & x_3 &     &   \\
        \hline
        0 &     & 0  & 1   & 5   & 4   &     & 0 \\
        1 & x_2 & 2  & 3   & 7   & 6   &     & 0 \\
        1 & x_2 & 10 & 11  & 15  & 14  & x_4 & 1 \\
        0 &     & 8  & 9   & 13  & 12  & x_4 & 1 \\
        \hline
          &     &    & x_1 & x_1 &     &     &   \\
          &     & 0  & 1   & 1   & 0   &     &   \\
    \end{array}
\end{equation}

\subsubsection{7 Segment Display - Aiken Code}

\begin{equation}
    \begin{array}{c|cccc|ccccccc}
        BCD & x_4 & x_3 & x_2 & x_1 & a & b & c & d & e & f & g \\
        \hline
        0   & 0   & 0   & 0   & 0   & 1 & 1 & 1 & 1 & 1 & 1 &   \\
        1   & 0   & 0   & 0   & 1   &   & 1 & 1 &   &   &   &   \\
        2   & 0   & 0   & 1   & 0   & 1 & 1 &   & 1 & 1 &   & 1 \\
        3   & 0   & 0   & 1   & 1   & 1 & 1 & 1 & 1 &   &   & 1 \\
        4   & 0   & 1   & 0   & 0   &   & 1 & 1 &   &   & 1 & 1 \\
            & 0   & 1   & 0   & 1   &   &   &   &   &   &   &   \\
            & 0   & 1   & 1   & 0   &   &   &   &   &   &   &   \\
            & 0   & 1   & 1   & 1   &   &   &   &   &   &   &   \\
            & 1   & 0   & 0   & 0   &   &   &   &   &   &   &   \\
            & 1   & 0   & 0   & 1   &   &   &   &   &   &   &   \\
            & 1   & 0   & 1   & 0   &   &   &   &   &   &   &   \\
        5   & 1   & 0   & 1   & 1   & 1 &   & 1 & 1 &   & 1 & 1 \\
        6   & 1   & 1   & 0   & 0   & 1 &   & 1 & 1 & 1 & 1 & 1 \\
        7   & 1   & 1   & 0   & 1   & 1 & 1 & 1 &   &   &   &   \\
        8   & 1   & 1   & 1   & 0   & 1 & 1 & 1 & 1 & 1 & 1 & 1 \\
        9   & 1   & 1   & 1   & 1   & 1 & 1 & 1 & 1 &   & 1 & 1 \\
    \end{array}
\end{equation}

\begin{equation}
    \begin{matrix}
        \begin{array}{cc|cccc}
                &     &   &     & x_3 & x_3 \\
                &     &   & x_1 & x_1 &     \\
            \hline
                &     & 0 & 1   &     & 4   \\
                & x_2 & 2 & 3   &     &     \\
            x_4 & x_2 &   & 5   & 9   & 8   \\
            x_4 &     &   &     & 7   & 6   \\
        \end{array} &
        \begin{array}{cc|cccc}
            a   &     &   &     & x_3 & x_3 \\
                &     &   & x_1 & x_1 &     \\
            \hline
                &     & 1 &     & d   &     \\
                & x_2 & 1 & 1   & d   & d   \\
            x_4 & x_2 & d & 1   & 1   & 1   \\
            x_4 &     & d & d   & 1   & 1   \\
        \end{array} &
        \begin{array}{cc|cccc}
            b   &     &   &     & x_3 & x_3 \\
                &     &   & x_1 & x_1 &     \\
            \hline
                &     & 1 & 1   & d   & 1   \\
                & x_2 & 1 & 1   & d   & d   \\
            x_4 & x_2 & d &     & 1   & 1   \\
            x_4 &     & d & d   & 1   &     \\
        \end{array} \\ \\
        \begin{array}{cc|cccc}
            c   &     &   &     & x_3 & x_3 \\
                &     &   & x_1 & x_1 &     \\
            \hline
                &     & 1 & 1   & d   & 1   \\
                & x_2 &   & 1   & d   & d   \\
            x_4 & x_2 & d & 1   & 1   & 1   \\
            x_4 &     & d & d   & 1   & 1   \\
        \end{array} &
        \begin{array}{cc|cccc}
            d   &     &   &     & x_3 & x_3 \\
                &     &   & x_1 & x_1 &     \\
            \hline
                &     & 1 &     & d   &     \\
                & x_2 & 1 & 1   & d   & d   \\
            x_4 & x_2 & d & 1   & 1   & 1   \\
            x_4 &     & d & d   &     & 1   \\
        \end{array} &
        \begin{array}{cc|cccc}
            e   &     &   &     & x_3 & x_3 \\
                &     &   & x_1 & x_1 &     \\
            \hline
                &     & 1 &     & d   &     \\
                & x_2 & 1 &     & d   & d   \\
            x_4 & x_2 & d &     &     & 1   \\
            x_4 &     & d & d   &     & 1   \\
        \end{array} \\ \\
        \begin{array}{cc|cccc}
            f   &     &   &     & x_3 & x_3 \\
                &     &   & x_1 & x_1 &     \\
            \hline
                &     & 1 &     & d   & 1   \\
                & x_2 &   &     & d   & d   \\
            x_4 & x_2 & d & 1   & 1   & 1   \\
            x_4 &     & d & d   &     & 1   \\
        \end{array} &
        \begin{array}{cc|cccc}
            g   &     &   &     & x_3 & x_3 \\
                &     &   & x_1 & x_1 &     \\
            \hline
                &     &   &     & d   & 1   \\
                & x_2 & 1 & 1   & d   & d   \\
            x_4 & x_2 & d & 1   & 1   & 1   \\
            x_4 &     & d & d   &     & 1   \\
        \end{array} &
    \end{matrix}
\end{equation}

\begin{equation}
    \begin{matrix}
        a= (\overline{x_3}+x_4)*(\overline{x_1}+x_2+x_4)    &
        b= (\overline{x_4}+x_3)*(\overline{x_4}+x_2+x_1)            \\
        c= x_3+\overline{x_2}+x_1                           &
        d= x_2+\overline{x_3}*\overline{x_1}+x_4*x_3*\overline{x_1} \\
        e= \overline{x_3}*\overline{x_1}+x_4*\overline{x_1} &
        f= x_4*x_2+\overline{x_2}*\overline{x_1}+\overline{x_4}*x_3 \\
        g=x_3*\overline{x_1}+x_2
    \end{matrix}
\end{equation}

Diese Funktionen habe ich in einem Logic Simulator nachgebaut und überprüft. Sie funktionieren.

\subsubsection{Minimale KDNF und KKNF}

Entweder durch das Zusammenfassen von 0 oder 1. Die DNF wird aus 1 gebildet und die KNF aus 0.

\subsection{Schaltnetze}

Ein Schaltnetz wird auch als Kombinatorische Logik bezeichnet. Ein Schaltnetz kann beliebig viele Eingänge und Ausgangsfunktionen haben.

Zudem haben Sie einen weiteren Input, welcher der control genannt wird. Dieser gibt an, welcher wert wir durchschalten wollen.

\subsubsection{1-aus-k Multiplexer}

Dieser schaltet den k-ten Wert durch. Wir können einen Multiplexer auch als Logikschaltung darstellen.

\subsubsection{1-zu-k Demultiplexer}

Dieser schalten den einen Eingang nur an einem der vielen Ausgänge durch.

\subsubsection{k-zu-n Kodierer}

Wir haben k Eingänge, n Ausgänge und keine Control Leitung.

Hier darf nur genau ein Eingang auf 1 sein. Als Ausgang, kommt dan der Index von dem gesetztem Eingang.

\subsubsection{n-zu-k Dekodierer}

Als Input kommt ein Index, und als Ausgang kommt dann $K_n$ als 1.

\subsubsection{Realisierung}

\subsection{Gatter}

Gatter haben in der Realen Welt eine kleine Verzögerung. Zudem springt der Wert nicht, sonder die Spannung verändert sich kontinuierlich.

Die Verzögerung nennen wir Tau: $\tau$ ist definiert als Zeitspanne zwischen den ZEitpunkten der Überschreitung eines 50$\%$ Pegel der realen Schaltung.

\subsubsection{Zeitverhalten}

Durch die kleine Verzögerung entsteht das sogenannte Strukturhazard.

\end{document}
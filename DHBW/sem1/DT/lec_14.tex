\documentclass[a4paper]{article}

%\usepackage{url}

%% Math
\usepackage{mathtools}
%% For Mengen like natural numbers
\usepackage{amsfonts}
%% Für spezielle Symbole
\usepackage{amssymb}

%% Images
\usepackage{import}
\usepackage{xifthen}
\usepackage{pdfpages}
%\usepackage{transparent}

%%% Command for simpler images
\newcommand{\incfig}[1]{%
    \def\svgwidth{\columnwidth}
    \import{./fig/}{#1.pdf_tex}
}

%% Links
\usepackage{hyperref}
\hypersetup{
    colorlinks=true,
    linkcolor=black,
    filecolor=magenta,
    urlcolor=cyan
}

%% Formatting
\usepackage{parskip}

\title{Digitaltechnik}
\author{Moritz}
\date{February 11, 2025}

\begin{document}
\maketitle
\tableofcontents

\section{Halbleitertechnik}

\subsection{Speicherhierarchie}

Es gibt verschieden Speicher welche je schnelle sie sind, sind sie auch kleiner und näher an der CPU. So din Register nur einzelne Bytes groß haben dafür eine Zugriffszeit von 1ns.

Die Caches sind auch unterschiedlich groß und sorgen dafür, dass die geschwindigkeitsdifferenz zwischen RAM (Hauptspeicher) und CPU "überbrückt".

\subsection{Begriffe}

Umwandlungen und Größen.

Es gibt die bekannten: kB, MB, GB. Diese sind aber in den 10er Potenzen.

Es gibt aber auch die Kibi Bytes: KiB, MiB, GiB, welche auf basis der 2er Potenzen sind.

\begin{center}
    \begin{tabular}{c c|c c}
        kB & $10^3$ & KiB & $2^{10}$ \\
        MB & $10^6$ & MiB & $2^{20}$ \\
        GB & $10^9$ & GiB & $2^{20}$ \\
    \end{tabular}
\end{center}

\subsubsection{Zugriffszeit}

\href{https://elearning.dhbw-stuttgart.de/moodle/pluginfile.php/834358/mod_resource/content/1/DT5mem.pdf#page=8}{Zugriffszeit}

\subsubsection{Flüchtiger Speicher}

Statischer Speicher: Speicherung durch Transistoren, mit SRAM, Dual-ported SRAM

Dynamische Speicher: Speicherung durch einen Transistor und einem Kondensator, welcher immer wieder aufgefrischt werden muss. Bausteine: DRAM, SDRAM, DDR-SDRAM

\subsection{SRAM}

Eine \href{https://elearning.dhbw-stuttgart.de/moodle/pluginfile.php/834358/mod_resource/content/1/DT5mem.pdf#page=10}{Typische SRAM-Zelle} besteht aus FET-Transistor. Dieser hat drei Eingänge: Gate, Source, Drain. Wenn Gate gesetzt ist, so gilt Source = Drain. Bei Gate=0 sind diese beiden getrennt.

\subsubsection{Architektur}

Diese Speicherzellen werden in Worten mit Wortlänge $n$ Organisiert. Jedes Wort ist dann eine Zeile in einer Speichermatrix $m\times n$. Die Zielen lassen sich über einen Adressdecoder auslesen lassen.

Es gibt dann verschieden Eingaben, um auf diesen Speicher zuzugreifen: Lesen und Schreiben geht nur getrennt/ nacheinander.

\subsubsection{Tristate-Treiber}

\href{https://elearning.dhbw-stuttgart.de/moodle/pluginfile.php/834358/mod_resource/content/1/DT5mem.pdf#page=17}{Tristate-Treiber} haben drei Zustände:

\begin{enumerate}
    \item G=0, Treiber nicht aktiv und Y ist offen (hochohmig)
    \item G=1, Treiber ist aktiv und Y=X mit X=1
    \item G=1, Treiber ist aktiv und Y=X mit X=0
\end{enumerate}

\end{document}
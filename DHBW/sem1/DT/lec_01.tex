\documentclass[a4paper]{article}

%\usepackage{url}

%% Math
\usepackage{mathtools}
%% For Mengen like natural numbers
\usepackage{amsfonts}
%% Für spezielle Symbole
\usepackage{amssymb}

%% Images
\usepackage{import}
\usepackage{xifthen}
\usepackage{pdfpages}
%\usepackage{transparent}

%%% Command for simpler images
\newcommand{\incfig}[1]{%
    \def\svgwidth{\columnwidth}
    \import{./fig/}{#1.pdf_tex}
}

%% Links
\usepackage{hyperref}
\hypersetup{
    colorlinks=true,
    linkcolor=black,
    filecolor=magenta,
    urlcolor=cyan
}

%% Formatting
\usepackage{parskip}

\title{Digitaltechnik}
\author{Moritz}

\begin{document}
\maketitle
\tableofcontents

\section{Kapitel 1 | Einführung}

Heißt auch Technische Informatik I im Modulhandbuch

\subsection{Lernziele}

Grundlagen
Zahlendarstellung
Booleschen Algebra Anwendung
Systematischen Entwurd von Schaltnezten
Typische Flip Flop Arten, aka Schaltwerke (Zähler/ Frequenzteiler)

\subsection{Bücher}

auf dem Skript

Digitaltechnik (Elektronik 4) Klaus BEuth, Olaf BEuth

Digitaltechnik: Lehr und Übungsbuch, Klaus Fricke für Elektrotechnik und Informatiker - gibt es in der online Bibi

Digitaltechnik: Grundlagen, VHDL, FPGAs, Mikrocontroller, Winfried
Gehrke - gibt es in der online Bibi.

\subsection{Aufgabe}

Gibt es jede Woche. Wird teilweise in der Vorlesung besprochen.

\subsection{Moodle}

Vorlesungsfolien sind als pdf auf moodle.

Die Aufgaben sind auch dort.

\subsection{Klausur}

90min entsprechen 90 Punkte.

Man braucht 45P um zu bestehen.

Schwerpunkt wird die Realisierung von Schaltwerken.

\section{Mitstudierende}

Viele sind bei Mercedes, Teilweise INformatik Firmem und Polizei.

Zudem habe einige eine Hintergrund mit der

\section{Einführung}

\subsection{Lernziele}

Vor und Nachteile analog vs. Digitaltechnik
Kodierungen in positiv, negativ.

\subsection{Motivation}

unterste Schicht der Rechnerachritektur

Digitale Logik (GAtter)

Mikroachritektur (ALU)

Instruktionssatz (Schnittstelle Hard- Software)

System (Busse, CPU und Chips)

\subsection{Historisches}

Bei bedarf aus dem Skript entnehmen

\subsection{Analog- und Digitaltechnik}

Analgotechnik hat kontinuierliche Signale (Kurven)

Digitale Signale sind diskrete (Stufenweise), sie haben nur k Zustände

Bei $k=2$ spricht man von binären Signalen

\subsection{Spannungen}

\begin{equation}
    \begin{split}
        U<0.8V \implies \text{logisch 1} \\
        U>3.2V \implies \text{logisch 0} \sum
    \end{split}
\end{equation}

\begin{figure}[ht]
    \centering
    \incfig{uri}
    \caption{URI}
    \label{fig:URI}
\end{figure}

\subsection{Vor und Nachteile}

%Nachtragen aus dem Script
Nachtragen aus dem Script

\subsection{Analog- und Digitaltechnik}

Kodierung erfolgt in Wortbreite n.

n repräsentiert ein Bit (also 1 oder 0)

$n=8$ ist ein Byte

$n=16$ ist ein Halbwort

$n=32$ ist ein Wort

$n=64$ ist ein Doppelwort (64Bit Prozessoren)

\subsection{Darstellung Positiver Zahlen}

\begin{equation}
    \begin{split}
        \sum_{i=0}^{n-1}x_i*10^i & \mid \text{Dezimalzahlen} \\
        \sum_{i=0}^{n-1}y_i*2^i  & \mid \text{Binärzahlen}
    \end{split}
\end{equation}

Es gibt auch noch andere Zahlensysteme: Oktalzahlen (Basis 8) oder Hexadezimal (Basis 16)

\subsection{Umwandlung von Zahlendarstellung}

Dividiere Decimalzahlen $x$ durch die größte Zahl $2^k$, welche $x<=2^k$ ist. Diese von x subtrahieren und diesen Schritt bis $x=0$ wiederholen.

Es gibt auch die Variante mit modulo Rechnung. Hier muss auf die Richtige Aufschreibreihenfolge geachtet werden.

\begin{equation}
    \begin{array}{c c c}
        29/2 & =14, & r_0=1 \\
        14/2 & =7,  & r_1=0 \\
        7/2  & =3,  & r_2=1 \\
        3/2  & =1,  & r_3=1 \\
        1/2  & =0,  & r_4=1
    \end{array}
\end{equation}

\end{document}
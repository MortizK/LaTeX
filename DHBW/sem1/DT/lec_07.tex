\documentclass[a4paper]{article}

%\usepackage{url}

%% Math
\usepackage{mathtools}
%% For Mengen like natural numbers
\usepackage{amsfonts}
%% Für spezielle Symbole
\usepackage{amssymb}

%% Images
\usepackage{import}
\usepackage{xifthen}
\usepackage{pdfpages}
%\usepackage{transparent}

%%% Command for simpler images
\newcommand{\incfig}[1]{%
    \def\svgwidth{\columnwidth}
    \import{./fig/}{#1.pdf_tex}
}

%% Links
\usepackage{hyperref}
\hypersetup{
    colorlinks=true,
    linkcolor=black,
    filecolor=magenta,
    urlcolor=cyan
}

%% Formatting
\usepackage{parskip}

\title{Digitaltechnik}
\author{Moritz}

\begin{document}
\maketitle
\tableofcontents

\section{Kombinatorische Logik}

\subsection{Normalform}

Gelegentlich sind gewisse Eingangskombinationen irrelevant, in diesem Fall wird in der Wahrheitstabelle für den Ausgang ein $d$ für "i don't care" eingetragen.

Diese Zeilen können für die Bildung der KDNF oder KKNF nicht berücksichtigt werden.

\subsubsection{Übung mit Primzahlen}

\begin{equation}
    \begin{array}{cccc|c}
          & A_2 & A_1 & A_0 & Y \\
        \hline
        0 & 0   & 0   & 0   & d \\
        1 & 0   & 0   & 1   & 0 \\
        2 & 0   & 1   & 0   & 1 \\
        3 & 0   & 1   & 1   & 1 \\
        4 & 1   & 0   & 0   & 0 \\
        5 & 1   & 0   & 1   & 1 \\
        6 & 1   & 1   & 0   & 0 \\
        7 & 1   & 1   & 1   & 1 \\
    \end{array}
\end{equation}

Die KDNF: $Y=\overline{A_2}A_1\overline{A_0}+\overline{A_2}A_1A_0+A_2\overline{A_1}A_0+A_2A_1A_0$

Die KKNF: $Y=(A_2+A_1+\overline{A_0})*(\overline{A_2}+A_1+A_0)*(\overline{A_2}+\overline{A_1}+A_0)$

\subsection{Minimale Darstellung}

Kanonische Normalformen sind eindeutig, jedoch nicht minimal.

Minimal kann unterschiedliches Bedeuten: Anzahl Gatter, Anzahl Leitungen, Anzahl der Min- Maxterme? Wir verwenden Anzahl Boolescher Operatoren

Wir können diese Kanonischen Normalformen auch durch Anwendung der Axiome und deren Gesetze vereinfache. Somit können wir weniger Boolesche Operationen Erreiche, verlieren hierfür aber die Normalform.

\subsubsection{Karnaugh-Veitch-Diagramme (KV-Dia.)}

Hieraus können wir Bereiche finden, die Überdeckungen haben. Hier haben wir die Bereiche bei $x_1=0$ und $x_2=1$, somit können wir $y=\overline{x_1}+x_2$ bilden.

\begin{equation}
    \begin{matrix}
        \begin{array}{c|cc}
                           & \overline{x_1}               & x_1               \\
            \hline
            \overline{x_2} & \overline{x_1}\overline{x_2} & x_1\overline{x_2} \\
            x_2            & \overline{x_1}x_2            & x_1x_2            \\
        \end{array} &
        \begin{array}{c|cc}
                           & \overline{x_1} & x_1 \\
            \hline
            \overline{x_2} & 1              & 0   \\
            x_2            & 1              & 1
        \end{array}
    \end{matrix}
\end{equation}

Diese Diagramm kann man erweitern auf mehrere Eingaben, nur für wenige Eingaben. Hier ist eins für 3 Eingabefelder

\begin{equation}
    \begin{array}{cc|cccc}
          &                & 0              & 0              & 1   & 1              \\
          &                & \overline{x_3} & \overline{x_3} & x_3 & x_3            \\
        \hline
        0 & \overline{x_2} & 1              & 1              & 0   & 1              \\
        1 & x_2            & 1              & 0              & 0   & 0              \\
        \hline
          &                & \overline{x_1} & x_1            & x_1 & \overline{x_1} \\
          &                & 1              & 0              & 0   & 1              \\
    \end{array}
\end{equation}

Hier kommt ein minimaler DNF raus: $Y=\overline{x_1}*\overline{x_3}+\overline{x_1}*\overline{x_2}+\overline{x_2}*\overline{x_3}$

Für 4 Eingabefelder

\begin{equation}
    \begin{array}{cc|cccc|cc}
          &                & 0              & 0              & 1   & 1              &   &                \\
          &                & \overline{x_3} & \overline{x_3} & x_3 & x_3            &   &                \\
        \hline
        0 & \overline{x_2} & 0              & 1              & 5   & 4              & 0 & \overline{x_4} \\
        1 & x_2            & 2              & 3              & 7   & 6              & 0 & \overline{x_4} \\
        1 & x_2            & 10             & 11             & 15  & 14             & 1 & x_4            \\
        0 & \overline{x_2} & 8              & 9              & 13  & 12             & 1 & x_4            \\
        \hline
          &                & \overline{x_1} & x_1            & x_1 & \overline{x_1} &   &                \\
          &                & 0              & 1              & 1   & 0              &   &                \\
    \end{array}
\end{equation}

\end{document}
\documentclass[a4paper]{article}

%\usepackage{url}

%% Math
\usepackage{mathtools}
%% For Mengen like natural numbers
\usepackage{amsfonts}
%% Für spezielle Symbole
\usepackage{amssymb}

%% Images
\usepackage{import}
\usepackage{xifthen}
\usepackage{pdfpages}
%\usepackage{transparent}

%%% Command for simpler images
\newcommand{\incfig}[1]{%
    \def\svgwidth{\columnwidth}
    \import{./fig/}{#1.pdf_tex}
}

%% Links
\usepackage{hyperref}
\hypersetup{
    colorlinks=true,
    linkcolor=black,
    filecolor=magenta,
    urlcolor=cyan
}

%% Formatting
\usepackage{parskip}

\title{Digitaltechnik}
\author{Moritz}
\date{December 16, 2024}

\begin{document}
\maketitle
\tableofcontents

\section{Kodierungen}

\subsection{Fehlererkennende Codes}

Ein Code mit der Hamming-Distanz $h$ kann $(h-1)$ Bitfehler erkennen.

Ein Code mit der Hamming-Distanz $h$ kann $(h-1)/2$ Bitfehler korrigieren.

\subsection{Hamming(7,4)-Code}

Hat 7 Bits und 3 Korrekturbits. Jedes Korrekturbit hat die Parität von unterschiedlichen Gruppen an Wertbits.

Alice sendet $10_{10}=1010_2=101\_0\_\_$. Somit wird $1010010$ gesendet. Leider kommt $1000010$ mit einem Fehler an. Wenn wir nun die Prüfbits dieser Zahl neu korrigieren, kommen wir auf $1001011$. Wir listen nur die falschen Prüfbits auf und kommen auf $101_2=5_{10}$. Somit ist das fünfte Bit, beginnend mit 1 von rechts nach links das Falsche Bit.

Wenn wir diesen Code nun erweitern, fällt auf, das die Paritätsbits an den stellen $2^n$ sind, somit: $1,2,4,8,\dots$.

Allgemein gilt: $(2^k-1,2^k-1-n)$-Hamming-Codes mit $k=$Paritätsbit, $n=$ Datenbits und $N=n+k=2^k-1$.

\subsubsection{Beispiel Fehlererkennenung}

Mithilfe der Übertragung werden die Prüfbits neu Berechnet. Wenn diese mit den Übertragenen Prüfbits nicht Übereinstimmen, wird eine 1 gemerkt.

Somit wird aus $1000010$ mit der Wertigkeit $D_4, D_3, D_2, P_3, D_1, P_2, P_1$. Anhand der Datenbits ermitteln wir, das die Prüfbits $1,1,1$ sein müssten, aber $010$ sind. Somit liegt er Fehler an dem Bit $101_2=5$ an der 5ten stelle der Übertragung. Hier wäre also die richtige Nachricht $1010010$

\section{Kombinatorische Logik}

Aus der Sicht der Logik ergeben sich 2 Zustände. wahr/ falsche, 1/0.

Elektronisch sind das kein (oder ein sehr geringer) Stromfluss. Und aus ist ein (oder ein großer) Stromfluss.

\subsection{Gatter}

Ein Gatter ist eine "Black Box". mit einem, zwei oder mehreren Eingängen $A,B,C\in \{0,1\}$.

Die gängige Form ein Gatter zu schreiben ist eine Box mit einem Symbol in der Box um das Gatter zu innenzitieren.

Ein Sinnvoller Link zu Sebastian Lague: \url{https://github.com/SebLague/Digital-Logic-Sim}

\subsubsection{UND-Gatter}

Auch Konjunktion genannt. $Y=A*B$ oder auch $Y=AB$, $Y=A\land B$ und $Y=A\& B$

Symbol in der Box ist $\&$

\begin{equation}
    \begin{array}{c|c|c}
        A & B & Y=A*B \\
        \hline
        0 & 0 & 0     \\
        0 & 1 & 0     \\
        1 & 0 & 0     \\
        1 & 1 & 1     \\
    \end{array}
\end{equation}

\subsubsection{ODER-Gatter}

Auch Disjunktion genannt. $Y=A+B$.

Symbol in der Box ist: $\geq 1$

\begin{equation}
    \begin{array}{c|c|c}
        A & B & Y=A+B \\
        \hline
        0 & 0 & 0     \\
        0 & 1 & 1     \\
        1 & 0 & 1     \\
        1 & 1 & 1     \\
    \end{array}
\end{equation}

\subsubsection{NICHT-Gatter}

Oder auch Negation oder Inverter genannt. $Y=\overline{A}$

Symbol in der Box ist $1$

\begin{equation}
    \begin{array}{c|c}
        A & Y=\overline{A} \\
        \hline
        0 & 1              \\
        1 & 0              \\
    \end{array}
\end{equation}

\subsubsection{NAND, NOR}

Die Negation wird auch durch einen Kreis bei dem Ausgang einer Box gekennzeichnet.

Diese beiden Nicht Elementaren Gatter können mit den oben genannten Elementaren Gattern gebaut werden.

\begin{equation}
    \begin{matrix}
        \begin{array}{c|c|c}
            A & B & Y=\overline{A*B} \\
            \hline
            0 & 0 & 1                \\
            0 & 1 & 1                \\
            1 & 0 & 1                \\
            1 & 1 & 0                \\
        \end{array} &
        \begin{array}{c|c|c}
            A & B & Y=\overline{A+B} \\
            \hline
            0 & 0 & 1                \\
            0 & 1 & 0                \\
            1 & 0 & 0                \\
            1 & 1 & 0                \\
        \end{array}
    \end{matrix}
\end{equation}

\subsubsection{XOR, XNOR}

Das XOR hat das Symbol $=1$. Auch Antivalenz genannt.

Das XNOR, auch Equivalent hat das Symbol $=1$ mit einem Punkt hinter der Box für die Negation.

Das XOR und XNOR sind auch Nicht Elementare Gatter und können durch die Elementaren Gatter AND, OR und NOT gebildet werden.

\begin{equation}
    \begin{matrix}
        \begin{array}{c|c|c}
            A & B & Y=A\oplus B \\
            \hline
            0 & 0 & 0           \\
            0 & 1 & 1           \\
            1 & 0 & 1           \\
            1 & 1 & 0           \\
        \end{array} &
        \begin{array}{c|c|c}
            A & B & Y=A\equiv B \\
            \hline
            0 & 0 & 1           \\
            0 & 1 & 0           \\
            1 & 0 & 0           \\
            1 & 1 & 1           \\
        \end{array}
    \end{matrix}
\end{equation}

\subsection{Boolesche Algebra}

Verschoben auf den nächsten Termin.

\end{document}
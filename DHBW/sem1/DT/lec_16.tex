\documentclass[a4paper]{article}

%\usepackage{url}

%% Math
\usepackage{mathtools}
%% For Mengen like natural numbers
\usepackage{amsfonts}
%% Für spezielle Symbole
\usepackage{amssymb}

%% Images
\usepackage{import}
\usepackage{xifthen}
\usepackage{pdfpages}
%\usepackage{transparent}

%%% Command for simpler images
\newcommand{\incfig}[1]{%
    \def\svgwidth{\columnwidth}
    \import{./fig/}{#1.pdf_tex}
}

%% Links
\usepackage{hyperref}
\hypersetup{
    colorlinks=true,
    linkcolor=black,
    filecolor=magenta,
    urlcolor=cyan
}

%% Formatting
\usepackage{parskip}

\title{Digitaltechnik}
\author{Moritz}
\date{February 18, 2025}

\begin{document}
\maketitle
\tableofcontents

\section{Halbleitertechnik}

\subsection{Aufgabenblatt 8.}

\begin{center}
    \begin{tabular}{l|l}
        SRAM                                   & DRAM                       \\
        \hline
        Speicherung in Flipflops               & Speicherung im Kondensator \\
        Kein Refresh                           & Refresh, weil sich der     \\ & Kondensator entlädt \\
        Zugriffszeit: 5ns                      & Zugriffszeit: 50ns         \\
        6 Transistoren                         & je Zelle 1 Kondensator     \\ $\implies$ größer als DRAM & + 1 Transistor\\
        unempfindlich                          & empfindlich gegenüber      \\ & Elektromagn. Strahlung \\
        Organisation $m\times n$ mit $m\in2^k$ & Organisation $m\times n$   \\
        1 Speichermatrix                       & n Speichermatrizen         \\
        Auswahl eine Zeile durch k-Bit Adresse & Adress-Multiplexing,       \\ & $\frac{k}{2}$ Adressleitungen, $\overline{RAS}, \overline{CAS}$
    \end{tabular}
\end{center}

\subsubsection{Refreshzeit Rechnung}

\begin{equation}
    \frac{32ms}{8192} = \frac{32000\mu s}{8192} = 3,9\mu s \implies 3,9\mu s > 3,5\mu s
\end{equation}

Somit Reicht dieser Refresh von $3,5\mu s$, da dieser Kleiner ist als die maximal erlaubte Refresh Zeit.

\subsubsection{Speicherbandbreite}

Wir bauen aus mehreren 64M-4 Bausteinen einen 128M-8 Baustein. Unser ursprüngliche Baustein hat eine Zykluszeit von 60ns

Da unsere Gebauter Baustein jeden Zyklus nun einen Byte überträgt. So muss nun einfach multipliziert werden:

\begin{equation}
    \frac{1Byte}{60ns} = 16,67 \frac{MByte}{s}
\end{equation}

\subsubsection{Adressraum}

Adressen werden in Hexadecimal angegeben.

\know{Ist Klausurrelevant!}{Skizze ist in meinem Heft unter DT 18.02 Adressraum.}

\section{Wandlung Digital $\leftrightarrow$ Analog}

Im Analogen sind wir kontinuierlich, sowohl in der Zeit, als auch in dem Amplituden.

Im Digitalen, können wir nur zu bestimmten Zeiten einen Wert auslesen und diesen auch nur mit einer festen Auflösung. Somit sind beide diskret.

\subsection{Digital-Analog-Umwandlung}

Funktioniert nur bei festen Bit-Wertigkeiten wie $2^n$. Hier entstehen Stufen. Und diese Stufenhöhe ist genau das 1 LSB.

Signale können mit einem Tiefpass geglättet werden. Dieser macht aus jeder Stufe ein Spline/ Kurve.

Die Genauigkeit ist $\pm \frac{1}{2}$LSB

In der Praxis ist die Genauigkeit doch größer bis zu 2LSB

\subsubsection{Gestuftes Widerstandsnetzwerk}

Hier schalten wir unsere Verschieden Bits (A, B, C, D) in Reihe in unterschiedliche Widerstände.

\begin{center}
    \begin{tabular}{r c c}
        $R_1$ & $R/1$ & 5000$\Omega$ \\
        $R_2$ & $R/2$ & 2500$\Omega$ \\
        $R_3$ & $R/4$ & 1250$\Omega$ \\
        $R_4$ & $R/8$ & 625$\Omega$  \\
        $R_x$ &       & 10$\Omega$   \\
    \end{tabular}
\end{center}

Somit kommen für $I_A, \dots, I_D$ unterschieldiche Spannungen. Bei $U=5V$

\begin{center}
    \begin{tabular}{r c}
        $I_A$ für A=1 & $I_A=1mA$ \\
        $I_B$ für B=1 & $I_B=1mA$ \\
        $I_C$ für C=1 & $I_C=1mA$ \\
        $I_D$ für D=1 & $I_D=1mA$ \\
    \end{tabular}
\end{center}

Wir können nun $I_G=I_A+I_B+I_C+I_D$ bilden und kommen zu diesen Ergbenissen:

\begin{center}
    \begin{tabular}{c c c c|c|c}
        \multicolumn{4}{c|}{Digital} &   & Analog                           \\
        D                            & C & B      & A & $I_G$/mA & $U_A$/mV \\
        \hline
        0                            & 0 & 0      & 0 & 0        & 0        \\
        0                            & 0 & 0      & 1 & 1        & 10       \\
        0                            & 0 & 1      & 0 & 2        & 20       \\
        0                            & 0 & 1      & 1 & 3        & 30       \\
        0                            & 1 & 0      & 0 & 4        & 40       \\
        0                            & 1 & 0      & 1 & 5        & 50       \\
        0                            & 1 & 1      & 0 & 6        & 60       \\
        0                            & 1 & 1      & 1 & 7        & 70       \\
        \vdots                       &   &        &   & \vdots   & \vdots   \\
        1                            & 1 & 1      & 1 & 15       & 150      \\
    \end{tabular}
\end{center}

\subsubsection{Bessere Implementierung}

WIr können diese Schaltung verbessern, indem wir die Eingaben (A, B, C, D) in einen Transistor als Schaltsignal verwenden. Diese Transistoren würden daraufhin ein konstantes (z.B. 2V) durchschalten.

Diese Signal kommt in einen Operationsverstärker, welcher das Signal verstärkt.

\begin{equation}
    U_A=-\frac{R_N}{R_V}*U_E
\end{equation}

Wir wählen $R_0 = 20k\Omega$ und verringern den in zweierpotenzen. Nun ergeben sich Spannungen:

\begin{center}
    \begin{tabular}{r c l}
        für D,C,B,A & = 0000 & $U_A = 0V$                                   \\
        für D,C,B,A & = 0001 & $U_A = -RN *(0V + 0V + 0V + 0,1mV)$          \\
        für D,C,B,A & = 1111 & $U_A = -RN *(0,8mV + 0,4mV + 0,2mV + 0,1mV)$ \\
                    &        & $U_A = -RN * 1,5mV$
    \end{tabular}
\end{center}

Wenn die maximale Spannung = 5V sein soll, so wählen wir $R_N=3,3k\Omega$

\subsection{Analog-Digital-Umwandlung}

Wir bestimmen das Zeitliche Intervall der Abtastungen, sowie die Diskretisierung der Amplitudenwerte in $2^n$ Stufen.

Genannte auch: Diskretisierung in der Zeit und Quantisierung.

Bei der Quantisierung entstehen Fehlen, dieser wird Auch Quantisierungsfehler genannt:

\begin{center}
    Fehler $e_Q=\pm \frac{\Delta}{2}$, mit $\Delta$ Höhe der Amplitude
\end{center}

\subsubsection{Nyquist-Shannan-Abtasttheorem}

Besagt, dass die die Abtatsfrequenz mindestens doppelt so hoch sein, wie die höchste Freuqnz des Eingangsignals.

Beispiel: Hifi hat 20kHz, somit müssten wir 40kHz Abtasten. Hier ist unser zeitintervall: $t=0,025ms$

\subsubsection{Sägezahnverfahren}

Wir haben einen Zähler, welche startet, wenn wir 0V überschreiten und, stoppen diesen Zähler, nach dem Überschreiten des Signals und geht weiter bis 10V.

Anhand des Zähler $\Delta t$ und den Winkel der Steigung, können wir $U=\tan(\alpha) *  \Delta t$ bestimmen.

Dieses Verfahren ist Langsam, da $U_E$ muss längere Zeit konstant anliegen.

Die Genauigkeit ist abhängig von dem Impulsgenerator und Sägezahngenerator.

Dafür ist dieses Verfahren sehr einfach umzusetzen, da alle Bausteine standardbausteine sind.

\end{document}
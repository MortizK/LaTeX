\documentclass[a4paper]{article}

%\usepackage{url}

%% Math
\usepackage{mathtools}
%% For Mengen like natural numbers
\usepackage{amsfonts}
%% Für spezielle Symbole
\usepackage{amssymb}

%% Images
\usepackage{import}
\usepackage{xifthen}
\usepackage{pdfpages}
%\usepackage{transparent}

%%% Command for simpler images
\newcommand{\incfig}[1]{%
    \def\svgwidth{\columnwidth}
    \import{./fig/}{#1.pdf_tex}
}

%% Links
\usepackage{hyperref}
\hypersetup{
    colorlinks=true,
    linkcolor=black,
    filecolor=magenta,
    urlcolor=cyan
}

%% Formatting
\usepackage{parskip}

\title{Digitaltechnik}
\author{Moritz}
\date{February 19, 2025}

\begin{document}
\maketitle
\tableofcontents

\section{Wandlung Digital $\Leftrightarrow$ Analog}

\subsection{Analog-Digital-Umwandlung}

\subsubsection{Kompensationsverfahren}

Ist ein näherungsverfahren, welches das aktuelle Digitale Signal in das Analoge umwandelt ($U_K$). Dieses $U_K$ wird mit dem Eingangssignal $U_E$ verglichen. Der Spannungsunterschied steuert die Geschwindigkeit des Zählers. Sollte $U_K>U_E$ sein, so zählt der Zähler Rückwerts.

Die Schaltung braucht einen DA-Wandler um einen AD-Wandel zu bauen, ist dafür aber schnell und hat für ein Vergrößerung der Wortbreite einen geringen Aufwand.

\section{Schaltkreistechnik}

\subsection{Diode}

Haben Drei Bereiche:

\begin{center}
    \begin{tabular}{|c|c|c|}
        \hline
        p-dotiert        & Raumladungszone & n-dotiert  \\
        Elektronenlöcher &                 & Elektronen \\
        \hline
    \end{tabular}
\end{center}

Je-nachdem, wie rum wir Strom an eine Diode anschließen, sperren wir diese oder Strom fließt.

Bei Silizium ist der Widerstand von der Diode $U_S=0,7V$, somit fließt Strom, wenn wir eine Spannung $>0,7V$ schalten.

\subsubsection{Gatter bauen mit Dioden}

Wir können ODER und UND-Gatter mit Dioden bauen. Ein ODER Gatter ist Erstmal einfacher zu verstehen:

Hier schaltet eine der Beiden Dioden die 5V durch, und wenn keine Durchschaltet, so gibt es einen PullDown wiederstand, der mit GND verbunden ist.

\subsubsection{Bipolar-Transistor}

Sperren, wenn die Eingangsspannung <= Emitterspannung und schaltet durch, wenn nicht.

Mit diesen lassen sich auch Schaltungen bauen. Diese nennt man TTL Schaltungen.

\subsection{Transistor-Transistor-Logik (TTL)}

Verbrauchen relativ viel Strom, da beim durchschalten von Transistoren viel Strom fließt.

\subsubsection{Familien}

Es gibt verschiedenen Familien: 74xx, 74LSxx, 74ALSxx, 74Fxx, 74ASxx

Die Abkürzungen stehen für: Standard, Low Power Schottky, Advanced LS, Fast, Advanced Schottky

Diese haben zwischen Geschwindigkeiten, Stromverbrauch und einiges mehr abwägen, welche Familie benutzt wird.

\subsection{Feldeffekttransistor}

\subsubsection{MOS-FET}

Es geht ums sperren und leiten. Es gibt aber den Unterschied zwischen selbstsperrenden MOS-FETs als p-Kanal (-) oder n-Kanal (+).

\subsubsection{CMOS}

Hier werden beide Arten von MOS-FETs benutzt. Also sowohl p-Kanal als auch n-Kanal.

Vorteil: Braucht nur Strom um den Zustand zu ändern.

\subsubsection{Familien}

Wieder mit ihren verschiedenen Vor-und Nachteile, von Geschwindigkeit und Spannung.

\end{document}
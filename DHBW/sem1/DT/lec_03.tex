\documentclass[a4paper]{article}

%\usepackage{url}

%% Math
\usepackage{mathtools}
%% For Mengen like natural numbers
\usepackage{amsfonts}
%% Für spezielle Symbole
\usepackage{amssymb}

%% Images
\usepackage{import}
\usepackage{xifthen}
\usepackage{pdfpages}
%\usepackage{transparent}

%%% Command for simpler images
\newcommand{\incfig}[1]{%
    \def\svgwidth{\columnwidth}
    \import{./fig/}{#1.pdf_tex}
}

%% Links
\usepackage{hyperref}
\hypersetup{
    colorlinks=true,
    linkcolor=black,
    filecolor=magenta,
    urlcolor=cyan
}

%% Formatting
\usepackage{parskip}

\title{Digitaltechnik}
\author{Moritz}

\begin{document}
\maketitle
\tableofcontents

\section{Kodierungen}

Haben Verschiedene Maße um eine Kodierung zu bewerten. Hier wiederhole ich nochmal die Redundanz.

\subsection{Redundanz}

Anteil einer Nachricht (die der Einheit Bit), die keine Informationen enthält; lässt sich abschätzen durch $R=\log{N/M}$

N: mögliche Anzahl der mit $n$ Bit kodierbaren Zeichen.

M: tatsächliche Anzahl der im Code verwendeten Zeichen.

\subsection{BCD-Code}

Teilt jede dezimalziffer in eine Tetrade. Diese haben 4 Bit und gehen ganz klassisch dann von $0_{10}=0000_2$ und $9_{10}=1001_2$.

Redundanz: $R=\log_2(\frac{16}{10}=\log_{2}16-\log_{2}10)=4-3,32=0,678$

\subsubsection{Logarithmik}

Beliebige Logarithmen zu einem anderen Basis umformen.

\begin{equation}
    \log_{b}x=\frac{\ln x}{\ln b}=\frac{\log_{10}x}{\log_{10}b}
\end{equation}

\subsection{Aiken-Code}

Hat auch je Dezimalziffer eine Tetrade. Hier ist die Wertigkeit der Bits aber anders.

\begin{equation}
    \begin{array}{c|cccc}
        \text{Wert} & 2 & 4 & 2 & 1 \\
        \hline
        0           & 0 & 0 & 0 & 0 \\
        1           & 0 & 0 & 0 & 1 \\
        2           & 0 & 0 & 1 & 0 \\
        3           & 0 & 0 & 1 & 1 \\
        4           & 0 & 1 & 0 & 0 \\
        5           & 1 & 0 & 1 & 1 \\
        6           & 1 & 1 & 0 & 0 \\
        7           & 1 & 1 & 0 & 1 \\
        8           & 1 & 1 & 1 & 0 \\
        9           & 1 & 1 & 1 & 1 \\
    \end{array}
\end{equation}

\subsection{Gray-Code}

Die Distanz der benachbarten Codewörtern ist gleich 1 und ist somit Stetig.

Um den nächst größeren Gray-Code zu erstellen, Spiegel ich die bisherigen Bits und habe das Original mit einer 0 vorne und die neue Ruflektion mit einer 1 vorne.

Somit: $G_1=0,1$, dann $G_2=?0, ?1 \mid ?1, ?0$. Nun die neuen Stellen füllen ergibt: $G_2=00,01\mid 11,10$.

\subsection{Fehlererkennende Codes}

Wir ergänzen den Dual-Code um ein weiteres Bit, welche angibt, ob eine ungerade Anzahl an Bits 1 ist. Diese Bit heißt Paritätsbit.

Hier ist die Hamming-Distanz (Die minimale Distanz zwischen Beliebigen Werten) $H=2$.

Die Redundanz ist: $\log_{2}\frac{32}{16}=1$

\subsection{Hamming(7,4)-Code}

Ist ein häufig verwendeter fehlerkorrigierender Code.

\end{document}
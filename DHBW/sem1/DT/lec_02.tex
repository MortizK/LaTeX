\documentclass[a4paper]{article}

%\usepackage{url}

%% Math
\usepackage{mathtools}
%% For Mengen like natural numbers
\usepackage{amsfonts}
%% Für spezielle Symbole
\usepackage{amssymb}

%% Images
\usepackage{import}
\usepackage{xifthen}
\usepackage{pdfpages}
%\usepackage{transparent}

%%% Command for simpler images
\newcommand{\incfig}[1]{%
    \def\svgwidth{\columnwidth}
    \import{./fig/}{#1.pdf_tex}
}

%% Links
\usepackage{hyperref}
\hypersetup{
    colorlinks=true,
    linkcolor=black,
    filecolor=magenta,
    urlcolor=cyan
}

%% Formatting
\usepackage{parskip}

\title{Digitaltechnik}
\author{Moritz}

\begin{document}
\maketitle
\tableofcontents

\section{Kapitel 1 | Einführung}

\subsection{Quiz}

Umrechnungen in verschiedene Zahlensysteme und Vor- und Nachteile vom Binärsystem (Digital vs. Analog)

\subsection{Darstellung von Festkommazahlen}

Die Zahlen werden aufgebrochen in einem ganzzahligen Teil und einen gebrochenen Teil.

\begin{equation}
    z=\sum_{i=-k}^{n-k-1}z_i*b^{i}
\end{equation}

Das heißt für eine 8-Bit Zahl mit k=3 gilt: $z=01101,110_2=2^3+2^2+2^0+1^{-1}+1^{-2}$ somit ergibt sich $13,75$

\subsubsection{Umwandeln in Hexadezimal}

\begin{equation}
    \begin{split}
        555:16     & = 34, r=11\cong B                    \\
        34:16      & = 2, r=2\cong 2                      \\
        2:16       & = 0, r=2\cong 2                      \\
        \hline                                            \\
        0,3*16     & =4,8 , \text{ü}=4\cong 4             \\
        0,8*16     & =12,8, \text{ü}=12\cong C            \\
        0,8*16     & =12,8, \text{ü}=12\cong C            \\
        \hline                                            \\
        555,3_{10} & \cong 22B,3CC\dots=22B,3\overline{C}
    \end{split}
\end{equation}

\subsection{Negative Zahlen}

Es gibt 3 Möglichkeiten

\subsubsection{Vorzeichen und Betrag}

Die Höchste Ziffer ist nun das Vorzeichen.

\subsubsection{(b-1)-Komplement}

Hier wird zur Bildung von $-z$ jede der n Ziffern $z_{i}$ gemäß $\overline{u_{i}}=b-1-z_{i}$ komplimentiert.

\begin{equation}
    \begin{split}
        -1234_{10} & =8765_{10,9er-Komplement} \\
        \hline                                 \\
        z          & =1234                     \\
        -z         & =8765                     \\
        \hline
                   & =9999
    \end{split}
\end{equation}

\begin{figure}[ht]
    \centering
    \incfig{zahlengerade}
    \caption{Zahlengerade mit dem (b-1)-Komplement}
    \label{fig:zahlengerade_(b-1)-Komplement}
\end{figure}

Rechnen am Beispiel: $2000-1234\cong2000+8765=1\mid 0765$ mit diesem Ergebnis können wir weiterarbeiten $0765+1=0766$. Dieses Ergebnis ist Richtig.

\subsubsection{b-Komplement}

Hier ist die Summe aus $z$ und $-z$ ergibt $b^{n}$

\begin{figure}[ht]
    \centering
    \incfig{zahlengerade_b-kompliment}
    \caption{Zahlengerade mit dem b-Komplement}
    \label{fig:zahlengerade_b-Komplement}
\end{figure}

\subsection{Negative Binärzahlen}

Bei einer Wortlänge von $n=8$ und einen Basis von $b=2$.

Hier gibt es wieder alle 3 Möglichkeiten.

Das mit dem Vorzeichen und Betrag, das Einerkomplement und das Zweier-komplement.

\subsubsection{Einerkomplement}

$z=00101101_2=45_{10} \implies -z=11010010_2=-45_{10}$

\subsubsection{Zweierkomplement}

$z=00101101_2=45_{10} \implies -z=11010011_2=-45_{10}$

\subsection{Vor- und Nachteile von diesen Darstellung}

Mit dem Vorzeichen und Betrag kann man nicht Rechnen.

Mit den anderen beiden kann man Rechnen.

\subsubsection{Einerkomplement}

Diese hat einen Symmetrischen Zahlenbereich. Deshalb gibt es auch 2 Nullen, die +0 und -0.

Zudem muss zum Rechnen noch ein Korrekturschritt gemacht werden.

Diese Problematik kann man gut in einem Zahlenkreis darstellen. Hier muss man zum subtrahieren von 5, den Zeiger um 6 nach hinten bewegen, statt 5.

\subsubsection{Zweierkomplement}

Dieser Zahlenbereich ist leicht asymmetrisch. Der negative Bereich hat eine Zahl mehr.

\subsection{Aufgaben}

Es gibt ein Aufgabenblatt auf Moodle, welches bis nächste WOche bearbeitet werden soll.

\section{Kapitel 2 | Codes}

\subsection{Lernziele}

Bewertung verschiedener Codes.

Aufbau und EIgenschaften verschiedener Codes

Verstehen von Fehlercodes und Fehlerkorrektioncodes.

\subsection{Kodierung}

Ein Code ist eine Abbildungsvorschrift, die jedem Zeichen einer Urbildmenge ein Zeichen/ Zeichenfolge der Bildenge zuordnet.

\subsubsection{Quellenkodierung}

Umwandlung, ggf. mit Kompression

\subsubsection{Kanalkodierung}

Kodierung für die Übertragung in einem Kanal, kann Fehlererkennung und -korrekur enthalten.

\subsubsection{Leitungskodierung}

Kodierung für den physischen Transport über ein Medium.

Es gibt eine Schöne Grafik auf dem Foliensatz.

Zudem sieht die Manchaster Kodierung so aus und ist Teil der Leitungskodierung.

\begin{figure}[ht]
    \centering
    \incfig{manchaster-kodierung}
    \caption{manchaster-kodierung}
    \label{fig:manchaster-kodierung}
\end{figure}

\subsection{Quellenkodierung}

Hier wird zwischen Numerischen Codes und Alphanumerischen Codes.

\subsubsection{Numerische Codes}

Zur Darstellung von Zahlen

\subsubsection{Alphanumerischen Codes}

Zur Darstellung von Buchstaben, Ziffern und anderen Sonderzeichen. Bekannte Codes umfassen Morsecode und ASCII.

Zudem gibt es den Unicode als UTF-8 und UTF-16

\subsection{Verschiedene Maße}

Wertigkeit: Gibt es eine Stellwert für jede Position.

Distanz: Anzahl der Bitstellen, in den sich zwei Codewörter unterscheiden. Durch die verschiedene Wertigkeiten der Bits, unterscheiden sich die Distanzen zwischen 2 benachbarten Werten um 1, 2, 3, 4, da sich bis zu allen 4 Bits gleichzeitig ändern können.

Hamming-Distanz: das Minimum aller Abstände zwischen Wörtern innerhalb eines Codes

Stetigkeit: Die Distanz zwei benachbarten Codewörtern eines numerischen Codes ist konstant. Durch unter Distanz beschrieben Verhalten, von Binärcodes, sind diese nicht stetig.

% von der Folie entnehmen
Redundanz: von der Folie entnehmen

\subsection{Binäre Codes}

Wir können alle gängigen mathematischen Operationen wie + - * / durchführen.

Aufgrund der festgelegten Wortlänge können durch einen Überlauf Fehler passieren. Da das Ergebnis evtl. eine größere Wortlänge bräuchte.

Bei der Multiplikation, kann das Ergebnis eine Wortlänge bis zu 2n haben, wenn die beiden Zahlen eine Wortlänge von n haben.

\subsubsection{Bewertung}

Wertigkeit: Die Stelle i hat den Stellenwert $2^{i}$

Die Distanz zwischen zwei benachbarten n-Bit Codewörtern ist minimal 1 und maximal n

Der Code ist nicht stetig

Die Hamming-Distanz $H$ ist 1

Die Redundanz $R$ ist 0 Bit

\subsection{BCD-Code}

Jede Ziffer aus dem Dezimalsystem wird in je 4 Bit kodiert. Jede Ziffer ist eine Tetrade.

Bei eine Addition, die in einem Pseudotetrade endet. muss mit diese Tetrade mit eine Binär 6 Addiert werden. Achte auf den Übertrag.

Wertigkeit: Es gibt stellen, den keinen Position zugeordnet ist. Die Pseudotetraden.

Distanz: ist minimal 1 und maximal n

Hamming-Distanz: ist 1

Stetigkeit: ist nicht stetig

Redundanz: hat eine Redundanz $R=\log(\frac{16}{10}=0,678)$

\end{document}
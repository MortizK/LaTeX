\documentclass[a4paper]{article}

%\usepackage{url}

%% Math
\usepackage{mathtools}
%% For Mengen like natural numbers
\usepackage{amsfonts}
%% Für spezielle Symbole
\usepackage{amssymb}

%% Images
\usepackage{import}
\usepackage{xifthen}
\usepackage{pdfpages}
%\usepackage{transparent}

%%% Command for simpler images
\newcommand{\incfig}[1]{%
    \def\svgwidth{\columnwidth}
    \import{./fig/}{#1.pdf_tex}
}

%% Links
\usepackage{hyperref}
\hypersetup{
    colorlinks=true,
    linkcolor=black,
    filecolor=magenta,
    urlcolor=cyan
}

%% Formatting
\usepackage{parskip}

\title{Digitaltechnik}
\author{Moritz}

\begin{document}
\maketitle
\tableofcontents

\section{Kombinatorische Logik}

\subsection{Boolesche Algebra}

Es gibt nur 3 Operatoren: +, * und $\overline{x}$. Übersetzt ODER, UND, NICHT.

\subsubsection{Axiome}

Kommutativität: $A+B=B+A$ und $A*B=B*A$

Distributivität: $A*(B+C)=(A*B)+(A*C)$ und $A+(B*C)=(A+B)*(A+C)$

\begin{equation}
    \begin{array}{ccc|c|c|c|c}
        A & B & C & A+(B*C) & (A+B) & (A+C) & (A+B)*(A+C) \\
        \hline
        0 & 0 & 1 & 0       & 0     & 0     & 0           \\
        0 & 0 & 0 & 0       & 0     & 1     & 0           \\
        0 & 1 & 1 & 0       & 1     & 0     & 0           \\
        0 & 1 & 0 & 1       & 1     & 1     & 1           \\
        1 & 0 & 1 & 1       & 1     & 1     & 1           \\
        1 & 0 & 0 & 1       & 1     & 1     & 1           \\
        1 & 1 & 1 & 1       & 1     & 1     & 1           \\
        1 & 1 & 0 & 1       & 1     & 1     & 1           \\
    \end{array}
\end{equation}

Neutrales Element: $0+A=A$ und $1*A=1$

Komplementäres Element: $A+\overline{A}=1$ und $A*\overline{A}=0$

\subsubsection{Sätze}

Idempotenz: $A+A=0$ und $A*A=A$. Beweis durch Nutzung der Axiome:

\begin{equation}
    A+A=1*A+A*1=A*1+A*1=A*(1+1)=A*1=A
\end{equation}

Assoziativität: $A+(B+C)=(A+B)+C$ und $A*(B*C)=(A*B)*C$

Substitutionsregeln: $A+1=1$ und $A*0=0$. Beweis durch Nutzung der Axiome:

\begin{equation}
    A*0=0+(A*0)=(A*\overline{A})+(A*0)=A*(\overline{A}+0)=A*\overline{A}=0
\end{equation}

Absorptionsgesetze: $A+(A*B)=A$ und $A*(A+B)=A$. Beweis durch Nutzung der Axiome und der Substitutionsregeln:

\begin{equation}
    A*(A+B)=(A+B)*A=(A+B)*(A+0)=A+(B*0)=A+0=A
\end{equation}

Doppelnegation: $\overline{\overline{A}}=A$

Komplementäre Werte: $\overline{0}=1$ und $\overline{1}=0$

Abgeschlossenheit: Boolesche Operationen liefern nur boolesche Werte als Ergebnis.

de Morgansche Regeln: $\overline{A+B}=\overline{A}*\overline{B}$ und $\overline{A*B}=\overline{A}+\overline{B}$

\begin{equation}
    \begin{array}{cc|c|c}
        A & B & \overline{A*B} & \overline{A}+\overline{B} \\
        \hline
        0 & 0 & 1              & 1                         \\
        0 & 1 & 1              & 1                         \\
        1 & 0 & 1              & 1                         \\
        1 & 1 & 0              & 0                         \\
    \end{array}
\end{equation}

Am Beispiel wird, gezeigt, dass um einen Term zu negieren, kann einfach jeder Operator durch sein gegenstück (UND und ODER) tausche und die Komplementierung aller Variablen durchführe.

\begin{equation}
    \begin{split}
        Y            & =A*B+C*\overline{D}                           \\
        \overline{Y} & =\overline{A*B+C*\overline{D}}                \\
        \overline{Y} & =\overline{A*B}*\overline{C*\overline{D}}     \\
        \overline{Y} & =(\overline{A}+\overline{B})*(\overline{C}+D) \\
    \end{split}
\end{equation}

Dualität: Für jede aus den Axiomen ableitbare Aussage gibt es eine duale Aussage, die durch Tausch der Operation + und * sowie durch Tausch der Werte 0 und 1 entsteht.

\subsection{Funktionale Vollständigkeit}

AND, OR, NOT sind vollständig.

OR, NOT ist auch vollständig, da $\overline{\overline{A}\lor\overline{B}}=A\land B$

AND, NOT ist auch vollständig, wie OR, NOT.

NAND ist auch vollständig, da AND = $\overline{\overline{A\land B}\land\overline{A\land B}}$, NOT = $\overline{A\land A}$ und OR = $\overline{\overline{A\land A}\land\overline{B\land B}}$

\subsection{Schaltfunktionen}

Funktionen $f:\{0,1\}^n\to\{0,1\}^m$ mit $n,m\geq 1$ werden auch als Schaltfunktionen bezeichnet.

Eine Schaltfunktion $f:\{0,1\}^n\to\{0,1\}$ heißt eine $n$-stellige Boolesche Funktion

Jede Schaltfunktion $f:\{0,1\}^n\to\{0,1\}^m$ kann durch $m$ Boolesche Funktionen ausgedrückt werden.

Schaltfunktion $f:\{0,1\}^3\to\{0,1\}^2$. $y_2$ ist eine 3-Stellige Boolesche Funktion und $y_1$ ist eine 3-Stellige Boolesche Funktion.

\begin{equation}
    \begin{array}{ccc|cc}
            & n=3 &     &     & m=2 \\
        x_3 & x_2 & x_1 & y_2 & y_1 \\
        \hline
        0   & 0   & 0   & 0   & 1   \\
        0   & 0   & 1   & 1   & 1   \\
        0   & 1   & 0   & 0   & 0   \\
        0   & 1   & 1   & 1   & 0   \\
        1   & 0   & 0   & 0   & 1   \\
        1   & 0   & 1   & 0   & 1   \\
        1   & 1   & 0   & 1   & 0   \\
        1   & 1   & 1   & 0   & 1   \\
    \end{array}
\end{equation}

Es gibt insgesamt $2^{2^n}$ $n$-stellige Boolesche Funktionen.

\end{document}
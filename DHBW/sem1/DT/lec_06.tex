\documentclass[a4paper]{article}

%\usepackage{url}

%% Math
\usepackage{mathtools}
%% For Mengen like natural numbers
\usepackage{amsfonts}
%% Für spezielle Symbole
\usepackage{amssymb}

%% Images
\usepackage{import}
\usepackage{xifthen}
\usepackage{pdfpages}
%\usepackage{transparent}

%%% Command for simpler images
\newcommand{\incfig}[1]{%
    \def\svgwidth{\columnwidth}
    \import{./fig/}{#1.pdf_tex}
}

%% Links
\usepackage{hyperref}
\hypersetup{
    colorlinks=true,
    linkcolor=black,
    filecolor=magenta,
    urlcolor=cyan
}

%% Formatting
\usepackage{parskip}

\title{Digitaltechnik}
\author{Moritz}
\date{January 8, 2025}

\begin{document}
\maketitle
\tableofcontents

\section{Kombinatorische Logik}

\subsection{Ausgezeichnete Terme}

Produktterm: $x, y, \overline{x}*y, x*\overline{y}, x*\overline{y}*z$

Summenterm: $x, y, \overline{x}+y, x+\overline{y}, x+\overline{y}+z$

Minterm: Produktterm, in dem jede Variable einer booleschen Funktion genau einmal vorkommt. $x*y*z$ einer Funktion $f(x,y,z)$

Maxterm: Summenterm, in dem jede Variable einer booleschen Funktion genau einmal vorkommt. $x+y+z$ einer Funktion $f(x,y,z)$

\subsection{Normalformen}

Disjunktive Normalform (DNF): Summe von Produkten wie: $(\overline{x}*y)+(x*\overline{y}*z)$

Konjunktive Normalform (KNF): Produkt von Summen wie: $(\overline{x}+y)*(x+\overline{y}+z)$

Kanonische Disjunktive Normalform (KDNF): Summe von Mintermen wie: $(x*\overline{y}*z)+(x*y*z)$ einer Funktion $f(x,y,z)$

Kanonische Konjunktive Normalform (KKNF): Produkte von Maxtermen wie: $(x+\overline{y}+z)*(x+y+z)$ einer Funktion $f(x,y,z)$

\subsubsection{Satz}

Jede Boolesche Funktion lässt sich als genau eine KDNF (Summe von Mintermen) darstellen. FÜr jede Zeile in der Wahrheitstabelle mit $f(x_1, x_2, \dots, x_n)=1$ wird ein Minterm aufgestellt. Diese Minterme werden alle mit einem ODER verknüpft.

\begin{equation}
    \begin{array}{ccc|c|cc}
        A & B & C & Y & DNF                                    & KNF                                    \\
        \hline
        0 & 0 & 0 & 1 & \overline{A}*\overline{B}*\overline{C} &                                        \\
        0 & 0 & 1 & 1 & \overline{A}*\overline{B}*C            &                                        \\
        0 & 1 & 0 & 1 & \overline{A}*B*\overline{C}            &                                        \\
        0 & 1 & 1 & 0 &                                        & A+\overline{B}+\overline{C}            \\
        1 & 0 & 0 & 1 & A*\overline{B}*\overline{C}            &                                        \\
        1 & 0 & 1 & 0 &                                        & \overline{A}+B+\overline{C}            \\
        1 & 1 & 0 & 0 &                                        & \overline{A}+\overline{B}+C            \\
        1 & 1 & 1 & 0 &                                        & \overline{A}+\overline{B}+\overline{C} \\
    \end{array}
\end{equation}

Beispiel zu KDNF: \begin{equation}
    Y_D=\overline{A}*\overline{B}*\overline{C}+\overline{A}*\overline{B}*C+\overline{A}*B*\overline{C}+A*\overline{B}*\overline{C}
\end{equation}

Beispiel zu KKNF und Umwandlung in das KDNF: \begin{equation}
    \begin{split}
        Y_K =  & (A+\overline{B}+\overline{C})*(\overline{A}+B+\overline{C})*(\overline{A}+\overline{B}+C)*(\overline{A}+\overline{B}+\overline{C})                                      \\
        Y_K  = & (A+\overline{B}+\overline{C})*(\overline{A}+B+\overline{C})*(\overline{A}+\overline{B}+\overline{C})*(\overline{A}+\overline{B}+(C*\overline{C}))                       \\
        0   =  & C*\overline{C}\text{ Diesen Schritt für die drei Terme wiederholen}                                                                                                     \\
        Y_K =  & (\overline{A}+\overline{B})*(\overline{B}+\overline{C})*(\overline{A}+\overline{C})                                                                                     \\
               & \text{Ausmultiplizieren}                                                                                                                                                \\
        Y_K =  & \overline{A}*\overline{B}+\overline{A}*\overline{C}+\overline{B}*\overline{C}+\overline{A}*\overline{B}*\overline{C}                                                    \\
        Y_K =  & \overline{A}*\overline{B}*1+\overline{A}*\overline{C}*1+\overline{B}*\overline{C}*1+\overline{A}*\overline{B}*\overline{C}                                              \\
        Y_K =  & \overline{A}*\overline{B}*(C+\overline{C})+\overline{A}*(B+\overline{B})*\overline{C}+(A+\overline{A})*\overline{B}*\overline{C}+\overline{A}*\overline{B}*\overline{C} \\
               & \text{Ausmultiplizieren und Duplikate entfernen}                                                                                                                        \\
        Y_K =  & \overline{A}*\overline{B}*\overline{C}+\overline{A}*\overline{B}*C+\overline{A}*B*\overline{C}+A*\overline{B}*\overline{C} = Y_D
    \end{split}
\end{equation}

\subsubsection{Umsetzung}

Eine Realisierung einer Booleschen Funktion $f$ kann erzeugt werden durch:

1. Aufstellung der Wahrheitstabelle von $f$

2. Bilder der KDNF (oder KKNF) von $f$

3. Realisierung der KDNF (oder KKNF) mit Gattern

Eine KDNF kann günstiger sein als eine KKNF, wenn die Anzahl der Notwendigen Minterme geringer ist als die Anzahl der Notwendigen Maxterme - Es gibt weniger 1 als 0.

\begin{equation}
    \begin{array}{cc|c}
        x_1 & x_2 & f(x_1,x_2) \\
        \hline
        0   & 0   & 1          \\
        0   & 1   & 1          \\
        1   & 0   & 0          \\
        1   & 1   & 1          \\
    \end{array}
\end{equation}

Hier ist eine KKNF günstiger, da weniger Terme genutzt werden müssen.

\end{document}
\documentclass[a4paper]{article}

%\usepackage{url}

%% Math
\usepackage{mathtools}
%% For Mengen like natural numbers
\usepackage{amsfonts}
%% Für spezielle Symbole
\usepackage{amssymb}

%% Images
\usepackage{import}
\usepackage{xifthen}
\usepackage{pdfpages}
%\usepackage{transparent}

%%% Command for simpler images
\newcommand{\incfig}[1]{%
    \def\svgwidth{\columnwidth}
    \import{./fig/}{#1.pdf_tex}
}

%% Links
\usepackage{hyperref}
\hypersetup{
    colorlinks=true,
    linkcolor=black,
    filecolor=magenta,
    urlcolor=cyan
}

%% Formatting
\usepackage{parskip}

\title{Digitaltechnik}
\author{Moritz}
\date{February 24, 2025}

\begin{document}
\maketitle
\tableofcontents

\section{Spickzettel}

Eine DIN-A4 Seite

\begin{enumerate}
    \item KV $4\times 4$ Wertetabelle
    \item Boolesche Algebra \href{https://elearning.dhbw-stuttgart.de/moodle/pluginfile.php/820065/mod_resource/content/1/DT3logic.pdf#page=13}{Gesetze}
    \item Kodierungen (Gray-Code, usw.)
    \item DNF: starten mit $X$; KNF: starten mit $\overline{X}$ und dan De-Morgan
    \item Verschieden Typen an Flip-Flops
    \item Moore und Mealy Automat
    \item Speicher (SRAM, DRAM, usw.)
    \item EEPROM und Organisation: Aufteilung des Adressraum,
\end{enumerate}



\end{document}
\documentclass[a4paper]{article}

%\usepackage{url}

%% Math
\usepackage{mathtools}
%% For Mengen like natural numbers
\usepackage{amsfonts}
%% Für spezielle Symbole
\usepackage{amssymb}

%% Images
\usepackage{import}
\usepackage{xifthen}
\usepackage{pdfpages}
%\usepackage{transparent}

%%% Command for simpler images
\newcommand{\incfig}[1]{%
    \def\svgwidth{\columnwidth}
    \import{./fig/}{#1.pdf_tex}
}

%% Links
\usepackage{hyperref}
\hypersetup{
    colorlinks=true,
    linkcolor=black,
    filecolor=magenta,
    urlcolor=cyan
}

%% Formatting
\usepackage{parskip}

\title{Forschungsprojekt}
\author{Moritz}
\date{January 9, 2025}

\begin{document}
\maketitle
\tableofcontents

\section{Aufgaben der Hochschulen}

\subsection{Forschung}

\subsection{Lehre}

\subsection{Third Mission}

Technologietransfer

Weiterbildung

Soziales und Regionales Engagement

\section{Wissenschaftliche Forschung}

Definition auf S.15 der Einführung

\section{Forschungsmethoden}

Auf S. 19 ist eine Schöne Grafik

\subsection{Quantitative}

Bestehende Hypothese prüfen mit Deduktion

\subsection{Qualitative}

Neues Schaffen

Ein Phänomen, welches umfassend Analysiert wird mit Induktion.

\section{Forschungsprojekt}

Ablaufgrafik auf S. 25

Startet in ca. 2 Terminen und wird erst so richtig im 2 Semester bearbeitet.

\subsection{Hypothese}

Sollte formuliert werden und in der praxis in einem online-termin mit Dozenten besprechen.

\subsection{Probanden}

Zielgruppe wählen: z.B. Die anderen Gruppen oder Kurse. Oder Sportgruppen, Vereine, usw.

Zeit ist kostbar für die Probanden.

\section{Forschungsfrage}

S. 30

Ich muss eine Theorie (Fachbereich) kennen

Bearbeiten ein Lücke in der bestehenden Theorie oder Brücken zwischen Theorien bilden.

Sie kann beschreiben, bewerten, vorhersagen und erklären.

Die Frage muss klar beschrieben werden. Umsetzung muss quasi direkt ableitbar sein. Und muss neu, relevant und nicht trivial sein.

\end{document}
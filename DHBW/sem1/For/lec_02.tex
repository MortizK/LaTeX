\documentclass[a4paper]{article}

%\usepackage{url}

%% Math
\usepackage{mathtools}
%% For Mengen like natural numbers
\usepackage{amsfonts}
%% Für spezielle Symbole
\usepackage{amssymb}

%% Images
\usepackage{import}
\usepackage{xifthen}
\usepackage{pdfpages}
%\usepackage{transparent}

%%% Command for simpler images
\newcommand{\incfig}[1]{%
    \def\svgwidth{\columnwidth}
    \import{./fig/}{#1.pdf_tex}
}

%% Links
\usepackage{hyperref}
\hypersetup{
    colorlinks=true,
    linkcolor=black,
    filecolor=magenta,
    urlcolor=cyan
}

%% Formatting
\usepackage{parskip}

\title{Forschungsprojekt}
\author{Moritz}
\date{January 14, 2025}

\begin{document}
\maketitle
\tableofcontents

\section{Forschungsfrage}

Wiederholung vom letztem Termin

Wenn man sich mit einem Thema beschäftigt, kommen meist mehr Fragen auf, als ich eigentlich beantworten sollte. Diese kann ich in einem Ausblick dokumentieren.

\section{Praktische Aufgabe}

Was brauche ich für die Wissenschaftliche Arbeit meiner Aufgaben im Unternehmen?

\subsection{Vorher}

Erhebung, Fallstudie, Action Research

1. Problem analysieren

2. Anforderungen ableiten

\subsection{Implementieren}

Programmieren eine Software

\subsection{Nachher}

Experiment, Fallstudie, Action Research

1. Lösung vorstellen

2. Lösung evaluieren

Das Ergebnis muss ausgewertet werden!

\section{Forschungsinteressen}

\subsection{y-basiertes Design}

Von einer Ursache zu einem System. Welche Ursachen haben dieses System

\subsection{x-basiertes Design}

Ich suche nach den Folgen/ Wirkung meines Systems.

\section{Messen von Tests}

\subsection{Objektivität}

Durchführung, unterschiedliche Testleiter kommen zu gleichen Ergebnis

Auswertung, verschiedene Anwender kommen zu gleichen Ergebnis

Interpretation, aus mehreren Ergebnisse die gleichen Schlüsse ziehen

\subsection{Reliabilität}

Mehrere Fehler werden sich schon ausgleichen um auf das Echte Mittel zu kommen.

\subsection{Validität}

In wie weit testet dieser Test was er testen soll? S. 48

\section{Datenerhebung}

\subsection{Die Befragung}

Fragebögen, Interview, Schriftlich oder Mündlich

\subsection{Soziale Erwünschtheit}

Lösungsansätze:

Was denken Sie, denken die anderen in deiner Umgebung über Thema X.

Antwortmodus: 3 Möglichkeiten

Durch zeitstress drei Harmlose Fragen stellen und 4te relevante Frage stellen.

\subsection{Fragebogengestaltung}

Ratingskalen

Zitieren von Rohrmann (1978) um die Statistik von Wörtern auf Zahlen auszuwerten.

Häufigkeiten: nie, selten, gelegentlich, oft, immer

\section{Beobachtung}

Offen oder Verdeckt,

Eingreifen oder Nicht



\end{document}
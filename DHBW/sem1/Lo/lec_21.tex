\documentclass[a4paper]{article}

%\usepackage{url}

%% Math
\usepackage{mathtools}
%% For Mengen like natural numbers
\usepackage{amsfonts}
%% Für spezielle Symbole
\usepackage{amssymb}

%% Images
\usepackage{import}
\usepackage{xifthen}
\usepackage{pdfpages}
%\usepackage{transparent}

%%% Command for simpler images
\newcommand{\incfig}[1]{%
    \def\svgwidth{\columnwidth}
    \import{./fig/}{#1.pdf_tex}
}

%% Links
\usepackage{hyperref}
\hypersetup{
    colorlinks=true,
    linkcolor=black,
    filecolor=magenta,
    urlcolor=cyan
}

%% Formatting
\usepackage{parskip}

\title{Logik}
\author{Moritz}
\date{February 26, 2025}

\begin{document}
\maketitle
\tableofcontents

\section{Prädikatenlogik}

\subsection{Resolutionskalkül}

Fast nur Übungen Gemacht, sind in Heft 2 unter Lo 26.02 Resolution.

Beispiel aus seinem \href{https://wwwlehre.dhbw-stuttgart.de/~sschulz/TEACHING/LGLI2024/Logic.pdf#page=813}{Skriptum}

Resolution:

\begin{tabular}{l c l}
    $\neg hat(john, Y)\lor\neg k(Y)\lor\neg m(maus)$ & 2+6  & $\{X\leftarrow john, Z\leftarrow maus\}$ \\
    $\neg hat(john, Y)\lor\neg k(Y)$                 & 9+8  & $\{\}$                                   \\
    $\neg k(tier)$                                   & 10+4 & $\{Y\leftarrow tier\}$                   \\
    $h(tier)$                                        & 11+7 & $\{\}$                                   \\
    $l(tier)$                                        & 12+1 & $\{X\leftarrow tier\}$                   \\
    $\neg e(X)\lor\neg hat(X, tier)$                 & 13+3 & $\{Y\leftarrow tier\}$                   \\
    $\neg e(john)$                                   & 14+4 & $\{X\leftarrow john\}$                   \\
    $\Box$                                           & 15+5 & $\{\}$                                   \\
\end{tabular}

\subsubsection{Resolution und Induktion}

\begin{equation*}
    p(0)\land\forall X(p(X)\to p(s(X))) \models p(s(s(0)))
\end{equation*}

Mit der Negiertem Erwartung $\neg p(s(s(0)))$ und den Klauseln aus dem vorderem Teil lässt sich Resolution durchführen.

Diese s-Türme lassen sich weiterführen und wir beweisen Induktion.

\section{Abschluss}

Die Zusammenfassung an Themen sind im Skriptum ab S. 562

\begin{enumerate}
    \item Mengenlehre
    \item Scheme: Unterschied cons und append
    \item Aussagenlogik
    \item Prädikatenlogik: Unifikation und Resolution
\end{enumerate}

\end{document}
\documentclass[a4paper]{article}

%\usepackage{url}

%% Math
\usepackage{mathtools}
%% For Mengen like natural numbers
\usepackage{amsfonts}
%% Für spezielle Symbole
\usepackage{amssymb}

%% Images
\usepackage{import}
\usepackage{xifthen}
\usepackage{pdfpages}
%\usepackage{transparent}

%%% Command for simpler images
\newcommand{\incfig}[1]{%
    \def\svgwidth{\columnwidth}
    \import{./fig/}{#1.pdf_tex}
}

%% Links
\usepackage{hyperref}
\hypersetup{
    colorlinks=true,
    linkcolor=black,
    filecolor=magenta,
    urlcolor=cyan
}

%% Formatting
\usepackage{parskip}

\title{Logik}
\author{Moritz}

\begin{document}
\maketitle
\tableofcontents

\section{Ziele}

Mengenlehre

\section{Definitionen}

\subsection{Definition}

Eine Definition ist einer genaue Beschreibung eines Objektes oder Konzepts.

Können einfach oder Komplex sein. Sie müssen Präzise sein.

Eine Definition versucht, ein "reales" Konzept formal zu beschreiben. Hier steckt oft die intuition fürs Verständnis hinter.

Die Definition darf aber keine Intuition enthalten

\subsection{Beweis}

Ein Beweis ist ein Argument, das einen verständigen (Empfänger muss folgen können) und unvoreingenommen Empfänge von der unbestreitbaren Wahrheit einer Aussage überzeugt.

Oft mindestens semi-formal.

Aussage ist fast immer ein Konditional (d.h. eine bedingte Aussage)

Wir benutzen Annahmen, wie z.B. was macht + oder was sind $\mathbb{N}$ Natürliche Zahlen.

Wir brauchen eine 100\% Sicherheit, sonst hilft uns dieser Beweis nicht so viel.

\section{Mengenlehre}

\subsection{Definition (Menge)}

Eine Menge ist eine Sammlung von Objekten, betrachtet als Einheit.

Die Objekte heißen auch Elemente der Menge.

Menge können auch Menge und Funktionen enthalten. Ein Objekt oder Element kann alles sein, auch Kurse an der DHBW \dots

Es gibt eine universelle Trägermenge, wo alles drin ist, was drin sein kann. Um das Russelsche Paradox aus dem Weg zu gehen.

\subsection{Notationen}

\subsubsection{Explezite Aufzählung}

\begin{equation}
    \begin{split}
        A={2, 3, 5, 7, 11, 13} \\
        \mathbb{N}={0,1,2,3,\dots}
    \end{split}
\end{equation}

\subsubsection{Beschreibung}

\begin{equation}
    A={x\mid x \text{ ist Primzahl und } x<13}
\end{equation}

\subsection{Mengenzugehörigkeit}

\begin{equation}
    2\in A
\end{equation}

\subsection{Eigenschaften}

Mengen sind ungeordnet. Sie sind also KEINE Listen.

Jedes Element kommt maximal einmal vor. Wenn wir also 1 aus ${1,1,1}$ entfernen ist diese Menge nun ${}$.

\subsection{Teilmengen und Mengengleichheit}

Wenn $M_1=M_2$, dann sind die Mengen Mengengleich.
% Hier nochmal checken, ob diese dann dieselbe Menge sind oder ob diese nur den gleichen Inhalt haben

Wenn es eine Teilmenge Prüfen wollen, können wir auch in beide Richtungen Prüfen, ob Sie jeweils eine Teilmenge der anderen sind.

\subsection{Echte Teilmengen}

Es gibt auch Obermengen

\subsection{Wichtige Menge}

Leere Menge $\{\}$ oder $\emptyset$. Sie ist auch eine Teilmenge von jeder Menge $\emptyset\subset M$.

$\mathbb{N}=\{0,1,2,3,\dots\}$ Die natürlichen Zahlen

$\mathbb{N}^+=\{1,2,3,\dots\}$ Die positiven natürlichen Zahlen

$\mathbb{Z}=\{\dots ,-2, -1, 0,1,2,\dots\}$ Die ganzen Zahlen

$\mathbb{Q}=\{\frac{p}{q}\mid p\in \mathbb{Z}, q\in\mathbb{N}^+\}$ Die rationalen Zahlen

$\mathbb{R}$ Die reellen Zahlen

\subsection{Mengenbeschreibung}

Alle geraden Zahlen $=\left\{x\mid \frac{x}{2}\in\mathbb{Z}\right\}=\{2x\mid x\in\mathbb{Z}\}$

Alle quadratzahlen $=\{x*x\mid x\in\mathbb{N}\}$

Alle Primzahlen $=\{x\mid x\neq 1\land\forall_{2<y<x}\frac{x}{y}\notin\mathbb{N}\}=\{x\mid x\in\mathbb{N}, x>1, \text{für alle }y, z\in\mathbb{N}\text{ gilt: Wenn }y*z=x\text{, dann }y=1\text{ oder }z=1\}$

\section{Mathematik: Die Grundlagenkrise}

Bertrand Russell und Alfred North Whitehead haben einen Boden für die Mathematik gebaut, sodass wir z.B. $\mathbb{N}$ Natürliche Zahlen haben.

\subsection{Grundlagen von $\mathbb{N}$}

Sie entstehen alle aus der leeren Menge $\emptyset$.

So ist $\emptyset$ entspricht $0$.

So ist $\{\emptyset\}$ entspricht $1$.

So ist $\{\{\emptyset\}\}$ entspricht $2$.

\subsection{Addition in Termalgebra}

Durch ein regelsystem

\begin{equation}
    \begin{split}
        a(x,0)    & =x          \\
        a(x, s(y) & =s(a(x,y)))
    \end{split}
\end{equation}

Durch wiederholte Anwendung dieser Regeln, kann man Addieren.

Hier wird die Addition von 1 und 2 ausgeführt und ergibt 3, durch Mechanik.

\begin{equation}
    a(s(0),s(s(0)))=s(s(s(0)))
\end{equation}


\subsection{Multiplikation in Termalgebra}

Als Aufgabe

\begin{equation}
    \begin{split}
        m(0,y)    & =0            \\
        m(s(x),y) & =a(ma(x,y),x)
    \end{split}
\end{equation}

Wird nicht als gesamt-ergebnis immer Null zurückgeben, da $m(0,y)$ als ganz kleine Schachtel in der Rekursion zurückgegeben wird.

\subsection{Kommentare zur Zahlenkonstruktion}

Die Rechnungen sind alle rein mechanisch. Und wir verwenden nur die einfachste Definition auf Basis des Mengenbegriffes, der leeren Menge $\emptyset$.

Die Konstruktion der Natürlichen Zahlen ist induktiv.

Im ersten Fall wird direkt eine Lösung zurückgegeben, der zweite ist rekursiv.

\begin{figure}[ht]
    \centering
    \incfig{rekursion_bsp}
    \caption{Rekursion Beispiel - Füllfunktion von Paint}
    \label{fig:rekursion_bsp}
\end{figure}

Die Füllfunktion geht Rekursiv vor. Überprüft, ist dieser Pixel schon gefüllt, wenn nicht: Füllen, sonst fertig. Danach Gehe die 4 Nachbarn ab.

\section{Venn-Diagramme}

Sind Kreise oder Menge, welche Schnittmenge haben. z.B. $\mathbb{N}$ und alle Primezahlen.

\subsection{Mengenoperatoren}

Vereinigung - beide, addition

Schnittmenge - beide, zusammen

Differenz

Symmetrische Differenz - entweder Oder

Kompliment - alles bis auf eine Menge

\subsection{Aufgaben}

Sei $T=\{1,2,\dots,12\},M_1=\{1,2,3,4,5,6,7,8\}, M_2=\{2,4,6,8,10,12\}$.

Berechnen und visualisieren Sie diese.

\begin{figure}[ht]
    \centering
    \incfig{venn_diagramm}
    \caption{Venn Diagramm}
    \label{fig:venn_diagramm}
\end{figure}

\begin{equation}
    \begin{split}
        M_1\cup M_2           & =\{1,2,3,4,5,6,7,8,10,12\} \\
        M_1\cap M_2           & =\{2,4,6,8\}               \\
        M_1/ M_2              & =\{1,3,5,7\}               \\
        M_1\bigtriangleup M_2 & =\{1,3,5,7,10,12\}         \\
        \overline{M_1}        & =\{9,10,11,12\}            \\
        \overline{M_2}        & =\{1,3,5,7,11\}
    \end{split}
\end{equation}

\section{Hausaufgaben in Termalgebra}

Die Multiplikation und Addition für Negative Zahlen mechanisch zu machen.

Auf S.470 im Skript

\subsection{Differenz}

Meine Lösung gilt nur dann, wenn $x>y$ ist. na steht für negative addition. Mir sit nichts besseres eingefallen.

\begin{equation}
    \begin{split}
        na(x,0)       & =0          \\
        na(s(x),s(y)) & =s(na(x,y))
    \end{split}
\end{equation}

Eine allgemeine Differenz könnte sonst auch so aussehen.

\begin{equation}
    \begin{split}
        d(x,0)       & =0         \\
        d(0,x)       & =0         \\
        d(s(x),s(y)) & =s(d(x,y))
    \end{split}
\end{equation}

\subsection{Negative Zahlen}

Wir definieren einen neuen Operator $n()$, welche eine Zahl aus Leeren Menge beinhaltet. Hier fehlt dann noch die Definition $n(0)=0$, somit sind die Nullen $-0=0$ identisch.

Mit dieser Regelungen können wir Addition auf Negative Zahlen erweitern und stellen damit auch eine Subtraktion dar. Mit $a(x,n(y))$.

\subsubsection{Addition mit Negativen Zahlen}

Die Reihenfolge ist hier wichtig, somit wird sich zuerst um negative Zahlen gekümmert.

\begin{equation}
    \begin{split}
        n(0)            & =0         \\
        a(x,0)          & =x         \\
        a(0,y)          & =y         \\
        a(n(x),n(y))    & =n(a(x,y)) \\
        a(n(x),y)       & =a(y,n(x)) \\
        a(s(x),n(s(y))) & =a(x,n(y)) \\
        a(x,s(y))       & =s(a(x,y))
    \end{split}
\end{equation}

\subsubsection{Multiplikation mit Negativen Zahlen}

Die Reihenfolge ist hier auch wichtig, aus dem selbem Grund. Zudem ist die Variante, wenn $m(n(x),y)$, redundant, da diese von der Addition mit Negativen Zahlen gelöst werden kann ohne eine Zahl zu Produzieren die $n(n())$ eine doppelte Negierung hat.

\begin{equation}
    \begin{split}
        m(x,0)       & =0           \\
        m(n(x),n(y)) & =m(x,y)      \\
        m(x,n(y))    & =n(m(x,y))   \\
        m(x,s(y))    & =a(x,m(x,y))
    \end{split}
\end{equation}


\subsection{Division}

Das Ergebnis dieser Funktion is immer Aufgerundet. Das heißt eine division von $5/2=2$ und $2/2=1$.

Eine Erweiterung auf negative Zahlen sieht ähnlich aus wie bei der Multiplikation. Hier müssen nur alle drei Möglichkeiten für Negative Zahlen ihre Bedingung bekommen.

\begin{equation}
    \begin{split}
        d(x,0)       & =e                 \\
        d(0,y)       & =0                 \\
        s(d(n(x),y)) & =0                 \\
        d(x,y)       & =s(d(a(x,n(y)),y))
    \end{split}
\end{equation}

\end{document}
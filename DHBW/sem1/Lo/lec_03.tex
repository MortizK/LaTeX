\documentclass[a4paper]{article}

%\usepackage{url}

%% Math
\usepackage{mathtools}
%% For Mengen like natural numbers
\usepackage{amsfonts}
%% Für spezielle Symbole
\usepackage{amssymb}

%% Images
\usepackage{import}
\usepackage{xifthen}
\usepackage{pdfpages}
%\usepackage{transparent}

%%% Command for simpler images
\newcommand{\incfig}[1]{%
    \def\svgwidth{\columnwidth}
    \import{./fig/}{#1.pdf_tex}
}

%% Links
\usepackage{hyperref}
\hypersetup{
    colorlinks=true,
    linkcolor=black,
    filecolor=magenta,
    urlcolor=cyan
}

%% Formatting
\usepackage{parskip}

\title{Logik}
\author{Moritz}

\begin{document}
\maketitle
\tableofcontents

\section{Termalgebra Ergänzung}

\subsection{Definitionen}

Ergänzung auf Negative Zahlen. $p(x)$ ist der vorgänger von $x$ und $n(x)$ ist die negative Zahl von $x$

\begin{equation}
    \begin{split}
        p(x)       & :=x-1 \\
        p(0)       & =-1   \\
        p(p(0))    & =-2   \\
        p(p(p(0))) & =-3   \\
        n(0)       & =0    \\
        n(x)       & :=-x  \\
        n(n(x))    & =x
    \end{split}
\end{equation}

\subsection{Regeln}

Ergänzende Regeln um die Terme zu kürzen.

\begin{equation}
    \begin{split}
        s(p(x)) & =x       \\
        p(s(x)) & =x       \\
        n(s(x)) & =p(n(x)) \\
        n(p(x)) & =s(n(x))
    \end{split}
\end{equation}

\subsection{Subtraktion}

\begin{equation}
    \begin{split}
        a(x,p(x))=p(a(x,y)) \\
    \end{split}
\end{equation}

\subsection{Division}

Ist eine unvollständige Funktion, da nicht durch 0 dividiert werden kann.

\section{Übung Mengenoperatoren}

$T=\mathbb{N}, M_1=\{3i\mid i\in \mathbb{N}\}, M_2=\{2i+1\mid i\in \mathbb{N}\}$

Eine Darstellung auf einem Zahlenstrahl hilft hierbei. Es zeigt sich ein Muster, welches sich alle 6 Werte wiederholt.

Ich kann auch die jeweiligen Mengen, kann auch umgangssprachlich formuliert werden, um die Einfachheit zu erkennen.

\begin{equation}
    \begin{split}
        M_1\cup M_2                 & =\{6i+j\mid i\in \mathbb{N}, j\in\{0,1,3,5\}\} \\
        M_1\cap M_2                 & =\{6i+3\mid i\in \mathbb{N}\}                  \\
        M_1\backslash M_2           & =\{6i\mid i\in \mathbb{N}\}                    \\
        M_1\backslash\overline{M_2} & =M_1\cap M_2=\{3(2i+1)\mid i\in \mathbb{N}\}   \\
        M_1\bigtriangleup M_2       & =\{6i+j\mid i\in \mathbb{N}, j\in \{0,1,5\}\}  \\
    \end{split}
\end{equation}

\section{Kartesische Produkt}

\begin{equation}
    \begin{split}
        M_1\times M_2=\{(x,y)\mid x\in M_1, y\in M_2\} \\
        M_1\times M_2\dots\times M_n={(x_1,x_2\dots,x_n)\mid x_i\in M_i}
    \end{split}
\end{equation}

\subsection{Tuple}

In einem Tuple ist die Reihenfolge wichtig.

\subsection{n-Tuple über einer Menge}

Sei $M$ eine beliebe Menge. Dann ist $M^n=M\times\dots\times M$, die Multiplikation wird n-mal wiederholt. Es wird auch n-Tuple über M.

Spezialfall: $n=1: M^1=\{(x)\mid x\in M\}$ wird oft auch $M$ genannt, obwohl diese verschieden sind, da: $\{0\}$ und $\{(0)\}$ unterschiedlich sind.

Spezialfall: $n=0: M^0=\{()\}$. Diese enthält das leere Tuple.

\subsection{Potenzmenge}

\begin{equation}
    \begin{split}
        M_1=\{1,2\}                              \\
        2^{M_1}=\{\emptyset,\{1\},\{2\}\{1,2\}\} \\
    \end{split}
\end{equation}

\section{Mengenalgebra}

Algebraische Strukturen $(\mathbb{Z},+)$ ist eine Gruppe. Es gibt eine neutrales Element und eine Inverses Element.

$(\mathbb{Z},+,*)$ ist ein Ring. Hier gibt es auch ein Neutrales Element, aber kein Inverses Element.

$(\{0,s(0),s(s(0)),\dots,\},a,m)$ ist eine Termalgebra.

\subsection{Algebraische Regeln}

Es gelten: Kommutativgesetze, Neutrale Elemente, Absorption, Assoziativgesetz, Distributivgesetz, Komplementäre Elemente, Idempotenz, Gesetze von De-Morgan, Doppelte Komplementbildung.

Das Distributivgesetz wurde an der Tafel Grafisch dargestellt, warum das funktioniert.

\section{Relation}

Seien $M_1,M_2,\dots M_n$ Mengen. Eine (n-stellige) Relation $R$ über $M_1,M_2,\dots M_n$ ist eine Teilmenge des kartesischen Produktes der Mengen, also $R\subseteq M_1\times M_2\times\dots\times M_n$ (oder äquivalent $R\in M^{M_1\times M_2\times\dots\times M_n}$)

\subsection{Homogene Relation}

$R$ heißt homogen, falls $M_i=M_j$ für alle $i,j\in\{1,\dots,n\}$

\subsection{Binäre Relation}

$R$ heißt binär, falls $n=2$

Beide Arten können auch kombiniert werden

\subsection{Beispiele}

$=$ über $\mathbb{N}$ ist also $\{(0,0),(1,1),(2,2),\dots\}$

Und weitere auf der Folie.

\subsection{Eigenschaften}

Linkstotal und Rechtseindeutig

Linkstotal heißt: Jedes erste Element hat mindestens eine Relation.

Rechtseindeutig: Jedes erste Element hat genau eins oder keine Relation.

\subsubsection{Äquivalenzrelation}

Reflexiv, symmetrisch, transitiv, Äquivalenzrelation heißt alle drei

Zug fahren ist meistens eine Äquivalenzrelation. Du kannst bleibe wo du bist, du kannst über drei Ecken einen Ort erreichen und meist auch wieder zurück.

\subsubsection{Übung}


Alle Äquivalenzrelation sind Rechtseindeutig - Nein, da wir ein $a=\{a,b\}$ und aus dieser eine Relation $R=\{(a,a),(b,b),(a,b),(b,a)\}$ bilden können. Somit haben wir ein Gegenbeispiel gefunden.

Alle Äquivalenzrelation sind Linkstotal - Ja, denn alle Äquivalenzrelation sind Reflexiv und somit ist jedem a ein a zugeordnet: $\{(a,a)\}$.

\end{document}
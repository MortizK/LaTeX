\documentclass[a4paper]{article}

%\usepackage{url}

%% Math
\usepackage{mathtools}
%% For Mengen like natural numbers
\usepackage{amsfonts}
%% Für spezielle Symbole
\usepackage{amssymb}

%% Images
\usepackage{import}
\usepackage{xifthen}
\usepackage{pdfpages}
%\usepackage{transparent}

%%% Command for simpler images
\newcommand{\incfig}[1]{%
    \def\svgwidth{\columnwidth}
    \import{./fig/}{#1.pdf_tex}
}

%% Links
\usepackage{hyperref}
\hypersetup{
    colorlinks=true,
    linkcolor=black,
    filecolor=magenta,
    urlcolor=cyan
}

%% Formatting
\usepackage{parskip}

\title{Logik}
\author{Moritz}
\date{February 19, 2025}

\begin{document}
\maketitle
\tableofcontents

\section{Prädikatenlogik}

\subsection{Übung}

Formalisierung Prädikat der Primzahl

\begin{center}
    $\forall x(prim(x)\iff x>1 \land \forall yz(mult(y, z)=x) \implies (y = 1\lor z = 1))$
\end{center}

Aussage: Zu jeder Primzahl gibt es eine größere Primzahl

\begin{center}
    $\forall x(prim(x)\implies \exists y(prim(y)\land y > x))$
\end{center}

Modelle:

\begin{enumerate}
    \item $U=\mathbb{N}$
    \item $I(prim)=\{(p)\mid p$ ist Primzahl$\}$
    \item $I(mult)=(x,y)\to x*y$
    \item $I(1)=1$
    \item $I(>)=>$
    \item $I(=)==$
\end{enumerate}

Das nicht-Modell: $I(>)=<$

\subsection{Grund- und geschlossene Formeln}

Wir haben Funktionen, die die Menge der Variablen zurückgibt: $Vars(t)$ und auch eine die nur die freien Variablen zurückgibt: $FVras(F)$, dies sind die, die nicht von Quantoren gebunden sind.

Eine Geschlossenen Formeln ist eine, wo $Vars(t)=\{\emptyset\}$ entspricht.

\subsection{Resolution}

\subsubsection{Quantoren Regeln}

Quantoren gleicher Art sind Kommutativ. $\forall x\forall y = \forall y\forall x$

Dualität gilt für beide: $\forall x \dots= \neg \exists x\neg \dots$

Distribution von $\forall$: $\forall x(\dots \land \dots)=\forall x \dots \land \forall x \dots$

Distribution von $\exists$: $\exists x(\dots \lor \dots)=\exists x \dots \lor \exists x \dots$

\subsubsection{Negationsnormalform (NNF)}

Erlaube nur $\neg, \land, \lor, \exists, \forall$ und $\neg$ nur direkt vor dem Atom.

Dies kann über De-Morgan erreicht werden.

\subsubsection{Variable Normierte Formel}

Ist eine Form, wo jeder Quantor seine eigenen Variable bindet.

\subsubsection{Prenex-Normalform (PNF)}

Alle Quantoren werden quasi ausgeklammert. Somit haben wir einen Prefix aus Quantoren mit unterschiedlichen Variablen, sowie der MAtrix $G$, welche die Formel enthält.

\subsubsection{Skolemisierung: Elemination von $\exists$}

Wir erweitern, die Interpretation durch die $sk_0$-Werte (Skolem Werte), welche den Wert haben, den der Existensquantor gefunden hätte.

Für Existensquantoren, die nicht an erster Stelle stehen, bauen wir Skolem-Funktionen. $sk_1(x)$

\subsubsection{Skolem-Normalform (SNF)}

Ist ein PNF ohne Existensquantoren. Durch diese Umformung sind die Formeln nicht mehr equivalent, aber noch erfüllbarequivalent!

Siehe Skolemisierung.

\end{document}
\documentclass[a4paper]{article}

%\usepackage{url}

%% Math
\usepackage{mathtools}
%% For Mengen like natural numbers
\usepackage{amsfonts}
%% Für spezielle Symbole
\usepackage{amssymb}

%% Images
\usepackage{import}
\usepackage{xifthen}
\usepackage{pdfpages}
%\usepackage{transparent}

%%% Command for simpler images
\newcommand{\incfig}[1]{%
    \def\svgwidth{\columnwidth}
    \import{./fig/}{#1.pdf_tex}
}

%% Links
\usepackage{hyperref}
\hypersetup{
    colorlinks=true,
    linkcolor=black,
    filecolor=magenta,
    urlcolor=cyan
}

%% Formatting
\usepackage{parskip}

\title{Logik}
\author{Moritz}
\date{December 12, 2024}

\begin{document}
\maketitle
\tableofcontents

\section{Funktionen}

Seien M, N Mengen. Eine (totale) Funktion $f:M\to N$ ist eine Relation $f\subseteq(M\times N)$ , die linkstotal und rechtseindeutig ist.

Eine partielle Funktion $f: M\to N$ ist eine Relation $f\subseteq(M\times N)$, die rechtseindeutig ist.

Eine Funktion, auch Abbildung, ordnet (jedem) Element aus $M$ höchstens ein Element aus $N$ zu.

\subsection{Begriffe}

$M$ heißt Definitionsmenge von $f$.

$N$ heißt Zielmenge von $f$.

\subsection{Bild, Urbild einer Menge}

Sei $M, N$ Mengen und $f : M \to N$ eine Funktion. Sei $M_0 \subseteq M$ und $N_ \subseteq N$

$f(M_0) = {y\in \mathbb{N}\mid\exists_{x\in M_0} : f (x) = y }$ ist das Bild von $M_0$ unter $f$.

$\{x\in M\mid \exists_{y\in N_0}:f(x)=y\}$ ist das Urbild von $N_0$ unter $f$.

\subsection{Eigenschaften}

Sei $f:M\to N$ eine totale Funktion.

$f$ heißt surjektiv, wenn $\forall_{y\in N}\exists_{x\in M}:f(x)=y$.

$f$ heißt injektiv, wenn $\forall_{x,z\in M}:f(x)=y \text{ und } f(z)=y\to x=z$

$f$ heißt bijektiv (oder "1-zu-1"), wenn $f$ inketiv und surjektiv ist.

\subsection{Beispielfunktion}

\begin{equation}
    f_1:\mathbb{Z}\to \mathbb{N},x\to|x|
\end{equation}

$\mathbb{Z}$ sind alle positiven und negativen ganze Zahlen und $\mathbb{N}$ sind alle positiven ganzen Zahlen.

Das heißt die Definitionsmenge ist $\mathbb{Z}$ und die Zielmenge ist $\mathbb{N}$. Angewendet auf die Funktion nach dem Komma ergibt sich: $\mathbb{Z}\to |\mathbb{Z}|$.

\subsubsection{Surjektiv}

Diese Funktion ist Surjektiv, weil jedem Wert aus der Zielmenge $\mathbb{N}$ mindestens einem Erzeugerwert aus der Definitionsmenge $\mathbb{Z}$ zugeordnet werden kann.

\subsubsection{Injektiv}

Ist nicht Injektiv, da es mehrere Erzeugerwerte aus der Definitionsmenge $\mathbb{Z}$ gibt, für einen Wert aus der Zielmenge $\mathbb{N}$.

Beispiel: $f(2)=2$ und $f(-2)=2$.

\subsubsection{Bijektiv}

Diese Funktion ist nicht Bijektiv, da sie nicht Surjektiv und Injektiv ist.

\section{Mengen}

\subsection{Kardinalität}

Die Mächtigkeit oder Kardinalität $|M|$ einer MEnge $M$ ist ein Maß für die Anzahl der Elemente in $M$.

Zwei Mengen $M, N$ sind gleichmächtig, wenn eine bijektive Abbildung $f:M\to N$ existiert.

Eine unendliche Menge $M$ heißt abzählbar, wenn Sie die selbe Kardinalität wie $\mathbb{N}$ hat.

\end{document}
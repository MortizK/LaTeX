\documentclass[a4paper]{article}

%\usepackage{url}

%% Math
\usepackage{mathtools}
%% For Mengen like natural numbers
\usepackage{amsfonts}
%% Für spezielle Symbole
\usepackage{amssymb}

%% Images
\usepackage{import}
\usepackage{xifthen}
\usepackage{pdfpages}
%\usepackage{transparent}

%%% Command for simpler images
\newcommand{\incfig}[1]{%
    \def\svgwidth{\columnwidth}
    \import{./fig/}{#1.pdf_tex}
}

%% Links
\usepackage{hyperref}
\hypersetup{
    colorlinks=true,
    linkcolor=black,
    filecolor=magenta,
    urlcolor=cyan
}

%% Formatting
\usepackage{parskip}

\title{Logik}
\author{Moritz}
\date{February 12, 2025}

\begin{document}
\maketitle
\tableofcontents

\section{Aussagenlogik}

\subsection{Resolution}

Ein Verfahren zum Nachweis der Unerfüllbarkeit einer Klauselmenge.

Eine Klausel ist somit eine Menge $C_1=\{\neg c, \neg s\}$, wir aber dennoch Geschrieben als:

\begin{center}
    $C_1=\neg c \lor \neg s$
\end{center}

\subsubsection{Interpretationen}

Eine Interpretation ist die Menge der Literalen, wo entweder $\neg p$ oder $p$ existiert für alle $p\in\Sigma$

Eine Klausel ist wahr unter $I$, falls eines ihrer Literale in $I$ vorkommt.

\subsubsection{Erfülbarkeit von Klauselmenge}

Eine Klauselmenge ist erfüllbar, wenn die Interpretation für jede Klausel wahr ist.

Der Umkehrschluss ist: Wenn eine der Klauseln die Leere Klausel ist, so ist die Klauselmenge unerfüllbar.

\subsubsection{Resolutionsregel}

\begin{center}
    \begin{tabular}{c c}
        $C\lor p$ \quad & \quad$D\lor \neg p$ \\
        \hline
        \multicolumn{2}{c}{$C\lor D$}         \\
    \end{tabular}
\end{center}

Wir haben zwei Klauseln, welche eine gegenteilige Variable $p$ und $\neg p$ haben, wenn das der Fall ist, können wir einen neue Klausel bilden aus den beiden vorherigen ver-odert.

\subsection{Spickzettel}

Gefunden auf S.312 im \href{https://wwwlehre.dhbw-stuttgart.de/~sschulz/TEACHING/LGLI2024/Logic.pdf#page=549}{Script}

\end{document}
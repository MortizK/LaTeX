\documentclass[a4paper]{article}

%\usepackage{url}

%% Math
\usepackage{mathtools}
%% For Mengen like natural numbers
\usepackage{amsfonts}
%% Für spezielle Symbole
\usepackage{amssymb}

%% Images
\usepackage{import}
\usepackage{xifthen}
\usepackage{pdfpages}
%\usepackage{transparent}

%%% Command for simpler images
\newcommand{\incfig}[1]{%
    \def\svgwidth{\columnwidth}
    \import{./fig/}{#1.pdf_tex}
}

%% Links
\usepackage{hyperref}
\hypersetup{
    colorlinks=true,
    linkcolor=black,
    filecolor=magenta,
    urlcolor=cyan
}

%% Formatting
\usepackage{parskip}

\title{Logik}
\author{Moritz}

\begin{document}
\maketitle
\tableofcontents

\section{Listenverarbeitung}

S. 120 sind alle Funktionen und Objekte für Listen

$'()$ Die Leere Liste

$(cons obj list)$ "constructor" $(cons '(0 1) '(2 3 4)) \implies ((0 1) 2 3 4)$ oder $(cons 1 '(2 3 4)) \implies (1 2 3 4)$

$(append list_1 list_2)$ Hängt beide Listen zusammen $(append '(1 2 3) '(4 5)) \implies (1 2 3 4 5)$

$(car list)$ oder $(first list)$ Das erste Element

$(cdr list)$ oder $(rest list)$ Die Liste ohne das erste Element

$(null? list)$ gibt \#t zurück, wenn $list$ leer ist.

$(list obj_1\dots obj_n)$ Gibt die Liste $(obj_1\dots obj_n)$ zurück

\subsection{Aufgaben}

Eine Liste umdrehen.

Bonus: Liste abwechselnd in zwei Listen aufteilen.

Bonus: Zwei Listen abwechselnd in eine Liste zusammenfügen.

\subsection{Speichermodell und Variablen}

Wir haben eine Namenstabelle und einen Speicher. Die Namen zeigen auf eine Stelle im Speicher.

Ich kann die Variable woanders im Speicher zeigen lassen. Oder den Speicher wo diese hinzeigt, ändern.

Temporäre Variablen existieren nur in ihrem Scope.

\subsection{Special Form: cond}

Nach cond kommt eine Liste an Klauseln. Jede Klausel hat einen Test (= x 10) und mehrere Ausdrücke.

\end{document}
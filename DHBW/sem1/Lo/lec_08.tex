\documentclass[a4paper]{article}

%\usepackage{url}

%% Math
\usepackage{mathtools}
%% For Mengen like natural numbers
\usepackage{amsfonts}
%% Für spezielle Symbole
\usepackage{amssymb}

%% Images
\usepackage{import}
\usepackage{xifthen}
\usepackage{pdfpages}
%\usepackage{transparent}

%%% Command for simpler images
\newcommand{\incfig}[1]{%
    \def\svgwidth{\columnwidth}
    \import{./fig/}{#1.pdf_tex}
}

%% Links
\usepackage{hyperref}
\hypersetup{
    colorlinks=true,
    linkcolor=black,
    filecolor=magenta,
    urlcolor=cyan
}

%% Formatting
\usepackage{parskip}
\usepackage{listings}

\title{Logik}
\author{Moritz}
\date{January 8, 2025}

\begin{document}
\maketitle
\tableofcontents

Hier fehlen kleine Ecken, die aus dem Script nachgetragen werden können.

\subsection{Special Form: and}

And kann beliebig viele Werte entgegen nehmen und überprüft solange, bis ein Falsche wahr ist, dann wird \#f zurückgegeben, sonst der Wert des letzten Wertes.

\subsection{Special Form: or}

Funktioniert genau wie das and. Nur wenn ein Wert True ist, wird dieser Wert zurückgegeben.

\subsection{Special Form: let}

let führt Temporäre Variablen für zwischenergebnisse ein. Hier werden erst alle Werte der Bindungen ausgewertet, dann die Variablen angelegt und dann beschrieben.

(let ((a 10)(b 20)) (+ a b)) gibt 30 zurück

(let ((a +)(b *)) (a (b 10 20) (b 5 10))) gibt 250 zurück

\subsubsection{let*}

Hier werden die Bindungen sequentiell ausgewertet

(let* ((a 10)(b (+ a 20))) (+ a b)) gibt 40 zurück

\section{Sortieren}

Sortieren mit Insert Sort

\begin{lstlisting}
    (define (insert k lst)
        (cond ((null? lst) (list k))
              ((< k (car lst)) (cons k lst))
              (else (cons (car lst) 
                          (insert k (cdr lst))))))

    (define (isort lst)
        (if (null? lst)
            lst
            (insert (car lst) (isort (cdr lst)))))
\end{lstlisting}

\end{document}
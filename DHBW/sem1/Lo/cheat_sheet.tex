\documentclass[a4paper]{article}

%\usepackage{url}

%% Math
\usepackage{mathtools}
%% For Mengen like natural numbers
\usepackage{amsfonts}
%% Für spezielle Symbole
\usepackage{amssymb}

%% Images
\usepackage{import}
\usepackage{xifthen}
\usepackage{pdfpages}
%\usepackage{transparent}

%%% Command for simpler images
\newcommand{\incfig}[1]{%
    \def\svgwidth{\columnwidth}
    \import{./fig/}{#1.pdf_tex}
}

%% Links
\usepackage{hyperref}
\hypersetup{
    colorlinks=true,
    linkcolor=black,
    filecolor=magenta,
    urlcolor=cyan
}

%% Formatting
\usepackage{parskip}

\title{Lineare Algebra}
\author{Moritz}

\begin{document}
\maketitle
\tableofcontents

\section{Ziele}

Diese Dokument soll nicht nochmal alles im Detail aufschrieben, was wir gemacht haben, sondern eine Sammlung sein um sich auf die Prüfung vorzubereiten.

Dies dient auch als Vorarbeit für meinen A4 Spickzettel.

\subsection{Altklausuren}

Ich werde Altklausuren analysieren um die Hauptaufgaben zu filtern und in mehr Detail zu analysieren und üben.

\subsection{Formelsammlung/ Tools}

Die Idee ist es eine Sortierte Formelsammlung zu jedem Unten aufgelisteten Themengebiet zu erstellen.

\subsection{Begriffsverzeichnis}

Zudem werde ich wie Formel auch Begriffe sammel und lernen, um nicht mehr darüber nachdenken zu müssen.

\know{Brown Fox}{The quick brown fox jumps right over the lazy dog. the quick brown fox jumps right over the lazy dog. the quick brown fox jumps right over the lazy dog. the quick brown fox jumps right over the lazy dog. the quick brown fox jumps right over the lazy dog. the quick brown fox jumps right over the lazy dog. the quick brown fox jumps right over the lazy dog. the quick brown fox jumps right over the lazy dog.}

\subsection{Bildverzeichnis}

Von hilfsreichen Grafischen Darstellungen, wie z.B. Alle Formel im Dreieck.

\subsection{Rechnungen}

Vorgerechnete Aufgaben in dem Korrekten Klausuraufschrieb.

\section{Grundlagen}

\subsection{Aussagenlogik}

\subsection{Mengen}

\section{Relationen und Funktionen}

\subsection{Allgemeine Eigenschaften von Relationen}

\subsection{Klassifikation von Relationen}

\subsection{Äquivalenzrelationen}

\subsection{Verkettung von Relation und die inverse Relation}

\subsection{Funktionen}

\section{Zahlentheorie}

\subsection{Teilbarkeitstheorie}

\subsection{Rechnen Modulo p}

\subsection{RSA-Verschlüsselung}

\subsection{Chinesische Restsatz}

\section{Kombinatorik}

\subsection{Die Urnenmodelle}

\section{Vektorräume}

\subsection{$\mathbb{R}^2$ und $\mathbb{R}^3$ als Vektorraum}

\subsection{Die komplexen Zahlen}

\subsection{$\mathbb{C}^n$ als Vektorraum}

\subsection{Der allgemeine Vektorraumbegriff$^*$}

\subsection{Vektorräume mit Norm und Skalarprodukt}

\section{Matrizen}

\subsection{Lineare Gleichungssysteme und das GAUSS-Verfahren}

\subsection{Die Matrix zum LGS}

\subsection{Das Schema zum GAUSS-Verfahren}

\subsection{Lineare Unabhängigkeit}

\subsection{Basis und Dimension}

\subsection{Matrizen als lineare Abbildung}

\subsection{Direkte Zerlegung eines Vektorraum}

\subsection{Die Dimensionsformel}

\subsection{Die inverse Matrix}

\subsection{Die adjungierte Matrix}

\subsection{Koordinatentransformation}

\subsection{Determinanten}

\end{document}
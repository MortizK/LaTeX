\documentclass[a4paper]{article}

%\usepackage{url}

%% Math
\usepackage{mathtools}
%% For Mengen like natural numbers
\usepackage{amsfonts}
%% Für spezielle Symbole
\usepackage{amssymb}

%% Images
\usepackage{import}
\usepackage{xifthen}
\usepackage{pdfpages}
%\usepackage{transparent}

%%% Command for simpler images
\newcommand{\incfig}[1]{%
    \def\svgwidth{\columnwidth}
    \import{./fig/}{#1.pdf_tex}
}

%% Links
\usepackage{hyperref}
\hypersetup{
    colorlinks=true,
    linkcolor=black,
    filecolor=magenta,
    urlcolor=cyan
}

%% Formatting
\usepackage{parskip}

\title{Logik}
\author{Moritz}
\date{February 5, 2025}

\begin{document}
\maketitle
\tableofcontents

\section{Aussagenlogik}

\subsection{Tableaux}

Ist ein Tool um zu beweisen, dass eine Formel unerfüllbar ist.

\subsection{Logische Äquivalenz}

Eine Formel heißt Äquivalent, wenn sie sich gegenseitig folgern lassen. Wir schreiben $F\equiv G$.

\subsection{Basis der Aussagenlogik}

Eine Basis ist die Menge der gültigen Operatoren womit es möglich ist zwei Formeln zu bauen, welche nur aus den Operatoren in der Basis bestehen.

Bekannte Basen sind: $\{\land, \lor, \neg\}$, $\{\to, \neg\}$ oder $\{\land, \neg\}$

\subsection{Beweis einer Basis}

\subsubsection{Behauptung}

Behauptung: $o=\{\land, \lor, \neg\}$ sit eine Basis der Aussagenlogik

zu zeigen: Zu jeder Formel $F\in For_0\Sigma$ existiert eine Formel $F'\in For_0\Sigma$ mit zwei Eigenschaften:

\begin{enumerate}[1.]
    \item $F\equiv F'$
    \item $F'$ enthält nur Operatoren aus $0$
\end{enumerate}

Beweis durch Induktion über den Aufbau

\subsubsection{Induktionsanfang}

Induktionsanfang: Sie $F$ eine elementare (atomare) Formel, dann gilt: $F=\top$ oder $F=\bot$ oder $F=a\in\Sigma$

\begin{enumerate}[Fall 1.]
    \item $F=\top$. Betrachte $F'=\top$. Dann gilt $F=F'$ und enthält nur Operatoren aus $o$.
    \item $F=\bot$. Betrachte $F'=\bot$. Dann gilt $F=F'$ und enthält nur Operatoren aus $o$.
    \item $F=a\in\Sigma$. Betrachte $F'=a\in\Sigma$. Dann gilt $F=F'$ und enthält nur Operatoren aus $o$.
\end{enumerate}

In allen 3 Fällen existiert das gesuchte F'. Also ist der Induktionsanfang gesichert.

\subsubsection{Induktionsvoraussetzung}

Induktionsvoraussetzung: Die Behauptung gelten für $A$ und $B\in For_0\Sigma$. Also: Es existiert $A', B'\in For_0\Sigma$ mit $A\equiv A'$, $B\equiv B'$ und $A', B'$ enthalten nur Operatoren aus $o$.

\subsubsection{Induktionsschritt}

Induktionsschritt: Sei $F$ eine zusammengesetze Formel, dann gilt: $F$ hat die Formen $(\neg A),(A\land B),(A\lor B),(A\to B)$ oder $(A\iff B)$

\begin{enumerate}[1.]
    \item $F\equiv F'$
    \item $F'$ enthält nur Operatoren aus $o$
\end{enumerate}

\begin{enumerate}[Fall 1.]
    \item $F=\neg A$. Dann existiert nach IV $A'$ mit 1 und 2. Betrachte $F'=\neg A'$. Dann gilt $F=F'$ Wahrheitstabelle und 2.
    \item $F=A\land B$. Dann existiert nach IV $A'$ und $B'$ mit 1 und 2. Betrachte $F'=A'\land B'$. Dann gilt $F=F'$ Wahrheitstabelle und 2.
    \item $F=A\lor B$. Dann existiert nach IV $A'$ und $B'$ mit 1 und 2. Betrachte $F'=A'\lor B'$. Dann gilt $F=F'$ Wahrheitstabelle und 2.
    \item $F=A\to B$. Betrachte $F'=\neg A'\lor B'$. Dann:
          \begin{center}
              \begin{tabular}{c c c|c c c}
                  $A$ & $B$ & $F$ & $A'$ & $B'$ & $F'$ \\
                  \hline
                  0   & 0   & 1   & 0    & 0    & 1    \\
                  0   & 1   & 1   & 0    & 1    & 1    \\
                  1   & 0   & 0   & 1    & 0    & 0    \\
                  1   & 1   & 1   & 1    & 1    & 1    \\
              \end{tabular}
          \end{center}
          Dann gilt $F=F'$ und 2.
    \item $F=A\iff B$. Betrachte $F'=(\neg A'\lor B')\land(A'\lor\neg B')$. Dann:
          \begin{center}
              \begin{tabular}{c c c|c c c}
                  $A$ & $B$ & $F$ & $A'$ & $B'$ & $F'$ \\
                  \hline
                  0   & 0   & 1   & 0    & 0    & 1    \\
                  0   & 1   & 0   & 0    & 1    & 0    \\
                  1   & 0   & 0   & 1    & 0    & 0    \\
                  1   & 1   & 1   & 1    & 1    & 1    \\
              \end{tabular}
          \end{center}
          Dann gilt $F=F'$ und 2.
\end{enumerate}

\subsection{Teilformel}

Die Menge aller Teilformel einer Formel sind z.B.:

\begin{center}
    $TF(A\land\neg B)=\{(A\land B),(A),(\neg B, (B))\}$
\end{center}

Wir können zudem Äquivalente Teilformeln eine Formel austauschen, ohne die gesamtformel zu ändern.

\subsubsection{Kalkül}

\begin{equation}
    \begin{split}
        \top & \vdash_{LU}\neg B\lor \top          \\
             & \vdash_{LU}\neg B\lor(A\lor\neg A)  \\
             & \vdash_{LU}\neg B\lor(\neg A\lor A) \\
             & \vdash_{LU}(\neg B\lor\neg A)\lor A \\
             & \vdash_{LU}(\neg A\lor\neg B)\lor A \\
             & \vdash_{LU}\neg A\lor(\neg B\lor A) \\
             & \vdash_{LU}A\to (\neg B\lor A)      \\
             & \vdash_{LU}A\to(B\to A)             \\
    \end{split}
\end{equation}

\subsubsection{Korrektheit}

Das Kalkül der logischen Umformungen ist korrekt und vollständig. Somit ist jede Tautologie durch dieses Kalkül herleitbar.

\end{document}
\documentclass[a4paper]{article}

%\usepackage{url}

%% Math
\usepackage{mathtools}
%% For Mengen like natural numbers
\usepackage{amsfonts}
%% Für spezielle Symbole
\usepackage{amssymb}

%% Images
\usepackage{import}
\usepackage{xifthen}
\usepackage{pdfpages}
%\usepackage{transparent}

%%% Command for simpler images
\newcommand{\incfig}[1]{%
    \def\svgwidth{\columnwidth}
    \import{./fig/}{#1.pdf_tex}
}

%% Links
\usepackage{hyperref}
\hypersetup{
    colorlinks=true,
    linkcolor=black,
    filecolor=magenta,
    urlcolor=cyan
}

%% Formatting
\usepackage{parskip}

\title{Logik}
\author{Moritz}
\date{December 12, 2024}

\begin{document}
\maketitle
\tableofcontents

\section{Relation}

Eine Relation ist eine Menge aus Tupeln, Welche eine Teilmenge von dem Kreuzprodukt zweier Menge ist.

\subsection{Graphendarstellung}

Grafik kann von der Folie entnommen werden.

Alle Eigenschaften von einer Relation, können einfach von einem Menschen abgelesen werden.

\subsection{Tabellendarstellung}

Diese Art der Darstellung ist auch einfach im Rechner realisierbar. Ein Lookup von $xRy$ ist schnell.

\begin{equation}
    \begin{array}{c|cccc}
          & 0 & 1 & 2 & 3 \\
        \hline
        0 & 1 & 1 & 0 & 0 \\
        1 & 0 & 0 & 1 & 0 \\
        2 & 0 & 0 & 0 & 1 \\
        3 & 0 & 1 & 0 & 1 \\
    \end{array}
\end{equation}

Ein Nachteil ist, dass die Tabelle sehr groß wird, denn wächst $n^2$. Wir verschwenden also viel zu viel Speicher.

\subsection{Inverse Relation}

Sei $R$ eine Relation. Die Inverse Relation (zu $R$) ist $R^{-1}=\{(y,x)\mid(x,y)\in R\}$

Für symmetrische Relation gilt $R^{-1}=R$.

\subsection{Relationsprodukt}

Seien $R_1,R_2$ zwei (binäre) Relationen. Das Relationsprodukt $R_2\circ R_1$ ist die Relation $\{(x,y)\mid\exists z: (x,z)\in R_1\text{ und }(z,y)\in R_12\}$.

\subsection{Identitätsrelation}

Sei $M$ eine Menge. Dann ist $id_M=\{(x,y)\mid x\in M\}$ (kurz: id) die Identitätsrelation über $M$.

\subsection{Relationsalgebra}

Sei H die Menge aller Menschen, die jemals gelebt haben.

Sei $V\subseteq H\times H$ die Vater-Relation (also $V=\{(x,y)\mid y\text{ ist Vater von }x\}$)

Sei analog $M\subseteq H\times H$ die Mutter-Relation.

Dann können wir auch eine Großmutter-Relation $G$ wie folgt beschreiben: $G\subseteq M\circ(M\cup H)$

\subsection{Potenzierung}

\subsubsection{Übung}

Sei $M=\{a,b,c,d\}$, $R=\{(a,b),(b,c),(c,d)\}$

Für die Klausur, entweder die Nullen eintragen, oder die Großen Flächen an Nullen markieren.

Die Pfeildarstellung kann helfen, da ein $R^2$ alle Pfaden enthält, die die Länge $2$ haben.

\begin{equation}
    \begin{matrix}
        \begin{array}{c|cccc}
            R^{-1} & a & b & c & d \\
            \hline
            a      &   &   &   &   \\
            b      & 1 &   &   &   \\
            c      &   & 1 &   &   \\
            d      &   &   & 1 &   \\
        \end{array} &
        \begin{array}{c|cccc}
            R^0 & a & b & c & d \\
            \hline
            a   & 1 &   &   &   \\
            b   &   & 1 &   &   \\
            c   &   &   & 1 &   \\
            d   &   &   &   & 1 \\
        \end{array}    &
        \begin{array}{c|cccc}
            R^1 & a & b & c & d \\
            \hline
            a   &   & 1 &   &   \\
            b   &   &   & 1 &   \\
            c   &   &   &   & 1 \\
            d   &   &   &   &   \\
        \end{array}            \\
                                  &   & \\
        \begin{array}{c|cccc}
            R^2 & a & b & c & d \\
            \hline
            a   &   &   & 1 &   \\
            b   &   &   &   & 1 \\
            c   &   &   &   &   \\
            d   &   &   &   &   \\
        \end{array}    &
        \begin{array}{c|cccc}
            R^3 & a & b & c & d \\
            \hline
            a   &   &   &   & 1 \\
            b   &   &   &   &   \\
            c   &   &   &   &   \\
            d   &   &   &   &   \\
        \end{array}    &
        \begin{array}{c|cccc}
            R^4 & a & b & c & d \\
            \hline
            a   &   &   &   &   \\
            b   &   &   &   &   \\
            c   &   &   &   &   \\
            d   &   &   &   &   \\
        \end{array}
    \end{matrix}
\end{equation}

Sei $S=\{(a,a),(b,b),(c,c)\}$. Wir gehen über das Rechten - $R$ - zu einem Element aus $S$, also $S\circ R$.

\begin{equation}
    \begin{matrix}
        \begin{array}{c|cccc}
            S\circ S & a & b & c & d \\
            \hline
            a        & 1 &   &   &   \\
            b        &   & 1 &   &   \\
            c        &   &   & 1 &   \\
            d        &   &   &   & 0 \\
        \end{array} &
        \begin{array}{c|cccc}
            S\circ R & a & b & c & d \\
            \hline
            a        &   & 1 &   &   \\
            b        &   &   & 1 &   \\
            c        &   &   &   & 0 \\
            d        &   &   &   &   \\
        \end{array}
    \end{matrix}
\end{equation}

\subsection{Erweiterung: "Hüllenbildung"}

Auf einer Neuen Folie, noch nachtragen

\begin{equation}
    \begin{matrix}
        \begin{array}{c|cccc}
            \text{reflexiv } R\cup R^0 & a & b & c & d \\
            \hline
            a                          & 1 & 1 &   &   \\
            b                          &   & 1 & 1 &   \\
            c                          &   &   & 1 & 1 \\
            d                          &   &   &   & 1 \\
        \end{array} &
        \begin{array}{c|cccc}
            \text{symmetrisch } R\cup R^{-1} & a & b & c & d \\
            \hline
            a                                & 0 & 1 &   &   \\
            b                                & 1 & 0 & 1 &   \\
            c                                &   & 1 & 0 & 1 \\
            d                                &   &   & 1 & 0 \\
        \end{array}    \\
        \begin{array}{c|cccc}
            \text{transitiv } R^{+} & a & b & c & d \\
            \hline
            a                       &   & 1 & 1 & 1 \\
            b                       &   &   & 1 & 1 \\
            c                       &   &   &   & 1 \\
            d                       &   &   &   &   \\
        \end{array}    &
        \begin{array}{c|cccc}
            \text{reflexiv \& transitiv } R^{*} & a & b & c & d \\
            \hline
            a                                   & 1 & 1 & 1 & 1 \\
            b                                   &   & 1 & 1 & 1 \\
            c                                   &   &   & 1 & 1 \\
            d                                   &   &   &   & 1 \\
        \end{array} \\
        \begin{array}{c|cccc}
            \text{alles }(R\cup R^{-1})^{*} & a & b & c & d \\
            \hline
            a                               & 1 & 1 & 1 & 1 \\
            b                               & 1 & 1 & 1 & 1 \\
            c                               & 1 & 1 & 1 & 1 \\
            d                               & 1 & 1 & 1 & 1 \\
        \end{array}
    \end{matrix}
\end{equation}

\end{document}
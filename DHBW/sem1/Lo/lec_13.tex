\documentclass[a4paper]{article}

%\usepackage{url}

%% Math
\usepackage{mathtools}
%% For Mengen like natural numbers
\usepackage{amsfonts}
%% Für spezielle Symbole
\usepackage{amssymb}

%% Images
\usepackage{import}
\usepackage{xifthen}
\usepackage{pdfpages}
%\usepackage{transparent}

%%% Command for simpler images
\newcommand{\incfig}[1]{%
    \def\svgwidth{\columnwidth}
    \import{./fig/}{#1.pdf_tex}
}

%% Links
\usepackage{hyperref}
\hypersetup{
    colorlinks=true,
    linkcolor=black,
    filecolor=magenta,
    urlcolor=cyan
}

%% Formatting
\usepackage{parskip}

\title{Logik}
\author{Moritz}
\date{January 29, 2025}

\begin{document}
\maketitle
\tableofcontents

\section{Aussagenlogik}

\subsection{Deduktionstheorem}

Eine Formel G folgt aus einer Formelmenge $\{F_1, \dots, F_n\}$ genau dann, wenn die Formel $(F_1\land \dots\land F_n)\to G$ allgemeingültig ist ...

... und also genau dann, wenn $neg((F_1\land \dots\land F_n)\to G)$ unerfüllbar ist.

\subsection{Widerspruchskalküle}

Es wird auf Unerfüllbarkeit eine Formel-(Menge) geprüft.

\subsection{Baum}

Ein Baum besteht aus Knoten, Vertices und Kanten, Edges.

Es gibt nur eine Wurzel und diese ist der einzige Knoten ohne Vorgänger. Alle anderen Knoten haben genau einen Vorgänger.

Es gibt noch verschieden Begriffe: Wurzel, Blatt, innerer Knoten, Pfad, Ast.

Ein Baum, wo jeder Knoten maximal 2 Nachfolger hat heißt Binärbaum.

\subsection{Tableaux}

Ist ein Widerlegungskalkül, welches durch vollständige Fallunterscheidung die Unerfüllbarkeit zu zeigen.

Wir teilen Formeln immer in zwei und versuchen über beide teile Aussagen zu treffen: z.b. $a\land b$:

\begin{center}
    a = 1 und\\
    b = 1
\end{center}

\subsubsection{Formelntyp}

Primitive Formeln: $a, \neg b$

Konjunktive Formel $\alpha$: ist zerlegbar und beide Seiten müssen Wahr sein.

Disjunktive Formel $\beta$: ist zerlegbar und hier muss mindestens eins Wahr sein.

\subsubsection{Zerlegung von $\alpha$ und $\beta$ Formeln}

Für Klausur auf S. 247 in dem Script

\begin{center}
    \begin{tabular}{c c}
        \begin{tabular}{c|c c}
            $\alpha$        & $\alpha_1$ & $\alpha_2$ \\
            \hline
            $A\land B$      & $A$        & $B$        \\
            $\neg(A\lor B)$ & $\neg A$   & $\neg B$   \\
            $\neg(A\to B)$  & $A$        & $\neg B$   \\
            $A\iff B$       & $A\to B$   & $B\to A$   \\
            $\neg\neg A$    & $A$        &
        \end{tabular} \quad & \quad
        \begin{tabular}{c|c c}
            $\beta$          & $\beta_1$      & $\beta_2$       \\
            \hline
            $\neg(A\land B)$ & $\neg A$       & $\neg B$        \\
            $A\lor B$        & $A$            & $B$             \\
            $A\to B$         & $\neg A$       & $B$             \\
            $\neg(A\iff B)$  & $A\land\neg B$ & $\neg A\land B$ \\
                             &                &
        \end{tabular}
    \end{tabular}
\end{center}

\subsubsection{Tableaux Konstruktion}

Startregel: Initialisiere Eine Formel

$\alpha$-Regel: Beide $\alpha$ untereinander in den Baum eintragen.

$\beta$-Regel: Unter allen Ästen dieses Zweigs, eine neue Gabelung mit $\beta_1$ und $\beta_2$ einfügen.

Abbruchbedingung: Atomare Konflikt in einem Ast oder eine Formel $A$ und $\neg A$ tritt auf.

\end{document}
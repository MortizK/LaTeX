\documentclass[a4paper]{article}

%\usepackage{url}

%% Math
\usepackage{mathtools}
%% For Mengen like natural numbers
\usepackage{amsfonts}
%% Für spezielle Symbole
\usepackage{amssymb}

%% Images
\usepackage{import}
\usepackage{xifthen}
\usepackage{pdfpages}
%\usepackage{transparent}

%%% Command for simpler images
\newcommand{\incfig}[1]{%
    \def\svgwidth{\columnwidth}
    \import{./fig/}{#1.pdf_tex}
}

%% Links
\usepackage{hyperref}
\hypersetup{
    colorlinks=true,
    linkcolor=black,
    filecolor=magenta,
    urlcolor=cyan
}

%% Formatting
\usepackage{parskip}

\title{Logik}
\author{Moritz}
\date{February 13, 2025}

\begin{document}
\maketitle
\tableofcontents

\section{Aussagenlogik}

\subsection{Prädikatenlogik erster Stufe}

\subsubsection{Grenzen der Aussagenlogik}

Mögliche Schlussfolgerungen:

\begin{enumerate}
    \item Alle Menschen sind sterblich, Sokrates ist ein Mensch $\to$ Also ist Sokrates sterblich.
    \item Alle vögel können fliegen, Ein Pinguin ist ein Vogel $\to$ Also kann ein Pinguin fliegen.
\end{enumerate}

Hier ist der Schluss 2. richtig, aber ein Pinguin kann nicht fliegen.

\begin{center}
    Somit ust unsere Annahme falsch!
\end{center}

\subsubsection{Idee}

Atome sind nicht mehr Atomar

Prädikatssymbol repräsentieren Relationen (rot, rund, prim, ist Bruder von, ...)

Funktionssymbole repräsentieren Funktionen (+, Mittelwert von, Vater von, ...)

Es gibt wieder unser Quantoren: $\forall$ und $\exists$

Und Object (Leute, ...)

\begin{center}
    \begin{tabular}{c}
        $\forall X(mensch(X)\to sterblich(X))$ \\
        $mensch(Sokrates)$                     \\
        \hline
        $sterblich(Sokrates)$
    \end{tabular}
\end{center}

\subsubsection{Definition}

Eine Signatur $\Sigma$ ist ein 3-Tupel (P, F, V), welche paarweise disjunkte Mengen.

\begin{enumerate}
    \item[P] - Prädikate: $P=\{mensch/1, sterblich/1, lauter/2\}$ sind Tupel welche eine Namen haben und eine Anzahl an Eingaben.
    \item[F] - Funktionen: $F=\{gruppe/2, mathelehrer/1, sokrates/0 aristoteles/0\}$ sind Tupel, wobei 0-stellige Funktionen, konstanten sind.
    \item[V] - Variablen: $V=\{x, y, z, u, x_1, x_2, \dots\}$
\end{enumerate}

\subsubsection{Terme}

Grundsätzlich bestehen Terme $T_{\Sigma}$ aus: Allen Variablen und allen Kombinationen von Funktionen mit Funktionen und Variablen.

Funktionen sind außen und Variablen sind ganz innen.

Mögliche Terme:

\begin{enumerate}
    \item X
    \item sokrates
    \item gruppe(X, sokrates)
    \item gruppe(lehrer(sokrates), gruppe(sokrates, X))
\end{enumerate}

\subsubsection{Atome}

Atome sind Außen EIN Prädikat und innen bestehen sie NUR aus Funktionen.

Beispiel:

\begin{enumerate}
    \item mensch(X)
    \item lauter(sokrates, X)
    \item lauter(sokrates, sokrates)
    \item lauter(sokrates, lehrer(lehrer(lehrer(sokrates))))
\end{enumerate}

\subsubsection{Syntax}

Ist Identisch, mit Extras. Somit ist die Aussagenlogik eine Teilmenge von der Prädikatenlogik.

Die Zusätze sind:

\begin{enumerate}
    \item Allquantor: $\forall$
    \item Existensquantor: $\exists$
\end{enumerate}

Die Formel werden somit auch um diese erweiter: Wenn A eine Formel ist und x eine Variabel:

\begin{enumerate}
    \item $\forall xA$
    \item $\exists xA$
    \item $\in For_{Sigma}$
\end{enumerate}

\subsubsection{Beispiel für Formeln}

\begin{enumerate}
    \item[P] $=\{studiert_in/2, schlau/1, prim/1\}$
    \item[F] $=\{stuttgart/0. mannheim/0, sum/2\}$
    \item[V] $=\{x, y\}$
\end{enumerate}

Alle die in Stuttgart studieren, sind schlaue! Es gibt jemanden, der in Mannheim studiert, der schlaue ist! Die Summe zweier Primzahlen, ist auch eine Primzahl!

\begin{center}
    $\forall x(studiert_in(x, stuttgart)\to schlau(x))$

    $\exists x(studiert_in(x, mannheim)\land schlau(x))$

    $\forall x\forall y(prim(x)\land prim(y)\to prim(sum(x, y)))$
\end{center}

\subsubsection{Semantik}

Die Interpretation ist auf Folie 335 im Skriptum!

\subsubsection{Auswertung}

Wir können Stück für Stück von innen nach außen auswerten. Relation enthalten nur die Tupel, wo die Interpretationen wahr ist (1).

\end{document}
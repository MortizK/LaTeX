\documentclass[a4paper]{article}

%\usepackage{url}

%% Math
\usepackage{mathtools}
%% For Mengen like natural numbers
\usepackage{amsfonts}
%% Für spezielle Symbole
\usepackage{amssymb}

%% Images
\usepackage{import}
\usepackage{xifthen}
\usepackage{pdfpages}
%\usepackage{transparent}

%%% Command for simpler images
\newcommand{\incfig}[1]{%
    \def\svgwidth{\columnwidth}
    \import{./fig/}{#1.pdf_tex}
}

%% Links
\usepackage{hyperref}
\hypersetup{
    colorlinks=true,
    linkcolor=black,
    filecolor=magenta,
    urlcolor=cyan
}

%% Formatting
\usepackage{parskip}

\title{Logik}
\author{Moritz}
\date{January 22, 2025}

\begin{document}
\maketitle
\tableofcontents

\section{Aussagenlogik}

\subsection{Syntax}

Was ist ein korrekter Satz?

\subsubsection{Atomare Aussagen}

Beispiele: C ist schuldig, C, Die Straße ist nass, StraßeNass

\subsubsection{Operatoren}

$land$ und, $lor$ oder, $\to$ impliziert, $\neq$ nicht

\subsubsection{Beispiel}

1. Wenn A schuldig und B unschuldig, so ist C schuldig

2. C arbeitet niemals allein

3. A arbeitet niemals mit C

4. Niemand außer A, B oder C war beteiligt und mindestens ein muss es gewesen sein.

1. $A\land \neq B\to C$

2. $C\to (A\lor B)$ (mit 4.)

3. $A\to\neg C$

4. $A\lor B\lor C$

\subsubsection{Definition}

Eine aussagenlogische Signatur $\sum$ ist eine (Nichtleere) abzählbare Menge von Symbolen, etwa

\subsubsection{Logische Symbole}

$\top$ für den Wahrheitswert "wahr"

$\bot$ für den Wahrheitswert "falsch"

$\neg$ Negations "nicht"

$\land$ Konjunktion "und"

$\lor$ Disjunktion "oder"

$\to$ Implikation "wenn ..., dann"

$\leftrightarrow$ Äquivalenz "genau dann, wenn"

() Klammern

\subsubsection{Definition: Menge der Formeln}

Die Menge aller Formeln. $\top$ und $\bot$ sind Teil der Menge, sowie alle Atome. Zudem gilt, wenn $P, Q\in For0_{\sum}$, dann sind auch $(\neg P), (P\land Q), (P\lor Q), (P\to Q), (P \leftrightarrow Q)$

\begin{equation}
    \sum=\{A_0,\dots,A_n\}
\end{equation}

Hier müssen wir darauf achten, dass der Syntax immer Klammern um Jede Formel hat.

\subsubsection{Klammern}

Wir erlauben uns durch Faulheit, manche Klammern wegzulassen. Somit können wir die äußersten Klammer weglassen.

Zudem haben wir wieder eine Präzedenz, welche den Verschiedenen Operatoren eine Gewichtung und somit eine Rechenreihenfolge gibt.

Die Reihenfolge ist: $\neg , \land, \lor, \to, \leftrightarrow$. Somit brauchen wir hier: $C\land(P\to A\lor C)$

\subsection{Semantik}

Wann ist ein Satz (inhaltlich) wahr oder falsch?

\subsubsection{Definition}

Sei $\sum$ eine aussagenlogische Signatur.

Eine Interpretation (über $\sum$) ist eine beliebige Abbildung $I: \sum\to\{1, 0\}$

Die Menge aller Interpretationen über $\sum$ bezeichnen wir als $I_{\sum}$

\subsubsection{Auswertung}

Wir Definieren einfach für alle Operatoren die Ergebnisse in Abhängigkeit von den Ergebnisse ihre Teile.

Somit bilden wir die Wertetabellen für alle Operatoren für zwei Atome.

\end{document}
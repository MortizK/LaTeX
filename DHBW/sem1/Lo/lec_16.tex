\documentclass[a4paper]{article}

%\usepackage{url}

%% Math
\usepackage{mathtools}
%% For Mengen like natural numbers
\usepackage{amsfonts}
%% Für spezielle Symbole
\usepackage{amssymb}

%% Images
\usepackage{import}
\usepackage{xifthen}
\usepackage{pdfpages}
%\usepackage{transparent}

%%% Command for simpler images
\newcommand{\incfig}[1]{%
    \def\svgwidth{\columnwidth}
    \import{./fig/}{#1.pdf_tex}
}

%% Links
\usepackage{hyperref}
\hypersetup{
    colorlinks=true,
    linkcolor=black,
    filecolor=magenta,
    urlcolor=cyan
}

%% Formatting
\usepackage{parskip}

\title{Logik}
\author{Moritz}
\date{February 6, 2025}

\begin{document}
\maketitle
\tableofcontents

\section{Aussagenlogik}

\subsection{Normalformen}

Sind Aussagenlogische Formeln, welche beliebig verschachtelt werden. In der Normalform, sind diese kompakter.

Algorithmen und Kalküle werden einfache für einfachere Sprachen.

\subsubsection{Negations-Normalform (NNF)}

Es gibt nur $\land, \lor, \neg$ und $\neg$ kommt nur vor Atomen vor. Zudem sind die Wahrheitswerte $\top, \bot$ nur bei Formeln die entweder Tautologien sind oder unerfüllbar sind.

Zur Umwandlung drei Schritte

\begin{enumerate}
    \item $A\iff B$ durch $(A\to B)\land(B\to A)$ austauschen.
    \item $A\to B$ durch $\neg A \lor B$ austauschen.
    \item Durch De Morgan die $\neg$ nach innen schieben, sowie alle $\top$ und $\bot$ Regeln anwenden.
\end{enumerate}

\know{Literal}{Ist ein Atom (aussagenlogische Formel) oder die Negation eiens Atom
    \begin{center}
        $A$ oder $\neg A$
    \end{center}}

\know{Klausel}{Ist eine Disjunktion von Literalen
    \begin{center}
        $(A\lor\neg B\lor C)$
    \end{center}
    Es gibt auch die leere Klause, diese ist unerfüllbar.}

\subsubsection{Konjunktive Normalform (KNF)}

Ist eine Konjunktion von Klauseln.

\begin{center}
    $(A\lor\neg B)\land(B\lor\neg C\lor\neg D)$

    $(A\lor B)$ und $(A)\land (B)$ sind auch KNF
\end{center}

Und die leere Konjunktion ist eine Tautologie.

\end{document}
\documentclass[a4paper]{article}

%\usepackage{url}

%% Math
\usepackage{mathtools}
%% For Mengen like natural numbers
\usepackage{amsfonts}
%% Für spezielle Symbole
\usepackage{amssymb}

%% Images
\usepackage{import}
\usepackage{xifthen}
\usepackage{pdfpages}
%\usepackage{transparent}

%%% Command for simpler images
\newcommand{\incfig}[1]{%
    \def\svgwidth{\columnwidth}
    \import{./fig/}{#1.pdf_tex}
}

%% Links
\usepackage{hyperref}
\hypersetup{
    colorlinks=true,
    linkcolor=black,
    filecolor=magenta,
    urlcolor=cyan
}

%% Formatting
\usepackage{parskip}
\usepackage{listings}

\title{Logik}
\author{Moritz}
\date{January 16, 2025}

\begin{document}
\maketitle
\tableofcontents

\section{Listen}

Wie sehen Listen unter der Haube aus?

Nichtleere Liste bestehen aus cons-Paaren. Diese cons-Paare sind zwei Pointer. Der erste zeigt auch car und der zweiter auf den Reste der Liste. Der zweiter zeigt somit entweder selber auf ein cons-Paar oder auf den Nullpointer (Leere Liste).

Somit sind Listen Rekursiv definiert.

\subsection{cons a b}

Erstelle ein Neues cons-Paar.

Schreibt den Pointer auf a in dessen car.

Schreibt den Pointer auf b in dessen cdr.

\subsection{car und cdr verallgemeinert}

Wir können auch Abkürzungen bis ebene 4 Zusammenfassen.

Somit ist (caar x) = (car (car x)) und Kombinationen wie (cadd x) = (car (cdr (cdr x)))

\subsection{Typen}

(pair? obj) prüft, ob obj ein cons-Paar ist.

(list? obj) prüft, ob obj eine Liste ist.

(null? obj) prüft, ob obj die leere Liste ist.

\subsection{Lambda}

\begin{lstlisting}
    ((lambda (x y)(+ (* 2 x) y)) 3 5) 
    => 11
\end{lstlisting}

lambda erstellt eine Funktion mit Parametern. Hier werden (3 5) für (x y) eingesetzt und der Body wird dann ausgeführt.

\subsection{map}

\begin{lstlisting}
    (map (lambda (x) (* x 3) '(1 2 3 4)))
    => '(3 6 9 12)
\end{lstlisting}

map kann auch mehrere Parameter haben, aber nur dann, wenn die Funktion in map mehrere Parameter braucht. Aso z.B.:

\begin{lstlisting}
    (map (lambda (x y) (if (> x y) x y)) 
    '(2 5) '(3 4))
    => '(3 5)
\end{lstlisting}

\subsection{apply}

\begin{lstlisting}
    (apply + '(11 7 14)) 
    => 32
\end{lstlisting}

Hier wird die Funktion zwischen jedem Element der Liste Ausgewertet.

\begin{lstlisting}
    (apply map (list (lambda ( x )(+ x 3 )) '(1 2 3)))
    (apply map '(x->x+3 '(1 2 3)))
    (map x->x+3 '(1 2 3)) 
    => '(4 5 6)
\end{lstlisting}

\subsection{eval}

eval nimmt einen Scheme-Ausdruck und wertet ihn in der aktuellen Umgebung aus. (eval (+ 2 11)) =$>$ 13

\subsection{set-mcar! und set-mcdr!}

In Scheme, aber nicht in Racket, gibt es diese Funktionen und "Modifizierbaren Listen".

\section{Namenskonvention}

Verändernde (destruktive) Funktionen enden in !
set!, set-car!, vector-set!, . . .

Prädikate (liefern $\#$t oder $\#$f) enden in ?
null?, pair?, equal?. . .

Konvertierungsfunktionen enthalten ->
string->number, list->vector

\section{Mut zur Lücke}

Was wir in Scheme nicht bearbeitet haben. Finde ich in der Folie auf S. 187

\end{document}
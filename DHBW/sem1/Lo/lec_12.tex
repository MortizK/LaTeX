\documentclass[a4paper]{article}

%\usepackage{url}

%% Math
\usepackage{mathtools}
%% For Mengen like natural numbers
\usepackage{amsfonts}
%% Für spezielle Symbole
\usepackage{amssymb}

%% Images
\usepackage{import}
\usepackage{xifthen}
\usepackage{pdfpages}
%\usepackage{transparent}

%%% Command for simpler images
\newcommand{\incfig}[1]{%
    \def\svgwidth{\columnwidth}
    \import{./fig/}{#1.pdf_tex}
}

%% Links
\usepackage{hyperref}
\hypersetup{
    colorlinks=true,
    linkcolor=black,
    filecolor=magenta,
    urlcolor=cyan
}

%% Formatting
\usepackage{parskip}

\title{Logik}
\author{Moritz}
\date{January 23, 2025}

\begin{document}
\maketitle
\tableofcontents

\section{Aussagenlogik}

\subsection{Modelle}

Modelle sind auf S. 213

Ein Modell einer Formel mit $A\in For=_{\sum}$ ist $val_I(A)=1$. Zudem können wir auch ein Modell für eine Formelmenge bilden:  $M\subset For=_{\sum}$ ist $val_I(A)=1$, für alle $A\in M$.

Eine Formelmenge ist auch die Konjunktion $(\land)$ ihrer einzelnen Formeln.

\subsubsection{Modellmenge}

Sei $A\in For0_{\sum}$ eine Formel. $Mod(A)=\{I\in I_{\sum}\mid I(A)=1\}$. Dies lässt sich auch auf die Formelmenge erweitern.

\subsubsection{Mengenlehre}

Dadurch, dass wir unsere Logik als Mengen ausdrücken können, sind wir in der Lage unsere Atome und Grundformeln der Aussagenlogik in Mengenoperatoren umzuwandeln. Dies ist auf S. 217.

\begin{equation}
    \begin{split}
        \sum     & = \{p, q, r\}             \\
        F        & = \{p\lor q\}             \\
        G        & = \{p\to r\}              \\
        I_{\sum} & = \{I_0, I_1,\dots, I_7\}
    \end{split}
\end{equation}

\begin{equation}
    \begin{array}{cc}
        \begin{array}{c|ccc}
            I_{\sum} & p & q & r \\
            \hline
            I_0      & 0 & 0 & 0 \\
            I_1      & 0 & 0 & 1 \\
            I_2      & 0 & 1 & 0 \\
            I_3      & 0 & 1 & 1 \\
            I_4      & 1 & 0 & 0 \\
            I_5      & 1 & 0 & 1 \\
            I_6      & 1 & 1 & 0 \\
            I_7      & 1 & 1 & 1 \\
        \end{array} &
        \begin{aligned}
            Mod(p)        & = \{I_4, I_5, I_6, I_7\}           \\
            Mod(q)        & = \{I_2, I_3, I_6, I_7\}           \\
            Mod(r)        & = \{I_1, I_3, I_5, I_7\}           \\
            Mod(F)        & = \{I_2, I_3,I_4, I_5, I_6, I_7\}  \\
            Mod(G)        & = \{I_0, I_1, I_2, I_3, I_5, I_7\} \\
            Mod(\neg F)   & = \{I_0, I_1\}                     \\
            Mod(\neg G)   & = \{I_4, I_6\}                     \\
            Mod(F\land G) & = \{I_2, I_3, I_5, I_7\}           \\
            Mod(F\to G)   & = \{I_0, I_1, I_2, I_3, I_5, I_7\} \\
        \end{aligned}
    \end{array}
\end{equation}

\subsection{Tautologie}

Eine Tautologie oder allgemeingültigkeit, falls $val_I(F)=1$ für jede Interpretation $I$. Schreibweise $\models I$.

Da eine Tautologie immer wahr ist, ist die Modellmenge $Mode(F)=I_{\sum}$.

\subsubsection{Beispiele}

$T$, $A\land\neg A$, $A\to A$

\subsection{Unerfüllbarkeit}

Eine Formel heißt unerfüllbar, wenn sie kein Modell hat. z.B. $\bot$, $A\land\neg A$ oder $\{(A\land B), (A\lor\neg B), (\neg A\lor B), (\neg A \lor\neg B)\}$

\subsection{Dualität}

Eine Formel $F$ ist genau dann eine Tautologie, wenn $\neg F$ unerfüllbar ist.

Hierzu machen wir zwei Beweise: Einen für jeder der beiden Richtungen.

Für den Beweis beweisen wir es für ein Beliebiges unbestimmtes $I$ und zeigen dadurch, dass es für alle $I$ gültig ist.

\subsection{Logische Folgerung}

Eine Formel $A$ folgt logisch aus Formelmenge $KB$ gdw. alle Modelle $KB$ auch Modelle von $A$ sind. $(Mod(KB)\subset Mod(A))$ oder geschrieben $KB\models A$.

\end{document}
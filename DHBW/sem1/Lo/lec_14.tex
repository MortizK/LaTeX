\documentclass[a4paper]{article}

%\usepackage{url}

%% Math
\usepackage{mathtools}
%% For Mengen like natural numbers
\usepackage{amsfonts}
%% Für spezielle Symbole
\usepackage{amssymb}

%% Images
\usepackage{import}
\usepackage{xifthen}
\usepackage{pdfpages}
%\usepackage{transparent}

%%% Command for simpler images
\newcommand{\incfig}[1]{%
    \def\svgwidth{\columnwidth}
    \import{./fig/}{#1.pdf_tex}
}

%% Links
\usepackage{hyperref}
\hypersetup{
    colorlinks=true,
    linkcolor=black,
    filecolor=magenta,
    urlcolor=cyan
}

%% Formatting
\usepackage{parskip}

\title{Logik}
\author{Moritz}
\date{January 30, 2025}

\begin{document}
\maketitle
\tableofcontents

\section{Aussagenlogik}

\subsection{Tableaux}

Sei $F$ eine Formeln und $T$ ein geschlossenes Tableau für $F$. Dann ist $F$ unerfüllbar. Wenn es geschlossen ist, ist die Formel unerfüllbar.

\subsubsection{Beweis mit Lemma}

Hierfür der Beweis mithilfe der Induktion:

Nach dem Lemma ist der Begin des Tableau mit nur der Ursprungsformel $F$. Diese ist WAHR.

Wenn wir nun durch unsere Induktionsvoraussetzung davon ausgehen, dass es einen Pfad gibt welche unter $I = 1$ ist.

Nun gibt es eine Fallunterscheidung. Wir expandieren einen Knoten, der Nicht auf dem Pfad, welcher unter $I = 1$ ist, verändert dieser den Pfad nicht.

Wenn wir einen Knoten auf unserem Pfad expandieren müssen wir unter $\alpha$ und $\beta$ unterschieden.

\begin{enumerate}
    \item[$\alpha$] Hier ist der neue Pfad unter $I(\langle M, \alpha_1, \alpha_2\rangle)=1$
    \item[$\beta$] Bei $\beta$ wissen wir das einer von beiden Ästen war sein muss. $I(\langle M, \beta_1\rangle)=1$ oder $I(\langle M, \beta_2\rangle)=1$
\end{enumerate}

\subsubsection{Satz}

Beweis des Lemmas durch Widerspruch.

Wenn unsere Tableau geschlossen ist und wir annehmen, dass F erfüllbar ist. Nach dem Lemma existiert nun ein Ast welcher unter $I(M)=1$ wahr ist. Dies kann aber nicht sein, da ein geschlossenes Tableau per Definition zwei komplementäre Formeln hat.

Widerspruch: Die Annahme war falsch $\to$ F ist unerfüllbar.

Das gleicher für ein vollständiges Tableau, welches erfüllbar sein muss.

\subsubsection{vollständigekeit und Korrektheit}

Sei $F\in For0_{\sum}$ eine Aussagenlogische Formel. Dann gilt:

\begin{enumerate}
    \item Das Tableau lässt sich in n Schritten vervollständigen.
    \item Wenn das Tableau offen ist, so ist $F$ erfüllbar.
    \item Wenn das Tableau geschlossen ist, so ist $F$ unerfüllbar.
\end{enumerate}

\end{document}
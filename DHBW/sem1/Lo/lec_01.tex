\documentclass[a4paper]{article}

%\usepackage{url}

%% Math
\usepackage{mathtools}
%% For Mengen like natural numbers
\usepackage{amsfonts}
%% Für spezielle Symbole
\usepackage{amssymb}

%% Images
\usepackage{import}
\usepackage{xifthen}
\usepackage{pdfpages}
%\usepackage{transparent}

%%% Command for simpler images
\newcommand{\incfig}[1]{%
    \def\svgwidth{\columnwidth}
    \import{./fig/}{#1.pdf_tex}
}

%% Links
\usepackage{hyperref}
\hypersetup{
    colorlinks=true,
    linkcolor=black,
    filecolor=magenta,
    urlcolor=cyan
}

%% Formatting
\usepackage{parskip}

\title{Logik}
\author{Moritz}

\begin{document}
\maketitle
\tableofcontents

\section{Deckblatt der Folie}

Rekursion ist von oben nach untern
Und Induktion ist von unten nach oben

\section{Einleitung}

Welche Erfahrung haben wir?

\subsection{Wenn zu Schnell}

Herr Schulz hat schon 11 mal diese Modul unterrichtet.
Wenn Er zu schnell ist, einfach nach haken.

\subsection{Problem der multiplen Generäle}

Es gibt Problem, die man nicht lösen kann.
Beschließen, durch Theorie, dass es nicht lösbar ist.
Also BEschließen, dass es ausreicht!

\subsection{Klausur}

8 bis 9 Aufgaben.

ca. 1 Punkt pro Minute, also 90P

Wir dürfen 2 Tragbare Gegenstände mitnehmen.

Zum Lesen, statt ausdrucken auch das Tablet.

Es gibt echte Klausuren aus den Vorjahren

\subsection{Übersicht}

Mengenlehre

Nichtprozeurale Programmiermodelle (Funktional: Racket)
Das Ziel ist es mal anders zu denken, statt den Standard

Aussagenlogik

Prädikatenlogik

\subsection{Literatur}

Einiges kann dem Skript entnommen werden. Wie die links zu den digitalen Versionen.

Theoretische Informatik, Dirk W. Hoffmann - eine Empfehlung ab 3 Auflage, wegen Fehlern

Theoretische Informatik I - Logik und MEngenlehre, Karl Stroetmann

Introduction to Mathematical Philosophy, Bertrand Russell (1918) - digital

First-Order Logic, Raymond M. Smullyan (1968) - ist quasi Prädikatenlogik

Grundlagen der Archimetric, \dots

Structure and Interpretation of COmputer Programs, G. J. Sussmann and H. Abelson - gibt es digital

\subsection{Ziele}

Vokabular beibringen um über Informatik reden zu können.

\subsubsection{Methoden}

Modellierung

Anwendungen von und über Logik und Deduktion

Argumentieren über Logik, Deduktion und Programme, sowie dessen Verhalten.

Gibt es requirements, welche zu viel Arbeit machen oder nicht lösbar sind. Wofür wird dass gebraucht oder gibt es eine andere Lösung.

\subsection{Das MIU-Rätsel}

Alle Worte bestehen aus den Buchstaben M, I und U.

mehrere Regeln für diese Worte.

Kommt aus einem Buchstaben

\subsubsection{Formale Aufschreibung}

\begin{equation}
    \begin{split}
        1. xI    & \implies xIU \\
        2. xIIIy & \implies xUy \\
        3. xUUy  & \implies xy  \\
        4. Mx    & \implies Mxx
    \end{split}
\end{equation}

Invariante: alle Worte starten mit einem M

\begin{equation}
    \begin{split}
        MI & -_4MII-_4MIIII-_2MUI-_1-MUIU                    \\
        MI & -_4^3MIIIIIIII-_1MIIIIIIIIU-_2MIIIIIUU-_3MIIIII \\
        MI & -_4^4MI^{16}-_2^5MUUUUUI-_3MUUUI                \\
        MI & - \text{Gibt keine Lösung} - MU
    \end{split}
\end{equation}

\subsubsection{Reflektion}

Wir haben verschiedene Notationen Eingeführt.

Wie $-_4^3$ zur mehrfach ausführung der Regel 4.
Sowie die kurze schreibweise um Buchstaben zu zählen, wie: $MI^4$ für $MIIII$.

\section{Logik}

Logik ist die Lehre von der Bedeutung oder Sinn der Lehre an Sich.

% Problem von Göttle ergänzen
Problem von Göttle ergänzen!

AND in Programm oder if() sind Prädikatenlogik!

\subsection{Deduktionsmethoden}

\subsubsection{Eindeutige Spezifikation}

Syntax

Sematik

\subsubsection{Objektiv richtige Ableitung von neuem Wissen}

Was bedeutet "Richtigkeit"?

Kann man RIchtigkeit automatisch sicherstellen?

Kann man neue Fakten automatisch herleiten?

\subsection{Exkurs Lotzen}

Logische Abfragen erstellen, welche alles Testen können, was ein Kunde haben möchte.

Daraufhin kann man jedem Kunden seine Aussagenkombination geben.

Die Entwickler meinten, viele Funktionen sind zu teuer. Somit wurde ein umstellung. Die Entwickler habe es umgestellt. Um es günstiger zu machen.

Diese Umstellung musste Bewiesen werden. Hierbei beachten: Es git nur LowerAirSpace und HigherAirSpace. Nichts weiteres.

\section{Aufgaben}

Installierung von Racket.

Und ausprobieren von mindestens "Hello World". Auf PDF S. 465

\section{Lehren von Heute}

Wir dürfen eigene Definition für z.B Schreibweisen erstellen.
Wie bei dem MIU-Rätsle.

Zudem kann mann auch mal das Ziel hinterfragen oder schwächen um überhaupt zu einem Ergebnis zu kömmen.
Hier das beispiel mit dem Gemüse der Ziege und dem Wolf, welche auf die andere Insel müsen. Leider passt ins Boot nur Ich und ein andere oder anderes.
Zudem kann die Ziege nicht mit dem Gemüse allein gelassen werden, der Wolf nicht mit der Ziege.
So ist es unlösbar, also lässt man den Wolf zurück.
Weil: wofür braucht man einen Wolf?

\end{document}
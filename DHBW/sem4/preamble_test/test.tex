\documentclass[a4paper,12pt]{article}

% ====================== PREAMBLE ============================
%\usepackage{url}

%% Math
\usepackage{mathtools}
%% For Mengen like natural numbers
\usepackage{amsfonts}
%% Für spezielle Symbole
\usepackage{amssymb}

%% Images
\usepackage{import}
\usepackage{xifthen}
\usepackage{pdfpages}
%\usepackage{transparent}

%%% Command for simpler images
\newcommand{\incfig}[1]{%
    \def\svgwidth{\columnwidth}
    \import{./fig/}{#1.pdf_tex}
}

%% Links
\usepackage{hyperref}
\hypersetup{
    colorlinks=true,
    linkcolor=black,
    filecolor=magenta,
    urlcolor=cyan
}

%% Formatting
\usepackage{parskip}
% ============================================================

\title{Mathematische Modellierung eines Empfehlungssystems\\
\large Ein Demonstrationsdokument für benutzerdefinierte LaTeX-Umgebungen}
\author{Chat GPT}
\date{\today}

\begin{document}
\maketitle
\tableofcontents
\newpage

% ============================================================
\section{Einleitung}
Dieses Dokument demonstriert alle Befehle und Umgebungen aus der angepassten Präambel.
Als Beispielthema dient die mathematische Struktur eines \emph{Recommender Systems}, das Benutzer und Produkte in einem hochdimensionalen Vektorraum modelliert.

Wir verwenden Mengen wie $\R$, Normen wie $\norm{\cdot}$, kodierte Elemente wie \code{User}, Pfade wie \filepath{/data/model/} und Buttons wie \button{Generate Model}.

% ============================================================
\section{Grundlagen der Vektorräume}

\subsection{Definitionen}

\begin{definition}
Ein \emph{Feature-Vektor} eines Benutzers ist ein Element $u \in \R^n$, das Präferenzen in $n$ Dimensionen beschreibt.
\end{definition}

\begin{definition}
Ein \emph{Score-Modell} ist eine Abbildung
\[
s : \R^n \times \R^n \to \R,\qquad s(u, p) = u^\top p.
\]
\end{definition}

\begin{remark}
Das Skalarprodukt ist implizit eine Ähnlichkeitsmetrik.
\end{remark}

% ============================================================
\section{Zentrale Theoreme}

\begin{theorem}
Für zwei normierte Vektoren $u,p \in \R^n$ gilt:
\[
-1 \le s(u,p) = u^\top p \le 1.
\]
\end{theorem}

\begin{proof}
Direkt aus der Cauchy-Schwarz-Ungleichung:
\[
\abs{u^\top p} \le \norm{u}\,\norm{p} = 1.
\]
\end{proof}

\begin{corollary}
Je größer $s(u,p)$ ist, desto ähnlicher sind Benutzer und Produktpräferenzen.
\end{corollary}

% ============================================================
\section{Ein Beispielalgorithmus}

Wir definieren einen einfachen Ranking-Algorithmus.

\begin{pseudocode}
function RankProducts(user u, product-list P):
    for each product p in P:
        compute score s = dot(u, p)
    sort P by score descending
    return P
\end{pseudocode}

Dieser Algorithmus könnte als \method{rankProducts} in einer Klasse \classname{RecommenderEngine} implementiert werden.

% ============================================================
\section{Ein praktisch relevantes Resultat}

\begin{resultbox}
Ein Empfehlungssystem lässt sich mathematisch vollständig durch die Struktur eines Vektorraums mit Skalarprodukt beschreiben.
Dies ermöglicht effiziente Ranking-Algorithmen mit Laufzeit $O(n\cdot m)$ für $n$ Features und $m$ Produkten.
\end{resultbox}

% ============================================================
\section{Wahrscheinlichkeitsbasierte Modellierung}

Wir modellieren Unsicherheit im Nutzungsverhalten durch Zufallsvariablen.

\begin{example}
Sei $X$ die Zufallsvariable: \enquote{Gefällt das Produkt?}.  
Dann gilt:
\[
\E[X] = \Prob(X=1), \qquad \Var(X) = \Prob(X=1)\Prob(X=0).
\]
\end{example}

Solche Variablen können in probabilistischen Empfehlungssystemen wie z.\,B. Bayesian Personalized Ranking benutzt werden.

% ============================================================
\section{Tabelle und Abbildung}

\subsection{Beispieltabelle}

\begin{table}[H]
\centering
\begin{tabular}{lcc}
\toprule
Produkt & Feature 1 & Feature 2 \\
\midrule
A & 0.2 & 0.9 \\
B & 0.7 & 0.1 \\
C & 0.4 & 0.6 \\
\bottomrule
\end{tabular}
\caption{Beispieldaten eines Feature-Raums.}
\label{tab:example}
\end{table}

\subsection{Beispielabbildung}

\begin{figure}[H]
    \centering
    \fbox{\rule{0pt}{4cm}\rule{6cm}{0pt}} % Dummy image
    \caption{Platzhalter-Abbildung eines Embedding-Raums.}
    \label{fig:embedding}
\end{figure}

Wir können darauf verweisen: siehe \figref{fig:embedding} und \tabref{tab:example}.

% ============================================================
\section{Fazit}

Dieses Dokument demonstriert alle in der Präambel definierten Werkzeuge:
mathematische Operatoren, Informatik-Makros, strukturierte Umgebungen, Theoreme und Verweise.

\end{document}
